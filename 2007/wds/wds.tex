\input ../../kaspiTeX/base.tex
\input ../../kaspiTeX/colors.tex
\input ../../kaspiTeX/biblio.tex
\input ../../kaspiTeX/slides.tex


\SetBackground{../../kaspiTeX/fig/bg2.eps}
%\def\FgColor{\rgbBl}
%\nocolor

\font\rpvi=pplr8z at 6pt

\def\comment#1{(\SmallerFonts #1)}
\def\comment#1{\message{* comment[\the\pageno]: #1}}
\def\ar{$\rightarrow$}
\def\Ar{$\Rightarrow$}

\newpage %-------------------------------------------------------------------------------------------
\hbox{}\vfil
\title{Jan Ka\v spar}
\title{TOTEM Experiment and Total Cross-Section at LHC}

\centerline{\figv{2.5cm}{../../kaspiTeX/fig/totemLogo.eps}\figv{2.5cm}{../../kaspiTeX/fig/mffLogo.eps}\figv{25mm}{../../kaspiTeX/fig/fzuLogo.eps}}

\newpage %-------------------------------------------------------------------------------------------
\title{Elastic Scattering at LHC}

\hbox{%
\fig{4cm}{fig/lhc2.eps}\bref{cern web}%
\hskip31mm\raise2.3cm
\vbox{\advance\hsize-7.1cm
\> Large Hadron Collider (LHC)\par
\>> proton-proton collider at CERN\par
\>> energies up to $14\un{TeV}$\par
\>> first test this year, real start probably 2008\par
}}

\vfil
\comment{do not say too much, just present it as an interesting topic}

\hbox{%
\fig{6.9cm}{fig/dsigma,isr.eps}\bref{isr}%
\hskip2mm\raise5mm
\vbox{\advance\hsize-7.1cm\parskip2mm
\> Elastic Scattering\par
\>> $p+p\rightarrow p+p$, i.e. the simplest process\par
\>> still challenge for theory - QCD cannot be used in perturbative regime\par
\>> do not know what to expect \Ar{} exciting\comment{history with ISR, new physics}\par
\>> probes structure of proton in a complementary way to the large experiments (such as ATLAS, CMS, etc.) \comment{it rather sees the proton as a whole than a set of constituents}\par
\>> clear signature process \Ar{} precise\break measurement\par
}}


\newpage %-------------------------------------------------------------------------------------------
\title{Total Cross-section}
\parskip1mm

\> The total pp cross-section is infinite (because of the Coulomb scattering).\par
\> The total cross-section of pp collisions due to the strong interaction is finite.\comment{I'll mean always this in what follows}\par
\> Important parameter of the strong interaction.

\> Interesting for theory -- Froissart-Martin theorem \bref{froissart}:
$$\lim_{E\to\infty} {\si_{\rm tot}(E)\over\log^2 E} < \infty$$\par
\>> present data: the bound might be saturated, i.e. $\si_{\rm tot} \propto \log^2 E$ \ asymptotically\par
\>> need for precise measurement in high energy region \comment{at the moment such a region was only accessible by cosmic ray experiments}

%\centerline{\fig{5cm}{fig/sigma,tot.eps}\bref{islam2002}}
\centerline{\fig{7.5cm}{fig/sigma,tot2.eps}\bref{deile,cudell}}

\parskip0mm


\newpage %-------------------------------------------------------------------------------------------
\title{TOTEM Experiment}
\unskip
\centerline{\rpvi (TOTal and Elastic Measurement)}
\vskip\baselineskip

\comment{smoothly come to its objectives, add the diffraction}
\> objectives are measurement of\par
\bitm
\itm Total Cross-Section -- precise measurement (error $\approx 1\percent$)
\itm Elastic Scattering in a wide kinematic range
\itm Diffractive production
\eitm
\vskip\baselineskip

\line{\hss\fig{15cm}{fig/diagrams.eps}\hss}

\vskip\baselineskip
\> Common property: particles in forward direction ($\eta > 3$) \Ar{} TOTEM needs special detectors.



\newpage %-------------------------------------------------------------------------------------------
\title{Inelastic Detectors}

\centerline{\fig{10cm}{fig/t1,t2.eps}\bref{tdr,deileDesy}}
\vskip\baselineskip
\line{%
\vbox{\divide\hsize2
T1\hfil\break
CSC, size $1-2\un{m}$\hfil\break
\line{\fig{6.5cm}{fig/t1.eps}\bref{tdr}\hss}
}\vbox{\divide\hsize2
T2\hfil\break
GEM, size $\sim 40\un{cm}$\hfil\break
\line{\fig{6.1cm}{fig/t2.eps}\bref{deileDesy}\hss}
}}


\newpage %-------------------------------------------------------------------------------------------
\title{Forward Proton Detectors}

\vskip-2mm
\> need to register very forward protons \Ar{} the detectors must:\par
\>> be far from the interaction point\comment{simple geometrical argument, position sensitive detectors}\par
\>> be as close to beam as possible (Roman Pot mechanism)\comment{why}\par
\vskip2mm

\> system of Roman Pots\comment{2 stations per arm, 147 and 220m stations}\par
\>> each station includes two substations ($4 \un{m}$ apart)\par
\>> each substation has 2 vertical RP and one horizontal RP\par
\>> each RP comprises 5+5 silicon strip detectors (pitch $66\un{\mu m}$) with perpendicular strips.\par
\>> RP manufactured by Vacuum Praha (the first time a Czech company supplies parts of accelerator)

\centerline{\figv{2.3cm}{fig/rp,ip.eps}\figv{2.3cm}{fig/substation.eps}}%
\centerline{\figv{4cm}{fig/station.eps}\hfil\figv{4cm}{fig/rp.eps}\hfil\figv{4cm}{fig/overlap.eps}}
\hbox{\bref{tdr,oriuonno}}



\newpage %-------------------------------------------------------------------------------------------
\title{Kinematics of Elastic Scattering}

\centerline{\fig{5cm}{fig/kinematics.eps}}

\> In a given coordinate frame, one needs 3 parameters to describe an elastic scattering event.\par
\vskip\baselineskip
\> Usually, the azimuthal angle $\ph$ (angle around the beam axis in CM system) is used.\par
\vskip\baselineskip
\> For the remaining 2 parameters, there are two common choices:
\bitm
\itm Mandelstam's variables (capital $P$ denotes four-momentum)
\eq{s = (P_1 + P_2)^2,\qquad t=(P_1' - P_1)^2}
\vskip\baselineskip
\itm CM system parameters ($\approx$ laboratory frame), $p$ denotes (three-)momentum of a proton
\eq{p = \sqrt{s - 4m^2\over 2}\approx {s\over 2},\qquad t = -2p^2 (1-\cos\th)\approx - (p\,\th)^2}
\eitm


\newpage %-------------------------------------------------------------------------------------------
\title{Proton Transport and Reconstruction of $t$}
\line{%
\raise3mm\hbox to5.6cm{\fig{5.5cm}{fig/effectiveLength.eps}}%
\vbox{\advance\hsize-5.6cm\parskip2mm
\> trajectory of a proton emitted from the interaction point (IP)\par
\eqref{\pmatrix{x(z)\cr\th_x(z)\cr} = T_x(z)\,\pmatrix{x(\rm IP)\cr \th_x(\rm IP)\cr}}{transport}
\vskip4mm
\>> analogically for $y$ coordinate\par
\>> $T$ given by magnet layout in the accelerator\par
\>> important matrix elements are:\hfil\break
magnification $T_{11}\equiv v_x$\hfil\break
effective length $T_{12}\equiv L_x$\comment{the transport matrix depends on the energy of the proton too}\par
}}

\vfil
\> positions of hits in detectors (in case of elastic scattering)\comment{symmetric optics}\hfil\break
right arm: $\displaystyle x_r = +L_x \,\th_x({\rm IP}) + v_x\, x({\rm IP}),\qquad \th_x\equiv \th\cos\ph$\hfil\break
left arm:\hskip4mm $\displaystyle x_l = -L_x \,\th_x({\rm IP}) + v_x\, x({\rm IP})$

\vfil
\> $t$-reconstruction
\eqref{t = t_x + t_y = - (p\,\th\cos\ph)^2 - (p\,\th\sin\ph)^2 = -p^2 (\th_x^2 + \th_y^2)}{t reconstruction}
\vskip2mm
\eqref{\th_x = {x_{r} - x_{l}\over 2 L_x},\quad \th_y = {y_{r} - y_{l}\over 2 L_y}}{theta reconstruction}


\newpage %-------------------------------------------------------------------------------------------
\title{Running Scenarios}

\line{%
\vbox{\advance\hsize-7.4cm
\> Physical programme of TOTEM too rich for one regime of the LHC.\par
\vskip\baselineskip
\> Key optics\par
\bitm
\itm $\be^*=1540\un{m}$ - low $|t|$ elastic scattering, total cross-section
$$L_y = 275\un{m},\qquad L_x = 100\un{m}$$
\comment{this is the dedicated optics for extrapolation; vx terms almost zero, i.e. parallel-to-point focusing}%
\vskip3mm
\itm $\be^*=90\un{m}$ - diffractive processes, elastic scattering, total cross-section
$$L_y = 265\un{m},\qquad L_x = 0\un{m}$$
\em{$\th_x$ can NOT be measured}
\comment{this is dedicated to diffraction, but still the extrapolation can be done}
\vskip3mm
\itm low $\be^*$ ($11\un{m}$, $2\un{m}$, $0.5\un{m}$) - diffractive processes, high $|t|$ elastic scattering
\eitm
}%
\hskip3mm
\raise-6mm\vbox{\fig{7cm}{fig/dsigma,models.eps}\hbox{\bref{deile}}}%
}



\newpage %-------------------------------------------------------------------------------------------
\title{Total Cross-Section Measurement}

\line{%
\vbox{\advance\hsize-6.7cm
\> Luminosity Independent method\par
\> based on the Optical Theorem (\bref{formanek qft})\par\vskip-2mm
\eqref{\Im {\cal M^{\rm pp\rightarrow pp}}(t=0) \propto \si^{\rm pp}_{\rm tot}}{optical}
\vskip1mm
\> common notation\par\vskip-3mm
$$\rh = {\Re {\cal M}(0)\over\Im {\cal M}(0)}$$
\> definition of cross-section and luminosity\par
$${\left.\d N\over\d t\right|_0} = {\cal L}{\left.\d \si\over\d t\right|_0} \propto {\cal L}\, |{\cal M}(0)|^2 \propto (1+\rh^2)\, {\cal L}\, \si_{\rm tot}^2 = $$
\vskip2mm
$$ = (1+\rh^2) (N_{\rm el} + N_{\rm inel}) \si_{\rm tot}$$\par
%
}\hskip6mm\raise3mm\vbox{\fig{6cm}{fig/rho.eps}\hbox to6cm{\hfil\bref{cudell}}}}


\eqref{\si_{\rm tot} = {1\over 1+\rh^2} {{\d N/ \d t|_0}\over N_{\rm el} + N_{\rm inel}}, \qquad\qquad {\cal L} = (1+\rh^2)\, {(N_{\rm el} + N_{\rm inel})^2\over {\d N/ \d t|_0}}}{sigma tot}

\> the inputs are:
\halign{\hfil#\ \ &#\hfil\cr
$\rho$			& either can be measured or prediction is used\cr
$\d N/\d t|_0$	& extrapolated from the measured differential rate\cr
$N_{\rm el}$	& elastic rate = integral form the measured differential rate\cr
$N_{\rm inel}$	& inelastic rate measured by T1 and T2\cr
}

\> advantages\par
\>> (quite large) uncertainty of luminosity does not influence the result\par
\>> luminosity is determined simultaneously and independently\par


\newpage %-------------------------------------------------------------------------------------------
\title{Exponential Extrapolation}

\vfil
\line{%
\fig{6cm}{fig/sigmaelastic.eps}\hskip2mm
\vbox{\advance\hsize-6.3cm
\> extrapolation \Ar a parameterisation is needed\par
\vskip\baselineskip
\> in the range $0 \leq |t| \leq 0.35\un{GeV^2}$ the differential cross-section exhibits almost purely exponential fall off\par
\vskip\baselineskip
\> convenient parameterization is
\eqref{{\d\si\over\d t} = |T(s, t)|^2,\qquad T(s, t) = e^{i P(t)}\, e^{M(t)}}{parameterisation}
\vskip1mm
$$\hbox{where } P(t) \hbox{ and } M(t) \hbox{ are polynomials}$$\par
}}
\vfil



\newpage %-------------------------------------------------------------------------------------------
\title{Inclusion of Coulomb Scattering}

\parskip0pt plus1fil
\> pp elastic scattering = strong (hadronic) + electro-magnetic force (Coulomb) interaction

\> interested in hadronic total cross-section only \comment{the one of Coulomb interaction is infinite}

\> A question: given an amplitude for hadronic scattering $T_H$, what is the complete amplitude $T_{H+C}$ describing hadronic+Coulomb interaction?

\> $T_{H+C}(t) \neq T_H(t) + T_C(t)$

\> 2 approaches: based on Feynman diagrams (\bref{wy}) or based on Eikonal representation (\bref{cahn,kundrat}). However, none gives clear solution in full generality.

\> The best solution on market (\bref{kundrat}):\par
\line{\hss
\fig{6.3cm}{fig/interference.eps}\comment{comment the curves,the gap between red and blue c. is rho sensitive}%
\vbox{\advance\hsize-5cm
$$\eqalign{
T^{C+H}(s, t) &= \mp {\alpha s\over t}\,F_C^2(t)\ +\ T^H(s, t) \left[1 \pm i \alpha (A + B) \right]\cr
A &= \int\limits_{t_{min}}^0 \!\!\d t'\ \log{t'\over t}\ {\d F_C^2(t')\over \d t}\,I(t, t')\cr
B &= \int\limits_{t_{min}}^0 \!\!\d t'\ \left({T^H(s, t')\over T^H(s, t)}  - 1\right)\,I(t, t')\cr
I(t, t') &= {1\over 2\pi} \int\limits_0^{2\pi} \d\ph {F_C^2(t'')\over t''}\cr
}\eqnoref{ckl}$$
}\hss}
\vskip\parskip

\> \Eq{parameterisation,ckl} define a model for fitting experimental data




\newpage %-------------------------------------------------------------------------------------------
\title{Errors, errors and errors}

\line{%
\vbox{\advance\hsize-5.2cm
\> bin content fluctuations - negligible\par
\> points placement - negligible (narrow bins)\par
\vskip\baselineskip
\> measured $t$ differs from the actual one\par
\bitm
\itm \em{beam divergence}, \em{crossing-angle}: the collision is not precisely head-to-head, protons scattered at $t=0$ do not propagate parallely to the beam axis
\itm \em{energy smearing}: the energy of colliding protons is not the nominal ($14\un{TeV}$)
\eitm
}\hskip2mm\fig{49mm}{fig/simulation.eps}\comment{show the smearing at t=0}%
}
\bitm\itc=2
\itm error in \em{effective length}: measurement precision $\approx 2\%$
\itm \em{strip pitch}: the pitch of dectors is $66\un{\mu m}$ which defines spacial resolution
\itm \em{alignment} errors: misalignments between single detectors, misalignment of the detector system with respect to the beam 
\eitm
\vfil

\line{%
\fig{5cm}{fig/smearing.eps}\hskip4mm
\raise5mm\vbox{\advance\hsize-55mm
\> errors in $t$

\>> constant shift: $\De t$ function of $t$ only

\>> smearing: $\De t$ depends on $\ph$ too

\vskip\baselineskip
\> antismearing correction\par
\>> an iterative attempt to reduce impact of smearing\par
}}


\newpage %-------------------------------------------------------------------------------------------
\title{Complication at $\mathbf{\be^*}$ = 90 m: only $t_y$ is measured}

\line{%
\raise10mm\vbox{\advance\hsize-6.5cm
\> $L_x = 0\un{m}$ \Ar{} only $t_y$ can be reconstructed \Ar{} $\d\si/\d {t_y}$ measured instead of $\d\si/\d t$\comment{emphasize the ty x t difference}\par
\vskip8mm
\> transformation between p.d.fs. of random variables $t,\ph$ and $t_y,\ph$\par
$$t_y(t, \ph) = t\sin^2 \ph \Rightarrow$$
\eqref{\Rightarrow\quad{\d\si\over\d t_y}(t_y) = {2\over\pi} \int\limits_{0}^{\pi/2} {\d\ph\over\sin^2\ph}\ \ {\d\si\over\d t}\!\left(t_y\over\sin^2\ph\right)}{ty}\par
%
}\hskip5mm\fig{5.95cm}{fig/autoconvolution.eps}}

\> inverse transformation (consequence of azimuthal symmetry)
\eqref{t = t_x + t_y,\qquad t_x = t\cos^2\ph,\quad t_y=t\sin^2\ph}{t decomposition}
\eqref{{\d\si\over\d t_y} = {\d\si\over\d t_x} \quad \Rightarrow \quad {\d\si\over\d t}(t) \propto \int\limits_t^0 \d u\, {\d\si\over\d t_y}(u)\, {\d\si\over\d t_y}(t - u)}{autoconvolution}\par
\>> can be well adapted for discrete case of histograms\par
\vskip3mm
\>> cannot be used because \em{information from low $|t_y|$ region is missing}


\newpage %-------------------------------------------------------------------------------------------
\title{Complication at $\mathbf{\be^*}$ = 90 m: a way out}
\parskip3mm

\> The effects of Coulomb interference are small at this optics.

\> The parameterization of the full cross-section is
$${\d\si\over\d t} = \exp(a + bt + ct^2 + \ldots)$$

\> if only $a$ and $b$ are non-zero: \Eq{ty} gives us
$${\d\si\over\d t}(t) = e^{a\, +\, bt}  \qquad\Rightarrow\qquad  {\d\si\over\d t_y}(t_y) = {1\over\sqrt{\pi}}\, {e^{a\, +\, bt_y}\over\sqrt{|b\,t_y|}}$$

\> if $c$ and higher coefficients do not contribute essentially (the actual situation): one may use the formula for the general case
$${\d\si\over\d t}(t) = e^{a\, +\, bt\, +\, ct^2\, +\, \ldots}  \qquad\Rightarrow\qquad  {\d\si\over\d t_y}(t_y) \approx {1\over\sqrt{\pi}}\, {e^{a\, +\, bt_y\, +\, ct_y^2\, +\, \ldots}\over\sqrt{|b\,t_y|}}$$

\parskip0mm

\newpage %-------------------------------------------------------------------------------------------
\title{Testing\&Tuning the Procedures}

\> test chain (contains a lot of C++)
\bitm
\itm MC generator based on a model of elastic scattering (5+1 models)
\itm adding errors: vertex position, beam divergence, crossing-angle
\itm transport of protons from IP to the detectors (using non-perfect optical functions)
\itm simulation of proton-detector interaction, digitization, tracking ...  (resulting in hit coordinates)
\itm reconstruction of $t$ or $t_y$ and building histogram of differential cross-section
\itm application of the fitting procedure
\itm comparison of the extrapolated value with the model
\eitm

\parskip1mm
\> important parameters\par
\>> upper/lower bound of fit, size of bins\par
\>> grades of polynomials in the fitting model (for modulus and for phase)\par
\>> number of anti-smearing iterations\par
\>> the way of Coulomb separation\par

\vskip\baselineskip
\> the goal - to find optimal parameters which minimize extrapolation deviations for all models used

\parskip0mm


\newpage %-------------------------------------------------------------------------------------------
\title{Sample Results (I)}

Preliminary results for the $\be^* = 90 m$ optics. Fitting performed with model
$${\d\si\over\d t} = \exp(a + bt + ct^2 + dt^3)$$
in $|t_y|$ range form the indicated lower bound to $0.25\un{GeV^2}$. The natural binning given by strip pitch was used.
\vskip\baselineskip

\line{\hfil\figv{6.5cm}{fig/extrapolationDeviation.eps}\hfil\figv{6.5cm}{fig/extrapolationUncertainty.eps}\hfil}

\comment{Band of models, Islam sticking out. General offset -2 percent}


\newpage %-------------------------------------------------------------------------------------------
\title{Sample Results (II)}
\parskip2mm

\> uncertainties\par

\>> $\rh$: prediction used (\bref{cudell}), relative error $4.3\%$, error of term $1/(1+\rh^2)$ approx. $0.16\%$

\>> $\d N/\d t|_0$: relative error $< 5\%$

\>> $N_{\rm el}$: high correlation with $\d N/\d t|_0$ \Ar{} partial error cancellation

\>> $N_{\rm inel}$: relative error $1\%$

\> error propagation
$$\si_{\rm tot} = {1\over 1+\rh^2} {{\d N/ \d t|_0}\over N_{\rm el} + N_{\rm inel}}$$

\>> $\si_{\rm tot}$: relative error $\approx 4.5\%$, i.e. error $\approx 5\un{mb}$

\parskip0mm

\newpage %-------------------------------------------------------------------------------------------
\hbox{}\vfil
\title{Thanks for attention.}
\vskip\baselineskip
\title{Any questions?}

\newpage %-------------------------------------------------------------------------------------------
\title{References}
\def\bc{, }
\baselineskip=15pt
\setbox\strutbox=\hbox{\vrule height8pt depth3pt width0pt}
\PrintReferences{ref.bib}

\end
