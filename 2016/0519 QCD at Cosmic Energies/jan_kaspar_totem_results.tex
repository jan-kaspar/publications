\input slides.tex
\input utf8-t1

\newpage %-------------------------------------------------------------------------------------------

\def\author{Jan Kašpar}
\def\caption{QCD at Cosmic Energies -- VII}
\def\date{19 May, 2016}

\newpage %-------------------------------------------------------------------------------------------
\hbox{}\vfil
\title{TOTEM}
\vfil
\centerline{\bf Jan Kašpar}
\centerline{on behalf of the TOTEM collaboration}
\vfil
\line{\hss\hss
	\fig[,25mm]{fig/logo_totem_blue.pdf}\hss
	\fig[,25mm]{fig/logo_cern_blue.pdf}\hss
	\fig[,25mm]{fig/logo_infn.pdf}\hss
	\hss}
\vfil
\centerline{\caption}
\centerline{\date}
%\vfil

\footline={}

\newpage %-------------------------------------------------------------------------------------------
\title{Outline}

\> projects and physics programme

\> detector apparatus, principle of proton measurement

\> physics analyses and results, TOTEM alone
\>> elastic scattering
\>> inelastic cross-section
\>> total cross-section
\>> single diffraction
\>> double diffraction
\>> central diffraction
\>> forward charged-particle multiplicity ($\d N/\d\eta$)

\> physics analyses and results, TOTEM + CMS
\>> forward charged-particle multiplicity ($\d N/\d\eta$)
\>> central diffraction
\>> inelastic event classification

\> outlook - future projects

\> \link{http://totem.web.cern.ch/Totem/publ_new.html}{list of TOTEM publications}

\newpage %-------------------------------------------------------------------------------------------
\title{TOTEM projects and physics programme}

\> add drawings + labels

\hrule

\> TOTEM
\>> si tot, ES, SD, DD, CD
\>> common features: rapidity gaps, particles in very forward region, intact protons

\> TOTEM + CMS
\>> independent experiments, exchange of triggers

\> CT-PPS (CMS-TOTEM Precision Proton Spectrometer)
\>> full integration, new dedicated 


\newpage %-------------------------------------------------------------------------------------------
\title{Detector apparatus}

\> \em{Inelastic telescopes T1 and T2}: charged particles from inelastic collisions

\centerline{%
	\fig[10cm]{fig/cms_totem_gray.pdf}%
	\hskip3mm
	\raise35mm\vtop{\hsize=5cm
		\> T1: $3.1 < |\et| < 4.7$,\\ $p_{\rm T} > 100 MeV$
		\vskip4mm
		\> T2: $5.3 < |\et| < 6.5$,\\ $p_{\rm T} > 40 MeV$
	}
}

\vfil

\> \em{Roman Pots (RP)}: elastic and diffractive protons close to outgoing beam

\centerline{\fig[13cm]{fig/RP_stations_run_1_2.pdf}}

\>> station at 147m in Run I $\rightarrow$ station 220m in Run II

\vfil

\> all detectors: symmetric about IP5, trigger capable, radiation tolerant


\newpage %-------------------------------------------------------------------------------------------
\title{Telescope 1 (T1)}

\SmallerFonts

\line{\fig[7cm]{fig/t1_installed.jpg}\hskip3mm\raise15mm\vbox{\hsize7.5cm
\> installed inside CMS end-caps
\> at $7.5$ to $10.5\un{m}$ from the IP
\> one \em{telescope} on each side of IP
\> each telescope consists of two \em{quarters}
}\hss}

\vfil
\line{\fig[7cm]{fig/t1_quarter.jpg}\hskip3mm\raise1mm\vbox{\hsize7.5cm
\> each quarter formed by 5 \em{planes} equally spaced along beam
\> each plane consists of 3 trapezoidal \em{CSC detectors}, each covering $60\un{^\circ}$ in azimuth
\> Cathode Strip Chamber: gaseous detector with 3 read-out coordinates (at $60\un{^\circ}$ wrt.~each other)
}\hss}

\newpage %-------------------------------------------------------------------------------------------
\title{Telescope 2 (T2)}

\SmallerFonts

\line{\AddBckg[0.2mm]{\fig[7.5cm]{fig/t2_installed.jpg}}\hskip3mm\raise15mm\vbox{\hsize7.0cm
\> installed inside CMS shielding between HF and Castor calorimeters
\> centred about $13.5\un{m}$ from the IP
\> one \em{telescope} on each side of IP
\> each telescope consists of two \em{quarters}
}\hss}

\vskip-5mm

\line{\AddBckg[0.2mm]{\fig[7.5cm]{fig/t2_quarter.jpg}}\hskip3mm\vbox{\hsize6.8cm
\> each quarter formed by 10 semi-circular \em{planes}, assembled in 5 back-to-back mounted pairs
\> each plane equipped with a \em{Gas Electron Multiplier} detector
\>> gaseous detector, electron multiplication by 3 perforated foils ($2\un{mm}$ separation)
\>> radial segmentation: \em{strips} (resolution $\approx 0.15\un{mm}$)
\>> coarse radial$\times$azimuthal segmentation: \em{pads} (for triggering, azimuthal resolution $0.8\un{^\circ}$)
}\hss}

\newpage %-------------------------------------------------------------------------------------------
\title{Roman Pots (RPs)}

\SmallerFonts

\line{\fig[5cm]{fig/rp_station.pdf}\hskip3mm\raise20mm\vbox{\hsize10cm
\> \em{stations} installed at $\pm 220\un{m}$ in the outgoing LHC beam-pipe
\> each station has two \em{units}, separated by $\approx 5\un{m}$
}\hss}

\line{\fig[5cm]{fig/rp_unit.pdf}\hskip3mm\raise1mm\vbox{\hsize10cm
\> each unit contains 3 \em{Roman Pots}: top, bottom and horizontal
\> Roman Pot = movable beam-pipe insertion
\>> \Em{beam unstable} $\Rightarrow$ RPs retracted to safe position
\>> \Em{beam stable} $\Rightarrow$ RPs as close to beam as reasonable
\> typical approach: $10\un{\si_{\rm beam}}$ (record $3\un{\si_{\rm beam}}$)
}\hss}

\line{\fig[5cm]{fig/rp_pot.pdf}\hskip3mm\raise3mm\vbox{\hsize10cm
\> Roman Pot: container for sensors
\> TODO: add picture of cylindrical ones and possibly new ferrites
}\hss}

\newpage %-------------------------------------------------------------------------------------------
\title{``Edgeless'' silicon sensors}

\SmallerFonts

\line{\fig[6cm]{fig/rp_package.pdf}\hskip3mm\raise39mm\vbox{\hsize10cm
\> each RP contains a \cYe\em{package}\cFg{} of 10 silicon sensors
\> 5 pairs of back-to-back mounted strip sensors
}\hss}

\vfil
\line{\fig[6cm]{fig/rp_hybrid_new.pdf}\hskip3mm\raise23mm\vbox{\hsize10cm
\> custom developed \cYe\em{``edgeless'' sensors}\cFg\\
$\Rightarrow$ \Em{insensitive edge $\approx 50\un{\mu m}$} (standard about $1\un{mm}$)
\> single-sided $\rm p^+$-n
\> 512 strips at pitch of $66\un{\mu m}$, at $45\un{^\circ}$ wrt.~cut edge
\> operated at $\approx -20\un{^\circ C}$, bias voltage $\approx 100\un{V}$
}\hss}


\newpage %-------------------------------------------------------------------------------------------
\title{Proton measurement with RPs}

\SmallerFonts

\> \em{proton transport}: described as in linear optics

\line{\hss\fig[11.5cm]{fig/ttm_proton_transport.pdf}\hss}

\vfil

\cBlack
\line{\hss$\displaystyle
\pmatrix{x\cr \th_x\cr y\cr \th_y\cr \xi}_{\rm RP} =
\underbrace{\pmatrix{
	v_x & L_x & \cdot & \cdot & D_x\cr
	\cdot & \cdot & \cdot & \cdot & \cdot\cr
	v_y & L_y & \cdot & \cdot & D_y\cr
	\cdot & \cdot & \cdot & \cdot & \cdot\cr
	\cdot & \cdot & \cdot & \cdot & 1\cr
}}_{\hbox{product from all lattice elements}}
\pmatrix{x^*\cr \th_x^*\cr y^*\cr \th_y^*\cr \xi}_{\rm IP}
$
\hskip10mm
\lower15mm\vbox{
\hbox{$\th_x^*, \th_y^*$: scattering angles}
\hbox{$x^*, y^*$: vertex}
\hbox{$\xi = \De p / p$: momentum loss}
\hbox{\strut}
\hbox{\Em{optical functions:}}
\hbox{effective length $L$}
\hbox{magnification $v$}
\hbox{dispersion $D$}
}
\hss}

\vfil

\> \em{proton reconstruction}: inverted transport RPs $\longrightarrow$ IP

\>> optical parameters functions of $\xi$ $\Rightarrow$ reconstruction is non-linear problem

\>> \cRed good knowledge of optics is crucial

\newpage %-------------------------------------------------------------------------------------------
\title{LHC optics}

\> simulation of centra diffraction for 2 different optics

\vskip2mm
\line{%
	\hss
	\hskip2mm
	\vtop{\hsize7.7cm
		\centerline{{\bf\cRed low $\bf\be^*$ (LHC standard)\cBlack}}
		\centerline{\SmallerFonts$\displaystyle L_x \approx 1.7\un{m},\quad L_y \approx 14\un{m},\quad D_x \approx 8\un{cm}$}
		
		\centerline{diffractive protons in \em{horizontal RPs}}
		\line{\hss\fig[,4.5cm]{fig/hit_dist_0p7.png}\hss}
	}%
	\vtop{\hsize7.7cm
		\centerline{\bf\cRed$\bf\be^* = 90\un{m}$ (special for RP)\cBlack}
		\centerline{\SmallerFonts$\displaystyle L_x \approx 0,\quad L_y \approx 260\un{m},\quad D_x \approx 4\un{cm}$}
		
		\centerline{diffractive protons in \em{vertical RPs}}
		\line{\hss\fig[,4.5cm]{fig/hit_dist_90.png}\hss}
	}%
	\hss
}

\vfil
\> optics typically ``labelled'' by \em{$\be^* \equiv$ betatron function at IP}
\>> beam width: $\sqrt{\ep \be}$, $\ep$: beam emittance
\>> beam angular divergence: $\sqrt{\ep / \be}$
\>> luminosity $\propto \hbox{(beam width at IP)}^{-2} \propto 1/\be^*$


\newpage %-------------------------------------------------------------------------------------------
\title{Typical run scenarios}

\centerline{\SmallerFonts ($t \approx -p^2 \th^2$: four-momentum transfer squared; $\xi = \De p / p$: fractional momentum loss)}
\vskip1mm

{

\advance\hsize\horizontalmargin\line{\kern-\horizontalmargin
\fig[,4.37cm]{fig/acceptance_2.pdf}
\fig[,4.37cm]{fig/acceptance_90.pdf}
\fig[,4.37cm]{fig/acceptance_1535.pdf}\hss
}

\line{\kern-\horizontalmargin\SmallerFonts\hss
\vtop{\hsize4.7cm\obeylines\leftskip0pt plus1fil\rightskip0pt plus1fil\parfillskip0pt
	\Em{\NormalFonts\bf $\bf\be^* = 0.55\un{m}$}
	\hrule\vskip1mm
	${\cal L} \approx 10^{33}\un{cm^{-2}s^{-1}}$
		\vskip\baselineskip
	\em{elastic scattering}: high $|t|$
		\vskip\baselineskip
	\em{diffraction}: $\xi > 0.01$, low cross-section processes
}
\hss
\vtop{\hsize4.7cm\obeylines\leftskip0pt plus1fil\rightskip0pt plus1fil\parfillskip0pt
	\Em{\NormalFonts\bf medium $\bf\be^* = 90\un{m}$}
	\hrule\vskip1mm
	${\cal L} \approx 10^{28}\un{cm^{-2}s^{-1}}$
		\vskip1\baselineskip
	\em{elastic scattering}: low to mid $|t|$
		\vskip\baselineskip
	\em{diffraction}: any $\xi$ for $|t| > 0.01\un{GeV^2}$
}
\hss
\vtop{\hsize4.7cm\obeylines\leftskip0pt plus1fil\rightskip0pt plus1fil\parfillskip0pt
	\Em{\NormalFonts\bf high $\bf\be^* = 1535\un{m}$}
	\hrule\vskip1mm
	${\cal L} \approx 10^{27}\un{cm^{-2}s^{-1}}$
		\vskip\baselineskip
	\em{elastic scattering}: very low $|t|$ (Coulomb-nuclear interference)
}
\hss}

}


\newpage %-------------------------------------------------------------------------------------------
\ctitle{Elastic scattering}{$p + p \rightarrow p + p$}

\> data overview

\centerline{\fig[14cm]{fig/es_summary.pdf}}

\> selection: two anti-collinear protons from the save vertex

\> (almost) purely data-driven analysis

\> different $|t|$ probe different physics regimes: Coulomb interference, diffractive cone, dip-bump, transition to pQCD

% ORAL: anticipate some features of the data: shift of dip position, cone steepening

\newpage %-------------------------------------------------------------------------------------------
\ctitle{Elastic scattering}{Trends}

\centerline{%
	\fig[,5cm]{fig/es_summary_detail.pdf}%
	\hfil
	\fig[,5cm]{fig/B_s.pdf}%
}

\> dip position: moves to lower $|t|$ as $\sqrt s$ increases ($8$ and $13\un{TeV}$ preliminary though!)

\> forward slope $B = {\d\over\d t} \log \left.{\d\si\over\d t} \right |_{t=0}$ increases with $\sqrt s$

\newpage %-------------------------------------------------------------------------------------------
\ctitle{Elastic scattering}{Non-exponentiality at low $|t|$}

\> diffraction cone: ``looks almost exponential''
\>> magnify deviations $\Rightarrow$ plot \cThird ${(\d\si/\d t - \hbox{ref. exp.}) / \hbox{ref. exp.}}$

\> $\be^* = 90\un{m}$ measurements at different energies (stat.~unc.~only):

\centerline{%
	\fig[,4.8cm]{fig/t_dist_rel_7.pdf}%
	\hskip-1mm
	\fig[,4.8cm]{fig/t_dist_rel_8.pdf}%
	\hskip-1mm
	\fig[,4.8cm]{fig/t_dist_rel_13.pdf}
}

\vskip-3mm

\line{\raise37mm\vtop{\hsize9.4cm
\> non-exponentiality observed at $8$ and $13\un{TeV}$!
\>> $8\un{TeV}$: \em{$7\un{\si}$ significance}\hfil $\longrightarrow$
\>> $13\un{TeV}$: preliminary results
\>> non-exponentiality of observed cross-section:

\cThird
\vskip-3mm
$$\d\si /\d t = \hbox{nuclear} + \hbox{Coulomb} + \hbox{interference}$$
}
	\hskip3mm
	\fig[5.6cm]{fig/t_dist_8TeV_non_exp.pdf}
	\hss
}


\newpage %-------------------------------------------------------------------------------------------
\ctitle{Elastic scattering}{Coulomb interference}

\> $8\un{TeV}$ data with $\be^* = 1000\un{m}$ optics
\>> RPs very close to the beam: $\approx 3\un{\si_{\rm beam}}$
\>> $|t|_{\rm min} \approx 6\cdot10^{-4}\un{GeV^2}$

\vfil
\centerline{\fig[11cm]{fig/coulomb_components.pdf}}


\newpage %-------------------------------------------------------------------------------------------
\ctitle{Elastic scattering}{Coulomb interference}

\> observed cross-section
\vskip-3mm
\cBlack
$${\d\si\over\d t} \propto \left|
\underbrace{\vcenter{\fig[,2cm]{fig/el_diagram_C.pdf}} + \cdots}_{\vbox to 0pt{\SmallerFonts\hbox{Coulomb}\hbox{amplitude}\vss}}
\quad +
\underbrace{\vcenter{\fig[,2cm]{fig/el_diagram_H.pdf}}}_{\vbox to 0pt{\SmallerFonts\hbox{hadronic}\hbox{amplitude}\vss}}
+
\underbrace{\vcenter{\fig[,2cm]{fig/el_diagram_C_H.pdf}} + \cdots}_{\vbox to 0pt{\SmallerFonts\hbox{``interference''}\hbox{terms}\vss}}
\right|^2$$

\vskip10mm

\> interference formula = summation for practical applications
\>> \Em{simplified West-Yennie (SWY)}: QFT framework, traditional but \em{heavy simplifications (constant hadronic phase, constant slope})
\>> \Em{Cahn} or \Em{Kundr\' at-Lokaj\' i\v cek (KL)}: eikonal framework, no explicit simplifications

\> interference $\Rightarrow$ phase of hadronic amplitude exposed in cross-section
\>> phase $t$-dependence needs to be considered in analysis
\>> constraints from data $\Rightarrow$ \Em{determination of $\rh$ parameter}

\vskip-3mm
\cThird
$$\rh = \left. {\Re {\cal A}^{\rm H}\over \Im {\cal A}^{\rm H}} \right|_{t=0}$$


\newpage %-------------------------------------------------------------------------------------------
\ctitle{Elastic scattering}{Coulomb interference -- Analysis strategy}

\centerline{\EM{central question:}}
\centerline{observed non-exponentiality -- due to hadronic, Coulomb or both?}

\vfil

\> fits with 2 different assumptions on hadronic component
\>> \em{purely-exponential} -- non-exponentiality due to Coulomb (+interference)\\
$\rightarrow$ degree-1 polynomial in exponent
\>> \em{flexible enough} to describe non-exponentiality even without Coulomb\\
$\rightarrow$ degree-3 polynomial in exponent

\vfil

\> role of hadronic phase $t$-dependence?
\>> largest impact: rate of change at low $|t|$
\>>> same quantity controls behaviour in impact-parameter space
\>> considered two families
\>>> \em{central}: preferably low impact parameters
\>>> \em{peripheral}: preferably higher impact parameters

\newpage %-------------------------------------------------------------------------------------------
\ctitle{Elastic scattering}{Coulomb interference -- Fits}

\SmallerFonts

\line{%
	\fig[7cm]{fig/coulomb_exp1_fit.pdf}%
	\raise45mm\vtop{\hsize8cm
	$\Leftarrow$ \EM{purely-exponential hadronic amplitude}
		\>> constant phase \em{excluded} (with both SWY and KL formulae) $\Rightarrow$ application of SWY formula excluded too
		\>> peripheral phase not excluded by data, but \em{disfavoured}
		\>>> $\rh$ value outside a consistent pattern of other fits and theoretical predictions
		\>>> number of theoretical reasons for non-exponential hadronic amplitude
	}
	\hss
}

\line{%
	\fig[7cm]{fig/coulomb_exp3_fit.pdf}%
	\raise30mm\vtop{\hsize8cm
	$\Leftarrow$ \EM{non-exponential hadronic amplitude}
		\>> both constant and peripheral phases compatible with data $\Rightarrow$ \em{centrality not necessity}
	}
	\hss
}

\newpage %-------------------------------------------------------------------------------------------
\ctitle{Elastic scattering}{Coulomb interference -- Result summary}

\> \em{$\rh$} -- first LHC determination from Coulomb-hadronic interference:
\vskip-4mm
\cBlack
$$\rh = 0.12 \pm 0.03$$

\centerline{\fig[11cm]{fig/rho_s.pdf}}

\> \em{$\si_{\rm tot}$} -- results consistent with previous publications, but for the first time:
\>> Coulomb component explicitly separated
\>> determined in the same analysis as $\rh$


\newpage %-------------------------------------------------------------------------------------------
\ctitle{Elastic scattering}{Exclusion power in Run I}

\> model predictions (prior to Run I) vs.~TOTEM data at $\sqrt s = 7\un{TeV}$:

\centerline{\fig[13cm]{fig/cmp_totem_data_7TeV.pdf}}

\>> \em{no model compatible with data!}

\> surprisingly?: little reaction after Run I (statement from 2015)
\>> most often: no change at all or simple parameter refit
\>> only Islam abandoned one mechanism of large $|t|$ scattering
\>> exclusion of physics mechanisms (model independent)?

\newpage %-------------------------------------------------------------------------------------------
\ctitle{Elastic scattering}{Structures at high $|t|$ ?}

\> $\sqrt s = 13\un{TeV}$: very preliminary, but already very strong results

\centerline{%
	\fig[,6.5cm]{fig/t_dist_13.pdf}%
	\hskip1mm
	\raise57mm\vtop{\hsize7cm
		\centerline{\SmallerFonts\cBlack model predictions:}
		\vskip1mm
		\fig[,4.5cm]{fig/models_13TeV.pdf}%
		\vskip-1mm
		\centerline{\SmallerFonts\em{oscillations in almost each model}}
	}%
}

\line{\vbox{\hsize10.8cm
\> \em{high-$|t|$: no structures!}
\>> rules out many models
\>> rules out physics mechanism: ``optical'' models
\>> physics interpretation: transition between diffraction and pQCD? \hfill $\Rightarrow$ \hfill e.g.~Donnachie-Landshoff:
}\hskip4mm\hbox{\fig[4cm]{fig/tripple_gluon.png}}\hss
}

\newpage %-------------------------------------------------------------------------------------------
\title{Inelastic and total cross-section}

\> \em{inelastic cross-section}: event counting with T2 (and T1)
\>> $95\un{\%}$ of inelastic events have at least 1 track in the T2 region
\>> only one significant MC correction: contribution from low mass diffraction

\> 3 methods to determine \em{total cross-section}

\centerline{\fig[13cm]{fig/tot_cs_methods.pdf}}

\>> $\rh$: from TOTEM ($8\un{TeV}$, $\be^* = 1000\un{m}$) or from external source (e.g.~COMPETE)


\newpage %-------------------------------------------------------------------------------------------
\title{Cross-section results}

\centerline{%
	\fig[,7cm]{fig/sigma_tot_el_inel_cmp.pdf}%
	\hskip2mm
	\fig[,7cm]{fig/sigma_el_to_sigma_tot.pdf}%
}

\> $\sqrt s = 2.76$ and $13\un{TeV}$ analyses ongoing


\newpage %-------------------------------------------------------------------------------------------
\title{Single diffraction: $p + p \rightarrow p + X$}

\centerline{\fig[13cm]{fig/topology_sd_T2_opp.pdf}}

\> double determination of $\xi \approx \e^{-\De\et}$
\>> from RPs
\>> from rapidity gap $\De\et$ (T1/T2)

\> mass of diffractive system:\cBlack $m(x) = \sqrt{s \xi}$ 

\> $\xi$ resolution from RPs: $\approx 0.9\un{\%}$ $\Rightarrow$ mass bins given by which T1/T2 active

\> available data
\>> $7\un{TeV}$: TOTEM-standalone analysis in progress
\>> $8$ and $13\un{TeV}$: common data with CMS

\newpage %-------------------------------------------------------------------------------------------
\ctitle{Single diffraction}{Preliminary results at $7\un{TeV}$}

\centerline{\fig[16cm]{fig/sd_7TeV.png}}

\newpage %-------------------------------------------------------------------------------------------
\title{Double diffraction: $p + p \rightarrow X + Y$}

\centerline{\fig[13cm]{fig/topology_dd_lab.pdf}}

\> experimental challenge: background (non-diffractive, SD pile-up)
\>> sub-sample with signal $\gg$ background:  $2\times$T2 and T1 veto
\>> non-diffractive background: control sample $2\times$T2 + $2\times$T1
\>> SD background: control sample $1\times$T2 + $0\times$T1

\> cross-section as function of $\et_{\rm min}$ on both sides
\>> challenge: reconstructed $\et_{\rm min}$ $\longrightarrow$ true/generator $\et_{\rm min}$\\
$\Rightarrow$ 2 $\et_{\rm min}$ bins only

\> available data
\>> $7\un{TeV}$: results published
\>> $8$ and $13\un{TeV}$: common data with CMS (improvement expected)

\newpage %-------------------------------------------------------------------------------------------
\ctitle{Double diffraction}{Results at $7\un{TeV}$}

\> measurement

\centerline{\fig[12cm]{fig/DD_meas.png}}

\> comparison to Monte Carlos

\centerline{\fig[12cm]{fig/DD_mc_comp.png}}

\newpage %-------------------------------------------------------------------------------------------
\title{Central diffraction: $p + p \rightarrow p + X + p$}

\centerline{\fig[13cm]{fig/topology_cd.pdf}}

\> available data
\>> $7\un{TeV}$: TOTEM only, analysis started
\>> $8\un{TeV}$: common data with CMS, analysis started
\>> $13\un{TeV}$: common data with CMS

\> $\be^* = 90\un{m}$: all $\xi$ visible, but resolution $\approx 0.9\un{\%}$

\> experimental challenge: background (pileup ES/beam halo + inelastic)
\>> anti-elastic cuts: anti-collinearity
\>> anti-beam-halo cuts: $|y| > 11\un{mm}$

\newpage %-------------------------------------------------------------------------------------------
\ctitle{Central diffraction}{First results at $7\un{TeV}$}

\> $|t_y|$ distribution: all $\xi$ values, only acceptance correction

\line{\raise8mm\vbox{\hsize70mm
\> estimate of $\si_{\rm CD}$

\cBlack\SmallerFonts
$${\d^2\si_{\rm CD}\over \d t_1 \d t_2} = C \e^{-B t_1} \e^{-B t_2}$$

$$\Downarrow$$

$$ \si_{\rm CD}
= \int\limits^{0}_{-\infty} \d t_1 \int\limits^{0}_{-\infty} \d t_2\ C \e^{-B t_1} \e^{-B t_2} \approx 1\un{mb}$$
}\hskip1mm\fig[8.5cm]{fig/cd_ty_dist_7TeV.png}\hss}

\newpage %-------------------------------------------------------------------------------------------
\title{Forward particle multiplicities, TOTEM and TOTEM + CMS}


\newpage %-------------------------------------------------------------------------------------------
\title{}


\newpage %-------------------------------------------------------------------------------------------
\title{}


\newpage %-------------------------------------------------------------------------------------------
\title{Central diffraction with CMS}


\newpage %-------------------------------------------------------------------------------------------
\title{}


\newpage %-------------------------------------------------------------------------------------------
\title{}


\newpage %-------------------------------------------------------------------------------------------
\title{Inelastic event classification}


\newpage %-------------------------------------------------------------------------------------------
\title{}

\newpage %-------------------------------------------------------------------------------------------
\title{Outlook: Odderon}


\newpage %-------------------------------------------------------------------------------------------
\title{Outlook: timing}


\newpage %-------------------------------------------------------------------------------------------
\title{}


\newpage %-------------------------------------------------------------------------------------------
\title{Outlook: CT-PPS}


\newpage %-------------------------------------------------------------------------------------------
\title{}


\newpage %-------------------------------------------------------------------------------------------
\title{Summary}


\newpage %-------------------------------------------------------------------------------------------
\hbox{}%
\vfil
\title{Backup}


\newpage %-------------------------------------------------------------------------------------------
\title{Backup topics}

\> RP alignment

\> Optics refinement with RP data

\> Coulomb analysis
\>> interference formulae
\>> phase choices

\vfil
\eject
\bye
