\input slides.tex
\input utf8-t1

\newpage %-------------------------------------------------------------------------------------------

\def\author{Jan Kašpar}
\def\caption{QCD at Cosmic Energies -- VII}
\def\date{19 May, 2016}

\newpage %-------------------------------------------------------------------------------------------
\hbox{}\vfil
\title{TOTEM}
\vfil
\centerline{\bf Jan Kašpar}
\centerline{on behalf of the TOTEM collaboration}
\vfil
\line{\hss\hss
	\fig[,25mm]{fig/logo_totem_blue.pdf}\hss
	\fig[,25mm]{fig/logo_cern_blue.pdf}\hss
	\fig[,25mm]{fig/logo_infn.pdf}\hss
	\hss}
\vfil
\centerline{\caption}
\centerline{\date}
%\vfil

\footline={}

\newpage %-------------------------------------------------------------------------------------------
\title{Outline}

\hbox{}
\vskip 0pt plus0.3fil

\centerline{\vtop{\hsize9cm
\itskip5mm
\> TOTEM projects and physics programme
%
\> detector apparatus,\\ principle of proton measurement
%
\> physics analyses and results: TOTEM alone
\iffalse
\>> elastic scattering
\>> inelastic cross-section
\>> total cross-section
\>> single diffraction
\>> double diffraction
\>> central diffraction
\>> forward charged-particle multiplicity ($\d N/\d\eta$)
\fi
%
\> physics analyses and results: TOTEM + CMS
\iffalse
\>> forward charged-particle multiplicity ($\d N/\d\eta$)
\>> central diffraction
\>> inelastic event classification
\fi
%
\> upgrades, outlook for 2016
}}

\vfil

\cBlack
\centerline{ list of TOTEM publications: }
\centerline{ \link{http://totem.web.cern.ch/Totem/publ_new.html}{http://totem.web.cern.ch/Totem/publ\_new.html} }

\newpage %-------------------------------------------------------------------------------------------
\title{TOTEM projects and physics programme}

\> \EM{TOTEM}
\>> LHC experiment dedicated to measurement of:

\centerline{\em{total cross-section, elastic scattering and diffractive processes}}
\>> common features: rapidity gaps, particles in very forward region, surviving protons $\Rightarrow$ special detectors

\vfil
\> \Em{TOTEM + CMS}
\>> both experiments at LHC Interaction Point 5
\>> excellent pseudorapidity coverage: optimal for hard diffraction studies
\>> cooperation mode: independent experiments, exchange of triggers

\vfil
\> \Em{CT-PPS} (CMS-TOTEM Precision Proton Spectrometer)
\>> all subdetectors fully integrated under CMS
\>> dedicated detectors for high-pileup environment (timing, pixels instead of strips)


\newpage %-------------------------------------------------------------------------------------------
\title{Detector apparatus}

\> \Em{Inelastic telescopes T1 and T2}: charged particles from inelastic collisions

\centerline{%
	\fig[10cm]{fig/cms_totem_gray.pdf}%
	\hskip3mm
	\raise35mm\vtop{\hsize=5cm
		\> \cBlack T1: $3.1 < |\et| < 4.7$,\\ $p_{\rm T} > 100 MeV$
		\vskip4mm
		\> \cBlack T2: $5.3 < |\et| < 6.5$,\\ $p_{\rm T} > 40 MeV$
	}
}

\vfil

\> \Em{Roman Pots (RP)}: elastic and diffractive protons close to outgoing beam

\centerline{\fig[13cm]{fig/RP_stations_run_1_2.pdf}}

\>> station at 147m in Run I $\rightarrow$ station 210m in Run II

\vfil

\> all detectors: symmetric about IP5, trigger capable, radiation tolerant


\newpage %-------------------------------------------------------------------------------------------
\title{Telescope 1 (T1)}

\SmallerFonts

\line{\fig[7cm]{fig/t1_installed.jpg}\hskip3mm\raise15mm\vbox{\hsize7.9cm
\> installed inside CMS end-caps
\> at $7.5$ to $10.5\un{m}$ from the IP
\> one \em{telescope} on each side of IP
\> each telescope consists of two \em{quarters}
}\hss}

\vfil
\line{\fig[7cm]{fig/t1_quarter.jpg}\hskip3mm\raise15mm\vbox{\hsize7.9cm
\> each quarter formed by 5 \em{planes} equally spaced along beam
\> each plane consists of 3 trapezoidal \em{CSC detectors}, each covering $60\un{^\circ}$ in azimuth
\> Cathode Strip Chamber: gaseous detector with 3 read-out coordinates (at $60\un{^\circ}$ wrt.~each other)
}\hss}

\newpage %-------------------------------------------------------------------------------------------
\title{Telescope 2 (T2)}

\SmallerFonts

\line{\fig[7.1cm]{fig/t2_installed.jpg}\hskip3mm\raise15mm\vbox{\hsize70mm
\> installed inside CMS shielding between HF and Castor calorimeters
\> centred about $13.5\un{m}$ from the IP
\> one \em{telescope} on each side of IP
\> each telescope consists of two \em{quarters}
}\hss}

\vfil
\line{\fig[7.1cm]{fig/t2_quarter.jpg}\hskip3mm\vbox{\hsize80mm
\> each quarter formed by 10 semi-circular \em{planes}, assembled in 5 back-to-back mounted pairs
\> each plane equipped with a \em{Gas Electron Multiplier} detector
\>> gaseous detector, electron multiplication by 3 perforated foils ($2\un{mm}$ separation)
\>> radial segmentation: \em{strips} (resolution\\ $\approx 0.15\un{mm}$)
\>> coarse radial$\times$azimuthal segmentation: \em{pads} (for triggering, azimuthal resolution $0.8\un{^\circ}$)
}\hss}

\newpage %-------------------------------------------------------------------------------------------
\title{Roman Pots (RPs)}

\SmallerFonts

\line{\fig[5cm]{fig/rp_station.pdf}\hskip3mm\raise20mm\vbox{\hsize10cm
\> \em{stations} installed at $\pm 220\un{m}$ in the outgoing LHC beam-pipe
\> each station has two \em{units}, separated by $\approx 5\un{m}$
}\hss}

\line{\fig[5cm]{fig/rp_unit.pdf}\hskip3mm\raise4mm\vbox{\hsize10cm
\> each unit contains 3 \em{Roman Pots}: top, bottom and horizontal
\> Roman Pot = movable beam-pipe insertion
\>> \Em{beam unstable} $\Rightarrow$ RPs retracted to safe position
\>> \Em{beam stable} $\Rightarrow$ RPs as close to beam as reasonable
\> typical approach: $10\un{\si_{\rm beam}}$ (record $3\un{\si_{\rm beam}}$)
}\hss}

\line{\fig[5cm]{fig/rp_pot_new.png}\hskip3mm\raise13mm\vbox{\hsize10cm
\> Roman Pot: container for sensors
\> LS1: improved RF shield $\Rightarrow$ possible close approach to high-intensity beam
}\hss}

\newpage %-------------------------------------------------------------------------------------------
\title{``Edgeless'' silicon sensors}

\SmallerFonts

\line{\fig[6cm]{fig/rp_package.pdf}\hskip3mm\raise39mm\vbox{\hsize10cm
\> each RP contains a \cYe\em{package}\cFg{} of 10 silicon sensors
\> 5 pairs of back-to-back mounted strip sensors
}\hss}

\vfil
\line{\fig[6cm]{fig/rp_hybrid_new.pdf}\hskip3mm\raise23mm\vbox{\hsize10cm
\> custom developed \cYe\em{``edgeless'' sensors}\cFg\\
$\Rightarrow$ \Em{insensitive edge $\approx 50\un{\mu m}$} (standard about $1\un{mm}$)
\> single-sided $\rm p^+$-n
\> 512 strips at pitch of $66\un{\mu m}$, at $45\un{^\circ}$ wrt.~cut edge
\> operated at $\approx -20\un{^\circ C}$, bias voltage $\approx 100\un{V}$
}\hss}


\newpage %-------------------------------------------------------------------------------------------
\title{Proton measurement with RPs}

\SmallerFonts

\> \em{proton transport}: described as in linear optics

\line{\hss\fig[11.5cm]{fig/ttm_proton_transport.pdf}\hss}

\vfil

\cBlack
\line{\hss$\displaystyle
\pmatrix{x\cr \th_x\cr y\cr \th_y\cr \xi}_{\rm RP} =
\underbrace{\pmatrix{
	v_x & L_x & \cdot & \cdot & D_x\cr
	\cdot & \cdot & \cdot & \cdot & \cdot\cr
	v_y & L_y & \cdot & \cdot & D_y\cr
	\cdot & \cdot & \cdot & \cdot & \cdot\cr
	\cdot & \cdot & \cdot & \cdot & 1\cr
}}_{\hbox{product from all lattice elements}}
\pmatrix{x^*\cr \th_x^*\cr y^*\cr \th_y^*\cr \xi}_{\rm IP}
$
\hskip10mm
\lower15mm\vbox{
\hbox{$\th_x^*, \th_y^*$: scattering angles}
\hbox{$x^*, y^*$: vertex}
\hbox{$\xi = \De p / p$: momentum loss}
\hbox{\strut}
\hbox{\Em{optical functions:}}
\hbox{effective length $L$}
\hbox{magnification $v$}
\hbox{dispersion $D$}
}
\hss}

\vfil

\> \em{proton reconstruction}: inverted transport RPs $\longrightarrow$ IP

\>> optical parameters functions of $\xi$ $\Rightarrow$ reconstruction is non-linear problem

\>> \cRed good knowledge of optics is crucial

\newpage %-------------------------------------------------------------------------------------------
\title{LHC optics}

\> simulation of central diffraction for 2 different optics

\vskip2mm
\line{%
	\hss
	\hskip2mm
	\vtop{\hsize7.7cm
		\centerline{{\bf\cRed low $\bf\be^*$ (LHC standard)\cBlack}}
		\centerline{\SmallerFonts$\displaystyle L_x \approx 1.7\un{m},\quad L_y \approx 14\un{m},\quad D_x \approx 8\un{cm}$}
		
		\centerline{diffractive protons in \em{horizontal RPs}}
		\line{\hss\fig[,4.5cm]{fig/hit_dist_0p7.png}\hss}
	}%
	\vtop{\hsize7.7cm
		\centerline{\bf\cRed$\bf\be^* = 90\un{m}$ (special for TOTEM)\cBlack}
		\centerline{\SmallerFonts$\displaystyle L_x \approx 0,\quad L_y \approx 260\un{m},\quad D_x \approx 4\un{cm}$}
		
		\centerline{diffractive protons in \em{vertical RPs}}
		\line{\hss\fig[,4.5cm]{fig/hit_dist_90.png}\hss}
	}%
	\hss
}

\vfil
\> optics typically ``labelled'' by \em{$\be^* \equiv$ betatron function at IP}
\>> beam width: $\sqrt{\ep \be}$, $\ep$: beam emittance
\>> beam angular divergence: $\sqrt{\ep / \be}$
\>> luminosity $\propto \hbox{(beam width at IP)}^{-2} \propto 1/\be^*$


\newpage %-------------------------------------------------------------------------------------------
\title{Typical run scenarios}

\centerline{\cRed $t \approx -p^2 \th^2$:\cBlue{} four-momentum transfer squared}
\centerline{\cRed $\xi = \De p / p$:\cBlue{} fractional momentum loss}

\vfil

{

\advance\hsize\horizontalmargin\line{\kern-\horizontalmargin
\fig[,4.37cm]{fig/acceptance_2.pdf}
\fig[,4.37cm]{fig/acceptance_90.pdf}
\fig[,4.37cm]{fig/acceptance_1535.pdf}\hss
}

\line{\kern-\horizontalmargin\SmallerFonts\hss
\vtop{\hsize4.7cm\obeylines\leftskip0pt plus1fil\rightskip0pt plus1fil\parfillskip0pt
	\Em{\NormalFonts\bf $\bf\be^* = 0.55\un{m}$}
	\hrule\vskip1mm
	${\cal L} \approx 10^{33}\un{cm^{-2}s^{-1}}$
		\vskip\baselineskip
	\em{elastic scattering}: high $|t|$
		\vskip\baselineskip
	\em{diffraction}: $\xi \gs 0.03$, low cross-section processes
}
\hss
\vtop{\hsize4.7cm\obeylines\leftskip0pt plus1fil\rightskip0pt plus1fil\parfillskip0pt
	\Em{\NormalFonts\bf medium $\bf\be^* = 90\un{m}$}
	\hrule\vskip1mm
	${\cal L} \approx 10^{28}\un{cm^{-2}s^{-1}}$
		\vskip1\baselineskip
	\em{elastic scattering}: low to mid $|t|$
		\vskip\baselineskip
	\em{diffraction}: any $\xi$ for $|t| \gs 0.01\un{GeV^2}$
}
\hss
\vtop{\hsize4.7cm\obeylines\leftskip0pt plus1fil\rightskip0pt plus1fil\parfillskip0pt
	\Em{\NormalFonts\bf high $\bf\be^* = 1535\un{m}$}
	\hrule\vskip1mm
	${\cal L} \approx 10^{27}\un{cm^{-2}s^{-1}}$
		\vskip\baselineskip
	\em{elastic scattering}: very low $|t|$ (Coulomb-nuclear interference)
}
\hss}

}

\vfil

\centerline{\cBlue details depend on RP approach to beam and precise optics}


\newpage %-------------------------------------------------------------------------------------------
\ctitle{Elastic scattering}{$p + p \rightarrow p + p$}

\> selection: two anti-collinear protons from the same vertex

\> (almost) purely data-driven analysis

\> data overview (selection), \cBlack gray = preliminary

\centerline{\fig[15cm]{fig/es_summary.pdf}}

\> different $|t|$ probe \em{different physics regimes} -- from lowest to highest $|t|$:
\\ \cThird Coulomb interference, diffractive cone, dip-bump, transition to pQCD

% ORAL: anticipate some features of the data: shift of dip position, cone steepening

\newpage %-------------------------------------------------------------------------------------------
\ctitle{Elastic scattering}{Trends}

\centerline{\cBlack (gray = preliminary)}

\centerline{%
	\fig[,55mm]{fig/es_summary_detail.pdf}%
	\hskip5mm
	\fig[,55mm]{fig/B_s.pdf}%
}

\> \em{dip position}
\>> $\sqrt s = 8\un{TeV}$: limited statistics
\>> $\sqrt s = 7 \rightarrow 13\un{TeV}$: dip moves to lower $|t|$
\>>> $13\un{TeV}$ results preliminary! (the yet missing unfolding correction likely to move the dip to lower $|t|$)

\> \em{forward slope} $B = {\d\over\d t} \log \left.{\d\si\over\d t} \right |_{t=0}$
\>> increase wrt.~previous experiments

\newpage %-------------------------------------------------------------------------------------------
\ctitle{Elastic scattering}{Non-exponentiality at low $|t|$}

\> diffraction cone: ``looks almost exponential''
\>> magnify deviations $\Rightarrow$ plot \cThird ${(\d\si/\d t - \hbox{ref. exp.}) / \hbox{ref. exp.}}$

\> $\be^* = 90\un{m}$ measurements at different energies (stat.~unc.~only):

\centerline{%
	\fig[,4.8cm]{fig/t_dist_rel_7.pdf}%
	\hskip-1mm
	\fig[,4.8cm]{fig/t_dist_rel_8.pdf}%
	\hskip-1mm
	\fig[,4.8cm]{fig/t_dist_rel_13.pdf}
}

\vskip-3mm

\line{\raise37mm\vtop{\hsize9.4cm
\> non-exponentiality observed at $8$ and $13\un{TeV}$!
\>> $8\un{TeV}$: \em{$7\un{\si}$ significance}\hfil $\longrightarrow$
\>> $13\un{TeV}$: preliminary results
\>> non-exponentiality of observed cross-section:

\cThird
\vskip-3mm
$$\d\si /\d t = \hbox{nuclear} + \hbox{Coulomb} + \hbox{interference}$$
}
	\hskip3mm
	\fig[5.6cm]{fig/t_dist_8TeV_non_exp.pdf}
	\hss
}


\newpage %-------------------------------------------------------------------------------------------
\ctitle{Elastic scattering}{Coulomb interference}

\> $8\un{TeV}$ data with $\be^* = 1000\un{m}$ optics
\>> RPs very close to the beam: $\approx 3\un{\si_{\rm beam}}$
\>> $|t|_{\rm min} \approx 6\cdot10^{-4}\un{GeV^2}$

\vfil
\centerline{\fig[11cm]{fig/coulomb_components.pdf}}


\newpage %-------------------------------------------------------------------------------------------
\ctitle{Elastic scattering}{Coulomb interference}

\> observed cross-section
\vskip-3mm
\cBlack
$${\d\si\over\d t} \propto \left|
\underbrace{\vcenter{\fig[,2cm]{fig/el_diagram_C.pdf}} + \cdots}_{\vbox to 0pt{\SmallerFonts\hbox{Coulomb}\hbox{amplitude}\vss}}
\quad +
\underbrace{\vcenter{\fig[,2cm]{fig/el_diagram_H.pdf}}}_{\vbox to 0pt{\SmallerFonts\hbox{hadronic}\hbox{amplitude}\vss}}
+
\underbrace{\vcenter{\fig[,2cm]{fig/el_diagram_C_H.pdf}} + \cdots}_{\vbox to 0pt{\SmallerFonts\hbox{``interference''}\hbox{terms}\vss}}
\right|^2$$

\vskip10mm

\>> 2 meanings of ``interference'': sum of amplitudes, additional terms

\> interference formula = summation for practical applications
\>> \Em{simplified West-Yennie (SWY)}: QFT framework, traditional but \em{heavy simplifications (constant hadronic phase, constant slope)}
\>> \Em{Cahn} or \Em{Kundr\' at-Lokaj\' i\v cek (KL)}: eikonal framework, no explicit simplifications

\> interference $\Rightarrow$ phase of hadronic amplitude exposed in cross-section
\>> phase $t$-dependence needs to be considered in analysis
\>> constraints from data $\Rightarrow$ \Em{determination of $\rh$ parameter}

\vskip-3mm
\cThird
$$\rh = \left. {\Re {\cal A}^{\rm N} / \Im {\cal A}^{\rm N}} \right|_{t=0}$$


\newpage %-------------------------------------------------------------------------------------------
\ctitle{Elastic scattering}{Coulomb interference -- Analysis strategy}

\centerline{\EM{central question:}}
\centerline{observed non-exponentiality -- due to hadronic, Coulomb or both?}

\vfil

\> fits with 2 different assumptions on hadronic component
\>> \em{purely-exponential} -- non-exponentiality due to Coulomb (+interference)\\
$\rightarrow$ $|{\cal A}^{\rm N}| = a\, \exp(b_1 t)$
\>> \em{flexible enough} to describe non-exponentiality even without Coulomb\\
$\rightarrow$ $|{\cal A}^{\rm N} = a\, \exp(b_1 t + b_2 t^2 + b_3 t^3)$

\vfil

\> role of hadronic phase $t$-dependence?
\>> largest impact: rate of change at low $|t|$
\>>> same quantity controls behaviour in impact-parameter space
\>> considered two families: \em{central} (black $\downarrow$), \em{peripheral} (blue $\downarrow$)

\centerline{%
	\raise30mm\vtop{\hsize48mm
	}
	\fig[,43mm]{fig/1000m_exp3_phase_cni_effect.pdf}%
	\fig[,43mm]{fig/1000m_exp3_b_dist.pdf}%
}


\newpage %-------------------------------------------------------------------------------------------
\ctitle{Elastic scattering}{Coulomb interference -- Fits}

\SmallerFonts

\line{%
	\fig[7cm]{fig/coulomb_exp1_fit.pdf}%
	\raise45mm\vtop{\hsize8cm
	$\Leftarrow$ \EM{purely-exponential hadronic amplitude}
		\>> \em{constant phase excluded} (with both SWY and KL formulae) $\Rightarrow$ application of SWY formula excluded too
		\>> \em{peripheral phase} not excluded by data, but \em{disfavoured}
		\>>> $\rh$ value outside a consistent pattern of\\ other fits and theoretical predictions
		\>>> number of theoretical reasons for\\ non-exponential hadronic amplitude
	}
	\hss
}

\line{%
	\fig[7cm]{fig/coulomb_exp3_fit.pdf}%
	\raise30mm\vtop{\hsize8cm
	$\Leftarrow$ \EM{non-exponential hadronic amplitude}
		\>> both constant and peripheral phases compatible with data $\Rightarrow$ \em{centrality not necessity}
	}
	\hss
}

\newpage %-------------------------------------------------------------------------------------------
\ctitle{Elastic scattering}{Coulomb interference -- $\rh$ parameter}

\> $\sqrt{s} = 8\un{TeV}$: first LHC determination from Coulomb-hadronic interference
\vskip-4mm
\cThird
$$\rh = 0.12 \pm 0.03$$

\centerline{\fig[11cm]{fig/rho_s.pdf}}

\>> leading uncertainty: statistics

\vfil

\> plans for $\sqrt s = 13\un{TeV}$: $\rh$ measurement with higher statistics


\newpage %-------------------------------------------------------------------------------------------
\ctitle{Elastic scattering}{Exclusion power in Run I}

\> model predictions (prior to Run I) vs.~TOTEM data at $\sqrt s = 7\un{TeV}$:

\centerline{\fig[13cm]{fig/cmp_totem_data_7TeV.pdf}}

\>> \em{no model compatible with data!}

\> surprisingly?: little reaction after Run I (statement from 2015)
\>> most often: no change at all or simple parameter refit
\>> only Islam abandoned one mechanism of large $|t|$ scattering
\>> exclusion of physics mechanisms (model independent)?

\newpage %-------------------------------------------------------------------------------------------
\ctitle{Elastic scattering}{Structures at high $|t|$ ?}

\> $\sqrt s = 13\un{TeV}$: very preliminary, but already very strong results

\centerline{%
	\fig[,6.5cm]{fig/t_dist_13.pdf}%
	\hskip1mm
	\raise57mm\vtop{\hsize7cm
		\centerline{\SmallerFonts\cBlack model predictions:}
		\vskip1mm
		\fig[,4.5cm]{fig/models_13TeV.pdf}%
		\vskip-1mm
		\centerline{\SmallerFonts\em{oscillations in almost each model}}
	}%
}

\line{\vbox{\hsize10.8cm
\> \em{high-$|t|$: no structures!}
\>> rules out many models
\>> rules out physics mechanism: ``optical'' models
\>> physics interpretation: transition between diffraction and pQCD? \hfill $\Rightarrow$ \hfill e.g.~Donnachie-Landshoff $\Rightarrow$
}\hskip4mm\hbox{\fig[4cm]{fig/tripple_gluon.png}}\hss
}

\newpage %-------------------------------------------------------------------------------------------
\title{Inelastic and total cross-section}

\> \em{inelastic cross-section}: event counting with T2 (and T1)
\>> $95\un{\%}$ of inelastic events have at least 1 track in the T2 region
\>> only one significant MC correction: contribution from low mass diffraction

\> 3 methods to determine \em{total cross-section}

\centerline{\fig[13cm]{fig/tot_cs_methods.pdf}}



\newpage %-------------------------------------------------------------------------------------------
\title{Cross-section results}

\centerline{%
	\fig[,7cm]{fig/sigma_tot_el_inel_cmp.pdf}%
	\hskip2mm
	\fig[,7cm]{fig/sigma_el_to_sigma_tot.pdf}%
}

\> $\sqrt s = 7\un{TeV}$: all 3 methods consistent

\> $\sqrt s = 8\un{TeV}$: results from CNI study superior
\>> Coulomb component explicitly separated
\>> determined in the same analysis as $\rh$

\> $\sqrt s = 2.76$ and $13\un{TeV}$ analyses ongoing

\newpage %-------------------------------------------------------------------------------------------
\title{Single diffraction: $p + p \rightarrow p + X$}

\centerline{\fig[13cm]{fig/topology_sd_T2_opp.pdf}}

\> double determination of $\xi \approx \e^{-\De\et}$
\>> from RPs
\>> from rapidity gap $\De\et$ (T1/T2)

\> mass of diffractive system: \cBlack $M_X = \sqrt{s \xi}$ 

\> available data
\>> $7\un{TeV}$: TOTEM-standalone analysis in progress
\>> $8$ and $13\un{TeV}$: common data with CMS

\> $\xi$ resolution from RPs: $\approx 0.9\un{\%}$ $\Rightarrow$ mass bins given by arms where T1/T2 active

\newpage %-------------------------------------------------------------------------------------------
\ctitle{Single diffraction}{Preliminary results at $7\un{TeV}$}

\centerline{\fig[16cm]{fig/sd_7TeV.png}}

\newpage %-------------------------------------------------------------------------------------------
\title{Double diffraction: $p + p \rightarrow X + Y$}

\centerline{\fig[13cm]{fig/topology_dd_lab.pdf}}

\> experimental challenge: background (non-diffractive, SD pile-up)
\>> sub-sample with signal $\gg$ background:  $2\times$T2 and T1 veto
\>> non-diffractive background: control sample $2\times$T2 + $2\times$T1
\>> SD background: control sample $1\times$T2 + $0\times$T1

\> cross-section as function of $\et_{\rm min}$ on both sides
\>> challenge: reconstructed $\et_{\rm min}$ $\longrightarrow$ true/generator $\et_{\rm min}$\\
$\Rightarrow$ 2 $\et_{\rm min}$ bins only

\> available data
\>> $7\un{TeV}$: results published
\>> $8$ and $13\un{TeV}$: common data with CMS (improvement expected)

\newpage %-------------------------------------------------------------------------------------------
\ctitle{Double diffraction}{Results at $7\un{TeV}$}

\> measurement

\centerline{\fig[12cm]{fig/DD_meas.png}}

\> comparison to Monte Carlos

\centerline{\fig[12cm]{fig/DD_mc_comp.png}}

\newpage %-------------------------------------------------------------------------------------------
\title{Central diffraction: $p + p \rightarrow p + X + p$}

\centerline{\fig[13cm]{fig/topology_cd.pdf}}

\> available data
\>> $7\un{TeV}$: TOTEM only, analysis started
\>> $8\un{TeV}$: common data with CMS, analysis started
\>> $13\un{TeV}$: common data with CMS

\> $\be^* = 90\un{m}$: all $\xi$ visible, but resolution $\approx 0.9\un{\%}$

\> experimental challenge: background (pileup ES/beam halo + inelastic)
\>> anti-elastic cuts: anti-collinearity
\>> anti-beam-halo cuts: $|y| > 11\un{mm}$

\newpage %-------------------------------------------------------------------------------------------
\ctitle{Central diffraction}{First results at $7\un{TeV}$}

\> $|t_y|$ distribution: all $\xi$ values, only acceptance correction

\line{\raise8mm\vbox{\hsize70mm
\> estimate of $\si_{\rm CD}$

\cBlack\SmallerFonts
$${\d^2\si_{\rm CD}\over \d t_1 \d t_2} = C \e^{-B t_1} \e^{-B t_2}$$

$$\Downarrow$$

$$ \si_{\rm CD}
= \int\limits^{0}_{-\infty} \d t_1 \int\limits^{0}_{-\infty} \d t_2\ C \e^{-B t_1} \e^{-B t_2} \approx 1\un{mb}$$
}\hskip1mm\fig[8.5cm]{fig/cd_ty_dist_7TeV.png}\hss}

\newpage %-------------------------------------------------------------------------------------------
\title{Forward charged-particle multiplicities}

\> \em{$\d N_{\rm ch} / \d\et$} : mean number of charged particles per event and per unit of pseudorapidity

\> probes (non-)perturbative strong interactions and hadronisation 


\centerline{
	\raise50mm\vtop{\hsize9cm
\> measurement based on T2
\>> $\approx 95\un{\%}$ of inelastic events seen
\>> almost all non-diffractive events visible
\>> diffraction with $M_X \gs 3.4\un{GeV}$\\ detected
\>> selection of primary particles:\\ lifetime $> 30\un{ps}$ (LHC convention)
	}
	\fig[6cm]{fig/dNdeta_zimpact.png}
}

\> available data
\>> $\sqrt s = 7\un{TeV}$: TOTEM only, published
\>> $\sqrt s = 8\un{TeV}$: TOTEM + CMS, TOTEM + shifted vertex, published
\>> $\sqrt s = 13\un{TeV}$ 


\newpage %-------------------------------------------------------------------------------------------
\ctitle{Forward charged-particle multiplicities}{$\sqrt s = 7\un{TeV}$}

% TODO: why the Monte-Carlo curves non-continuous ?

\centerline{
	\fig[,5cm]{fig/dNdeta_totem_7TeV.png}%
	\hskip3mm
	\fig[,5cm]{fig/dNdeta_7TeV_allLHC.png}%
}
\centerline{
	\hskip8cm
	\vtop{\hsize7cm
		\SmallerFonts $\uparrow$ NB: each experiment has different event selection!
	}
}

\> main contributions to systematic uncertainty ($\approx 10\un{\%}$)
\>> subtraction of a large fraction of secondaries (about $80\un{\%}$ of all T2 tracks)
\>> track efficiency and misalignment uncertainties

% no MC describes data well

%\> gap LHCb -- TOTEM T2 will be filled
%\>> analysis of T1 data in progress
%\>> data with shifted IP by $11\un{m}$ $\Rightarrow$ shift of T2 acceptance: $6.0 < \et < 7.3$ or  $3.8 < \et < 4.8


\newpage %-------------------------------------------------------------------------------------------
\ctitle{Forward charged-particle multiplicities}{$\sqrt s = 8\un{TeV}$}

\vskip-1mm
\> TOTEM + CMS
\>> \em{non-single-diffractive enhanced}: requiring both hemispheres of T2 on
\>> \em{single-diffractive enhanced}: requiring only one hemisphere of T2 on

\centerline{
	\fig[,45mm]{fig/dNdeta_8TeV_with_CMS.png}%
	\hskip3mm
	\fig[,45mm]{fig/dNdeta_8TeV_with_CMS_SD_enhanced.png}%
}


\vskip-1mm
\line{
	\raise42mm\vtop{\hsize8cm
		\> TOTEM only, with \em{displaced vertex}\\ ($z = 11.25\un{m}$)
	}
	\fig[,45mm]{fig/dNdeta_8TeV_displaced_vertex.png}%
	\hss
}


\newpage %-------------------------------------------------------------------------------------------
\title{Inelastic event classification}

\> $\sqrt s = 8\un{TeV}$: analysis in review
\>> classes considered: non-diffractive (ND), single diffractive (SD) with proton left/right, double diffractive (DD)
\>>> central diffraction not take into account: low cross-section not worth added complexity
\>> experimental definition of diffraction: rapidity gap $\De\et \ge 3$
\>> boosted decision tree
\>>> ordered binarisation: ND $\rightarrow$ SD (left) $\rightarrow$ SD (right) $\rightarrow$ DD
\>> training sample: Pythia 8-4C
\>> control samples: Pythia 8-4C+MBR, QGSJET-II-04, Pythia 8-Monash
\>> discriminators: $\De\et$, $\et_{\rm min}$, $\et_{\rm max}$, $N_{\rm CMS}$, $N_{\rm T2+}$, $N_{\rm T2-}$, $\xi_L$, $\xi_R$, ...

\> possible improvement: use of CMS FSC ($6 < |\et| < 8$) $\Rightarrow$ better distinction between low-mass DD (more forward than T2) and SD with undetected proton


\> other available data
\>> $\sqrt s = 13\un{TeV}$



\newpage %-------------------------------------------------------------------------------------------
\title{(Exclusive) central diffraction with CMS}

\centerline{\fig[11cm]{fig/diagram_cd_cms_lab.pdf}}


\> exchange of colour singlets with vacuum quantum numbers $\Rightarrow$ selection rules for system X: $J^{PC} = 0^{++}$, $2^{++}$, ...

\> double-arm proton tagging: mass reach and luminosity depending on optics

\> event selection via comparison CMS to RP protons:

\vskip-3mm
\cThird
$$M(pp) \hbox{ vs. } M(CMS)\ ,\quad p_T(pp)\hbox{ vs. }p_T(CMS)\ ,\quad	\hbox{vertex}(pp)\hbox{ vs. }\hbox{vertex}(CMS)$$

\> prediction of rapidity gap from proton $\xi$: $\De\et_{1,2} = −\log \xi_{1,2}$


\> analysis examples:
\>> studies of glueball candidates
\>> exclusive dijets: mainly $gg$ ($p_T > 30GeV$: $\si_{gg} \approx 100 pb$)
\>> exclusive $\ch_{c}$ and $J/\Ps$ production: $\O{10 pb - 10 nb}$
\>> search for missing mass signals of $\O{pb}$ $\Rightarrow$ SUSY searches

\newpage %-------------------------------------------------------------------------------------------
\title{Glueball searches}

\> CD production at LHC
\>> $M_X = 1\hbox{ to }4\un{GeV}$ $\Rightarrow$ $x\sim 10^{-4}$ $\Rightarrow$ pure gluon content
\>> Pomeron $\sim$ colour-less gluon ladder $\Rightarrow$ fusion likely to produce glueballs

\> $0^{++}$ glueball candidates: $f_0(1500)$, $f_0(1710)$, $f_0(1370)$
\>> lattice QCD: m(0++) glueball $\sim$ $1700 (\pm 100)\un{MeV}$

\> CMS + TOTEM:
\>> both protons measured and tagged by TOTEM
\>> effective selection with high purity ($p_T$ balance + $x$-vertexing) in required $\xi$-range.
\>> CMS tracker: charged particle invariant mass with $\si(M) \sim 20-30\un{MeV}$

\> available data
\>> 8 TeV, $\be^* = 90\un{m}$, Jul 2012: ${\cal L} \approx 1\un{nb^{-1}}$: proof of principle
\>> 13 TeV, $\be^* = 90\un{m}$, Oct 2015: ${\cal L} \approx 0.4\un{pb^{-1}}$ for (CMS + TOTEM): should allow full production and decay characterisation


\newpage %-------------------------------------------------------------------------------------------
\ctitle{Glueball searches}{2015 data}

\> CMS + TOTEM: trigger exchange, independent DAQ, offline data merging

\> trigger: RP double arm \& T2 veto \& at least one track in CMS:

\centerline{\fig[12cm]{fig/run_2015_90m_trigger.png}}

\> analysis strategy -- verify the following glueball conditions
\>> resonance \em{enhancement with increasing collision} energy (increasingly pure gluon contribution to CEP)
\>> final states \em{branching ratio to $\pi\pi$, $KK$, ...},  with equi-flavour partitioning selection rules (or with proportionality of
gluon coupling to quarks)
\>> \em{suppression of photon-photon} channel (in production and final state)



\newpage %-------------------------------------------------------------------------------------------
\title{Outlook}

\> \Em{upgrade projects}: interest in lower cross-section processes\\ $\Rightarrow$ higher luminosity needed $\Rightarrow$ higher pile-up
\>> need \em{timing}: association of RP proton vertex with CMS vertex
\>> need \em{pixels}: more tracks in RPs

\cThird
\centerline{%
\tab{\ln
\hbox{project} & \hbox{\bf CT-PPS} & \hbox{\bf timing in vertical RP} \cr\ln
\hbox{optics} & \hbox{low (standard) }\be^* & \hbox{high lumi }\be^* = 90\un{m}\cr
\hbox{RPs used} & \hbox{horizontal} & \hbox{vertical} \cr
\hbox{tracking} & \hbox{3D pixel (}10\un{\mu m}\hbox{)} & \hbox{current Si strip (}20\un{\mu m}\hbox{/RP)} \cr
\hbox{timing} & \hbox{L-bar cherenkov (}30\un{ps}\hbox{/module)} & \hbox{diamond (}100\un{ps}\hbox{/plane)}\cr
			 & \hbox{or ultrafast silicon or diamond} & \cr\ln
}%
}



\> \Em{priorities for 2016}
\>> \em{glueball} searches (2015 data)
\>> \em{odderon} searches: TOTEM only, special run with $be^* = 2500\un{m}$
\>> \em{di-photon} searches with ``accelerated'' CT-PPS: standard low $\be^*$ runs
\>>> motivated by ``the $750\un{GeV}$ may-be-resonance``, start CT-PPS programme with current detectors now


\newpage %-------------------------------------------------------------------------------------------
\ctitle{Outlook}{CT-PPS plan for 2016}

\line{%
	\raise21mm\vtop{\hsize11cm
\> motivation: di-photon excess at $750\un{GeV}$ observed by ATLAS and CMS
\>> CT-PPS: can see \em{exclusive and complete} events
	}%
	\hskip3mm
	\fig[4cm]{fig/diagram_750GeV_candidate.png}
	\hss
}

\vskip-5mm

\> background suppressible even \em{without timing} detectors
\>> large mass: low background from inelastic photon pair + pileup
\>> strong correlation between proton pair (RP) and photon pair (CMS ECal)
\>> nevertheless: install 2 RPs with 4 diamond planes (originally for vertical RPs)
\>>> can also be used for (coarse) tracking

\> \em{tracking: use current Si strip} detectors in station 210m
\>> functional at least for $10/fb$, then replaceable with detectors from 220m
\>> detector/beam shift: irradiation distributed

\line{%
	\raise9mm\vbox{\hsize9.3cm
\> \em{CMS-TOTEM integration} successfully ongoing
\>> trigger, DAQ, offline SW, DQM, ...
\>> DQM snapshot: from Monday, beam with 900 bunches, RPs inserted in physics positions, RPs included in global CMS DAQ, data processed with CMSSW 8.1.X (next release) $\rightarrow$
	}
	\hss
	\fig[5cm]{fig/ctpps_hit_dist.pdf}
}

\newpage %-------------------------------------------------------------------------------------------
\ctitle{Outlook}{Odderon searches}

\> Odderon = (hypothetical) cross-odd partner of Pomeron

\> overview of past Odderon searches
\>> comparison pp vs.~anti-pp (dip): not applicable at LHC
\>> spin analyses: not applicable at LHC
\>> structures in $\d\si / \d t$: where Pomeron contribution small
\>>> high-$|t|$: disfavoured by $13\un{TeV}$ measurements
\>>> low-$|t|$: shifts of $\rh$ value $\Rightarrow$ within reach of TOTEM


\line{%
	\raise45mm\vtop{\hsize73mm
	\> Coulomb-nuclear interference at\\ $\sqrt s = 13\un{TeV}$
	\>> needs special optics: $\be^* = 2500\un{m}$
	\>> $|t| = 6\cdot10^{-4}\un{GeV^2}$ reachable
	\>> $\sim 1\un{week}$ data-taking time\\ approved in 2016
	}
	\hskip2mm
	\fig[8.0cm]{fig/2500m_interference_effect.pdf}
}


\newpage %-------------------------------------------------------------------------------------------
\title{Summary}

\> \em{performance} in Run II
\>> all detectors functional
\>> RPs equipped with RF shields $\Rightarrow$ less impedance $\Rightarrow$ close approach possible
\>> DAQ throughput increased $50\times$ wrt.~Run I

\vfil

\> many \em{analyses} ongoing
\>> Run I and Run II
\>> TOTEM standalone, TOTEM + CMS

\vfil

\> \em{upgrade} projects
\>> timing in vertical RPs (diamonds)
\>> CT-PPS (pixel tracking, timing with fast silicon, diamond or Cherenkov)

\vfil

\> \em{priorities} for 2016
\>> glueball analysis of 2015 data
\>> TOTEM special run: $\be^* = 2500\un{m}$ $\Rightarrow$ Odderon searches via CNI
\>> CT-PPS runs with LHC standard optics: di-photon searches with existing RPs, ...



\newpage %-------------------------------------------------------------------------------------------
\hbox{}%
\vfill
\centerline{ $\downarrow$ Backup}
\footline={}


\iffalse
\newpage %-------------------------------------------------------------------------------------------
\title{Backup topics}

\> Optics refinement with RP data

\> Coulomb analysis
\>> interference formulae
\>> phase choices

\fi

\newpage %-------------------------------------------------------------------------------------------
\centerline{\fig[16cm]{fig/mario_slide_11.png}}

\newpage %-------------------------------------------------------------------------------------------
\centerline{\fig[16cm]{fig/mario_slide_12.png}}

\newpage %-------------------------------------------------------------------------------------------
\centerline{\fig[16cm]{fig/mario_slide_14.png}}

\newpage %-------------------------------------------------------------------------------------------
\centerline{\fig[16cm]{fig/mario_slide_15.png}}

\vfil
\eject
\bye
