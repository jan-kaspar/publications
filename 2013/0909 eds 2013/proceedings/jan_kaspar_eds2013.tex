\documentclass{desyproc}

%------------------------------------

\def\un#1{\,{\rm #1}}

%------------------------------------

\begin{document}
\title{TOTEM Results on Elastic Scattering and Total Cross-Section}

\author{{\slshape Jan Ka\v spar} of behalf of the TOTEM collaboration\\[1ex]
CERN, 1211 Geneva 23, Switzerland,\\
Institute of Physics, AS CR, v.v.i., 182 21 Prague 8, Czech Republic
}

% if the proceedings are available online (e.g. at Indico)
% please enter the contribution ID or file_name below for the DOI
%\contribID{smith\_joe}
 
\acronym{EDS'13} % if you want the Acronym in the page footer uncomment this line

\maketitle

\begin{abstract}
TOTEM \cite{totem_jinst} is an LHC experiment dedicated to forward hadronic physics. In this contribution, the status of its elastic scattering and total cross-section analyses is reviewed.
\end{abstract}

\section{Status of elastic scattering analyses}

At the centre-of-mass energy $\sqrt s = 7\un{TeV}$, there are three datasets available. The lowest four-momentum transfer squared, $|t|$, values (from $0.005$ to $0.4\un{GeV^2}$) were reached with $\beta^*=90\un{m}$ optics \cite{si_el_7_90b}. The dip-bump structure was observed in the $\beta^{*}=3.5\un{m}$ data \cite{si_el_7_3p5}, where the Roman Pot detectors (RPs) were placed at $7\un{\sigma_{\rm beam}}$ from the beam. The highest $|t|$ reached at this run was about $2.5\un{GeV^2}$. This value will most likely be extended to about $3.5\un{GeV^2}$ with another dataset recorded at $\beta^{*}=3.5\un{m}$ optics and RPs at $18\un{\sigma_{\rm beam}}$.

At $\sqrt s = 8\un{TeV}$, there are analyses of two datasets ongoing. With $\beta^* = 1000\un{m}$ optics the $|t|$ values as low as $6\cdot10^{-4}\un{GeV^2}$ were reached and the Coulomb-nuclear interference was observed for the first time at the LHC. This interference makes it possible to determine the phase of the nuclear amplitude -- an analysis in this direction is progress. The data recorded at $\beta^* = 90\un{m}$ extend the observable $|t|$ range up to about $1.4\un{GeV^2}$. Thanks to the high statistics of these data, both the dip and bump are clearly visible.

TOTEM recorded data at the energy of $2.76\un{TeV}$ as well (with $\beta^*=11\un{m}$ optics). The analysis is ongoing and the expected $|t|$ range is from about $0.06$ to about $0.4\un{GeV^2}$. Therefore the dip is unlikely to be covered.

\section{Status of total cross-section analyses}

At $\sqrt s =7\un{TeV}$, two total-cross section analyses were published, both exploiting the $\beta^*=90\un{m}$ optics. The first paper \cite{si_el_7_90a} used a method based on extrapolating the differential cross-section to $t = 0$ and applying the optical theorem. The second publication \cite{si_el_7_90b} used a dataset with much higher statistics and two additional methods were applied: sum of elastic and inelastic cross-sections \cite{si_inel_7} and luminosity-independent method \cite{si_tot_7}. All four total cross-section results are well compatible.

At the energy of $8\un{TeV}$, the luminosity-independent results on elastic, inelastic and total cross-section were published \cite{si_tot_8}. Moreover, the analysis of the $\beta^*=1000\un{m}$ data is in progress. With these data, the separation of Coulomb and nuclear effects is at reach, thus yielding methodically more accurate results.

At $\sqrt s = 2.76\un{TeV}$ TOTEM aims at applying all three total cross-section determination methods. The inelastic part of the analysis is almost completed, the elastic part is ongoing.


 
% ****************************************************************************
% BIBLIOGRAPHY AREA
% ****************************************************************************

\def\Name#1{#1,}
\def\Review#1#2#3#4{#1 {\bf#2} #4 (#3)}

\begin{footnotesize}
\begin{thebibliography}{99}

% TODO: put in order

\bibitem{totem_jinst}
	%The TOTEM Experiment at the CERN Large Hadron Collider
	\Name{The TOTEM collaboration}
	\Review{JINST}{3}{2008}{S08007}

\bibitem{si_el_7_90b}
	%Measurement of proton-proton elastic scattering and total cross-section at sqrt s = 7 TeV
	\Name{The TOTEM collaboration}
	\Review{EPL}{101}{2013}{21002}

\bibitem{si_el_7_3p5}
	%Proton-proton elastic scattering at the LHC energy of sqrt s = 7 TeV
	\Name{The TOTEM collaboration}
	\Review{EPL}{95}{2011}{41001}

\bibitem{si_el_7_90a}
	%First measurements of the total proton-proton cross section at the LHC energy of sqrt s = 7TeV
	\Name{The TOTEM collaboration}
	\Review{EPL}{96}{2011}{21002}

\bibitem{si_inel_7}
	%Measurement of proton-proton inelastic scattering cross-section at sqrt s = 7 TeV
	\Name{The TOTEM collaboration}
	\Review{EPL}{101}{2013}{21003}

\bibitem{si_tot_7}
	%Luminosity-independent measurements of total, elastic and inelastic cross-sections at sqrt s = 7 TeV
	\Name{The TOTEM collaboration}
	\Review{EPL}{101}{2013}{21004}

\bibitem{si_tot_8}
	%A luminosity-independent measurement of the proton-proton total cross-section at sqrt s = 8 TeV
	\Name{The TOTEM collaboration}
	\Review{Phys. Rev. Lett.}{111}{2013}{012001}


\end{thebibliography}
\end{footnotesize}
\end{document}
