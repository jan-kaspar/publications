\input slides.tex
\input utf8-t1

\def\Section#1{%
	\centerline{\EM{#1}}
}

%\def\Em#1{{\cRe#1\cFg}}
%\def\EM#1{{\cRe\bf #1\cFg}}

\def\author{Jan Kašpar}
\def\caption{EDS Blois 2013, Saariselkä, Finland}
\def\date{9 September, 2013}


\newpage %-------------------------------------------------------------------------------------------
\hbox{}\vfil
\title{TOTEM Results on Elastic Scattering and Total Cross-Section}
\vfil
\centerline{\bf Jan Kašpar}
\centerline{on behalf of the TOTEM collaboration}
\vfil
\line{\hss\hss
	\fig[,25mm]{fig/logo_totem_blue.pdf}\hss
	\fig[,25mm]{fig/logo_cern_blue.pdf}\hss
	\fig[,25mm]{fig/logo_fzu_small_blue.pdf}\hss
	\hss}
\vfil
\centerline{\caption}
\centerline{\date}
%\vfil

\footline={}


\newpage %-------------------------------------------------------------------------------------------
\title{Elastic scattering and total cross-section}

\vfil
\cBlue
\line{\hskip17mm Elastic scattering \hskip41mm Total cross-section\hss}
\vskip1mm
\line{\hss
\hbox to12cm{%
	\hss
	\fig[,3cm]{fig/diagram_es.pdf}
	\hss
	\raise14mm\hbox{\fig{fig/optical_theorem.pdf}}
	\hss
	\fig[,3cm]{fig/diagram_tot.pdf}
	\hss
}%
\hss}
\vfil
\vfil

\centerline{\vbox{\hsize10cm
\section{Outline}
\bitm\cBlue
\itm Introduction: \cBlack{\it Roman Pot detectors, ...}\cBlue
\itm Elastic scattering: \cBlack{\it analysis method and results}\cBlue
\itm Total cross-section: \cBlack{\it analysis method and results}\cBlue
\itm Study of Coulomb-nuclear interference
\eitm
}}


\newpage %-------------------------------------------------------------------------------------------

\title{TOTEM and elastic scattering}

\> TOTEM shares the LHC interaction point (IP5) with CMS

\> elastic scattering = 2 anti-collinear protons from the same vertex:

\line{\hss\fig[15.5cm]{fig/elastic_diagram.pdf}\hss}

\vskip-31mm

\line{%
	\vbox{\hsize9cm
		\> 2 diagonals $\Rightarrow$ control of systematics
		\>> left {\it bottom} -- right {\it top}
		\>> left {\it top} -- right {\it bottom}
	}%
	\hskip2mm
	\raise0cm\vbox{\hsize6cm
		\cBlack
		\line{\hskip29mm$\Downarrow$\hss}
		\vskip1mm
		\fig[6.0cm]{fig/rp_station.jpg}%

		\> 2 units $\Rightarrow$ improved:
		\>> event selection
		\>> kinematics reconstruction
	}%
	\hss
}


\newpage %-------------------------------------------------------------------------------------------
\title{Roman Pot detectors}

\line{%
\raise105mm\vtop{\hsize10.5cm
\itskip5mm
\> each station: \cBlack near and far units

\> each unit: \cBlack top, bottom and horizontal Roman Pots

\> Roman Pot
\>> movable beam-pipe insertion
\>>> retracted when beam unstable
\>>> close to beam for data taking
\>> contains: $5\times 2$ back-to-back mounted silicon sensors

\> edge-less silicon sensors
\>> insensitive edge (facing beam): $\approx 50\un{\mu m}$
\>> strips with pitch $66\un{\mu m}$ oriented at $45^\circ$ wrt.~active edge

\> VFAT: \cBlack trigger-capable read-out chip
}\hskip2mm
\vbox{\hsize5cm
	\line{\hss\fig[33mm]{fig/rp_unit.pdf}\hss}%
	\line{\hss\fig[33mm]{fig/rp_pot.pdf}\hss}%
	\line{\hss\fig[33mm]{fig/rp_package.pdf}\hss}%
	\line{\hss\fig[33mm]{fig/rp_hybrid.pdf}\hss}%
}}


\newpage %-------------------------------------------------------------------------------------------
\title{Proton measurement with RPs}
\vskip-2mm

\> proton transport IP5 $\rightarrow$ RP detectors:

\line{\hss\fig[10cm]{fig/ttm_proton_transport.pdf}\hss}

\> optics

\line{\hss\fig[8.8cm]{fig/optics.pdf}\hss}


\> example: elastic sample seen with 3 different optics:

\line{%
	\hss
	\fig[6.8cm]{fig/ttm_hit_distribution.pdf}%
	\hskip10mm
	\raise10mm\vbox{\hsize6cm
		\cBlack
		\hbox{$\Rightarrow$ \em{optics knowledge essential}}%
		\centerline{$\Downarrow$}%
		\hbox{TOTEM can improve optics accuracy}%
	}%
	\hss
}


\newpage %-------------------------------------------------------------------------------------------
\title{Elastic scattering analysis I}

\> entirely data-driven
\> two diagonals, several LHC fills $\simeq$ different experiments $\Rightarrow$ control of systematics

\vfil
\noindent\cRe{\bf 1. Alignment}\cFg
\> prior to data-taking: collimator-like beam-based alignment
\> offline alignment: \em{relative} (analysis of track fit residuals) and\\ \em{absolute wrt.~beam} (symmetries of elastic scattering)

\vfil
\noindent\cRe{\bf 2. Kinematics reconstruction}\cFg

\> tracks in RPs $\mathop{\longrightarrow}$ kinematics at IP ($\xi = 0$ $\Rightarrow$ relatively easy)
\> choice of formulae (using {\it N}ear and {\it F}ar RPs) $\rightarrow$ minimisation of systematics, typically:

$$\th_x^* = {x^{\rm F} - x^{\rm N}\over L_x^{\rm F} - L_x^{\rm N}} ,\qquad \th_y^* = {1\over 2} \left ( {y^{\rm N}\over L_y^{\rm N}} + {y^{\rm F}\over L_y^{\rm F}} \right )$$

\newpage %-------------------------------------------------------------------------------------------
\title{Elastic scattering analysis II}

\line{\vbox{\hsize10cm

\noindent\cRe{\bf 3. Elastic tagging}\cFg

\> angles left = angles right \cBlack(tolerance set by beam divergence: higher $\be^*$ $\Rightarrow$ more stringent cut)
\> vertex left = vertex right
\> protons $\xi \approx 0$ $\Rightarrow$ correlation hit position vs.~track angle at RP

\vskip3mm
\noindent\cRe{\bf 4. Background subtraction}\cFg

\> typically needed only for low $\be^*$ optics

\> interpolation of event distribution surrounding the signal (tagged) region

\vskip3mm
\noindent\cRe{\bf 5. Acceptance corrections}\cFg

\> RP sensors have finite size $\Rightarrow$ low $|\th_y^*|$ cut
\> LHC apertures $\Rightarrow$ high $|\th_y^*|$ cut
\> azimuthal symmetry (verified) $\Rightarrow$\\ geometrical correction (+ smearing around edges)

}%
\vbox{\hsize5.1cm
	\centerline{\em{$\th_x^*$ collinearity\cFg}}%
	\fig[5cm]{fig/cuts_th_x_L_vs_th_x_R.pdf}%

	\vskip5mm

	\centerline{\em{acceptance correction\cFg}}
	\fig[5cm]{fig/acc_corr_phi_lab.pdf}
}%
}


\newpage %-------------------------------------------------------------------------------------------
\title{Elastic scattering analysis III}

\noindent\cRe{\bf 6. Unfolding of resolution effects}\cFg

\> angular resolution \cBlack (better for high $\be^*$)\cBlue: left-right proton comparison
\> Monte Carlo calculation $\Rightarrow$ impact on $t$-distribution

\vfil
\noindent\cRe{\bf 7. Inefficiency corrections}\cFg

\> uncorrelated 1-RP inefficiencies\cBlack: repeat tagging with 3 RPs only and check the signal in 4th RP
\> near-far correlated RP inefficiencies \cBlack(showers from near to far RP)
\> ``pile-up'' = elastic event + another track in a RP \cBlack (prob.~from zero-bias stream)

\vfil
\noindent\cRe{\bf 8. Luminosity}\cFg

\> from CMS (if available), uncertainty $\approx 4\%$
\> from TOTEM (details later on)

\vfil
\noindent\cRe{\bf 9. Study of systematic uncertainties}\cFg

\> final $\d\si/\d t$ $\Rightarrow$ input to Monte-Carlo simulation
\> any analysis parameter: discrepancy simulation vs.~reconstruction $\Rightarrow$ study impact on $t$-distribution


\newpage %-------------------------------------------------------------------------------------------
\ctitle{Elastic scattering results}{$\sqrt{s}$ = 7 TeV}

\line{\hss\AddBox{\vbox{\tab{\bln
\be^* & \hbox{RP approach} & \hbox{$|t|$ range} & \hbox{el.~events} & \hbox{status}\cr\bln
90\un{m} & 4.8 \hbox{ to }6.5\un{\si}	& 0.005\hbox{ to } 0.4\un{GeV^2} 		& 1\un{M}	& \hbox{[EPL 101 (2013) 2100]}\cr\ln
3.5\un{m} & 7\un{\si}					& 0.4\hbox{ to } 2.5\un{GeV^2}			& 66\un{k}	& \hbox{[EPL 95 (2011) 41001]}\cr\ln
3.5\un{m} & 18\un{\si}					& \approx 2\hbox{ to } 3.5\un{GeV^2}	& 10\un{k}	& \hbox{anal.~advanced}\cr\bln
}}}\hss}

\line{\hss\fig[,78mm]{fig/el_results_7.pdf}\hss}

% SAY: 18sigma data compatible with continued exponential-like decrease



\newpage %-------------------------------------------------------------------------------------------
\ctitle{Elastic scattering results}{$\sqrt{s}$ = 8 TeV}

\cBlack

\line{\hss\AddBox{\vbox{\tab{\bln
\be^* & \hbox{RP approach} & \hbox{$|t|$ range} & \hbox{el.~events} & \hbox{status}\cr\bln
1000\un{m} & 3\hbox{ or }10\un{\si}	& 0.0006\hbox{ to }0.2\un{GeV^2}	& 352\un{k}	& \hbox{publ.~in prep.}\cr\ln
90\un{m} & 6\hbox{ to }9.5\un{\si}	& 0.01\hbox{ to }0.3\un{GeV^2}		& 0.68\un{M}& \hbox{[PRL 111 (2013)]}\cr\ln % 012001
90\un{m} & 9.5\un{\si}				& 0.02\hbox{ to }1.4\un{GeV^2}		& 7.2\un{M}	& \hbox{anal.~advanced}\cr\ln
}}}\hss}


\line{\hss\fig[,6cm]{fig/el_results_8.pdf}\hss}

\> dip well visible in the combined $\be^* = 90\un{m}$ data

% SAY: data tables for 90m data will be published together


\newpage %-------------------------------------------------------------------------------------------
\ctitle{Elastic scattering results}{$\sqrt{s}$ = 2.76 TeV}

\line{\hss\AddBox{\vbox{\tab{\bln
\be^* & \hbox{RP approach} & \hbox{$|t|$ range} & \hbox{el.~events} & \hbox{status}\cr\bln
11\un{m} & 5\un{\si} & 0.06\hbox{ to }0.4\un{GeV^2} & 45\un{k} & \hbox{anal.~started}\cr\ln
11\un{m} & 13\un{\si} & 0.4\hbox{ to }0.5\un{GeV^2} & 2\un{k} & \hbox{anal.~started}\cr\bln
}}}\hss}

\line{\hss\fig[,6cm]{fig/el_results_2p76.pdf}\hss}

\> $\be^*=11\un{m}$ optics tuning in progress ($\rightarrow$ $t$ values preliminary)
\> LHC aperture(s) at $\approx 14\un{\si}$
\> dip (expected at $|t| \approx 0.6\un{GeV^2}$) unlikely to be visible


\newpage %-------------------------------------------------------------------------------------------
\title{Total cross-section}

\line{\em{3 complementary methods:}\hss \cBlack$\displaystyle\rh \equiv \left . {\Re {\cal A}_{\rm el}\over \Im {\cal A}_{\rm el}} \right |_{t = 0}$}

\vfil

\line{\hss\fig[13cm]{fig/tot_cs_methods.pdf}\hss}

\vfil

\centerline{\hskip-5mm \cRe$N_{\rm el}$\cBlack{} from RPs\hss \cRe$N_{\rm inel}$\cBlack{} from T2\hss \cRe$\cal L$\cBlack{} from CMS\hss \cRe$\rh$\cBlack{} from COMPETE or TOTEM\hskip-5mm}

\newpage %-------------------------------------------------------------------------------------------
\title{Inelastic cross-section}

\line{%
	\hskip-4mm
	\fig[8cm]{fig/ttm_det_overview.pdf}%
	\hskip5mm
	\raise12mm\vbox{\hsize6.5cm
		\section{Forward inelastic telescope T2}
		\> detects charged particles at\\ $5.3 < |\et| < 6.5$
		\> $\approx 95\un{\%}$ of inelastic events seen (enough to detect 1 track!)
	}
	\hss
}

\section{Inelastic cross-section analysis}

\vfil
\noindent 1) \em{Raw rate}: event counting with T2

\vskip2mm
\line{%
	\hskip10mm
	\hbox{$\downarrow$}%
	\hskip3mm
	\vtop{\hsize5cm
		\noindent \cBlack{experimental corrections}:
	}%
	\hskip3mm
	\vtop{\hsize8cm
		\noindent\SetFontSizesX\it\cDGray trigger and reconstruction inefficiencies, beam-gas event suppression, pile-up consideration
		\cBlack
	}%
	\hss
}

\vskip2mm

\noindent 2) \em{Visible rate}: visible with T2 in perfect conditions

\vskip2mm
\line{%
	\hskip10mm
	\lower2mm\hbox{$\downarrow$}%
	\hskip3mm
	\vtop{\hsize5cm
		\noindent\cBlack{recovery of events with no tracks in T2}:
	}%
	\hskip3mm
	\vtop{\hsize8cm
		\noindent\SetFontSizesX\it\cDGray T1-only events, events with gap over T2, low-mass diffraction, cen.~diff. without tracks in T1 and T2
		\cBlack
	}%
	\hss
}
\vskip2mm

\noindent 3) \em{Physics rate}: true rate of inelastic events

\vfil
\> only one major Monte-Carlo-based correction: \em{low-mass diffraction}\\
$\Rightarrow$ but can be constrained from data ($\si_{\rm tot}^{RP} - \si_{\rm el}^{RP} - \si_{\rm visible}^{T2}$)


\newpage %-------------------------------------------------------------------------------------------
\ctitle{Total cross-section}{$\sqrt{s}$ = 7 and 8 TeV results}


\line{%
	\hss
	\vtop{\hsize7.7cm
		\centerline{\bf\cRe$\sqrt s = 7\un{TeV}$\cFg}
		\vskip3mm
		\line{\hss\fig{fig/tot_cs_methods_res7.pdf}\hss}
	}%
	\hskip2mm
	\vrule
	\hskip2mm
	\vtop{\hsize7.7cm
		\centerline{\bf\cRe$\sqrt s = 8\un{TeV}$\cFg}
		\vskip3mm
		\line{\hss\fig{fig/tot_cs_methods_res8.pdf}\hss}
		\SetFontSizesX
		\> CMS luminosity unavailable
		\> $\cal L$ from luminosity-independent method\\ $\Rightarrow$ normalisation of $\d\si/\d t$ both at $\be^* = 90$ and $1000\un{m}$
	}%
	\hss
}

\newpage %-------------------------------------------------------------------------------------------
\ctitle{Total cross-section}{Results in context}

\line{\hss\fig[13.5cm]{fig/sigma_tot_el_inel_cmp_big.pdf}\hss}

\vskip-6mm

\> analysis at $\sqrt s = 2.76\un{TeV}$: all three methods planned
\>> elastic analysis: ongoing
\>> inelastic analysis: almost finished

\newpage %-------------------------------------------------------------------------------------------
\title{Coulomb-nuclear interference at 8 TeV}

\> $\be^* = 1000\un{m}$ : \cBlack $|t|$ as low as $6\cdot10^{-4}\un{GeV^2}$ $\Rightarrow$ \em{observed Coulomb-nuclear interference}
\cBlack (between Coulomb/electromagnetic and nuclear/strong interactions):

\vfil
\line{\hss\fig[85mm]{fig/plotFitExampleComponents.pdf}\hss}

\vskip-5mm

\> interesting aspects
\>> interference $\Rightarrow$ \em{determination of phase} of nuclear amplitude
\>> separation of Coulomb/nuclear effects $\Rightarrow$ \em{methodically better determination of $\si_{\rm tot}$}
$$\cDGrayRed\si_{\rm tot}^{\rm(nuclear)} \propto \Im {\cal A_{\rm el}^{\rm nuclear}}(t = 0)\cFg$$


\newpage %-------------------------------------------------------------------------------------------
\ctitle{Coulomb-nuclear interference}{Theory}

\vskip-3mm

\cBlue
$${\d\si\over \d t} \propto |{\cal A}^{\rm C+N}|^2\ ,\qquad
{\cal A}^{\rm C+N} = \hbox{interference formula}({\cal A^{\rm C}}, {\cal A}^{\rm N})$$
\vskip1mm
\line{\hss\fig[,21mm]{fig/el_diagrams.pdf}\hss}

\> \em{Coulomb amplitude ${\cal A}^{\rm C}$}: well known (QED, form-factors measured)

\> \em{Nuclear amplitude ${\cal A}^{\rm N}$}
\>> \em{modulus}: constrained by TOTEM data $\Rightarrow$ parametrised:
$$\cBlue\exp(b_1 t + b_2 t^2 + ...)\qquad N_b = \hbox{number of $b_i$ parameters} = 1 \hbox{ to }3$$
\>> \em{phase}: weak guidance from data $\Rightarrow$ test a range of theoretical alternatives 

\> \em{interference formula}
\>> \cBlue simplified West-Yennie \cBlack(SWY) [Phys.~Rev.~172 (1968) 1413-1422]
\>>> traditional but
\>>> only compatible with constant phase and purely exponential modulus
\>> \cBlue Kundrát-Lokajíček \cBlack(KL) [Z.~Phys.~C63 (1994) 619-630]
\>>> no ${\cal A}^{\rm N}$ limitations

\vfil

\centerline{\Em{general approach: exploration}}


\newpage %-------------------------------------------------------------------------------------------
\ctitle{Coulomb-nuclear interference}{Phase of nuclear amplitude}

\line{\hskip-2mm\raise-20mm\vbox{\hsize8.1cm
\itskip10mm

\> \Em{constant phase} -- the simplest choice
\cDGrayRed
$$\SmallerFonts\arg {\cal A}^{\rm N} = p_0$$

\> \Em{central phase} -- similar shape as in many phenomenological models

\cDGrayRed
$$\SmallerFonts\arg {\cal A}^{\rm N} = {\pi\over 2} - \atan {\rh_0\over 1 - {t\over t_{\rm d}}},\ \rh_0 = {1\over \tan p_0}$$
$$\SmallerFonts t_{\rm d} \approx -0.53\un{GeV^2}$$

\> \Em{peripheral phase} [Z.~Phys.~C63 (1994)\break 619-630] -- expected order in impact\break parameter space: elastic collisions more peripheral than inelastic $\langle b^2 \rangle^{\rm el} > \langle b^2 \rangle^{\rm inel}$

\cDGrayRed
$$\SmallerFonts\arg {\cal A}^{\rm N} = p_0 + A \exp\left[ \ka \left( \log {t\over t_{\rm m}} - {t\over t_{\rm m}} + 1 \right) \right]$$
$$\SmallerFonts A \approx 5.53,\ \ka \approx 4.01,\ t_{\rm m} \approx -0.310\un{GeV^2}$$
}\hskip2mm\fig[7.5cm]{fig/compareHadronicPhases.pdf}\hss}


\newpage %-------------------------------------------------------------------------------------------
\ctitle{Coulomb-nuclear interference}{Data fits}

\> data fits $\rightarrow$ for every parameter: value and uncertainty
\>> full $|t|$-range: $6\cdot10^{-4}$ to $0.2\un{GeV^2}$
\>> generalised $\ch^2$ (full covariance matrix)
\>> typical $\ch^2/\hbox{``ndf''} \approx 1$
\>> nuclear phase: only $p_0$ (value at $t=0$) free parameter

\> fits with constant and central phase: undistinguishable
\> fits with $N_b = 1$ and KL or SWY interference formula: undistinguishable

%$$\rh = \left. {\Re {\cal A^{\rm H}}\over \Im {\cal A}^{\rm H}}\right |_{t = 0} =
%0.110 \pm 0.027^{\rm (stat)} \pm 0.010^{\rm (syst)}\ \hbox{}^{\hbox to 7pt{\hss+\hss} 0.013}_{\hbox to 7pt{\hss-\hss} 0.012}\ \hbox{}^{\rm (model)}
%$$

\line{%
	\hss
	\fig[,6cm]{fig/plotDataResults.pdf}%
	\hskip3mm
	\raise50mm\vtop{\hsize56mm
		\let\itcol\cDGrayRed
		\> indications that $N_b = 1$\break insufficient to describe data
		\vskip10mm
		\> $\si_{\rm tot}$: very stable results
		\> green line and band:\\
		previous $\be^* = 90\un{m}$ results
		[PRL 111 (2013) 012001]
	}%
	\hss
}

\newpage %-------------------------------------------------------------------------------------------
\ctitle{Coulomb-nuclear interference}{Variation of peripheral phase}

\> data fits: only $p_0$ is free

\> probe influence of phase shape: vary peak amplitude, width and position:

\vskip2mm

\line{%
	\hss
	\fig[15.5cm]{fig/plotDataResults_per_ranges.pdf}%
	\hss
}

\vfil

\centerline{$\Downarrow$}

\vfil

\centerline{range of fit results}

\vskip 0pt plus 2fil

\newpage %-------------------------------------------------------------------------------------------
\ctitle{Coulomb-nuclear interference}{Results}


\line{\raise25mm\hbox{\SetFontSizesXVI\EM{$\rh \rightarrow$}}\hss\fig[,5cm]{fig/rho_s.pdf}\hss}

\line{\raise25mm\hbox{\SetFontSizesXVI\EM{$\si_{\rm tot} \rightarrow$}}\hss\fig[,5cm]{fig/sigma_tot_details.pdf}\hss}

\newpage %-------------------------------------------------------------------------------------------
\title{Summary}

{
\SmallerFonts

\Section{Elastic differential cross-section}

\> \em{7 TeV}
\>> $\be^*=90\un{m}$ and medium-$|t|$ at $\be^{*}=3.5\un{m}$: published
\>> high-$|t|$ at $\be^{*}=3.5\un{m}$: advanced analysis

\> \em{8 TeV}
\>> $\be^*=1000\un{m}$: publication ongoing
\>> $\be^*=90\un{m}$: advanced analysis

\> \em{2.76 TeV}
\>> $\be^*=11\un{m}$: analysis ongoing

\vfil
\Section{Total cross-section}

\> \em{7 TeV}
\>> $\be^*=90\un{m}$: published

\> \em{8 TeV}
\>> $\be^*=90\un{m}$: published
\>> $\be^*=1000\un{m}$: publication ongoing (+ separation Coulomb/nuclear effects)

\> \em{2.76 TeV}
\>> $\be^*=11\un{m}$: elastic analysis started, inelastic ready

\vfil
\Section{Coulomb-nuclear interference studies}

\> \em{8 TeV}
\>> $\be^*=1000\un{m}$: publication ongoing
}

\newpage %-------------------------------------------------------------------------------------------

\hbox{}
\vfill
\centerline{Backup}
\centerline{$\downarrow$}

\footline={}


\newpage %-------------------------------------------------------------------------------------------
\ctitle{Backup}{RP Alignment I}

\> RPs = movable insertions $\Rightarrow$ each run at different positions
\> required angular precision $\rm \mu rad$ $\Rightarrow$ $\rm\mu m$ alignment precision needed

\vfil
\centerline{$\Downarrow$}
\vfil

\> two types of alignment needed
\>> alignment of mechanical RP edges $\rightarrow$ for machine protection
\>> alignment of RP sensors $\rightarrow$ for physics

\> need alignment \em{wrt.~the beam}

\vfil
\centerline{$\Downarrow$}
\vfil

\centerline{\EM{3-step alignment procedure}:}

\vfil

\noindent1) \em{Collimation alignment}: RP alignment wrt.~beam, rough sensor alignment

\line{\hss\vbox{\hsize8cm
\> procedure prior to data taking
\> standard procedure for LHC collimators

\fig[,25mm]{fig/al_collim_scheme.pdf}}\hskip1mm\hfil\fig[,45mm]{fig/al_collim_ex.pdf}\hss}

\newpage %-------------------------------------------------------------------------------------------
\ctitle{Backup}{RP Alignment II}

\noindent2) \em{Track-based alignment}: relative alignment among sensors

\line{\raise3mm\vbox{\hsize10cm
{\itskip1mm
\> RP station: no magnetic field $\rightarrow$ straight tracks
\> misalignments $\rightarrow$ residuals
\> residual analysis $\rightarrow$ alignment corrections
\> overlap between horizontal and vertical RPs $\rightarrow$ relative alignment among all sensors
\> singular/weak modes: e.g.~overall shift/rotation\\
$\Rightarrow$ need further alignment step
}}\hss\fig[,3.8cm]{fig/overlap.pdf}}

\vfil
\noindent3) \em{Alignment with elastic scattering}: sensor alignment wrt.~beam

\line{\hss\fig[16cm]{fig/al_el_plots_sum_slides.pdf}\hss}

\newpage %-------------------------------------------------------------------------------------------
\ctitle{Backup}{Optics imperfections}

\vfil
\> optics imperfection sources
\>> power-converter error: $\De I / I \approx 10^{-4}$
\>> magnet transfer function: $\De B / B \approx 10^{-3}$
\>> magnet rotation ($<1\un{mrad}$) and displacements ($<0.5\un{mm}$)
\>> magnet harmonics ($\De B / B \approx 10^{-4}$)
\>> beam momentum offset: $\De p / p \approx 10^{-3}$
\>> beam crossing-angle uncertainty

\vfil
\> optics determination
\>> direct measurement -- difficult
\>> indirect from TOTEM observables

\vfil
\> TOTEM optics determination -- variation of magnet/beam parameters (within tolerances) to match TOTEM observables:
\>> $L_y^L / L_y^R$
\>> ${\d L_y\over \d s} / L_y$
\>> $s(L_x = 0)$
\>> $xy$ coupling (tilts in $xy$ plane)
\>> ...

\vfil

\newpage %-------------------------------------------------------------------------------------------
\ctitle{Backup}{Optics refinement with TOTEM data}

\centerline{\em{example for $\be^* = $ 3.5\ m optics}}

\vfil
\line{\hss\hskip0mm\fig[15.5cm]{fig/optics_refinement.png}\hss}

\vfil

\> optics uncertainty reduced:

\cBlack
\centerline{x projection: from $1.6\%$ to $0.17\%$}
\centerline{y projection: from $4.2\%$ to $0.16\%$}

\vfil

\cBlue
{\SetFontSizesVIII
\noindent [H. Niewiadomski, Roman Pots for beam diagnostic, OMCM, CERN, 20-23.06.2011]

\noindent [H. Niewiadomski, F. Nemes, LHC Optics Determination with Proton Tracks, IPAC'12, Louisiana, USA, 20-25.05.2012]
}

\newpage %-------------------------------------------------------------------------------------------
\ctitle{Backup}{Systematic studies}

\> Monte-Carlo: offset in 1 parameter $\Rightarrow$ impact on $t$-distribution

\vskip1mm

\centerline{\em{example for $\be^* = $ 1000\ m optics}}

\centerline{\fig[11.5cm]{fig/predefined_scenarios_plots_eb.pdf}}

\newpage %-------------------------------------------------------------------------------------------
\ctitle{Backup}{Interference formulae}

\> {simplified West-Yennie formula} (SWY)
\>> \em{limitation}: derived for \em{constant slope $B$} (1 $b_i$ parameter only)\\ and \em{constant hadronic phase}
\>> acts as simple {interference phase} (i.e.~$\Ph$ is real-valued)

$$F^{\rm C+H} = F^{\rm C} \e^{\i \al \Ph} + F^{\rm H}\ ,\qquad \Ph = - \left( {B |t|\over 2} + \ga \right)$$

\> {Kundrát-Lokajíček formula} (KL)
\>> any slope $B$, any hadronic phase
\>> more complicated effect ($\Ps$ complex in general)

$$F^{\rm C+H} = F^{\rm C} + F^{\rm H} \e^{\i \al \Ps}$$

$$\Ps(t) =
	\mp \int\limits_{t_{\rm min}}^{0} \d t' \log {t'\over t}\ {\d\phantom{t'}\over\d t'} {\cal F}^2(t')
	\pm \int\limits_{t_{\rm min}}^{0} \d t' \left ( {F^{\rm H}(t')\over F^{\rm H}(t)} - 1 \right)\, {I(t, t')\over 2\pi}
$$

$$I(t, t') = \int\limits_{0}^{2\pi}\d\ph\ {{\cal F}^2(t'')\over t''}\ ,\qquad t'' = t + t' + 2\sqrt{t t'} \cos\ph$$


\newpage %-------------------------------------------------------------------------------------------
\ctitle{Backup}{More elastic observables}

\hbox{}\vfil

\centerline{%
	\fig[,7.5cm]{fig/B_s.pdf}%
	\hskip5mm
	\fig[,7.5cm]{fig/sigma_el_to_sigma_tot.pdf}%
}


\vfil
\eject
\bye
