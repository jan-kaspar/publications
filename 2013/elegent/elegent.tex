%% Use the option review to obtain double line spacing
%% \documentclass[preprint,review,12pt]{elsarticle}

%% Use the options 1p,twocolumn; 3p; 3p,twocolumn; 5p; or 5p,twocolumn
%% for a journal layout:
%% \documentclass[final,1p,times]{elsarticle}
%% \documentclass[final,1p,times,twocolumn]{elsarticle}
%% \documentclass[final,3p,times]{elsarticle}
%%\documentclass[final,3p,times,twocolumn]{elsarticle}
\documentclass[preprint,12pt]{elsarticle}
%% \documentclass[final,5p,times]{elsarticle}
%\documentclass[final,5p,times,twocolumn]{elsarticle}


%% The amssymb package provides various useful mathematical symbols
\usepackage{amssymb}
%% The amsthm package provides extended theorem environments
%% \usepackage{amsthm}

%% The lineno packages adds line numbers. Start line numbering with
%% \begin{linenumbers}, end it with \end{linenumbers}. Or switch it on
%% for the whole article with \linenumbers after \end{frontmatter}.
%% \usepackage{lineno}

%% natbib.sty is loaded by default. However, natbib options can be
%% provided with \biboptions{...} command. Following options are
%% valid:

%%   round  -  round parentheses are used (default)
%%   square -  square brackets are used   [option]
%%   curly  -  curly braces are used      {option}
%%   angle  -  angle brackets are used    <option>
%%   semicolon  -  multiple citations separated by semi-colon
%%   colon  - same as semicolon, an earlier confusion
%%   comma  -  separated by comma
%%   numbers-  selects numerical citations
%%   super  -  numerical citations as superscripts
%%   sort   -  sorts multiple citations according to order in ref. list
%%   sort&compress   -  like sort, but also compresses numerical citations
%%   compress - compresses without sorting
%%
%% \biboptions{comma,round}

% \biboptions{}

%----------------------------------------------------------------------------------------------------

\def\class#1{{\tt #1}}
\def\e{{\rm e}}

%----------------------------------------------------------------------------------------------------


\journal{Computer Physics Communications}

\begin{document}

\begin{frontmatter}

\title{Elegent -- an elastic event generator}

\author{J.~Ka\v spar\corref{cor1}\fnref{fn1}}
\ead{jan.kaspar@cern.ch}
\address{CERN, Geneva, Switzerland}

\cortext[cor1]{Corresponding author}
\fntext[fn1]{On leave of absence from Institute of Physics, AS CR, v.v.i., 182 21 Prague 8, Czech Republic.}


\begin{abstract}
Elastic scattering: simple process but challenge for theory. Lacking hard scale, perturbative QCD can not be applied. Instead, many phenomenological/theoretical models. This package: unified implementation of a number of such models as a C++ library. Ready made t- and b-distributions for most interesting energies. A class performing Monte-Carlo generation of elastic scattering events. This class can easily embedded in any other program (e.g. TOTEM offline software).

TODO: add LHC somewhere (prediction for LHC energies?)
\end{abstract}

\begin{keyword}
elastic hadron scattering \sep Coulomb-nuclear interference \sep Monte-Carlo generator \sep TODO
%% keywords here, in the form: keyword \sep keyword

%% MSC codes here, in the form: \MSC code \sep code
%% or \MSC[2008] code \sep code (2000 is the default)
\end{keyword}

\end{frontmatter}

%% Start line numbering here if you want
%\linenumbers

%----------------------------------------------------------------------------------------------------

\newpage

\noindent
{\bf PROGRAM SUMMARY}

\begin{small}
\noindent
{\em Manuscript Title:} Elegent -- an elastic event generator\\
{\em Authors:} Jan Ka\v spar\\
{\em Program Title:} Elegent\\
{\em Journal Reference:}\\ % Leave blank, supplied by Elsevier.
{\em Catalogue identifier:}\\ % Leave blank, supplied by Elsevier.
{\em Licensing provisions:} TODO\\ %enter "none" if CPC non-profit use license is sufficient.
{\em Programming language:} C++\\
{\em Computer:} any\\
{\em Operating system:} any\\
{\em RAM:} bytes\\
  %RAM in bytes required to execute program with typical data.
%{\em Supplementary material:}\\ % Fill in if necessary, otherwise leave out.
{\em Keywords:} Keyword one, Keyword two, Keyword three, etc.\\ % Please give some freely chosen keywords that we can use in a cumulative keyword index.
{\em Classification:} 11.6 Phenomenological and Empirical Models and Theories\\ % http://cpc.cs.qub.ac.uk/subjectIndex/SUBJECT_index.html)
{\em External routines/libraries:} ROOT\\
  % Fill in if necessary, otherwise leave out.
{\em Nature of problem:}\\
  %Describe the nature of the problem here.
\\
{\em Solution method:}\\
  %Describe the method solution here.
\\
{\em Restrictions:}\\
  %Describe any restrictions on the complexity of the problem here.
\\
{\em Unusual features:}\\
  %Describe any unusual features of the program/problem here.
\\
{\em Additional comments:}\\
  %Provide any additional comments here.
\\
{\em Running time:}\\
  %Give an indication of the typical running time here.
\\

\iffalse
\begin{thebibliography}{0}
\bibitem{1}Reference 1         % This list should only contain those items referenced in the                 
\bibitem{2}Reference 2         % Program Summary section.   
\bibitem{3}Reference 3         % Type references in text as [1], [2], etc.
                               % This list is different from the bibliography at the end of 
                               % the Long Write-Up.
\end{thebibliography}
\fi

\end{small}

\newpage

%----------------------------------------------------------------------------------------------------

\section{Introduction}\label{s:inc}

* why interesting

* why need models

* elastic scattering: strong (hadronic) and Coulomb (and interference)


%----------------------------------------------------------------------------------------------------

\section{Hadronic models}\label{s:had mod}

The following models are currently implemented, for a more detailed model overview see \cite[section 1.1]{jan_thesis} or \cite[section 4]{dremin13}.

{\bf The Model of Block et al.}~is a QCD-inspired model formulated in eikonal formalism \cite{bh99,block06}. The eikonal receives four contributions due to quark-quark, quark-gluon, gluon-gluon interactions and an odderon exchange. The latter is responsible for the difference between $\rm pp$ and $\rm \bar p p$ scattering and is negligible at LHC energies. Each of the eikonal contributions is factorised into energy dependence term (proportional to the integral cross-section of a given sub-process) and impact-parameter profile. All four profiles have the same form but are scaled with a parameter reflecting the areas occupied by quarks and gluons in the nucleon. The gluon-gluon cross-section is modelled after QCD, adopting a gluon distribution function that behaves as $1/x^{1+\epsilon}$ at low momentum fractions $x$. It is this high soft-gluon content that is responsible for the cross-section rise with energy. This model is implemented as class \class{BHModel} in the library source code, using parameters from publication \cite{block06}.

%TODO: problem with 1/2 in eikonal


{\bf The Model of Bourrely et al.}~\cite{bsw79,bsw84,bsw03,bsw11} is also formulated in eikonal description. The eikonal includes two terms corresponding to two scattering mechanisms: pomeron and reggeon exchange. The pomeron term is factorised into a product of $s$- and $b$-dependent functions. The energy dependence is deduced from asymptotic quantum field theory behaviour \cite{wu70}. The impact-parameter profile of the pomeron exchange is derived by assuming similar distributions of electric charge and nuclear matter in the nucleon. The charge distribution is extracted from electromagnetic form factor and is further modified by a slowly varying function in order to get the complete profile function. The reggeons considered in this model include $\rm A_2$, $\rho$ and $\omega$. The trajectories are described with a traditional parametrisation with $s^{\alpha - 1}$ energy dependence and $\e^{-bt}$ momentum-transfer dependence. The reggeon contribution is responsible for the $\rm pp$ and $\rm \bar pp$ difference at low energies and is negligible at LHC energies and higher. This model is implemented as class \class{BSWModel} in the library source code, using parameters from publication \cite{bsw03}.

%TODO: problem with complex logarithm, factor in front of the regeon term, energy dependence S0 does actually depend on t.


 
In {\bf the Model of Islam et al.}~the proton is pictured in an effective quantum field theory model (gauged linear sigma model) \cite{islam06}. The proposed soliton solution has a mass comparable to the proton and divides the proton to two distinct layers: an outer cloud of quark-antiquark condensate and an inner core of topological charge. Later on \cite{islam05}, a third, the inner-most layer has been added: a valence-quark bag. These three layers give rise to three mechanisms contributing to elastic proton-proton scattering, each being dominant at a different region of $t$. At the lowest $|t|$ it is the diffraction scattering originating from a glancing overlapping of the outer clouds \cite{islam84,islam87}, at medium $|t|$ the core-core scattering \cite{islam06} and at high $|t|$ the valence-quark scattering \cite{islam05,islam09}. The diffraction scattering is described in impact-parameter picture using a Fermi profile function. As a consequence of the underlying quantum field theory model, the core-core scattering is mediated by the $\omega$ meson behaving as an elementary particle. The corresponding amplitude thus contains a Feynman propagator and a form factor ensuring an exponential fall off at higher $|t|$ values. Regarding the valence-quark scattering, the model comes with two alternative mechanisms: interaction via a BFKL gluon ladder (denoted {\em hard pomeron} version, HP) or interaction mediated by low-$x$ gluons surrounding the valence quark (variant {\em low-x}, LxG). This model is implemented as class \class{IslamModel} in the library source code, using parameters from publication TODO.

In {\bf the Model of Jenkovszky et al.}~\cite{jenkovszky11} the elastic scattering of nucleons is attributed to $t$-channel exchange of a pomeron, an odderon and reggeons $\rm f$ and $\omega$. The reggeon-exchange amplitudes have the traditional form with $s^{\alpha - 1}$ energy dependence and residua $\e^{-bt}$ and are negligible at LHC energies and above. The pomeron and odderon amplitudes have a more complicated form arising from the assumed double Regge pole nature of the pomeron \cite[section 2]{jenkovszky86}. Moreover, the model comes with several parametrisations of the pomeron Regge trajectory $\alpha(t)$. The linear parametrisation is discussed in the greatest detail in the publication as is the only implemented in Elegent for the moment. This model is implemented as class \class{JenkovszkyModel} in the library source code, using parameters from publication \cite{jenkovszky11}.

{\bf The Model of Petrov et al.}~\cite{petrov02} describes the collision of nucleons as an exchange of reggeons: pomerons, an odderon, $\rm f$ and $\omega$. Each of these trajectories contributes to the eikonal with a term proportional to $s^{\alpha - 1}$ and with a Gaussian profile $\e^{-c b^2}$ ($b$ being the impact parameter). The model has two variants: including two pomerons (denoted {\em 2P}\/) and three pomerons ({\em 3P}\/). This model is implemented as class \class{PPPModel} in the library source code, using parameters from publication \cite{petrov02}.


A {\bf simple exponential model} with a constant phase is implemented (class \class{ExpModel}) for convenience.



%----------------------------------------------------------------------------------------------------

\section{Coulomb amplitudes}\label{s:coul mod}

* calculable from QED, here limit ourselves to OPE approximation, anything beyond that is not anymore QED (proton can get excited in loops), more detailed discussion \cite[section 1.3.6]{jan_thesis}

* limit to a low |t| range?

* QED formula with form factors (electric and magnetic) -- experimentally determined, different parameterizations and fits (More details \cite[section 1.3.1]{jan_thesis}):

* formula with an effective formfactor (from \cite[equation (31)]{block06})

\begin{itemize}
\item Hofstadter \cite{hofstadter58}
\item Borkowski \cite{borkowski74,borkowski75}
\item Kelly \cite{kelly04}
\item Arrington \cite{arrington07}
\item Puckett \cite{puckett10}
\end{itemize}



%----------------------------------------------------------------------------------------------------

\section{Coulomb-hadronic interference}\label{s:int mod}

* same physics, but two calculation approaches: QFT and eikonal

West and Yennie \cite{wy68}, \cite[section 1.3.4]{jan_thesis}

Kundr\' at and Lokaj\' i\v cek, \cite{kl94}, \cite[section 1.3.5]{jan_thesis};
numerical considerations \cite[section 1.3.6]{jan_thesis}

%----------------------------------------------------------------------------------------------------

\section{Program package}\label{s:prog}

package \cite{elegent}, ref to Wiki as user's and implementation manual

* modules: implementation of models (C++) library, class for Monte-Carlo event simulation, data sets (t- and b-distributions, plots) for some energies of interest

standard init

* relation between diff. cs., total cs. and amplitudes: to define the amplitude convention used; iF NEEDED

* table: model <--> tag -- remove ??

%----------------------------------------------------------------------------------------------------


\begin{thebibliography}{00}

%TODO: put them in order

\def\Name#1{#1, }
\def\REVIEW#1#2#3#4{{\it #1} {\bf #2} (#3) #4}

\bibitem{islam84}
	\Name{ISLAM, M. M., FEARNLEY, T. and GUILLAUD, J. P.}
	\REVIEW{Nuovo Cim.}{A81}{1984}{737}
	%title          = "$\bar pp$ and $pp$ elastic scattering from $10\,GeV$ to $1000\,GeV$ center-of-mass energy",
	%doi            = "10.1007/BF02724225",

\bibitem{islam87}
	\Name{ISLAM, M. M., INNOCENTE V., FEARNLEY T. and SANGUIETTI, G.}
	\REVIEW{Europhys. Lett.}{4}{1987}{189--196}
	%title="High energy $pp$ and $\bar p p$ Elastic Scattering and Nucleon Structure",

%\bibitem{islam03}
%	\Name{ISLAM, M. M., LUDDY, R. J. and PROKUDIN, A. V.}
%	\REVIEW{Mod. Phys. Lett.}{A18}{2003}{743--752}
%	%title="pp elastic scattering at LHC and nucleon structure",
%	%number="11",

\bibitem{islam05}
	\Name{ISLAM, M. M., LUDDY, R. J. and PROKUDIN, A. V.}
	\REVIEW{Phys. Lett.}{B605}{2005}{115--122}
	%title          = "$pp$ elastic scattering at LHC in near forward direction",
	%doi            = "10.1016/j.physletb.2004.11.006",
    
% the review
\bibitem{islam06}
	\Name{ISLAM, M. M., LUDDY, R. J. and PROKUDIN, A. V.}
	\REVIEW{Int. J. Mod. Phys.}{A21}{2006}{1--42}
	%title     = "Near forward $p p$ elastic scattering at LHC and nucleon structure",
	%doi       = "10.1142/S0217751X06033271",


\bibitem{islam09}
	\Name{ISLAM, M. M., KA\v SPAR, J. and LUDDY, R. J.}
	\REVIEW{Mod. Phys. Lett.}{A24}{2009}{485--496}
	%title          = "Deep-elastic p p scattering at LHC from low-x gluons",
	%doi            = "10.1142/S0217732309030266",


\bibitem{wu70}
	\Name{CHENG H. and WU, T. T.}
	\REVIEW{Phys. Rev. Lett.}{24}{1970}{1456--1460}
	%title     = "Limit of Cross-Sections at Infinite Energy",
	%doi       = "10.1103/PhysRevLett.24.1456",


\bibitem{bsw79}
	\Name{BOURRELY C., SOFFER, J. and WU, T. T.}
	\REVIEW{Phys. Rev.}{D19}{1979}{3249}
	%title     = "A New Impact Picture for Low and High-Energy Proton Proton Elastic Scattering",
	%doi       = "10.1103/PhysRevD.19.3249",

\bibitem{bsw84}
	\Name{BOURRELY C., SOFFER, J. and WU, T. T.}
	\REVIEW{Nucl. Phys.}{B247}{1984}{15}
	%title     = "Impact Picture Expectations for Very High-Energy Elastic $p p$ and $p \bar p$ Scattering",
	%doi       = "10.1016/0550-3213(84)90369-9",

\bibitem{bsw03}
	\Name{BOURRELY C., SOFFER, J. and WU, T. T.}
	\REVIEW{Eur. Phys. J.}{C28}{2003}{97--105}
	%title     = "Impact picture phenomenology for $\pi^+$-$p$, $K^+$-$p$ and $p p$, $\bar p p$ elastic scattering at high-energies",
	%doi       = "10.1140/epjc/s2003-01159-7",


\bibitem{bsw11}
	\Name{BOURRELY C., SOFFER, J. and WU, T. T.}
	\REVIEW{Eur. Phys. J.}{C71}{2011}{1601}
	%title          = "Determination of the forward slope in $pp$ and $\bar  pp$ elastic scattering up to LHC energy",
	%doi            = "10.1140/epjc/s10052-011-1601-x",


\bibitem{bh99}
	\Name{BLOCK, M. M., GREGORES, E. M., HALZEN, F. and PANCHERI, G.}
	\REVIEW{Phys. Rev.}{D60}{199}{054024}
	%title     = "Photon - proton and photon-photon scattering from nucleon-nucleon forward amplitudes",
	%doi       = "10.1103/PhysRevD.60.054024",

\bibitem{block06}
	\Name{BLOCK, M. M.}
	\REVIEW{Phys. Rept.}{436}{2006}{71--215}
	%title     = "Hadronic forward scattering: Predictions for the Large Hadron Collider and cosmic rays",
	%doi       = "10.1016/j.physrep.2006.06.003",

%\bibitem{bh11}
%	\Name{BLOCK, M. M. and HALZEN, F.}
%	\REVIEW{Phys. Rev.}{D83}{2011}{077901}
%	%title          = "Forward hadronic scattering at 7 TeV: Predictions for the LHC: An Update",
%	%doi            = "10.1103/PhysRevD.83.077901",


\bibitem{petrov02}
	\Name{PETROV, V. A. and PROKUDIN, A. V.}
	\REVIEW{Eur. Phys. J.}{C23}{2002}{135--143}
	%title          = "The First three pomerons...",
	%doi            = "10.1007/s100520100838",

\bibitem{jenkovszky86}
	\Name{JENKOVSZKY, L.~L.}
	\REVIEW{Fortchr. Phys.}{34}{1986}{791 -- 816} % No. 12


\bibitem{jenkovszky11}
	\Name{JENKOVSZKY, L.~L., LENGYEL, A.~I. and LONTKOVSKYI, D.~I.}
	\REVIEW{Int. J. Mod. Phys. A}{26}{2011}{4755--4771}
	%DOI: 10.1142/S0217751X11054760


\bibitem{jan_thesis}
    \Name{KA\v SPAR J.}
    PhD Thesis, CERN-THESIS-2011-214, {\tt http://cdsweb.cern.ch/record/1441140}

\bibitem{hofstadter58}
	\Name{HOFSTADTER, R., BUMILLER, F. and YEARIAN, M. R.}
	\REVIEW{Rev. Mod. Phys.}{30}{1958}{482--497}
	%title = "Electromagnetic Structure of the Proton and Neutron",
	%doi = "10.1103/RevModPhys.30.482",

\bibitem{borkowski74}
	\Name{BORKOWSKI F., PEUSER, P., SIMON, G. G., WALTHER et al.}
	\REVIEW{Nucl. Phys.}{A222}{1974}{269--275}
	%title = "Electromagnetic form factors of the proton at low four-momentum transfer",
	%doi = "DOI: 10.1016/0375-9474(74)90392-3",

\bibitem{borkowski75}
	\Name{BORKOWSKI F., PEUSER, P., SIMON, G. G., WALTHER et al.}
	\REVIEW{Nucl. Phys.}{B93}{1975}{461}
	%title          = "Electromagnetic Form-Factors of the Proton at Low Four-Momentum Transfer",
	%doi            = "10.1016/0550-3213(75)90514-3",

\bibitem{kelly04}
	\Name{KELLY, J. J.}
	\REVIEW{Phys. Rev.}{C70}{2004}{068202}
	%title = "Simple parametrization of nucleon form factors",
	%doi = "10.1103/PhysRevC.70.068202",

\bibitem{arrington07}
	\Name{ARRINGTON, J., MELNITCHOUK, W. and TJON, J. A.}
	\REVIEW{Phys. Rev.}{C76}{2007}{035205}
	%title          = "Global analysis of proton elastic form factor data with two-photon exchange corrections",
	%doi            = "10.1103/PhysRevC.76.035205",

\bibitem{puckett10}
	\Name{PUCKETT, A. J. R. (GEp-III collaboration)}
	arXiv: nucl-ex/1008.0855
	%title          = "Final Results of the GEp-III Experiment and the Status of the Proton Form Factors",
	%year           = "2010",
	%eprint         = "nucl-ex/1008.0855",
	%archivePrefix  = "arXiv",

\bibitem{wy68}
	\Name{WEST, G. B. and YENNIE, D. R.}
	\REVIEW{Phys. Rev.}{172}{1968}{1413-1422}
	%title     = "Coulomb interference in high-energy scattering",
	%doi       = "10.1103/PhysRev.172.1413",

\bibitem{kl94}
	\Name{KUNDR\' AT, V. and LOKAJ\' I\v CEK, M.}
	\REVIEW{Z. Phys.}{C63}{1994}{619--630}
	%title     = "High-energy scattering amplitude of unpolarized and
	%doi       = "10.1007/BF01557628",

\bibitem{elegent}
	http://elegent.hepforge.org/

\bibitem{dremin13}
	\Name{DREMIN, I. M.}
	\REVIEW{Physics - Uspekhi}{56}{2013}{3--28}
	


\end{thebibliography}

\end{document}
