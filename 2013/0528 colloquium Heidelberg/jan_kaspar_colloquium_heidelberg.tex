\input slides.tex
\input utf8-csf

\iftrue
	\StdBackground
\else
	\SetBackground{fig/bg_prepare.pdf}
	\def\cYe{\cBlack}
	\def\cLRe{\cBlack}
	\def\cRe{\cBlack}
\fi

%\font\TitleFont = cs-qplb-sc at 15pt
%\font\ContextFont = cs-qplb-sc at 10pt
\font\ExtraTitleFont = pzcmi8t at 20pt
\font\TitleFont = pzcmi8t at 15pt
\font\ContextFont = pzcmi8t at 10pt
\font\PartFont = cs-qplb-sc at 15pt

%%%%%%%%%%%%%%%%%%%%
%\SetFontSizesXII
%\itskip5mm
%\iitskip0mm
%\itindent3mm


%\def\NormalSettings{}
%%%%%%%%%%%%%%%%%%%%

\def\Em#1{\em{\cLRe#1\cFg}}
\def\EM#1{{\cYe\TitleFont #1\cFg}}


\def\date{28 May, 2013}


\newpage %-------------------------------------------------------------------------------------------
\hbox{}\vfil
\title{\ExtraTitleFont The TOTEM Experiment}
\vfil
\font\NameFont=cs-qplb at 12pt
\centerline{\NameFont Jan Kašpar}
\centerline{on behalf of the TOTEM collaboration}
\vfil
\line{\hss\hss
	\fig[,25mm]{fig/logo_totem_white.pdf}\hss
	\fig[,25mm]{fig/logo_cern_white.pdf}\hss
	\fig[,25mm]{fig/logo_fzu_small_white.pdf}\hss
	\hss}
\vfil
\centerline{Heidelberg Particle Physics Colloquium, \date}
\vfil

{
\footline={}



\newpage %-------------------------------------------------------------------------------------------
\title{Outline}
}

\vfil

\line{\hss\vbox{\hsize8cm
\itindent5mm
\bitm
\itm Physics programme
\itm Detector apparatus
\> forward-proton measurement with Roman Pots
\itm Analyses and results
\> list the analyses here?
\itm Outlook
\itm Summary
\eitm
}\hss}

TODO: roman numerals

\vfil
\vfil

\newpage %-------------------------------------------------------------------------------------------
\hbox{}
\vfil

\line{\hss\TitleFont\cYe part I\cFg\hss}
\vskip2mm
\centerline{\PartFont\cYe Physics programme\cFg}

\vfil


\newpage %-------------------------------------------------------------------------------------------
\title{}

Forward hadronic phenomena at LHC

\> low momentum transfers --> non perturbative QCD --> interesting

\> forward particles

\> rapidity gaps

\> the traditional process figures

\> xi ~ De eta


\newpage %-------------------------------------------------------------------------------------------
\title{}


\newpage %-------------------------------------------------------------------------------------------
\title{}

(non-)diffractive interactions

the classical plot -- colour vs. colourless exchange 

\newpage %-------------------------------------------------------------------------------------------
\title{}

\fig*[,10.5cm]{fig/ttm_processes.pdf}

+ comments from Hubert's presentation

\newpage %-------------------------------------------------------------------------------------------
\title{}


\newpage %-------------------------------------------------------------------------------------------
\hbox{}
\vfil

\line{\hss\TitleFont\cYe part II\cFg\hss}
\vskip2mm
\centerline{\PartFont\cYe Detector apparatus\cFg}

\vfil


\newpage %-------------------------------------------------------------------------------------------
\title{LHC access points}

\line{\hss\fig*[,10cm]{fig/lhc.pdf}\hss}

\> TOTEM shares IP with CMS --> collaboration possible

\newpage %-------------------------------------------------------------------------------------------
\title{TOTEM Detectors}

\vfil

\line{%
	\hss
	\fig*[11.5cm]{fig/ttm_det_overview.pdf}
	\hskip2mm
	\raise9mm\vbox{\hsize4.1cm
		$\leftarrow$ \cOr telescopes T1 and T2\cFg\\
		\cYe{\it charged particles from\\ inelastic collisions}\cFg

		\centerline{T1: $3.1 < |\et| < 4.7$}
		\centerline{T2: $5.3 < |\et| < 6.5$}

		\vskip8mm
		$\leftarrow$ \cOr Roman Pots\cFg{} at the LHC\\
		\cYe{\it elastic and diffractive protons}\cFg
	}
	\hss
}

\> all detectors symmetrically on both sides of IP5
\> all detectors trigger-capable
\> all detectors radiation tolerant

\vfil

\newpage %-------------------------------------------------------------------------------------------
\title{Telescope T1}



\newpage %-------------------------------------------------------------------------------------------
\title{Telescope T2}



\newpage %-------------------------------------------------------------------------------------------
\title{Roman Pots}

\> typical approach: $10\un{\si}$


\newpage %-------------------------------------------------------------------------------------------
\title{Proton measurement with Roman Pots}

\vskip-5mm

\> LHC lattice between IP5 and RPs at $220\un{m}$

\vskip1mm
%\line{\hss 5 quadrupoles, 2 dipoles, corrector magnets and drift spaces:\hss}
\line{\hss\fig*[13.5cm]{fig/RP_stations.pdf}\hss}

\> \Em{proton transport}: described as in linear optics

\line{\hss\fig*[11.5cm]{fig/ttm_proton_transport.pdf}\hss}

\vskip1mm	
\line{\hss$\displaystyle
\pmatrix{x\cr \th_x\cr y\cr \th_y\cr \xi}_{\rm RP} =
\underbrace{\pmatrix{
	v_x & L_x & \cdot & \cdot & D_x\cr
	\cdot & \cdot & \cdot & \cdot & \cdot\cr
	v_y & L_y & \cdot & \cdot & D_y\cr
	\cdot & \cdot & \cdot & \cdot & \cdot\cr
	\cdot & \cdot & \cdot & \cdot & 1\cr
}}_{\hbox{product from all lattice elements}}
\pmatrix{x^*\cr \th_x^*\cr y^*\cr \th_y^*\cr \xi}_{\rm IP}
$
\hskip10mm
\lower10mm\vbox{
\hbox{\Em{optical functions:}}
\hbox{effective length $L$}
\hbox{magnification $v$}
\hbox{dispersion $D$}
\hbox{\strut}
\hbox{$\xi = \De p / p$: momentum loss}
}
\hss}

\> \Em{proton reconstruction}: inverted transport RPs $\longrightarrow$ IP

\>> optical parameters functions of $\xi$ $\Rightarrow$ reconstruction is non-linear problem

\>> \cYe good knowledge of optics is crucial\cFg

%\fig*[,2.8cm]{fig/optics.pdf}%

\newpage %-------------------------------------------------------------------------------------------
\title{LHC optics}

\> optics defines \Em{what} and \Em{how} can be observed -- a CD sample seen with 2 different optics

\vskip2mm
\line{%
	\hss
	\vtop{\hsize7.7cm
		\centerline{\BiggerFonts\bf\cOr$\bf\be^* = 90\un{m}$\cFg}
		\centerline{$\displaystyle L_x \approx 0,\quad L_y \approx 260\un{m},\quad D_x \approx 4\un{cm}$}
		
		\centerline{diffractive protons in \cYe vertical RPs\cFg}
		\line{\hss\fig[,4.5cm]{fig/hit_dist_90.png}\hss}
	}%
	\hskip2mm
	\vtop{\hsize7.7cm
		\centerline{\cOr{\BiggerFonts\bf low $\bf\be^*$\cFg{}}}
		\centerline{$\displaystyle L_x \approx 1.7\un{m},\quad L_y \approx 14\un{m},\quad D_x \approx 8\un{cm}$}
		
		\centerline{diffractive protons in \cYe horizontal RPs\cFg}
		\line{\hss\fig[,4.5cm]{fig/hit_dist_0p7.png}\hss}
	}%
	\hss
}

%\line{\hss elastic proton sample seen with different optics:\hss}
%\line{\hss \fig*[,6cm]{fig/ttm_hit_distribution.pdf}\hss}

\vfil
\> optics carefully optimised for TOTEM special runs

\vfil
\> optics typically ``labelled'' by $\be^* \equiv$ betatron function at IP
\>> beam width: $\sqrt{\ep \be}$, beam angular divergence: $\sqrt{\ep / \be}$; $\ep$: a measure of beam size/divergence
\>> luminosity $\propto \hbox{(beam width at IP)}^{-2} \propto 1/\be^*$
\>> example: \cYe high $\be^*$ $\Rightarrow$ reduced luminosity but protons ``more parallel''\cFg


\newpage %-------------------------------------------------------------------------------------------
\title{Run scenarios}

{

\advance\hsize\horizontalmargin\line{\kern-\horizontalmargin
\fig[,4.47cm]{fig/acceptance_2.pdf}
\fig[,4.47cm]{fig/acceptance_90.pdf}
\fig[,4.47cm]{fig/acceptance_1535.pdf}\hss
}

\line{\kern-\horizontalmargin\SmallerFonts\hss
\vtop{\hsize4.7cm\obeylines\leftskip0pt plus1fil\rightskip0pt plus1fil\parfillskip0pt
	\cYe{\NormalFonts\bf low $\be^*$}\cFg
	\hrule\vskip1mm
	$\be^* = 0.5$ to $3\un{m}$
	\vskip1mm
	${\cal L} \approx 10^{30}\un{cm^{-2}s^{-1}}$
		\vskip\baselineskip
	elastic data available
	$0.4 \ls |t/{\rm GeV^2}| \ls 3.5$
		\vskip\baselineskip
	resolution
	$\si(\th^*) \approx 15\un{\mu rad}$
	$\si(\xi) \approx 0.2\un{\%}$
		\vskip3\baselineskip
	\em{\cYe diffraction, high $|t|$ elastic scattering, low cross-section processes\cFg}
}
\hss
\vtop{\hsize4.7cm\obeylines\leftskip0pt plus1fil\rightskip0pt plus1fil\parfillskip0pt
	{\cYe\NormalFonts\bf medium $\be^*$\cFg}
	\hrule\vskip1mm
	$\be^* = 90\un{m}$
	\vskip1mm
	${\cal L} \approx 10^{28}\un{cm^{-2}s^{-1}}$
		\vskip1\baselineskip
	elastic data available
	$ 10^{-2} < |t/{\rm GeV^2}| \ls 1.3 $
		\vskip\baselineskip
	resolution
	$\si(\th^*) \approx 1.7\un{\mu rad}$
	$\si(\xi) \approx 0.4$ to $0.6\un{\%}$
		\vskip\baselineskip
	all $\xi$ seen, universal optics
		\vskip\baselineskip
	\em{\cYe diffraction, mid $|t|$ elastic scattering, total cross section\cFg}
}
\hss
\vtop{\hsize4.7cm\obeylines\leftskip0pt plus1fil\rightskip0pt plus1fil\parfillskip0pt
	{\cYe\NormalFonts\bf high $\be^*$\cFg}
	\hrule\vskip1mm
	$\be^* \gs 1000\un{m}$
	\vskip1mm
	${\cal L} \approx 10^{27}\un{cm^{-2}s^{-1}}$
		\vskip\baselineskip
	elastic data available
	$ 6\cdot10^{-4} < |t/{\rm GeV^2}| < 0.3$
		\vskip\baselineskip
	resolution
	$\si(\th^*) \approx 0.4\un{\mu rad}$
	%$\si(\xi) \approx 2\div 10\cdot10^{-3}$
		\vskip2\baselineskip
	all $\xi$ seen
		\vskip\baselineskip
	\em{\cYe total cross section, low $|t|$ elastic scattering\cFg}
}
\hss}

}

\newpage %-------------------------------------------------------------------------------------------
\title{Optics imperfections}

\centerline{\Em{good optics knowledge essential for reconstruction}}

\vfil
\> optics imperfection sources
\>> power-converter error: $\De I / I \approx 10^{-4}$
\>> magnet transfer function: $\De B / B \approx 10^{-3}$
\>> magnet rotation (< $1\un{mrad}$) and displacements ($<0.5\un{mm}$)
\>> magnet harmonics ($\De B \approx 10^{-4}$)
\>> beam momentum offset: $\De p / p \approx 10^{-3}$
\>> beam crossing-angle uncertainty

\vfil
\> optics determination
\>> direct measurement -- difficult
\>> indirect from TOTEM observables

\vfil
\> TOTEM optics determination -- variation of magnet/beam parameters (within tolerances) to\\ match TOTEM observables:
\>> $L_y^L / L_y^R$
\>> ${\d L_y\over \d s} / L_y$
\>> $s(L_x = 0)$
\>> $xy$ coupling (tilts in $xy$ plane)
\>> ...

\vfil


\newpage %-------------------------------------------------------------------------------------------
\title{Optics refinement with TOTEM data}

\centerline{\Em{example for $\be^* = $ 3.5\ m optics}}

\vfil
\line{\hss\hskip-5mm\fig[16cm]{fig/optics_refinement.png}\hss}

\vfil

\> optics uncertainty reduced:

\centerline{x projection: from $1.6\%$ to $0.17\%$}
\centerline{y projection: from $4.2\%$ to $0.16\%$}

\vfil

{\SmallerFonts
\> details:
\>> H. Niewiadomski, Roman Pots for beam diagnostic, OMCM, CERN, 20-23.06.2011
\>> H. Niewiadomski, F. Nemes, LHC Optics Determination with Proton Tracks, IPAC'12, Louisiana, USA, 20-25.05.2012
}

\vfil

\newpage %-------------------------------------------------------------------------------------------
\title{Alignment of Roman Pots}

\> RPs = movable insertions $\Rightarrow$ each run at different positions
\> required angular precision micro-radians $\Rightarrow$ micro-metre alignment precision needed

\vfil
\centerline{$\Downarrow$}
\vfil

\> two types of alignment needed
\>> alignment of mechanical RP edges $\rightarrow$ for machine protection
\>> alignment of RP sensors $\rightarrow$ for physics

\> need alignment \Em{wrt.~the beam}

\vfil
\centerline{$\Downarrow$}
\vfil

\centerline{\bf\cOr 3-step alignment procedure\cFg}

\vfil

\noindent1) \em{\cYe Collimation alignment}\cFg: RP alignment wrt.~beam, rough sensor alignment

\line{\hss\vbox{\hsize8cm
\> standard procedure for LHC collimators

\fig[,3cm]{fig/al_collim_scheme.pdf}}\hskip1mm\hfil\fig[,4.5cm]{fig/al_collim_ex.pdf}\hss}

\newpage %-------------------------------------------------------------------------------------------
\title{Alignment of Roman Pots}


\noindent2) \em{\cYe Track-based alignment\cFg}: relative alignment among sensors

\line{\raise3mm\vbox{\hsize10cm
{\itskip1mm
\> RP station: no magnetic field $\rightarrow$ straight tracks
\> misalignments $\rightarrow$ residuals
\> residual analysis $\rightarrow$ alignment corrections
\> overlap between horizontal and vertical RPs $\rightarrow$ relative alignment among all sensors
\> singular/weak modes: e.g.~overall shift/rotation\\
$\Rightarrow$ need further alignment step
}}\hss\fig[,3.8cm]{fig/overlap.pdf}}

\vfil
\noindent3) \em{\cYe Alignment with physics processes (elastic scattering)\cFg}: sensor alignment wrt.~beam

\line{\hss\fig[16cm]{fig/al_el_plots_sum_slides.pdf}\hss}

\vfil

\newpage %-------------------------------------------------------------------------------------------
\hbox{}
\vfil

\line{\hss\TitleFont\cYe part III\cFg\hss}
\vskip2mm
\centerline{\PartFont\cYe Analyses and results\cFg}

\vfil

\newpage %-------------------------------------------------------------------------------------------
\hbox{}
\vfil
\title{Elastic scattering at $\sqrt{\hbox{s }}$ = 7 TeV}

\vskip0pt plus0.5fil
\line{\hss\fig[,3cm]{fig/diagram_es_lab.pdf}\hss}

\vskip0pt plus0.5fil
\line{\hss\fig[,3cm]{fig/topology_es.pdf}\hss}

\newpage %-------------------------------------------------------------------------------------------
\title{Pre-TOTEM status}


\newpage %-------------------------------------------------------------------------------------------
\title{}

\> ES dsigma/dt, plus extrapol. for si tot

\> pre-TOTEM situation -- models with large differences
\>> different t regions -- probing different physics/QCD regimes

\> compare to older data: trends (peak shrinking and diff. min move) confirmed

\newpage %-------------------------------------------------------------------------------------------
\title{}

\> si tot in all three ways
\>> si inel
\> ES rho

\newpage %-------------------------------------------------------------------------------------------
\hbox{}
\vfil
\title{Total cross-section}

\vskip0pt plus0.5fil
\line{\hss\fig[,3cm]{fig/diagram_tot.pdf}\hss}

\vfil

\newpage %-------------------------------------------------------------------------------------------
\title{Pre-TOTEM status}


\newpage %-------------------------------------------------------------------------------------------
\hbox{}
\vfil
\title{Elastic scattering at $\sqrt{\hbox{s }}$ = 8 TeV and $\SetFontSizesXII\rh$ value}

%\vfil
%\vskip0pt plus0.5fil
%\line{\hss\fig[,3cm]{fig/diagram_tot.pdf}\hss}

% TODO: a figure?

\newpage %-------------------------------------------------------------------------------------------
\hbox{}
\vfil
\title{Forward charged-particle multiplicities}

% TODO: a figure

\newpage %-------------------------------------------------------------------------------------------
\title{}

\> dN/deta
\>> 7 TeV + 8 TeV

\newpage %-------------------------------------------------------------------------------------------
\title{Intermezzo: Optics for diffractive studies}
\vskip-8mm
$$x_{\rm RP} = v_x x^* + L_x \th_x^* + \xi D_x\ ,\qquad \xi = {\De p\over p_0}$$
\vskip-2mm

\line{%
	\hss
	\vtop{\hsize7.7cm
		\centerline{\BiggerFonts\bf\cOr$\bf\be^* = 90\un{m}$\cFg}
		%\> vertex: $160\un{\mu m}$
		%\> beam-divergence: $2.6\un{\mu rad}$
		\> optical functions at RP 220:
		$$L_x \approx 0,\quad L_y \approx 260\un{m},\quad D_x \approx 4\un{cm}$$
		
		\centerline{$\Downarrow$}
		\centerline{diffractive protons in \cYe vertical RPs\cFg}
		\centerline{\SmallerFonts (a CD sample)}
		\line{\hss\fig[,4.5cm]{fig/hit_dist_90.png}\hss}

		\> $|\xi|_{\rm min} = 0\%$ $\Rightarrow$ \cYe low masses \cFg
		\> $\xi$-resolution
		\>> RPs only: $(0.4 \hbox{ to } 1)\%$ ($t$-dependent)
		\>> with CMS vertex: $\approx 2\times$ better
		\vskip\baselineskip
		\centerline{\Em{used so far}\cFg}
	}%
	\hskip2mm
	\vrule
	\hskip2mm
	\vtop{\hsize7.7cm
		\centerline{\cOr{\BiggerFonts\bf low $\bf\be^*$\cFg{}} ($0.7\un{m}$ here)}
		%\> vertex: $14\un{\mu m}$
		%\> beam-divergence: $29\un{\mu rad}$
		\> optical functions at RP 220:
		$$L_x \approx 1.7\un{m},\quad L_y \approx 14\un{m},\quad D_x \approx 8\un{cm}$$
		
		\centerline{$\Downarrow$}
		\centerline{diffractive protons in \cYe horizontal RPs\cFg}
		\centerline{\SmallerFonts (a CD sample)}
		\line{\hss\fig[,4.5cm]{fig/hit_dist_0p7.png}\hss}

		\> $|\xi|_{\rm min} = 2.8\%$  $\Rightarrow$ \cYe higher masses \cFg
		\> $\xi$-resolution
		\>> RPs only: $\approx 0.2\%$
		\vskip\baselineskip
		\vskip\baselineskip
		\centerline{\Em{planned after long shutdown}\cFg}
	}%
	\hss
}

\newpage %-------------------------------------------------------------------------------------------
\hbox{}
\vfil
\title{Single diffraction}

\vskip0pt plus0.5fil
\line{\hss\fig[,3cm]{fig/diagram_sd_lab.pdf}\hss}

\vskip0pt plus0.5fil
\line{\hss\fig[,3cm]{fig/topology_sd_T1_opp.pdf}\hss}

\vfil
\line{\raise2mm\vtop{\hsize7cm
\> {\BiggerFonts \cYe $\xi \approx e^{-\De \et}$\cFg} $\Rightarrow$ double ``determination'' of $\xi$
\>> from proton (Roman Pots)
\>> from rapidity gap (T1/2)
}\hfil\vtop{\hsize6.3cm
\> mass of diffractive system X
$$m_X \approx \sqrt{s\xi}$$
\> minimal mass visible (T2 acceptance):
$$m_X \ge 3.4\un{GeV}$$
}\hfil}

\newpage %-------------------------------------------------------------------------------------------
\context{Single diffraction}
\title{Topologies / diffractive-mass classes}

\vskip-5mm

%\noindent\EM{topologies}
\> \Em{T2 opposite arm}: $m_X \approx 3.4$ to $7\un{GeV}$, $2\cdot10^{-7} < \xi < 1\cdot 10^{-6}$

\line{\hss\fig[7.1cm]{fig/topology_sd_T2_opp.pdf}\hss}

\> \Em{T1 opposite arm}: $m_X \approx 7$ to $350\un{GeV}$, $1\cdot10^{-6} < \xi < 2.5\cdot 10^{-3}$

\line{\hss\fig[7.1cm]{fig/topology_sd_T1_opp.pdf}\hss}

\> \Em{T1 same arm}: $m_X \approx 350$ to $1100\un{GeV}$, $2.5\cdot10^{-3} < \xi < 2.5\cdot 10^{-2}$

\line{\hss\fig[7.1cm]{fig/topology_sd_T1_same.pdf}\hss}

\> \Em{T2 same arm}: $m_X \gs 1100\un{GeV}$, $\xi > 2.5\cdot 10^{-2}$

\line{\hss\fig[7.1cm]{fig/topology_sd_T2_same.pdf}\hss}

\newpage %-------------------------------------------------------------------------------------------
\context{Single diffraction}
\title{Analysis steps}

\> available data: $\sqrt s = 7$ and $8\un{TeV}$, $\be^* = 90\un{m}$ (proton in vertical RPs)
\>> 7 TeV analysis used here for illustration
\>> 8 TeV data: also CMS data available

\> trigger: RP \& T2
% already quite good selection

\> four RP combinations (left/right $\times$ top/bottom) $\Rightarrow$ each analysed separately $\Rightarrow$ confidence

\> background: pile-up

\centerline{beam halo (RP) + inelastic (T1/2) \hskip10mm or \hskip10mm SD/DPE (RP) + inelastic (T1/2)}

$\Rightarrow$ proton and inelastic products independent

$\Rightarrow$ subtraction method: TODO

\line{\hss\fig[9cm]{fig/topology_sd_T1_opp_bckg.pdf}\hss}

\>> T2 same arm complicated

\newpage %-------------------------------------------------------------------------------------------
\context{Single diffraction}
\title{Analysis steps}

\> corrections
\>> RP proton acceptance --> FIG?
\>> efficiencies (trigger, reconstruction, ...)
\>> smearing in $t$ and $\xi$ (yet to be applied)

\> FIG: dsigma/dxi for T2 opposite arm ==> experimental xi resolution

\line{\hss\fig[7cm]{fig/sd_xi_resolution.png}\hss}

\>> fit: $\si(\xi) = 0.008$

\newpage %-------------------------------------------------------------------------------------------
\context{Single diffraction}
\title{Analysis steps}

\vskip-5mm

\> $|t|$-distributions (unfolding not yet applied) fitted with
\cLRe$$\BiggerFonts \displaystyle \d\si / \d t = C\, \e^{-B t}$$\cFg

\vskip-1mm
\line{\hss\fig[,5cm]{fig/sd_t_dist_t2_opp.png}\hss\fig[,5cm]{fig/sd_t_dist_t1_opp.png}\hss}

% B of SD = B of ES / 2?

\> cross-section per class, including the invisible low-$|t|$ contribution (exploiting the fit above)
\>> already for both proton sides
\>> preliminary

\centerline{\AddBckg[0.5mm]{\cBlack\tab{\bln
\hbox{topology} & \hbox{mass range} & \hbox{slope } B & \hbox{extrapolated cross-section} \cr\bln
\hbox{T2 opposite}	& 3.4 \hbox{ to } 7\un{GeV}		&  10.1\un{GeV^{-2}}	& 1.8 \un{mb} \cr\ln 
\hbox{T1 opposite}	& 7 \hbox{ to } 350\un{GeV}		&  8.5\un{GeV^{-2}}	& 3.3 \un{mb} \cr\ln
\hbox{T1 same}		& 350 \hbox{ to } 1100\un{GeV}	&  6.8\un{GeV^{-2}}	& 1.4 \un{mb} \cr\ln
\hbox{T2 same}		& \hbox{above } 1100\un{GeV}	& \multispan2\vrule \hfil\hbox{\it effort ongoing}\hfil \cr\ln
}}}

\> very preliminary

$$\si_{\rm SD}(3.4 < m_X < 1100\un{GeV}) = (6.5 \pm 1.3)\un{mb}\ ,\qquad \si_{\rm SD}(m_X < 3.4\un{GeV}) = \O{3\un{mb}} \hbox{TODO: verify, ref}$$
% for T2 same: about 1 to 2 mb

\> in progress: xi and t double differential distribution (Ken?)

\newpage %-------------------------------------------------------------------------------------------
\hbox{}
\vfil
\title{Double diffraction}

\vskip0pt plus0.5fil
\line{\hss\fig[,3cm]{fig/diagram_dd_lab.pdf}\hss}

\vskip0pt plus0.5fil
\line{\hss\fig[,3cm]{fig/topology_dd.pdf}\hss}

\newpage %-------------------------------------------------------------------------------------------
\title{Double diffraction}

\vskip-3mm

\> method
$$\si_{\rm DD} = {\hbox{(experimental corrections)}\cdot\hbox{(raw data)} - \hbox{(background)}\over {\cal L}}$$

\> experimental challenge: background (non-diffractive, SD pile-up)

\line{\hss\fig[,3cm]{fig/topology_dd_lab.pdf}\hss}

\centerline{\cYe sub-sample with signal $\gg$ background $\Rightarrow$  $2\times$T2 and T1 veto\cFg}
%\> trigger T2

%\centerline{(2T2 + 2T1 swamped with background)}

\> non-diffractive background
\>> based on control sample $2\times$T2 + $2\times$T1
\>> transferred to $2\times$T2 + $0\times$T1 using Monte-Carlo

\> SD background
\>> based on control sample $1\times$T2 + $0\times$T1
\>> transferred to $2\times$T2 + $0\times$T1 using the measured SD distributions


\> outputs
\>> integral visible cross-section
\>> cross-section as function of $\et_{\rm min}$ on both sides $\Rightarrow$ challenge:

\centerline{reconstructed $\et_{\rm min}$ $\longrightarrow$ true/generator $\et_{\rm min}$ \hskip5mm (bin migration $\Rightarrow$ limited number of bins)}


\newpage %-------------------------------------------------------------------------------------------
\title{Double diffraction}

%\> ``acceptance'' correction = efficiencies + eta bin migration + extrapolation to non-visible regions
%\>> based on Pythia 8

\noindent\EM{7 TeV results}

\> measurement

\centerline{\fig[11cm]{fig/DD_meas.png}}

\centerline{(T1: $3.1 < \et < 4.7$, T2: $5.3 < \et < 6.5$)}

\> comparison to Monte Carlos

\centerline{\fig[11cm]{fig/DD_mc_comp.png}}

%\> soon to be published

\vfil
\noindent\EM{8 TeV results}

{\itskip0pt
\> similar analysis to be repeated
\> improvement expected: data from CMS available
}


\newpage %-------------------------------------------------------------------------------------------
\hbox{}
\vfil
\title{Central diffraction -- TOTEM alone}

\vskip0pt plus0.5fil
\line{\hss\fig[,3cm]{fig/diagram_cd_lab.pdf}\hss}

\vskip0pt plus0.5fil
\line{\hss\fig[,3cm]{fig/topology_cd.pdf}\hss}

\newpage %-------------------------------------------------------------------------------------------
\title{}

\> DPE
\>> dsigma/dt

$$m_x \approx \sqrt{\xi_1 \xi_2 s}$$

\newpage %-------------------------------------------------------------------------------------------
\hbox{}
\vfil
\title{Central diffraction -- TOTEM + CMS}

\vskip0pt plus0.5fil
\line{\hss\fig[,3cm]{fig/diagram_cd_cms_lab.pdf}\hss}

\vskip0pt plus0.5fil
\line{\hss\fig[,3cm]{fig/topology_cd.pdf}\hss}

\> X = particles/jets

\newpage %-------------------------------------------------------------------------------------------
\title{}

\> DPE: TOTEM + CMS; how it works (trigger exchange etc.)
\>> soft DPE: p+p -> p+X+p
\>> hard DPE: p+p -> p+X+jj+p

\newpage %-------------------------------------------------------------------------------------------
\title{}


\newpage %-------------------------------------------------------------------------------------------
\hbox{}
\vfil

\line{\hss\TitleFont\cYe part IV\cFg\hss}
\vskip2mm
\centerline{\PartFont\cYe Outlook\cFg}

\vfil


\newpage %-------------------------------------------------------------------------------------------
\title{Upgrade plans}

RP upgrade --> to cope with pile-up

\> multiple tracks in RPs per event --> pixels or a 3rd rotated RP
\> timing detector --> match RP tracks with CMS vertices

\newpage %-------------------------------------------------------------------------------------------
\title{}


\newpage %-------------------------------------------------------------------------------------------
\title{Summary}

TODO: make a summary

\vfil
\eject
\bye
