\input slides.tex
\input utf8-csf

\iftrue
	\StdBackground
\else
	\SetBackground{fig/bg_prepare.pdf}
	\def\cYe{\cBlack}
	\def\cLRe{\cBlack}
	\def\cRe{\cBlack}
\fi

%\font\TitleFont = cs-qplb-sc at 15pt
%\font\ContextFont = cs-qplb-sc at 10pt
\font\ExtraTitleFont = pzcmi8t at 20pt
\font\TitleFont = pzcmi8t at 15pt
\font\ContextFont = pzcmi8t at 10pt
\font\PartFont = cs-qplb-sc at 15pt

%%%%%%%%%%%%%%%%%%%%
%\SetFontSizesXII
%\itskip5mm
%\iitskip0mm
%\itindent3mm


%\def\NormalSettings{}
%%%%%%%%%%%%%%%%%%%%

\def\Em#1{\em{\cLRe#1\cFg}}
\def\EM#1{{\cYe\TitleFont #1\cFg}}


\def\date{28 May, 2013}


\newpage %-------------------------------------------------------------------------------------------
\hbox{}\vfil
\title{\ExtraTitleFont The TOTEM Experiment}
\vfil
\font\NameFont=cs-qplb at 12pt
\centerline{\NameFont Jan Kašpar}
\centerline{on behalf of the TOTEM collaboration}
\vfil
\line{\hss\hss
	\fig[,25mm]{fig/logo_totem_white.pdf}\hss
	\fig[,25mm]{fig/logo_cern_white.pdf}\hss
	\fig[,25mm]{fig/logo_fzu_small_white.pdf}\hss
	\hss}
\vfil
\centerline{Heidelberg Particle Physics Colloquium, \date}
\vfil

{
\footline={}



\newpage %-------------------------------------------------------------------------------------------
\title{Outline}
}

\vfil

\line{\hss\vbox{\hsize6cm
\itindent5mm
\BiggerFonts
\noindent I) Physics programme

\vskip2mm
\noindent II) Detector apparatus
%\> forward-proton measurement with Roman Pots

\vskip2mm
\noindent III) Analyses and results
%\> list the analyses here?

\vskip2mm
\noindent IV) Upgrade plans

\vskip2mm
\noindent V) Summary
}\hss}


\vfil
\vfil

\newpage %-------------------------------------------------------------------------------------------
\hbox{}
\vfil

\line{\hss\TitleFont\cYe part I\cFg\hss}
\vskip2mm
\centerline{\PartFont\cYe Physics programme\cFg}

\vfil


\newpage %-------------------------------------------------------------------------------------------
\title{}

Forward hadronic phenomena at LHC

\> low momentum transfers --> non perturbative QCD --> interesting

\> forward particles

\> rapidity gaps

\> the traditional process figures

\> xi ~ De eta

\> non-diffractive event: exponential suppression of rapidity gaps

\fig*[,5cm]{fig/ppInteraction.png}


\newpage %-------------------------------------------------------------------------------------------
\title{}

\fig*[,10.5cm]{fig/ttm_processes.pdf}

+ comments from Hubert's presentation

\newpage %-------------------------------------------------------------------------------------------
\hbox{}
\vfil

\line{\hss\TitleFont\cYe part II\cFg\hss}
\vskip2mm
\centerline{\PartFont\cYe Detector apparatus\cFg}

\vfil


\newpage %-------------------------------------------------------------------------------------------
\title{TOTEM at LHC}

\line{\hss\fig*[,9cm]{fig/lhc.pdf}\hss}

\vfil
\centerline{TOTEM shares IP with CMS $\Rightarrow$ collaboration possible}

\newpage %-------------------------------------------------------------------------------------------
\title{TOTEM Detectors}

\vfil

\line{%
	\hss
	\fig*[11.5cm]{fig/ttm_det_overview.pdf}
	\hskip2mm
	\raise9mm\vbox{\hsize4.1cm
		$\leftarrow$ \cOr telescopes T1 and T2\cFg\\
		\cYe{\it charged particles from\\ inelastic collisions}\cFg

		\centerline{T1: $3.1 < |\et| < 4.7$}
		\centerline{T2: $5.3 < |\et| < 6.5$}

		\vskip8mm
		$\leftarrow$ \cOr Roman Pots\cFg{} at the LHC\\
		\cYe{\it elastic and diffractive protons}\cFg
	}
	\hss
}

\> all detectors symmetrically on both sides of IP5
\> all detectors trigger-capable
\> all detectors radiation tolerant

\vfil

\newpage %-------------------------------------------------------------------------------------------
\title{Telescope T1}



\newpage %-------------------------------------------------------------------------------------------
\title{Telescope T2}



\newpage %-------------------------------------------------------------------------------------------
\title{Roman Pots}

\> typical approach: $10\un{\si}$


\newpage %-------------------------------------------------------------------------------------------
\title{Proton measurement with Roman Pots}

\vskip-5mm

\> LHC lattice between IP5 and RPs at $220\un{m}$

\vskip1mm
%\line{\hss 5 quadrupoles, 2 dipoles, corrector magnets and drift spaces:\hss}
\line{\hss\fig*[13.5cm]{fig/RP_stations.pdf}\hss}

\> \Em{proton transport}: described as in linear optics

\line{\hss\fig*[11.5cm]{fig/ttm_proton_transport.pdf}\hss}

\vskip1mm	
\line{\hss$\displaystyle
\pmatrix{x\cr \th_x\cr y\cr \th_y\cr \xi}_{\rm RP} =
\underbrace{\pmatrix{
	v_x & L_x & \cdot & \cdot & D_x\cr
	\cdot & \cdot & \cdot & \cdot & \cdot\cr
	v_y & L_y & \cdot & \cdot & D_y\cr
	\cdot & \cdot & \cdot & \cdot & \cdot\cr
	\cdot & \cdot & \cdot & \cdot & 1\cr
}}_{\hbox{product from all lattice elements}}
\pmatrix{x^*\cr \th_x^*\cr y^*\cr \th_y^*\cr \xi}_{\rm IP}
$
\hskip10mm
\lower10mm\vbox{
\hbox{\Em{optical functions:}}
\hbox{effective length $L$}
\hbox{magnification $v$}
\hbox{dispersion $D$}
\hbox{\strut}
\hbox{$\xi = \De p / p$: momentum loss}
}
\hss}

\> \Em{proton reconstruction}: inverted transport RPs $\longrightarrow$ IP

\>> optical parameters functions of $\xi$ $\Rightarrow$ reconstruction is non-linear problem

\>> \cYe good knowledge of optics is crucial\cFg

%\fig*[,2.8cm]{fig/optics.pdf}%

\newpage %-------------------------------------------------------------------------------------------
\title{LHC optics}

\> optics defines \Em{what} and \Em{how} can be observed -- a CD sample seen with 2 different optics

\vskip2mm
\line{%
	\hss
	\vtop{\hsize7.7cm
		\centerline{\BiggerFonts\bf\cOr$\bf\be^* = 90\un{m}$\cFg}
		\centerline{$\displaystyle L_x \approx 0,\quad L_y \approx 260\un{m},\quad D_x \approx 4\un{cm}$}
		
		\centerline{diffractive protons in \cYe vertical RPs\cFg}
		\line{\hss\fig[,4.5cm]{fig/hit_dist_90.png}\hss}
	}%
	\hskip2mm
	\vtop{\hsize7.7cm
		\centerline{\cOr{\BiggerFonts\bf low $\bf\be^*$\cFg{}}}
		\centerline{$\displaystyle L_x \approx 1.7\un{m},\quad L_y \approx 14\un{m},\quad D_x \approx 8\un{cm}$}
		
		\centerline{diffractive protons in \cYe horizontal RPs\cFg}
		\line{\hss\fig[,4.5cm]{fig/hit_dist_0p7.png}\hss}
	}%
	\hss
}

%\line{\hss elastic proton sample seen with different optics:\hss}
%\line{\hss \fig*[,6cm]{fig/ttm_hit_distribution.pdf}\hss}

\vfil
\> optics carefully optimised for TOTEM special runs

\vfil
\> optics typically ``labelled'' by $\be^* \equiv$ betatron function at IP
\>> beam width: $\sqrt{\ep \be}$, beam angular divergence: $\sqrt{\ep / \be}$; $\ep$: a measure of beam size/divergence
\>> luminosity $\propto \hbox{(beam width at IP)}^{-2} \propto 1/\be^*$
\>> example: \cYe high $\be^*$ $\Rightarrow$ reduced luminosity but protons ``more parallel''\cFg


\newpage %-------------------------------------------------------------------------------------------
\title{Run scenarios}

{

\advance\hsize\horizontalmargin\line{\kern-\horizontalmargin
\fig[,4.47cm]{fig/acceptance_2.pdf}
\fig[,4.47cm]{fig/acceptance_90.pdf}
\fig[,4.47cm]{fig/acceptance_1535.pdf}\hss
}

\line{\kern-\horizontalmargin\SmallerFonts\hss
\vtop{\hsize4.7cm\obeylines\leftskip0pt plus1fil\rightskip0pt plus1fil\parfillskip0pt
	\cYe{\NormalFonts\bf low $\be^*$}\cFg
	\hrule\vskip1mm
	$\be^* = 0.5$ to $3\un{m}$
	\vskip1mm
	${\cal L} \approx 10^{30}\un{cm^{-2}s^{-1}}$
		\vskip\baselineskip
	elastic data available
	$0.4 \ls |t/{\rm GeV^2}| \ls 3.5$
		\vskip\baselineskip
	resolution
	$\si(\th^*) \approx 15\un{\mu rad}$
	$\si(\xi) \approx 0.2\un{\%}$
		\vskip3\baselineskip
	\em{\cYe diffraction, high $|t|$ elastic scattering, low cross-section processes\cFg}
}
\hss
\vtop{\hsize4.7cm\obeylines\leftskip0pt plus1fil\rightskip0pt plus1fil\parfillskip0pt
	{\cYe\NormalFonts\bf medium $\be^*$\cFg}
	\hrule\vskip1mm
	$\be^* = 90\un{m}$
	\vskip1mm
	${\cal L} \approx 10^{28}\un{cm^{-2}s^{-1}}$
		\vskip1\baselineskip
	elastic data available
	$ 10^{-2} < |t/{\rm GeV^2}| \ls 1.3 $
		\vskip\baselineskip
	resolution
	$\si(\th^*) \approx 1.7\un{\mu rad}$
	$\si(\xi) \approx 0.4$ to $0.6\un{\%}$
		\vskip\baselineskip
	all $\xi$ seen, universal optics
		\vskip\baselineskip
	\em{\cYe diffraction, mid $|t|$ elastic scattering, total cross section\cFg}
}
\hss
\vtop{\hsize4.7cm\obeylines\leftskip0pt plus1fil\rightskip0pt plus1fil\parfillskip0pt
	{\cYe\NormalFonts\bf high $\be^*$\cFg}
	\hrule\vskip1mm
	$\be^* \gs 1000\un{m}$
	\vskip1mm
	${\cal L} \approx 10^{27}\un{cm^{-2}s^{-1}}$
		\vskip\baselineskip
	elastic data available
	$ 6\cdot10^{-4} < |t/{\rm GeV^2}| < 0.3$
		\vskip\baselineskip
	resolution
	$\si(\th^*) \approx 0.4\un{\mu rad}$
	%$\si(\xi) \approx 2\div 10\cdot10^{-3}$
		\vskip2\baselineskip
	all $\xi$ seen
		\vskip\baselineskip
	\em{\cYe total cross section, low $|t|$ elastic scattering\cFg}
}
\hss}

}

\newpage %-------------------------------------------------------------------------------------------
\title{Optics imperfections}

\centerline{\Em{good optics knowledge essential for reconstruction}}

\vfil
\> optics imperfection sources
\>> power-converter error: $\De I / I \approx 10^{-4}$
\>> magnet transfer function: $\De B / B \approx 10^{-3}$
\>> magnet rotation (< $1\un{mrad}$) and displacements ($<0.5\un{mm}$)
\>> magnet harmonics ($\De B \approx 10^{-4}$)
\>> beam momentum offset: $\De p / p \approx 10^{-3}$
\>> beam crossing-angle uncertainty

\vfil
\> optics determination
\>> direct measurement -- difficult
\>> indirect from TOTEM observables

\vfil
\> TOTEM optics determination -- variation of magnet/beam parameters (within tolerances) to\\ match TOTEM observables:
\>> $L_y^L / L_y^R$
\>> ${\d L_y\over \d s} / L_y$
\>> $s(L_x = 0)$
\>> $xy$ coupling (tilts in $xy$ plane)
\>> ...

\vfil


\newpage %-------------------------------------------------------------------------------------------
\title{Optics refinement with TOTEM data}

\centerline{\Em{example for $\be^* = $ 3.5\ m optics}}

\vfil
\line{\hss\hskip-5mm\fig[16cm]{fig/optics_refinement.png}\hss}

\vfil

\> optics uncertainty reduced:

\centerline{x projection: from $1.6\%$ to $0.17\%$}
\centerline{y projection: from $4.2\%$ to $0.16\%$}

\vfil

{\SmallerFonts
\> details:
\>> H. Niewiadomski, Roman Pots for beam diagnostic, OMCM, CERN, 20-23.06.2011
\>> H. Niewiadomski, F. Nemes, LHC Optics Determination with Proton Tracks, IPAC'12, Louisiana, USA, 20-25.05.2012
}

\vfil

\newpage %-------------------------------------------------------------------------------------------
\title{Alignment of Roman Pots}

\> RPs = movable insertions $\Rightarrow$ each run at different positions
\> required angular precision micro-radians $\Rightarrow$ micro-metre alignment precision needed

\vfil
\centerline{$\Downarrow$}
\vfil

\> two types of alignment needed
\>> alignment of mechanical RP edges $\rightarrow$ for machine protection
\>> alignment of RP sensors $\rightarrow$ for physics

\> need alignment \Em{wrt.~the beam}

\vfil
\centerline{$\Downarrow$}
\vfil

\centerline{\bf\cOr 3-step alignment procedure\cFg}

\vfil

\noindent1) \em{\cYe Collimation alignment}\cFg: RP alignment wrt.~beam, rough sensor alignment

\line{\hss\vbox{\hsize8cm
\> standard procedure for LHC collimators

\fig[,3cm]{fig/al_collim_scheme.pdf}}\hskip1mm\hfil\fig[,4.5cm]{fig/al_collim_ex.pdf}\hss}

\newpage %-------------------------------------------------------------------------------------------
\title{Alignment of Roman Pots}


\noindent2) \em{\cYe Track-based alignment\cFg}: relative alignment among sensors

\line{\raise3mm\vbox{\hsize10cm
{\itskip1mm
\> RP station: no magnetic field $\rightarrow$ straight tracks
\> misalignments $\rightarrow$ residuals
\> residual analysis $\rightarrow$ alignment corrections
\> overlap between horizontal and vertical RPs $\rightarrow$ relative alignment among all sensors
\> singular/weak modes: e.g.~overall shift/rotation\\
$\Rightarrow$ need further alignment step
}}\hss\fig[,3.8cm]{fig/overlap.pdf}}

\vfil
\noindent3) \em{\cYe Alignment with physics processes (elastic scattering)\cFg}: sensor alignment wrt.~beam

\line{\hss\fig[16cm]{fig/al_el_plots_sum_slides.pdf}\hss}

\vfil

\newpage %-------------------------------------------------------------------------------------------
\hbox{}
\vfil

\line{\hss\TitleFont\cYe part III\cFg\hss}
\vskip2mm
\centerline{\PartFont\cYe Analyses and results\cFg}

\vfil

\newpage %-------------------------------------------------------------------------------------------
\hbox{}
\vfil
\title{Elastic scattering}

\vskip0pt plus0.5fil
\line{\hss\fig[,3cm]{fig/diagram_es_lab.pdf}\hss}

\vskip0pt plus0.5fil
\line{\hss\fig[,3cm]{fig/topology_es.pdf}\hss}

\line{\vtop{\hsize8cm
\> interesting process on its own
\> two anti-collinear protons $\Rightarrow$ excellent tool for
\>> alignment
\>> understanding detector effects
\>> optics tuning, etc.
}\hskip10mm\vtop{\hsize6cm
\> $\th$: scattering angle
\> $\ph$: azimuthal angle
\> $t$: four-momentum transfer squared
$$t \simeq - p^2 \th^2$$
}\hss}

\newpage %-------------------------------------------------------------------------------------------
\context{Elastic scattering}
\title{Pre-TOTEM status}

\> theoretical/phenomenological models: very different predictions at larger $|t|$

\vfil
\line{\hss\fig[14cm]{fig/es_models.pdf}\hss}

\vfil
\> different $|t|$ regions: different scattering mechanisms/QCD regimes

\newpage %-------------------------------------------------------------------------------------------
\context{Elastic scattering}
\title{Analysis}

\vskip-3mm

\line{\vbox{\hsize10cm
\itskip0mm
\itindent7mm

\centerline{\Em{entirely data-driven}}
\vskip1mm

\noindent\cYe{\bf 1. Kinematics reconstruction}\cFg

\> tracks in RPs $\mathop{\longrightarrow}$ kinematics at IP ($\xi = 0$ $\Rightarrow$ relatively easy)
\> choice of formulae $\rightarrow$ minimisation of systematics

\vskip1mm
\noindent\cYe{\bf 2. Elastic tagging}\cFg

\> angles left = angles right, vertex left = vertex right
\> protons $\xi \approx 0$ $\Rightarrow$ correlation hit position vs.~track angle at RPs

\vskip1mm
\noindent\cYe{\bf 3. Background subtraction}\cFg

\vskip1mm
\noindent\cYe{\bf 4. Acceptance corrections}\cFg

\> RP sensors have finite size, LHC apertures
\> azimuthal symmetry $\Rightarrow$ geometrical correction (+ smearing around edges)

\vskip1mm
\noindent\cYe{\bf 5. Unfolding of resolution effects}\cFg

\> angular resolution: left-right proton comparison
\> Monte Carlo calculation $\Rightarrow$ impact on $t$-distribution

\vskip1mm
\noindent\cYe{\bf 6. Inefficiency corrections}\cFg

\> 3-out-of-4 efficiency
\> near-far correlated RP inefficiencies
\> ``pile-up'' = elastic event + another track in a RP

\vskip1mm
\noindent\cYe{\bf 7. Luminosity}\cFg

\> from CMS (if available), uncertainty $\approx 4\%$
\> from TOTEM (details later on)
}\hskip3mm\vbox{\hsize5.1cm

\centerline{{\bf\cOr$\th_x^*$ collinearity\cFg}}

\fig[5cm]{fig/cuts_th_x_L_vs_th_x_R.pdf}

\vskip5mm

\centerline{{\bf\cOr acceptance correction\cFg}}

\fig[5cm]{fig/acc_corr_phi_lab.pdf}
}\hss}

\newpage %-------------------------------------------------------------------------------------------
\context{Elastic scattering}
\title{Results}

\centerline{\Em{building a puzzle from measurements with different $\be^*$}}

\vfil

\line{%
	\hss
	\vtop{\hsize7.7cm
		\centerline{\bf\cOr$\bf\sqrt s = 7\un{\bf TeV}$\cFg}
		\centerline{\SmallerFonts[EPL 96 (2011) 21002, EPL 101 (2013) 21002]}
		\vskip3mm
		\line{\hss\fig{fig/es_results_7.pdf}\hss}
	}%
	\hskip2mm
	\vrule
	\hskip2mm
	\vtop{\hsize7.7cm
		\centerline{\bf\cOr$\bf\sqrt s = 8\un{\bf TeV}$\cFg}
		\centerline{\SmallerFonts[CERN-PH-EP-2012-354]}
		\vskip3mm
		\line{\hss\fig{fig/es_results_8.pdf}\hss}
	}%
	\hss
}


\newpage %-------------------------------------------------------------------------------------------
\context{Elastic scattering}
\title{First conclusions}

\vskip-3mm

\line{\raise12mm\vbox{\hsize6.7cm
\> no theoretical/phenomenological model describes completely TOTEM data

\> at low $|t|$: nearly exponential decrease
$${\d\si\over \d t} \approx \e^{-B |t|}$$

}\hss\fig[,5cm]{fig/es_results_7_mod.pdf}\hss}

\line{\raise17mm\vbox{\hsize5cm
\> previously observed trends\\ confirmed: as $\sqrt{s}$ grows
\>> \Em{``forward peak'' shrinks}\\
$\Rightarrow$ minimum moves to lower values
\>> \Em{intercept at $t = 0$ increases}\\
$\Rightarrow$ related to $\si_{\rm tot}$ increase
\>> \Em{forward slope $B$ increases}
}\hskip2mm\fig[,5cm]{fig/dcs_older_experiments1.png}\hskip1mm\fig[,5cm]{fig/es_B_s.pdf}\hss}

%\line{\hss\fig[,4cm]{fig/dcs_older_experiments1.png}\hss\fig[,4cm]{fig/es_B_s.pdf}\hss}
%\line{\hss\fig[3cm]{fig/dcs_older_experiments2.png}\hss}


\newpage %-------------------------------------------------------------------------------------------
\context{Elastic scattering}
\title{Very low $|t|$: Coulomb-hadronic interference}

\vskip-3mm

\> $|t|$ as low as $6\cdot10^{-4}\un{GeV^2}$ accessible thanks to
\>> $\be^* = 1000\un{m}$ optics: large effective lengths, low beam divergence ($\approx 0.5\un{\mu rad}$)
\>> RPs approach of $3\un{\si_{\rm beam}}$ from beam

\vfil
\line{\hss\fig[8cm]{fig/plotFitExampleComponents.pdf}\hss}

\> interesting aspects
\>> Coulomb-hadronic interference $\Rightarrow$ \Em{determination of phase} of hadronic amplitude
\>> Coulomb/hadronic separation $\Rightarrow$ hadronic extrapolation to $t = 0$\\
$\Rightarrow$ \Em{total-cross section implications} via optical theorem

\vskip-2mm
$$\cYe\si_{\rm tot} \propto \Im {\cal A_{\rm el}}(t = 0)\cFg$$

\vfil

\newpage %-------------------------------------------------------------------------------------------
\context{Elastic scattering}
\title{Hadronic phase: first results}

\noindent\EM{Theory}

\line{\hss\raise2mm\vbox{\hsize5.5cm
$${\d\si\over \d t} \propto |{\cal A}^{\rm C+H}|^2$$

$${\cal A}^{\rm C+H} = \hbox{COMBINATION}({\cal A^{\rm C}}, {\cal A}^{\rm H})$$
}\hskip3mm\AddBckg[1mm]{\fig[,1.8cm]{fig/el_diagrams.pdf}}\hss}

\> \Em{COMBINATION}: 2 theoretical alternatives
\> \Em{${\cal A}^{\rm C}$}: well known
\> \Em{${\cal A}^{\rm H}$}
\>> \Em{modulus}: constrained by TOTEM data $\Rightarrow$ parametrised $\exp(Bt + \ldots)$
\>> \Em{phase}: test a range of theoretical predictions


\vfil


\line{\vbox{\hsize9cm
\noindent\EM{Fits}

\> various fit metrics: generalised $\ch^2$, Kolmogorov-like

\> combination of choices: little impact on the fit

\> very PRELIMINARY result

$$\rh = \left. {\Re {\cal A^{\rm H}}\over \Im {\cal A}^{\rm H}}\right |_{t = 0} =
0.110 \pm 0.027^{\rm (stat)} \pm 0.010^{\rm (syst)}\ \hbox{}^{\hbox to 7pt{\hss+\hss} 0.013}_{\hbox to 7pt{\hss-\hss} 0.012}\ \hbox{}^{\rm (model)}
$$

}\hskip3mm\vbox{\vskip-5mm\fig[,5cm]{fig/rho.pdf}}\hss}

\newpage %-------------------------------------------------------------------------------------------
\hbox{}
\vfil
\title{Total cross-section}

\vskip0pt plus0.5fil
\line{\hss\fig[,3cm]{fig/diagram_tot.pdf}\hss}

\vfil
\newpage %-------------------------------------------------------------------------------------------
\context{Total cross-section}
\title{Pre-TOTEM status}

\line{\hss various $\si_{\rm tot}$ extrapolations by COMPETE\hss}
%\line{\hss\fig*[10cm]{fig/compete2002_sigma_tot.pdf}\hss}
\line{\hss\fig*[8.5cm]{fig/sigma,tot.pdf}\hss}

\vfil
\> various models/theories:
$$\si_{\rm tot} \sim \log s\ ,\qquad \si_{\rm tot} \sim \log^2 s\ ,\qquad \si_{\rm tot} \sim s^{\al - 1}$$

\> predictions for $\sqrt{s} = 14\un{TeV}$
$$90\un{mb} < \si_{\rm tot} < 130\un{mb} \Rightarrow 40\un{\%}\hbox{ uncertainty}$$


\> available data not decisive (incompatible CDF/E810 measurements)

\vfil

\newpage %-------------------------------------------------------------------------------------------
\context{Total cross-section}
\title{Methods}

\vskip-5mm

\> consequence of optical theorem

$$\si_{\rm tot}^2 \propto \left[ \Im {\cal A}_{\rm el}(t = 0) \right]^2 = {1\over 1 + \rh^2} |{\cal A_{\rm el}}(t = 0)|^2 \propto
{1\over 1 + \rh^2} \left. \d\si_{\rm el}\over \d t \right|_{t=0}
$$


\vfil
\> 3 methods available
\vskip2mm
\line{\hss\fig[11cm]{fig/tot_cs_methods.pdf}\hss}

\vfil
\> $\rh$ value from TOTEM or from an external source, e.g.~COMPETE
\>> enters as $1+\rh^2$ $\Rightarrow$ limited impact

\vfil
\> by-product: by method combination luminosity $\cal L$ can be determined

\newpage %-------------------------------------------------------------------------------------------
\context{Total cross-section}
\title{Inelastic cross-section measurement}

\vskip-3mm
\centerline{[EPL 101 (2013) 21003]}

\vfil
\> T2 sees $\approx 95\un{\%}$ of inelastic events (enough to detect 1 track!)

\vfil
\noindent 1) \cYe{\bf Raw rate}\cFg: event counting with T2

\vskip2mm
\line{\hskip10mm\vtop{\hsize6cm
\noindent$\downarrow$ \em{experimental corrections}:
}\hskip1mm\vtop{\hsize8cm
\noindent trigger and reconstruction inefficiencies, beam-gas event suppression, pile-up consideration
}\hss}
\vskip2mm

\noindent 2) \cYe{\bf Visible rate}\cFg: visible with T2 in perfect conditions

\vskip2mm
\line{\hskip10mm\vtop{\hsize6cm
\noindent$\downarrow$ \em{recovery of events with no tracks in T2}:
}\hskip1mm\vtop{\hsize8cm
\noindent T1-only events, events with gap over T2, low-mass diffraction
}\hss}
\vskip2mm

\noindent 3) \cYe{\bf Physics rate}\cFg: true rate of inelastic events

\vfil
\> only one major Monte-Carlo-based correction: \Em{low-mass diffraction}\\
$\Rightarrow$ but can be constrained from data

\newpage %-------------------------------------------------------------------------------------------
\context{Total cross-section}
\title{Results}

\vskip-3mm

\line{%
	\hss
	\vtop{\hsize7.7cm
		\centerline{\bf\cOr$\bf\sqrt s = 7\un{\bf TeV}$\cFg}
		\centerline{\SmallerFonts[EPL 101 (2013) 21004]}
		\vskip3mm
		\line{\hss\fig{fig/tot_cs_methods_res7.pdf}\hss}
		\> low mass diffraction
		\vskip0mm
		$$\si_{\rm inel}^{|\et| > 6.5} = \si_{\rm tot}^{EOO} - \si_{\rm el}^{EOO} - \si_{\rm inel}^{\rm visible} = (2.6 \pm 2.2)\un{mb}$$
	}%
	\hskip2mm
	\vrule
	\hskip2mm
	\vtop{\hsize7.7cm
		\centerline{\bf\cOr$\bf\sqrt s = 8\un{\bf TeV}$\cFg}
		\centerline{\SmallerFonts[CERN-PH-EP-2012-354]}
		\vskip3mm
		\line{\hss\fig{fig/tot_cs_methods_res8.pdf}\hss}
		\centerline{(greyed: CMS luminosity unavailable)}
	}%
	\hss
}

\newpage %-------------------------------------------------------------------------------------------
\context{Total cross-section}
\title{Results in context}

\line{\hss\fig[14cm]{fig/sigma_tot_el_inel_cmp_big.pdf}\hss}



\newpage %-------------------------------------------------------------------------------------------
\title{Intermezzo: Optics for diffractive studies}
\vskip-8mm
$$x_{\rm RP} = v_x x^* + L_x \th_x^* + \xi D_x\ ,\qquad \xi = {\De p\over p_0}$$
\vskip-2mm

\line{%
	\hss
	\vtop{\hsize7.7cm
		\centerline{\BiggerFonts\bf\cOr$\bf\be^* = 90\un{m}$\cFg}
		%\> vertex: $160\un{\mu m}$
		%\> beam-divergence: $2.6\un{\mu rad}$
		\> optical functions at RP 220:
		$$L_x \approx 0,\quad L_y \approx 260\un{m},\quad D_x \approx 4\un{cm}$$
		
		\centerline{$\Downarrow$}
		\centerline{diffractive protons in \cYe vertical RPs\cFg}
		\centerline{\SmallerFonts (a CD sample)}
		\line{\hss\fig[,4.5cm]{fig/hit_dist_90.png}\hss}

		\> $|\xi|_{\rm min} = 0\%$ $\Rightarrow$ \cYe low masses \cFg
		\> $\xi$-resolution
		\>> RPs only: $(0.4 \hbox{ to } 1)\%$ ($t$-dependent)
		\>> with CMS vertex: $\approx 2\times$ better
		\vskip\baselineskip
		\centerline{\Em{used so far}\cFg}
	}%
	\hskip2mm
	\vrule
	\hskip2mm
	\vtop{\hsize7.7cm
		\centerline{\cOr{\BiggerFonts\bf low $\bf\be^*$\cFg{}} ($0.7\un{m}$ here)}
		%\> vertex: $14\un{\mu m}$
		%\> beam-divergence: $29\un{\mu rad}$
		\> optical functions at RP 220:
		$$L_x \approx 1.7\un{m},\quad L_y \approx 14\un{m},\quad D_x \approx 8\un{cm}$$
		
		\centerline{$\Downarrow$}
		\centerline{diffractive protons in \cYe horizontal RPs\cFg}
		\centerline{\SmallerFonts (a CD sample)}
		\line{\hss\fig[,4.5cm]{fig/hit_dist_0p7.png}\hss}

		\> $|\xi|_{\rm min} = 2.8\%$  $\Rightarrow$ \cYe higher masses \cFg
		\> $\xi$-resolution
		\>> RPs only: $\approx 0.2\%$
		\vskip\baselineskip
		\vskip\baselineskip
		\centerline{\Em{planned after long shutdown}\cFg}
	}%
	\hss
}

\newpage %-------------------------------------------------------------------------------------------
\hbox{}
\vfil
\title{Single diffraction}

\vskip0pt plus0.5fil
\line{\hss\fig[,3cm]{fig/diagram_sd_lab.pdf}\hss}

\vskip0pt plus0.5fil
\line{\hss\fig[,3cm]{fig/topology_sd_T1_opp.pdf}\hss}

\vfil
\line{\raise2mm\vtop{\hsize7cm
\> {\BiggerFonts \cYe $\xi \approx e^{-\De \et}$\cFg} $\Rightarrow$ double ``determination'' of $\xi$
\>> from proton (Roman Pots)
\>> from rapidity gap (T1/2)
}\hfil\vtop{\hsize6.3cm
\> mass of diffractive system X
$$m_X \approx \sqrt{s\xi}$$
\> minimal mass visible (T2 acceptance):
$$m_X \ge 3.4\un{GeV}$$
}\hfil}

\newpage %-------------------------------------------------------------------------------------------
\context{Single diffraction}
\title{Topologies / diffractive-mass classes}

\vskip-5mm

%\noindent\EM{topologies}
\> \Em{T2 opposite arm}: $m_X \approx 3.4$ to $7\un{GeV}$, $2\cdot10^{-7} < \xi < 1\cdot 10^{-6}$

\line{\hss\fig[7.1cm]{fig/topology_sd_T2_opp.pdf}\hss}

\> \Em{T1 opposite arm}: $m_X \approx 7$ to $350\un{GeV}$, $1\cdot10^{-6} < \xi < 2.5\cdot 10^{-3}$

\line{\hss\fig[7.1cm]{fig/topology_sd_T1_opp.pdf}\hss}

\> \Em{T1 same arm}: $m_X \approx 350$ to $1100\un{GeV}$, $2.5\cdot10^{-3} < \xi < 2.5\cdot 10^{-2}$

\line{\hss\fig[7.1cm]{fig/topology_sd_T1_same.pdf}\hss}

\> \Em{T2 same arm}: $m_X \gs 1100\un{GeV}$, $\xi > 2.5\cdot 10^{-2}$

\line{\hss\fig[7.1cm]{fig/topology_sd_T2_same.pdf}\hss}

\newpage %-------------------------------------------------------------------------------------------
\context{Single diffraction}
\title{Analysis I}

\> available data: $\sqrt s = 7$ and $8\un{TeV}$, $\be^* = 90\un{m}$ (proton in vertical RPs)
\>> 7 TeV analysis used here for illustration
\>> 8 TeV data: also CMS data available

\> trigger: RP \& T2
% already quite good selection

\> four RP combinations (left/right $\times$ top/bottom) $\Rightarrow$ each analysed separately $\Rightarrow$ confidence

\> background -- pile-up:

\centerline{beam halo (RP) + inelastic (T1/2) \hskip10mm or \hskip10mm SD/DPE (RP) + inelastic (T1/2)}

$\Rightarrow$ proton and inelastic products independent

$\Rightarrow$ background estimation: events with proton on the ``wrong'' side

\vskip1mm
\line{\hss\fig[10cm]{fig/topology_sd_T1_opp_bckg.pdf}\hss}
\vskip1mm

\>> complicated for class {T2 same arm}

\newpage %-------------------------------------------------------------------------------------------
\context{Single diffraction}
\title{Analysis II}

\vskip-7mm

\> corrections
\>> \Em{RP proton acceptance}

\line{\hss\fig[,4cm]{fig/rp_hit_patterns_90m.pdf}\hskip3mm\vbox{\hsize5cm
\> grey regions = sensors\\ $\longrightarrow$ protons detected
\> $L_x$ (ellipse width) strongly dependent on $\xi$
\> $L_y$ (ellipse width) weakly dependent on $\xi$
\> ellipse centre moves right with $|\xi|$ (dispersion $D_x$)
}\hss}

\>> \Em{efficiencies} (trigger, reconstruction, ...)
\>> \Em{smearing in $t$ and $\xi$} (yet to be applied)

\> experimental $\xi$ resolution from RPs

\line{\hss\fig[,4cm]{fig/sd_xi_resolution.png}\hskip10mm\raise10mm\vbox{\hsize7cm
\> class: T2 opposite arm\\
$\Rightarrow$ T2: $2\cdot10^{-7} < \xi < 1\cdot 10^{-6}$

\> plotted $\xi$ from RP reconstruction

\> Gaussian fit: $\si(\xi) = 0.008$

}\hss}



\newpage %-------------------------------------------------------------------------------------------
\context{Single diffraction}
\title{First results}

\vskip-5mm

\> $|t|$-distributions (unfolding not yet applied) fitted with
\cLRe$$\BiggerFonts \displaystyle \d\si / \d t = C\, \e^{-B t}$$\cFg

\vskip-1mm
\line{\hss\fig[,5cm]{fig/sd_t_dist_t2_opp.png}\hss\fig[,5cm]{fig/sd_t_dist_t1_opp.png}\hss}

% B of SD = B of ES / 2?

\> cross-section per class, including the invisible low-$|t|$ contribution (exploiting the fit above)
\>> already for both proton sides
\>> preliminary

\centerline{\AddBckg[0.5mm]{\cBlack\tab{\bln
\hbox{topology} & \hbox{mass range} & \hbox{slope } B & \hbox{extrapolated cross-section} \cr\bln
\hbox{T2 opposite}	& 3.4 \hbox{ to } 7\un{GeV}		&  10.1\un{GeV^{-2}}	& 1.8 \un{mb} \cr\ln 
\hbox{T1 opposite}	& 7 \hbox{ to } 350\un{GeV}		&  8.5\un{GeV^{-2}}	& 3.3 \un{mb} \cr\ln
\hbox{T1 same}		& 350 \hbox{ to } 1100\un{GeV}	&  6.8\un{GeV^{-2}}	& 1.4 \un{mb} \cr\ln
\hbox{T2 same}		& \hbox{above } 1100\un{GeV}	& \multispan2\vrule \hfil\hbox{\it effort ongoing ...}\hfil \cr\ln
}}}

\> very preliminary

$$\si_{\rm SD}(3.4 < m_X < 1100\un{GeV}) = (6.5 \pm 1.3)\un{mb}\ ,\qquad \si_{\rm SD}(m_X < 3.4\un{GeV}) = \O{2.5\un{mb}}$$
% for T2 same: about 1 to 2 mb

\> final goal: $\xi$ and $t$ double-differential distribution

\newpage %-------------------------------------------------------------------------------------------
\hbox{}
\vfil
\title{Double diffraction}

\vskip0pt plus0.5fil
\line{\hss\fig[,3cm]{fig/diagram_dd_lab.pdf}\hss}

\vskip0pt plus0.5fil
\line{\hss\fig[,3cm]{fig/topology_dd.pdf}\hss}

\newpage %-------------------------------------------------------------------------------------------
\context{Double diffraction}
\title{Method}

\vskip-3mm

\> method
$$\si_{\rm DD} = {\hbox{(experimental corrections)}\cdot\hbox{(raw data)} - \hbox{(background)}\over {\cal L}}$$

\> experimental challenge: background (non-diffractive, SD pile-up)

\line{\hss\fig[,3cm]{fig/topology_dd_lab.pdf}\hss}

\centerline{\cYe sub-sample with signal $\gg$ background $\Rightarrow$  $2\times$T2 and T1 veto\cFg}
%\> trigger T2

%\centerline{(2T2 + 2T1 swamped with background)}

\> non-diffractive background
\>> based on control sample $2\times$T2 + $2\times$T1
\>> transferred to $2\times$T2 + $0\times$T1 using Monte-Carlo

\> SD background
\>> based on control sample $1\times$T2 + $0\times$T1
\>> transferred to $2\times$T2 + $0\times$T1 using the measured SD distributions


\> outputs
\>> integral visible cross-section
\>> cross-section as function of $\et_{\rm min}$ on both sides $\Rightarrow$ challenge:

\centerline{reconstructed $\et_{\rm min}$ $\longrightarrow$ true/generator $\et_{\rm min}$ \hskip5mm (bin migration $\Rightarrow$ limited number of bins)}


\newpage %-------------------------------------------------------------------------------------------
\context{Double diffraction}
\title{Results}

%\> ``acceptance'' correction = efficiencies + eta bin migration + extrapolation to non-visible regions
%\>> based on Pythia 8

\noindent\EM{7 TeV results}

\> measurement

\centerline{\fig[11cm]{fig/DD_meas.png}}

\centerline{(T1: $3.1 < \et < 4.7$, T2: $5.3 < \et < 6.5$)}

\> comparison to Monte Carlos

\centerline{\fig[11cm]{fig/DD_mc_comp.png}}

%\> soon to be published

\vfil
\noindent\EM{8 TeV results}

{\itskip0pt
\> similar analysis to be repeated
\> improvement expected: data from CMS available
}


\newpage %-------------------------------------------------------------------------------------------
\hbox{}
\vfil
\title{Central diffraction -- TOTEM alone}

\vskip0pt plus0.5fil
\line{\hss\fig[,3cm]{fig/diagram_cd_lab.pdf}\hss}

\vskip0pt plus0.5fil
\line{\hss\fig[,3cm]{fig/topology_cd.pdf}\hss}

\line{\vtop{\hsize7cm
\> both protons detected (unprecedented) 
}\hfil\vtop{\hsize6.3cm
\> mass of diffractive system X
$$m_X \approx \sqrt{s\xi_1 \xi_2}$$
}\hfil}

\newpage %-------------------------------------------------------------------------------------------
\context{Central diffraction -- TOTEM alone}
\title{First results}

\> available data: $\sqrt s = 7\un{TeV}$, $\be^* = 90\un{m}$\\
$\Rightarrow$ almost complete $\xi$ acceptance, but resolution sufficient only $\xi \gs 0.03$

\> trigger/event selection: $2\times$ RP

\> background: ES + inelastic, beam halo + inelastic
\>> ES: anti-elastic cuts or use forbidden topologies only (top-top, bottom-bottom)
\>> beam-halo: cut $|y| > 11\un{\si_{\rm beam}}$ $\Rightarrow$ halo negligible

\> $|t_y|$ distribution: all $\xi$ values, only acceptance correction

\line{\raise8mm\vbox{\hsize7cm
\> estimate of $\si_{\rm CD}$

$${\d^2\si_{\rm CD}\over \d t_1 \d t_2} = C \e^{-B t_1} \e^{-B t_2}$$

$$\Downarrow$$

$$ \si_{\rm CD}
= \int\limits^{0}_{-\infty} \d t_1 \int\limits^{0}_{-\infty} \d t_2\ C \e^{-B t_1} \e^{-B t_2} \approx 1\un{mb}$$

\> final goal: $\displaystyle {\d^4\,\si_{\rm CD}\over \d t_1\, \d t_2\, \d\xi_1\, \d\xi_2}$
}\hss\fig[8cm]{fig/cd_ty_dist_7TeV.png}\hss}




\newpage %-------------------------------------------------------------------------------------------
\hbox{}
\vfil
\title{Central diffraction -- TOTEM + CMS}

\vskip0pt plus0.5fil
\line{\hss\fig[,3cm]{fig/diagram_cd_cms_lab.pdf}\hss}

\vskip0pt plus0.5fil
\line{\hss\fig[,3cm]{fig/topology_cd.pdf}\hss}

\line{\vtop{\hsize7cm
\> both protons detected (unprecedented) 
}\hfil\vtop{\hsize6.3cm
\> mass of diffractive system X -- double determination (unprecedented)
\>> by TOTEM RPs
$$m_X \approx \sqrt{s\xi_1 \xi_2}$$

\>> by CMS

}\hfil}

\newpage %-------------------------------------------------------------------------------------------
\context{Central diffraction -- TOTEM + CMS}
\title{Combined TOTEM+CMS analyses}

\> TOTEM and CMS independent experiments -- common runs need:
\>> exchange of triggers (TOTEM developed a faster electrical trigger)
\>> offline data merging (based on BunchCrossing and Orbit counters)

\> TOTEM + CMS = unprecedented rapidity coverage
\>> CMS tracker: $|η| < 2.5$
\>> CMS calorimeters: $|η| < 5.5$
\>> TOTEM-T1: $3.1 < |η| < 4.7$
\>> TOTEM-T2: $5.3 < |η| < 6.5$
\>> CMS-FSC: $6 < |η| < 8$

\> data available: $\sqrt s = 8\un{TeV}$, $\be^* = 90\un{m}$

\> two direction of studies
\>> \Em{soft CD}: inclusive $X$\\
$\Rightarrow$ analysis as with TOTEM alone

\>> \Em{hard CD}: $X = \hbox{jets} + \ldots$\\
$\Rightarrow$ interesting interplay between soft/non-perturbative and hard/perturbative QCD effects


\newpage %-------------------------------------------------------------------------------------------
\context{Central diffraction -- TOTEM + CMS}
\title{Hard CD}

\vskip-3mm

\line{\raise30mm\vbox{\hsize4.7cm
\> low cross-section processes
\>> \Em{background critical} (typically pile-up)
\>> \Em{more data needed}\\
$\Rightarrow$ $90\un{m}$ optics with more bunches or low-$\be^*$ optics
}\hskip3mm\fig[10cm]{fig/cd_totem_cms_event_display.png}\hss}

\vskip-5mm
\> pile-up removal:
\>> 0 or 1 vertex in CMS
\>> RP near edge area removed (1 elastic proton + beam halo or SD)
\>> $\xi > 1.5\un{\%}$ (far enough from resolution effects)
\>> RP $\xi$ predict event topology in central detectors
\>> FSC empty: QCD background protection
\>> $M_X^{\rm CMS} < M_X^{\rm TOTEM\ RPs}$

\newpage %-------------------------------------------------------------------------------------------
\hbox{}
\vfill
\title{Forward charged-particle multiplicities}


\vfill
\> $\d N_{\rm ch} / \d\et$ : mean number of charged particles per event and per unit of pseudorapidity

\> probes (non-)perturbative strong interactions and hadronisation 

\> primary particles only: primary = lifetime $> 30\un{ps}$ (definition consistent with other LHC experiments)

\> T1 and T2 extend $\et$ range to forward directions
\>> T1 analyses yet in early phase

\> measurement based on T2 only
\>> still $\approx 95\un{\%}$ of inelastic events seen
\>> almost all non-diffractive events visible
\>> almost all diffraction with $m_X \gs 3.4\un{GeV}$ detected

\vskip0pt plus0.1fill

\newpage %-------------------------------------------------------------------------------------------
\context{Forward charged-particle multiplicities}
\title{Method}

\line{\vbox{\hsize9cm
$$
\left. {\De N_{\rm ch} \over \De\et} \right|_{\et = \et_0}
	= {1\over N_{\rm ev}} \sum\limits_{\rm events} \ \sum\limits_{\rm \vbox{\SmallerFonts\hbox{tracks}\hbox{in bin $\et_0$}} } {\hbox{corrections}\over \De\et}
$$

$$
\hbox{corrections} = {W(\et_0, z_{\rm impact})\over \ep(\et_0, m)}\, \sum\limits_j B_j(\et_0)\ G(\et_0)\ S_p(\et_0)\ {2\pi\over \Ph}\ H\ P
$$

\vskip3mm

\> \Em{$W(\et_0, z_{\rm impact})$}: probability of a track to be primary

\> \Em{$\ep(\et_0, m)$}: primary-track efficiency, function of pad multiplicity $m$; value: $0.7$ to $0.9$

\> \Em{$B_j(\et_0)$}: bin migration (MC based)

}\hskip3mm\fig[6.2cm]{fig/dNdeta_zimpact.png}\hss}


\> \Em{$G(\et_0)$}: primary particles not reaching T2 (MC based); value $\approx 1.05$

\> \Em{$S_p(\et_0)$}: impurity of primary selection, mainly due to $K_S^0$ and $\pi^0\rightarrow \ga$'s; MC based; value $0.8$ to $0.9$

\> \Em{$2\pi/\Ph$}: geometrical acceptance of 1 quarter; analyses performed per quarter $\Rightarrow$ confidence

\> \Em{$H$}: correction for large-shower events discarded in the analysis (MC based); value $\approx 1.08$

\> \Em{$P$}: pile-up correction (estimated from zero-bias data stream); value $\approx 1.03$

\newpage %-------------------------------------------------------------------------------------------
\context{Forward charged-particle multiplicities}
\title{7 TeV Results}

\> published: EPL 98 (2012) 31002

\> main contributions to systematic uncertainty ($\approx 10\un{\%}$)
\>> subtraction of a large fraction of secondaries (about $80\un{\%}$ of all T2 tracks)
\>> track efficiency and misalignment uncertainties

\line{\hss\fig[,5.4cm]{fig/dNdeta_totem_7TeV.png}\hskip1mm\fig[,5.4cm]{fig/dNdeta_7TeV_allLHC.png}\hss}

% no MC describes data well

\> gap LHCb -- TOTEM T2 will be filled
\>> analysis of T1 data in progress
\>> data with shifted IP by $11\un{m}$ $\Rightarrow$ shift of T2 acceptance: $6.0 < \et < 7.3$ or  $3.8 < \et < 4.8$


\newpage %-------------------------------------------------------------------------------------------
\context{Forward charged-particle multiplicities}
\title{8 TeV: combined analysis TOTEM + CMS}

\vskip-3mm

\> combined TOTEM + CMS analysis
\>> the same T2-triggered data sample
\>> unified track selection

\> number of improvements wrt.~7 TeV analysis
\>> improved simulation of T2 detector response
\>> vertex information from CMS reduces pile-up correction
\>> MC better tuned to reproduce LHC data

\vfil
\line{\fig[9.5cm]{fig/dNdeta_totem_cms_8TeV.png}\hskip2mm\raise30mm\vbox{\hsize5.3cm
\centerline{PRELIMINARY RESULT:}
\noindent corrections and correlated systematics between CMS and TOTEM\break under study
}\hss}

\newpage %-------------------------------------------------------------------------------------------
\context{Forward charged-particle multiplicities}
\title{8 TeV: event classes}

\> TOTEM stand-alone analysis performed also for different event classes:
\>> \Em{inclusive}: as before
\>> \Em{non-single-diffractive enhanced}: requiring both hemispheres of T2 on
\>> \Em{single-diffractive enhanced}: requiring only one hemisphere of T2 on

\vskip2mm
\line{\hss\fig[10cm]{fig/dNdeta_totem_8TeV.png}\hss}

\vfil
\> in future: also correlations left/centre/right

\newpage %-------------------------------------------------------------------------------------------
\title{Upgrade plans}

TODO:

RP upgrade --> to cope with pile-up

\> multiple tracks in RPs per event --> pixels or a 3rd rotated RP
\> timing detector --> match RP tracks with CMS vertices

\newpage %-------------------------------------------------------------------------------------------
\title{Summary}

\noindent\Em{Number of results published}

\> elastic differential cross-section ($\sqrt s = 7\un{TeV}$)
\> total, elastic and inelastic cross-section ($\sqrt s = 7$ and $8\un{TeV}$)
\> forward charged-particle pseudorapidity density ($\sqrt s = 7\un{TeV}$)

\vfil
\noindent\Em{Number of analyses in progress, some of them combined with CMS}

\> double diffraction ($\sqrt s = 7$ and $8\un{TeV}$)
\> low-$|t|$ elastic cross-section and Coulomb-interference ($\sqrt s = 8\un{TeV}$)
\> elastic differential cross-section ($\sqrt s = 8\un{TeV}$)
\> forward charged-particle pseudorapidity density ($\sqrt s = 8\un{TeV}$), with CMS
\> single diffraction ($\sqrt s = 7$ and $8\un{TeV}$)
\> central diffraction, soft and hard ($\sqrt s = 7$ and $8\un{TeV}$), some with CMS


\vfil
\noindent\Em{More data available}

\> $\rm p + p$ data at $\sqrt s = 2.76\un{TeV}$
\> $\rm p + A$ data



\vfil
\noindent\Em{Roman Pot upgrade ongoing}
\> preparation for higher luminosities (higher pile-up)

\vfil
\eject
\bye
