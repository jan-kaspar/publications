\section{Experimental Apparatus}

\todo{shorter? leave out T1 and T2?}\\
\todo{the same as in 90m paper?}

The TOTEM experiment is dedicated to the measurement of the total 
cross-section, elastic scattering
and diffractive processes at the LHC. The experimental
apparatus, composed of three subdetectors
(Roman Pots (RP), T1 and T2 forward tracking telescopes), 
is placed symmetrically on both sides of Interaction Point (IP) 5, shared
with the CMS experiment. All three subdetectors have
trigger capability. The Roman Pot stations, equipped with
silicon detectors and placed at 220\,m from the IP,
detect elastically and diffractively scattered protons with
small scattering angles down to a few $\mu$rad. 
Each RP station is composed of two units separated
by a distance of about 5\,m. A unit consists of 3 RPs, two
approaching the outgoing beam vertically and one horizontally.
Each RP is equipped with a stack of 10 silicon
strip detectors designed with the specific objective of
reducing the insensitive area at the edge facing the beam
to only a few tens of micrometers. The long lever arm
between the near and the far RP units has two important
advantages: the local track angles in the x and y projections
perpendicular to the beam direction can be reconstructed
with a precision of about $10\,\mu$rad, and a high trigger efficiency
($> 99$\%) can be achieved as the proton trigger selection
uses all RPs independently.

The T1 and T2 telescopes, placed at about 8 and 14\,m from the IP, 
respectively, detect charged particles produced in the polar
angular range from a few mrad to about 100 mrad. The T1 telescope
($3.1 < |\eta| < 4.7$) consists of Cathode Strip Chambers,
while the T2 telescope ($5.3 < |\eta| < 6.5$) is made of
triple-GEM (Gas Electron Multipliers) chambers.

A complete description of the TOTEM detector layout is given in~\cite{totem-jinst}.
