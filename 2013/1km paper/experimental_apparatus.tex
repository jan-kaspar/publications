\section{Experimental Apparatus}
\label{sec:exp apparatus}

The TOTEM experiment, located at the LHC Interaction Point (IP) 5 together with
the CMS experiment, is dedicated to the measurement of the total 
cross-section, elastic scattering
and diffractive processes. The experimental
apparatus, symmetric with respect to the IP, is 
composed of a Roman Pot (RP) proton spectrometer and the forward tracking 
telescopes T1 and T2. All subdetectors have trigger capability. 
While a complete description of the TOTEM detector instrumentation 
and its performance is given in~\cite{totem-jinst} and~\cite{totem-ijmp}, 
the present article focusses on the data from the RPs, movable beam-pipe
insertions that approach the LHC beam very closely to detect protons with
scattering angles of only a few $\mu$rad. 
The RPs are organised in two stations: one on the left side of the IP 
(LHC sector 45) and one on the right (LHC sector 56).
Each RP station, located between 214 and 220\,m from the IP, is composed of two 
units: ``near'' (214\,m from the IP) and ``far'' (220\,m). 
A unit consists of 3 RPs, two
approaching the outgoing beam vertically and one horizontally.
Each RP is equipped with a stack of 10 silicon
strip detectors designed with the specific objective of
reducing the insensitive area at the edge facing the beam
to only a few tens of micrometers. The 5\,m long lever arm 
between the near and the far RP units has two important
advantages: the local track angles can be reconstructed
with a precision of about $10\,\mu$rad, and a high trigger efficiency
($> 99$\%) can be achieved as the proton trigger selection
uses all RPs independently. Since elastic scattering events consist of two anti-parallel protons, the detected events can have two topologies, called diagonals: 45 bottom -- 56 top and 45 top -- 56 bottom.

This report will use a reference frame where $x$ denotes the horizontal axis (pointing out of the LHC ring), $y$ the vertical axis (pointing against gravity) and $z$ the beam axis (in the clockwise direction).
