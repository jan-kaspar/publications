\section{Experimental Apparatus}
\label{sec:exp apparatus}

The TOTEM experiment, located at the LHC Interaction Point (IP) 5 together with
the CMS experiment, is dedicated to the measurement of the total 
cross-section, elastic scattering
and diffractive processes. The experimental
apparatus, symmetric with respect to the IP, is 
composed of a forward proton spectrometer (Roman Pots, RPs) and the 
forward tracking telescopes T1 and T2. 
A complete description of the TOTEM detector instrumentation 
and its performance is given in~\cite{totem-jinst} and~\cite{totem-ijmp}. 
The data analysed here come from the RPs only. A RP is a movable beam-pipe
insertion capable of approaching the LHC beam to less than a millimeter, in 
order to detect protons with scattering angles of only a few microradians. 
The proton spectrometer is organised in two RP stations: one on the left side of the IP 
(LHC sector 45) and one on the right (LHC sector 56).
Each RP station, located between 215 and 220\,m from the IP, is composed of two 
units: ``near'' (215\,m from the IP) and ``far'' (220\,m). 
A unit consists of 3 RPs, one
approaching the outgoing beam from the top, one from the bottom, and one 
horizontally.
Each RP houses a stack of 10 silicon
strip detectors designed with the specific objective of
reducing the insensitive area at the edge facing the beam
to only a few tens of micrometers. Due to the 5\,m long lever arm 
between the near and the far RP units 
the local track angles can be reconstructed
with a precision of about $10\,\mu$rad. A high trigger efficiency
($> 99$\%) is achieved by using all RPs independently. 
Since elastic scattering events consist of two collinear protons emitted in 
opposite directions, the detected events can have two topologies, called 
diagonals: 45 bottom -- 56 top and 45 top -- 56 bottom.

This report uses a reference frame where $x$ denotes the horizontal axis (pointing out of the LHC ring), $y$ the vertical axis (pointing against gravity) and $z$ the beam axis (in the clockwise direction).
