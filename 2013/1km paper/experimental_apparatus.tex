\section{Experimental Apparatus}
%
The TOTEM experiment, located at the LHC Interaction Point (IP) 5 together with
the CMS experiment, is dedicated to the measurement of the total 
cross-section, elastic scattering
and diffractive processes. The experimental
apparatus, symmetric with respect to the IP, is 
composed of a Roman Pot (RP) proton spectrometer and the forward tracking 
telescopes T1 and T2. All subdetectors have trigger capability. 
While a complete description of the TOTEM detector instrumentation 
and its performance is given in~\cite{totem-jinst} and~\cite{totem-ijmp}, 
the present article focusses on the data from the RP system, detecting 
elastically and diffractively scattered protons with
scattering angles of only a few $\mu$rad. 
Each RP station, located between 215 and 220\,m from the IP, is composed of two 
units separated
by a distance of about 5\,m. A unit consists of 3 RPs, two
approaching the outgoing beam vertically and one horizontally.
Each RP is equipped with a stack of 10 silicon
strip detectors designed with the specific objective of
reducing the insensitive area at the edge facing the beam
to only a few tens of micrometers. The long lever arm
between the near and the far RP units has two important
advantages: the local track angles in the x and y projections
perpendicular to the beam direction can be reconstructed
with a precision of about $10\,\mu$rad, and a high trigger efficiency
($> 99$\%) can be achieved as the proton trigger selection
uses all RPs independently.

