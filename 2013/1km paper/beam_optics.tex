\section{Beam optics}
\label{sec:beam optics}

The beam optics with $\beta^{*} = 1000\,$m
has been specifically developed for measuring low-$|t|$ elastic scattering.
Like the previously used $\beta^{*} = 90\,$m optics~\cite{epl96,epl101-el,epl101-tot,prl111},
this new optics configuration
provides parallel-to-point focussing in the vertical plane at the RP position 
$s = 220\un{m}$, implemented by tuning the transport matrix elements via the
LHC magnet currents: the magnification $v_{y}$ is made to vanish whereas the 
effective length $L_{y}$ is maximised ($285.6\un{m}$ as compared to $260\un{m}$ for 
$\beta^{*} = 90\,$m).
The resulting large displacement of a scattered proton with a given $t$-value 
away from the beam centre at the RP, together with a very small beam 
divergence in the interaction point and hence a tiny beam width 
at the RP ($\sigma_{y} = 260\,\mu$m as compared to 700\,$\mu$m for 
$\beta^{*} = 90\,$m) translates into 
an acceptance for $|t|$-values down to $10^{-4}\,\rm GeV^{2}$, provided the 
vertical RPs can approach the beam centre to only a few nominal beam standard 
deviations~\footnote{The nominal standard deviation is defined as 
the beam width for a normalised emittance 
$\varepsilon \gamma = 3.5\,\mu$m\,rad.}.
A further improvement relative to $\beta^{*} = 90\,$m is the non-vanishing 
effective length in the horizontal plane, $L_{x} = 46.8\,$m at $s = 220\,$m, 
enabling a better reconstruction of the horizontal component of the 
scattering angle.

\todo{integrate the formulae for proton transport}
\begin{equation}
\label{eq:scatt angle}
\theta_x^* = \theta^* \cos\phi^*\ ,\qquad \theta_y^* = \theta^* \sin\phi^*\ .
\end{equation}

\begin{equation}
\label{eq:prot trans}
	x = L_x\, \theta_x^* + v_x\, x^* + D_x\, \xi\ ,\qquad y = L_y\, \theta_y^* + v_y\, y^* + D_y\, \xi\ ,
\end{equation}


\todo{add table with optical values?}

