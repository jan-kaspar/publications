\section{Beam Optics}
\label{sec:beam optics}
%
The beam optics relates the proton state at the IP to its state at the RP location. At the IP, the direction of a proton can be described by the scattering angle $\theta^*$ (with respect to the $z$ axis) and azimuthal angle $\phi^*$ (about the $z$ axis). Alternatively, the horizontal ($x$) and vertical ($y$) projections of the scattering angle can be used:
\begin{equation}
\label{eq:scatt angle}
\theta_x^* = \theta^* \cos\phi^*\ ,\qquad \theta_y^* = \theta^* \sin\phi^*\ .
\end{equation}
A proton emerging from the vertex $(x^*$, $y^*)$ at the angle $(\theta_x^*,\theta_y^*)$ and with momentum $p\,(1+\xi)$, where $p$ is the nominal initial-state proton momentum, arrives at the RPs in the transverse position
\begin{equation}
\label{eq:prot trans}
x(z_{\rm RP}) = L_x(z_{\rm RP})\, \theta_x^*\ +\ v_x(z_{\rm RP})\, x^*\ +\ D_x(z_{\rm RP})\, \xi\ ,\quad y(z_{\rm RP}) = L_y(z_{\rm RP})\, \theta_y^*\ +\ v_y(z_{\rm RP})\, y^*\ +\ D_y(z_{\rm RP})\, \xi \quad
\end{equation}
relative to the beam centre. This position is determined by the optical functions: effective length $L_{x,y}(z)$, magnification $v_{x,y}(z)$ and dispersion $D_{x,y}(z)$. Since protons lose no momentum in elastic collisions, the values of $\xi$ follow from the initial state momentum offset and variations, see Section 4 in~\cite{8tev-90m}. Due to the collinearity of the two elastically scattered protons and the symmetry of the optics, the impact of the dispersion terms $D\,\xi$ on the reconstructed scattering angles (Eq.~(\ref{eq:kin 2a})) is negligible compared to other uncertainties and thus the terms can be ignored.

The special optics characterised by $\beta^{*} = 1000\,$m, where $\beta^{*}$ is the value of the $\beta$-function at the interaction point,
has been specifically developed for measuring low-$|t|$ elastic scattering.
Like the previously used $\beta^{*} = 90\,$m optics~\cite{epl96,epl101-el,epl101-tot,prl111}, this new optics configuration
provides parallel-to-point focussing in the vertical plane at the RP position 
$z = 220\un{m}$, implemented by tuning the transport matrix elements 
(Table~\ref{tab:optics}) via the
LHC magnet currents: the magnification $v_{y}$ is made to vanish whereas the 
effective length $L_{y}$ is maximised ($285\un{m}$ as compared to $260\un{m}$ for $\beta^{*} = 90\,$m). These settings optimise the sensitivity to the scattering angle -- and hence to $|t|$ -- by maximising the displacement of a scattered proton from the beam centre at the RP. In addition, the beam divergence in the 
interaction point is minimised, resulting in a tiny beam width 
at the RP ($\sigma_{y} = 260\,\mu$m as compared to 700\,$\mu$m for 
$\beta^{*} = 90\,$m, assuming a normalised transverse beam emittance of $\varepsilon \gamma = 3.5\,\mu$m\,rad). By approaching the beam centre with the vertical RPs to a distance of only a few $\sigma_{y}$, an acceptance for $|t|$-values down to the order of $10^{-4}\,\rm GeV^{2}$ is obtained.

A further improvement relative to $\beta^{*} = 90\,$m is the non-vanishing 
effective length in the horizontal plane, $L_{x} = 46\,$m at $z = 220\,$m, 
enabling a better reconstruction of the horizontal component of the 
scattering angle.

\begin{table}
\caption{
Optical functions for elastic proton transport for the $\beta^{*} = 1000\,$m optics. The values refer to the right arm, for the left one they are very similar.
}
\label{tab:optics}
\begin{center}
\vskip-3mm
\begin{tabular}{ccccc}\hline\hline
RP unit & $L_x$ & $v_x$ & $L_y$ & $v_y$ \cr\hline
near & $59.37\un{m}$  & $-0.867$ & $255.87\un{m}$ & $0.003$ \cr
far  & $45.89\un{m}$ & $-0.761$ & $284.62\un{m}$ & $-0.017$ \cr
\hline\hline
\end{tabular}
\end{center}
\end{table}


