\section{Beam optics}
\label{sec:beam optics}
%
The beam optics relates the proton state at the IP to its state at the RP location. At the IP, the direction of a proton can be described by the scattering angle $\theta^*$ (with respect to the $z$ axis) and azimuthal angle $\phi^*$ (about the $z$ axis). Alternatively, the horizontal ($x$) and vertical ($y$) projections of the scattering angle can be used:
\begin{equation}
\label{eq:scatt angle}
\theta_x^* = \theta^* \cos\phi^*\ ,\qquad \theta_y^* = \theta^* \sin\phi^*\ .
\end{equation}
A proton with such scattering angles, emerging at the vertex $(x^*$, $y^*)$ with momentum $p\,(1+\xi)$, where $p$ is the nominal initial-state proton momentum, arrives at the RPs in a transverse position
\begin{equation}
\label{eq:prot trans}
	x = L_x\, \theta_x^* + v_x\, x^* + D_x\, \xi\ ,\qquad y = L_y\, \theta_y^* + v_y\, y^* + D_y\, \xi\ ,
\end{equation}
with respect to the beam centre. The position is determined by the optical functions: effective length $L$, magnification $v$ and dispersion $D$. Since protons lose no momentum in elastic collisions, the values of $\xi$ follow from the initial state momentum offset and variations, see Section 4 in \cite{8tev-90m}. Due to the collinearity of the two elastically scattered protons and the symmetry of the optics, the impact of the dispersion terms $D\,\xi$ on the reconstructed scattering angles (Eq.~(\ref{eq:kin 2a})) is negligible compared to other uncertainties and thus the terms can be ignored.

The special optics characterised by $\beta^{*} = 1000\,$m~\footnote{$\beta^{*}$ is the value of the $\beta$-function at the interaction point; for the practical purposes of this article, it can be considered as a name label for the optics configuration.}
has been specifically developed for measuring low-$|t|$ elastic scattering.
Like the previously used $\beta^{*} = 90\,$m optics~\cite{epl96,epl101-el,epl101-tot,prl111},
this new optics configuration
provides parallel-to-point focussing in the vertical plane at the RP position 
$z = 220\un{m}$, implemented by tuning the transport matrix elements 
(Table~\ref{tab:optics}) via the
LHC magnet currents: the magnification $v_{y}$ is made to vanish whereas the 
effective length $L_{y}$ is maximised ($285\un{m}$ as compared to $260\un{m}$ for $\beta^{*} = 90\,$m).
The resulting large displacement of a scattered proton with a given $t$-value 
away from the beam centre at the RP, together with a very small beam 
divergence in the interaction point and hence a tiny beam width 
at the RP ($\sigma_{y} = 260\,\mu$m as compared to 700\,$\mu$m for 
$\beta^{*} = 90\,$m) translates into 
an acceptance for $|t|$-values down to $10^{-4}\,\rm GeV^{2}$, provided the 
vertical RPs can approach the beam centre to only a few nominal beam standard 
deviations~\footnote{The nominal standard deviation is defined as 
the beam width for a normalised emittance 
$\varepsilon \gamma = 3.5\,\mu$m\,rad.}.
A further improvement relative to $\beta^{*} = 90\,$m is the non-vanishing 
effective length in the horizontal plane, $L_{x} = 46\,$m at $z = 220\,$m, 
enabling a better reconstruction of the horizontal component of the 
scattering angle.

\begin{table}
\caption{
Optical functions for elastic proton transport for the $\beta^{*} = 1000\,$m optics. The values refer to the right arm, for the left one they are very similar.
}
\label{tab:optics}
\begin{center}
\vskip-3mm
\begin{tabular}{ccccc}\hline\hline
RP unit & $L_x$ & $v_x$ & $L_y$ & $v_y$ \cr\hline
near & $59.37\un{m}$  & $-0.867$ & $255.87\un{m}$ & $0.003$ \cr
far  & $45.89\un{m}$ & $-0.761$ & $284.62\un{m}$ & $-0.017$ \cr
\hline\hline
\end{tabular}
\end{center}
\end{table}


