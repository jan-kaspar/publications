\documentclass[pdftex,twocolumn,epjc3]{svjour3}

\RequirePackage[T1]{fontenc}

\RequirePackage{graphicx}
\RequirePackage{mathptmx}      % use Times fonts if available on your TeX system
\RequirePackage{flushend}
\RequirePackage[numbers,sort&compress]{natbib}
\RequirePackage[colorlinks,citecolor=blue,urlcolor=blue,linkcolor=blue]{hyperref}
\RequirePackage{amsmath}
\RequirePackage{amssymb}
\RequirePackage{enumitem}

%----------------------------------------------------------------------------

\def\d{{\rm d}}
\def\un#1{\,{\rm #1}}
\def\ung#1{\quad[{\rm #1}]}
\def\unt#1{[{\rm #1}]}
\def\e{{\rm e}}
\def\I{{\rm i}}
\def\T{{\rm T}}
%\def\vec#1{\mathbf{#1}}	% available in macro set
\def\mat#1{\tens{#1}}
\def\etal{et al.}
\def\todo#1{{\color{red}TODO: #1}}
\def\TODO#1{{\color{red}TODO: #1}}

\setbox123\hbox{$0$}
\setbox124\hbox{$.$}
\def\S{\hbox to\wd123{\hss}}
\def\.{\hbox to\wd124{\hss}}

\def\Name#1{#1, }
\def\Review#1#2#3#4{{\it #1} {\bf #2} (#3) #4}

%-----------------------------------------------------------------------------

\journalname{Eur. Phys. J. C}

\begin{document}

\title{Measurement of Elastic pp Scattering at $\sqrt{\hbox{s}} = \hbox{8}$\,TeV in the Coulomb-Nuclear Interference Region -- Determination of the $\mathbf{\rho}$-Parameter and the Total Cross-Section}

\titlerunning{Meas.~of Elastic pp Scatt.~at $\sqrt{\hbox{s}} = \hbox{8}$\,TeV in the CNI Region -- Determination of $\rho$ and $\sigma_{\rm tot}$}

\input authorlist_epjc

\authorrunning{The TOTEM Collaboration}

\date{Manuscript date: \today}

\maketitle


\begin{abstract}
\input abstract
\keywords{proton-proton interactions \and elastic scattering \and Coulomb-Nuclear Interference \and total cross-section \and rho parameter \and TOTEM \and LHC}
\PACS{
13.85.Dz % Elastic scattering
\and
13.85.Lg % Total cross sections
\and
13.40.Ks % Electromagnetic corrections to strong- and weak-interaction processes
}
\end{abstract}

%--------------------------------------------------

\input introduction.tex

\input experimental_apparatus.tex

\input beam_optics.tex

\input data_taking.tex

% TODO
\input differential_cross_section.tex

\input coulomb.tex

\input conclusions.tex

\begin{acknowledgements}
\input acknowledgements.tex
\end{acknowledgements}

%--------------------------------------------------

\begin{thebibliography}{99}
\input bibliography.tex
\end{thebibliography}


\end{document}
