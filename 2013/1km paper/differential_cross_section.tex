\section{Differential Cross-Section}
\label{sec:differential cross-section}

The analysis method is very similar to the previously published one \cite{8tev-90m} and is presented in two main blocks. Section~\ref{sec:event analysis} covers all aspects related to the reconstruction of a single event. Section~\ref{sec:diff cs} describes the steps of transforming a raw $t$-distribution into the differential cross-section. The $t$-distributions for the two diagonals are analysed separately. After comparison (Section~\ref{sec:cross checks}) they are finally merged (Section~\ref{sec:final data merging}).

%----------------------------------------------------------------------------------------------------
\subsection{Event Analysis}
\label{sec:event analysis}

Event kinematics are determined from track hits in RPs (after proper alignment, see Sec.~\ref{sec:alignment}) using the LHC optics (see Sec.~\ref{sec:optics}).

%------------------------------

\subsubsection{Kinematics Reconstruction}
\label{sec:kinematics}

The scattering angles and vertex position are first determined for each proton (i.e.~from each arm) separately by inverting the proton transport, Eq.~(\ref{eq:prot trans}), assuming $\xi = 0$. The following formulae optimise the robustness against optics imperfections:
\begin{equation}
\label{eq:kin 1a}
	\begin{aligned}
		\theta_x^{*\rm L,R} &= {v_x^{\rm N} x^{\rm F} - v_x^{\rm F} x^{\rm N}\over v_x^{\rm N} L_x^{\rm F} - v_x^{\rm F} L_x^{\rm N}}\ ,\quad
		\theta_y^{*\rm L,R} = {1\over 2} \left( {y^{\rm N}\over L_y^{\rm N}} + {y^{\rm F}\over L_y^{\rm F}} \right)\ ,\\
		x^{*\rm L,R} &= {L_x^{\rm N} x^{\rm F} - L_x^{\rm F} x^{\rm N}\over L_x^{\rm N} v_x^{\rm F} - L_x^{\rm F} v_x^{\rm N}}\ , \\
	\end{aligned}
\end{equation}
where the N and F superscripts refer to the near and far units, L and R to the left and right arm, respectively. This one-arm reconstruction is used for tagging elastic events, where the left and right arm protons are compared.

Once an event is selected, all four RPs can be used to reconstruct the kinematics of the event, in addition optimising for angular resolution (see Section \ref{sec:resolution}):
\begin{equation}
\label{eq:kin 2a}
		\theta_x^* = {
				\sum {v_x^i}^2 \sum L_x^i x^i - \sum L_x^i v_x^i \sum v_x^i x^i
				\over
				\sum {v_x^i}^2 \sum {L_x^i}^2 - \sum L_x^i v_x^i \sum v_x^i L_x^i
			}\ ,\quad
		\theta_y^* = {1\over 4} \sum {y^i\over L_y^i}\ ,
\end{equation}
where the sums go over the superscripts $i$ representing the four RPs of a diagonal.

Eventually, the full scattering angle, $\theta^*$, and four-momentum transfer squared, $t$, are calculated:

\begin{equation}
\label{eq:th t}
\theta^* = \sqrt{{\theta_x^*}^2 + {\theta_y^*}^2}\ ,\quad t = - p^2 ({\theta_x^*}^2 + {\theta_y^*}^2)\ ,
\end{equation}
where $p$ denotes the beam momentum.


%------------------------------

\subsubsection{Alignment}
\label{sec:alignment}

The standard three-stage procedure \cite{totem-ijmp} has been applied: beam-based alignment prior to the run (as for LHC collimators) followed by two off-line methods. First, track-based alignment for relative positions among RPs, and second, alignment with elastic events for absolute position with respect to the beam (performed in 15 minutes time periods to check for possible beam movements).

This procedure has been further extended to improve the vertical alignment. The new steps exploit the fact that the elastic scattering with its two anti-collinear protons can relate the alignment in the left and right arm (with an uncertainty of $20\un{\mu m}$). Furthermore, the horizontal RPs in the right arm recorded a hit distribution usable for vertical alignment in addition to the standard technique based on the vertical RPs, see Figure~\ref{fig:align meth}.

\begin{figure}
\begin{center}
\includegraphics{fig/alignment_method.pdf}
\caption{%
Hit scatter plot in the right-far unit, corresponding to a period of 15 minutes. Black (blue) dots and solid lines represent track hits and sensor contours of vertical (horizontal) RPs. Track hits close to the sensor edges are discarded because of possible bias due to acceptance effects.
The magenta histogram shows the horizontal profile of hits in the vertical RPs, the dashed magenta line interpolates the profiles between the top and bottom RPs. Similarly, the red histogram gives the vertical profile of the hits in the horizontal RP and the dashed red line its extrapolation to the beam region. The green dashed line indicates the vertical centre of symmetry of the hits in the vertical RPs (see \cite{totem-ijmp} for details). The crossing of the dashed lines represents the position the beam centre.
}
\label{fig:align meth}
\end{center}
\end{figure}

Exploiting all the methods, the alignment uncertainties have been estimated to $30\un{\mu m}$ (horizontal shift), $70\un{\mu m}$ (vertical shift) and $2\un{m rad}$ (rotation about the beam axis). Propagating them through Eq.~(\ref{eq:kin 2a}) to reconstructed scattering angles yields gives $0.28\un{\mu rad}$ ($0.19\un{\mu rad}$) for horizontal (vertical) angle. RP rotations induce a bias in the reconstructed horizontal scattering angle:
\begin{equation}
\label{eq:alig rot bias}
	\theta_x^* \rightarrow \theta_x^* + c \theta_y^*\ ,
\end{equation}
where the proportionality constant $c$ has a standard deviation of $0.005$.


% the observed hit inefficiencies -- possible asymmetries -- discuss ???
% mention final alignment check = 2D Gaussian fit of $\theta_x^*$ vs.~$\theta_y^*$ from both diagonals ??


%------------------------------

\subsubsection{Optics}
\label{sec:optics}
In order to reduce the impact of imperfect optics knowledge, the optics matching technique \cite{totem-optics} has been applied. This method uses various RP observables to determine fine corrections to the optical functions presented in Eq.~(\ref{eq:prot trans}).

The residual errors induce a bias in the reconstructed scattering angles:
\begin{equation}
\label{eq:opt bias}
	\theta_x^* \rightarrow (1 + d_x)\, \theta_x^*\ ,\qquad
	\theta_y^* \rightarrow (1 + d_y)\, \theta_y^*\ .
\end{equation}
For the two-arm reconstruction, Eq.~(\ref{eq:kin 2a}), the biases $d_x$ and $d_y$ have uncertainties of $0.34\un{\%}$ and $0.25\un{\%}$, respectively, and correlation factor of $-0.89$. These estimates include the effects of magnet harmonics. For evaluating the impact on the $t$-distribution, it is convenient to decompose the correlated biases $d_x$ and $d_y$ into eigenvectors of the covariance matrix:
\begin{equation}
\label{eq:opt bias modes}
\begin{pmatrix} d_x\cr d_y \end{pmatrix} =
	\eta_1 \underbrace{\begin{pmatrix} +0.338\un{\%} \cr -0.234\un{\%} \end{pmatrix}}_{\rm mode\ 1}
	\ +\ \eta_2 \underbrace{\begin{pmatrix} -0.053\un{\%} \cr -0.076\un{\%} \end{pmatrix}}_{\rm mode\ 2}
\end{equation}
normalised such that the factors $\eta_{1,2}$ have unit variance.

%	mode 0			mode 1
%	3.382E-03		-5.278E-04
%	-2.343E-03		-7.618E-04

%------------------------------

\subsubsection{Resolution}
\label{sec:resolution}

\TODO{Which plots interesting?}

The statistical fluctuations in $\theta_y^*$ are mostly due to the beam divergence and can be studied by comparing the angles reconstructed from the left and right arm. If the divergences of the two beams are the same (verified by emittance measurements with uncertainty of $25\un{\%}$), then the beam divergence distribution can be de-convoluted from the distribution of the left-right differences. The beam divergence has shown very small deviations from Gaussian shape, \TODO{decreasing width with time?}. The $\theta_y^*$ (Eq.~(\ref{eq:kin 2a})) resolution degraded from initial $0.43$ to $0.48\un{\mu rad}$ at the end of the fill.

In the horizontal projection, a different method is used since the one-arm reconstruction, Eq.~(\ref{eq:kin 1a}), is strongly influenced by sensor resolution (strip pitch). First, the beam divergence is estimated by dividing the vertex spread $\sigma(x^*)$ by $\beta^*$: it grew from $0.75$ to $0.9\un{\mu rad}$ during the fill. Then the beam-divergence component is subtracted quadratically \TODO{check} 
from the standard deviation of the right-left difference in $\theta_x^*$, yielding the (time-independent \TODO{experimentally verified}) contribution from the sensor resolution: $10.7\un{\mu m}$ (45 top -- 56 bottom) and $12.1\un{\mu m}$ (45 bottom -- 56 top). This can be propagated to the two-arm resolution in $\theta_x^*$, Eq.~({\ref{eq:kin 2a}}), which is by far dominated by the beam divergence component: it rose from $0.54$ to $0.65\un{\mu rad}$ during the fill. 

\begin{figure}
\begin{center}
\includegraphics{fig/resolutions_vs_time.pdf}
\caption{%
\TODO{final two-arm resolutions}. The yellow bands indicate regions of uninterrupted data-taking.
}
\label{fig:resol final}
\end{center}
\end{figure}

%----------------------------------------------------------------------------------------------------
\subsection{Differential Cross-Section Reconstruction}
\label{sec:diff cs}

For a given $t$ bin, the differential cross-section is evaluated by selecting and counting elastic events:
\begin{equation}
{\d\sigma\over \d t}(\hbox{bin}) =
	{\cal N}\, {\cal U}({\rm bin})\, {\cal B}\ 
	{\sum\limits_{t \in \hbox{bin}} {\cal A}(\theta^*, \theta_y^*)\ {\cal E}(\theta_y^*)\over \Delta t}\ ,
\end{equation}
where $\Delta t$ is the width of the bin, ${\cal N}$ is a normalisation factor and the other symbols stand for various correction factors:
 ${\cal U}$ for unfolding of resolution effects, ${\cal B}$ for background subtraction, ${\cal A}$ for acceptance correction and ${\cal E}$ for detection and reconstruction efficiency.

%-------------------------

\subsubsection{Event Tagging}
\label{sec:tagging}

\begin{table}
\caption{The elastic selection cuts. The superscripts R and L refer to the right and left arm. The $\alpha \theta_x^*$ term in cut 3 is intended to absorb possible effects of residual optics imperfections \TODO{[update at the end; give $\alpha$ value]}. The right-most column gives a typical RMS of the cut distribution.
}
\label{tab:cuts}
\begin{center}
\vskip-3mm
\begin{tabular}{ccc}\hline\hline
number & cut & RMS ($\equiv 1\sigma$)\cr\hline
1 & $\theta_x^{*\rm R} - \theta_x^{*\rm L}$				& $3.9\un{\mu rad}$	\cr
2 & $\theta_y^{*\rm R} - \theta_y^{*\rm L}$				& $1.0\un{\mu rad}$	\cr
3 & $x^{*\rm R} - x^{*\rm L} - \alpha \theta_x^*$		& $250\un{\mu m}$ 	\cr\hline\hline
\end{tabular}
\end{center}
\end{table}

The cuts used to select the elastic events are summarised in Table~\ref{tab:cuts}. Cuts 1 and 2 require the reconstructed-track collinearity between the left and right arm. Cut 3 ensures that the protons come from the same vertex (horizontally). The correlation plots corresponding to these cuts are shown in Figure~\ref{fig:cuts}. Thanks to the very low beam divergence, the collinearity cuts are very powerful, and consequently the other conceivable cuts (cf. Table~2 in~\cite{epl101-el}) bring no significant improvement.

All are applied at $4\sigma$-level - \TODO{Why}.

The tagging efficiency has been studied by applying the cuts also at the $5\un{\sigma}$-level. This selection has yielded $0.3\un{\%}$ more events in every $|t|$-bin. This kind of inefficiency therefore only contributes to a global scale factor and is irrelevant for the presented analysis because the normalisation is taken from a different data set (cf. Section~\ref{sec:normalisation}).


\begin{figure}
\begin{center}
\includegraphics{fig/cuts.pdf}
\caption{%
Correlation plots for the event selection cuts presented in Table~\ref{tab:cuts}, showing events with diagonal topology 45 top -- 56 bottom.
}
\label{fig:cuts}
\end{center}
\end{figure}


%-------------------------

\subsubsection{Background}
\label{sec:background}

To study the background (i.e.~impurity of the tagging), the distributions of right-left angular differences (see Table \ref{tab:cuts}) are plotted under various cut combinations \TODO{vague}. While the central part (signal) remains essentially constant, the tails (background) are dramatically affected. Interpolating the background smoothly into the signal region ($\pm 4\un{\sigma}$) permits to conclude negligible background: $1 - {\cal B} < 0.1\un{\%}$ (independent of whether the $\theta_x^*$ or the $\theta_y^*$ distribution is used). As an additional test, if a cut is released, the recuperated events (less than $0.3\un{\%}$) are distributed almost uniformly over the $|t|$ range.

% background = impurity of the cuts above

\begin{figure}
\begin{center}
\includegraphics{fig/cut_distributions.pdf}
\caption{%
\TODO{DS2b, 45t -- 56b, cuts from Table~\ref{tab:cuts}}
}
\label{fig:tag bckg}
\end{center}
\end{figure}

%-------------------------

\subsubsection{Acceptance Correction}
\label{sec:acc corr}

Beyond the usual acceptance limitations -- sensor edge (relevant for low $|\theta^*_y|$) and LHC apertures (high $|\theta_y^*|$) -- two additional ones have been identified. Both stem from the horizontal RPs interfering with the elastic protons, resulting in a phase space region with uncertain detection efficiency. Consequently, an additional acceptance restriction, $-50 < \theta_x^* < 80\un{\mu rad}$, has been adopted to avoid the interference regions. In the far vertical RPs, the restriction corresponds to about $-2.3 < x < 3.7\un{mm}$.

The expected azimuthal symmetry of elastic scattering is verified for the data within acceptance. Assuming the symmetry globally yields a geometrical correction ${\cal A}_{\rm geom}(\theta^*)$ calculated from the fraction of circle with radius $\theta^*$ falling into the acceptance region in the ($\theta_x^*$, $\theta_y^*$) space. The value of the correction drops rapidly from $10$ at the lowest $|t|$ to about $2$ at $|t| = 0.04\un{GeV^2}$ and then increases slowly to about $10$ as $|t|$ tends to $0.2\un{GeV^2}$.

The other acceptance correction, ${\cal A}_{\rm fluct}$, accounts for fluctuations around horizontal acceptance boundaries. It is calculated analytically from the probability that any of the two elastic protons is pushed outside the acceptance due to the beam divergence. The beam divergence distribution is modelled as Gaussian with spread determined in Section~\ref{sec:resolution}. This correction contribution is only relevant for low $|t|$ (below $0.002\un{GeV^2}$ for diagonal 45 top -- 56 bottom) and does not exceed $2.5$. The uncertainties are related to the resolution parameters. For the lowest $|t|$ bin their relative values are: vertical beam divergence: $2\un{\%}$, left-right asymmetry: $1\un{\%}$ and non-Gaussian shape: $1\un{\%}$.

The final acceptance correction combines the two contributions:
\begin{equation}
{\cal A}(\theta^*, \theta_y^*) = {\cal A}_{\rm geom}(\theta^*)\ {\cal A}_{\rm fluct}(\theta_y^*)\ .
\end{equation}

\iffalse
\> A_g
\>> at low |t|: correction < 10 (for 1 diagonal)
\>> at high |t|: correction < 20 (for 1 diagonal)

\> A_s
\>> at low |t|: correction < 2.5
\>> uncertainties only important for few low |t| bins, giving the value for the lowest |t| bin
\>>> sigma beam div: 2%
\>>> L-R asymmetry: 1%
\>>> non-Gauss: 1%
\fi


\begin{figure}
\begin{center}
\includegraphics{fig/acc_corr_phi_lab.pdf}
\caption{%
\TODO{DS2b, top (bottom) part from diagonal 45 bot -- 56 top (45 top -- 56 bot)}
\TODO{red = LHC apertures, blue = sensor edges, magenta = horizontal RPs}
\TODO{central theta in urad}
}
\label{fig:acc corr princ}
\end{center}
\end{figure}


\begin{figure}
\begin{center}
\includegraphics{fig/acc_corr_hist_lab.pdf}
\caption{%
\TODO{DS2b, 45 top -- 56 bot, control the numbers, mean correction per bin + std.~dev. per bin}
}
\label{fig:acc corr res}
\end{center}
\end{figure}

%-------------------------

\subsubsection{Inefficiency Corrections}
\label{sec:ineff corr}

Due to the external normalisation (Section~\ref{sec:normalisation}) any inefficiency correction that does not alter the $t$-distribution shape does not need to be considered in this analysis (e.g.~the pile-up inefficiency discussed in~\cite{prl111}).

The uncorrelated 1-RP inefficiency ${\cal I}_{3/4}$ is evaluated by removing a given RP from the tagging cuts, Table~\ref{tab:cuts}, and calculating the fraction of recovered events. This fraction has been found to depend gently on the vertical scattering angle: decreasing by about $0.4\un{\%}$ from the lowest to the highest $|\theta_y^*|$. The uncertainty of this dependence has been included in the final systematic uncertainty assessment.

The 1-RP inefficiencies are complemented by near-far correlated inefficiencies ${\cal I}_{2/4}$ from proton interactions in the near RP affecting also the far one. This contribution is determined by counting events with corresponding shower signatures and is validated with a Monte-Carlo simulation.

The full correction is calculated as
\begin{equation}
\label{efficiency}
	{\cal E}(\theta_y^*) = {1\over 1 - \left( \sum\limits_{i\in \rm RPs} {\cal I}^i_{3/4}(\theta_y^*) + 2 {\cal I}_{2/4} \right) } \ ,
\end{equation}
where the first term in parentheses typically grows from about $16$ to $18\un{\%}$ from the lowest to the highest $|\theta_y^*|$ and the second term amounts to about $3\un{\%}$.

\iffalse
\> 3-out-of-4 results
\>> right arm: typical results (near $\approx 98\un{\%}$, far $\approx 96.5\un{\%}$ efficiency)
\>> left arm: efficiency in far RP unexpectedly low ($\approx 90\un{\%}$) -- due to showers in horizontal RPs (horizontal in left arm closer that in the right one) -- but experimentally determinable and thus fully correctable

\> sum of all 3/4 inefficiencies (DS2b, 45 bottom)
\>> lowest |th_y|: 15.96%
\>> highest |th_y|: 17.88%

\> 2/4: 1.5% for one RP

\fi

\begin{figure}
\begin{center}
\includegraphics{fig/eff3outof4_fits.pdf}
\caption{%
\TODO{DS2b, 45t -- 56b, right near}
}
\label{fig:eff 3/4}
\end{center}
\end{figure}


%-------------------------

\subsubsection{Unfolding of Resolution Effects}
\label{sec:unfolding}

\TODO{numerical integration method, in principle similar method as in the 90m paper, but different implementation with better performance}

Due to the very small beam divergence, the correction for resolution effects can be safely determined by the following iterative procedure. The differential cross-section data are fitted by a smooth curve which serves as an input to a Monte-Carlo simulation using the resolution parameters determined above. Making a ratio of simulated histograms with / without smearing effects gives a set of per-bin correction factors. Applying them to the yet uncorrected differential cross-section yields a better estimate of the true $t$-distribution, which can be used as input to the next iteration. The final correction is negligible (${\cal U} \approx 1$) for all bins except at very low $|t|$ where the rapid cross-section growth occurs ($ {\cal U} - 1 < 2.5\un{\%}$).

For the uncertainty estimate, the uncertainties of $\theta_x^*$ and $\theta_y^*$ resolutions (accommodating the full time variation) as well as fit-model dependence have been considered, each contribution giving a few per-mille for the lowest-$|t|$ bin.


\begin{figure}
\begin{center}
\includegraphics{fig/unfolding.pdf}
\caption{%
Unfolding correction for diagonal 45 bottom -- 56 top is negligible everywhere except the very low-$|t|$ region. The vertical dashed line indicates the position of the acceptance cut.
}
\label{fig:unfolding}
\end{center}
\end{figure}

%-------------------------

\subsubsection{Normalisation}
\label{sec:normalisation}

The normalisation ${\cal N}$ is determined by requiring the same cross-section integral between $|t| = 0.014$ and $0.203\un{GeV^2}$ as for dataset 1 from \cite{prl111}, where the luminosity-independent calibration was applied. The leading uncertainty of the scaling factor $4.2\un{\%}$ comes from the luminosity-independent method.
% transfer negligible ($0.5\un{\%}$).
% stat: 0.3%, syst: 0.3% (DS2 at 90m), 0.3% (1km)


%-------------------------

\subsubsection{Binning}
\label{sec:binning}

The binning has been optimised in view of two aspects. At low $|t|$ (below $0.11\un{GeV^2}$), the bin size has been set to the triple of the resolution in $t$. At higher $|t|$, a fixed statistical fluctuation of $4\un{\%}$ has been targeted. For $|t| > 0.16\un{GeV^2}$, a fixed bin size of $0.01\un{GeV^2}$ has been adopted in order to avoid excessively large bins.


%-------------------------

\subsubsection{Systematic Uncertainties}
\label{sec:systematics}

Besides the systematic uncertainties mentioned at the above analysis steps, the beam momentum uncertainty of $0.1\un{\%}$ needs to be considered when the scattering angles are translated to $t$, see Eq.~(\ref{eq:th t}). The systematic effects are propagated to the $t$-distribution by inserting the final cross-section to a Monte-Carlo simulation where any analysis parameter can be artificially biased. 
%The results are validated with a semi-analytic calculation eventually exploiting numerical integration.


%----------------------------------------------------------------------------------------------------

\subsection{Systematic Cross-Checks}
\label{sec:cross checks}

Compatible results have been obtained from data originating from different bunches, different diagonals and different time periods -- in particular those right after and right before the beam cleanings.



%----------------------------------------------------------------------------------------------------
\subsection{Final Data Merging}
\label{sec:final data merging}

Finally, the differential cross-section histograms from both diagonals are merged. This is accomplished by a per-bin weighted average, with the weight given by inverse squared statistical uncertainty. The statistical and systematic uncertainties are propagated accordingly. For the systematic ones, the correlation between the diagonals is taken into account. For example the vertical (mis-)alignment of the RPs within one unit is almost fully correlated, thus the effect on differential cross-section is opposite in the two diagonals and consequently its impact is strongly reduced once the diagonals are merged.

The final cross-section is tabulated in Table~\ref{tab:data} and visualised in Figure~\ref{fig:dsdt}. The figure clearly shows a rapid cross-section rise below $|t| \lesssim 0.002\un{GeV^2}$, which will later be interpreted as an effect due to electromagnetic interaction.

The systematic uncertainties are summarised in Figure~\ref{fig:syst unc} where their impact on the differential cross-section is shown. The leading uncertainties include normalisation, optics imperfections, beam momentum offset and residual misalignment. The vertical misalignment is the dominant systematic effect in the very low $|t|$ region. The leading uncertainties are quantified in Table~\ref{tab:data} and can be used to approximate the covariance matrix of systematic uncertainties:
\begin{equation}
\label{eq:covar mat}
\mat V = \sum_{i=1}^{6} \vec v_i \vec v_i^{\rm T}\ ,
\end{equation}
where the $\vec v_i$'s represent the uncertainty vectors (columns) from the table.




\input data_table.tex


\begin{figure*}
\vskip-5mm
\begin{center}
\includegraphics[width=18cm]{fig/t_dist_tabulation.pdf}
\vskip-3mm
\caption{%
Differential cross-section from Table \ref{tab:data} with statistical (bars) and systematic uncertainties (bands). The blue band represents all systematic uncertainties, the yellow one all but normalisation \TODO{how centred}. LEFT: a low-$|t|$ zoom of the differential cross-section, with the rise due to the Coulomb interaction clearly visible. RIGHT: differential cross-section over the full $|t|$ range plotted relative to a reference exponential.
}
\label{fig:dsdt}
\end{center}
\end{figure*}


\begin{figure*}
\begin{center}
\includegraphics{fig/systematic_uncertainties.pdf}
\caption{%
Impact of systematic uncertainties on the differential cross-section. 
LEFT: full $|t|$ range, RIGHT: low $|t|$ zoom.
The two contributions due to optics correspond to the two eigenvectors of the $\theta_x^*$, $\theta_y^*$ scaling covariance matrix (see section \ref{sec:optics}).
}
\label{fig:syst unc}
\end{center}
\end{figure*}
