\section{Differential cross-section}

\todo{As for 90m:} 
The analysis method is very similar to the previously published ones \cite{prl111,epl101-el}. Each diagonal is analysed
separately, some steps are performed independently for each bunch. In this analysis, a different normalisation
approach is used, consequently making all $t$-independent scaling factors (e.g.~inefficiency corrections)
irrelevant.

%----------------------------------------------------------------------------------------------------
\subsection{Event reconstruction}

guide for reconstruction formulae = robustness against error sources: beam divergence, sensor pitch, misalignment, vertex term neglected, optics imperfections

One arm -- for cuts
\begin{equation}
\label{eq:kin 1a}
	\begin{aligned}
		\theta_x^* &= {v_x^{\rm N} x^{\rm F} - v_x^{\rm F} x^{\rm N}\over v_x^{\rm N} L_x^{\rm F} - v_x^{\rm F} L_x^{\rm N}}\ ,\qquad
		\theta_y^* = {1\over 2} \left( {y^{\rm N}\over L_y^{\rm N}} + {y^{\rm F}\over L_y^{\rm F}} \right)\ ,\\
		x^* &= {L_x^{\rm N} x^{\rm F} - L_x^{\rm F} x^{\rm N}\over L_x^{\rm N} v_x^{\rm F} - L_x^{\rm F} v_x^{\rm N}}\ , \\
	\end{aligned}
\end{equation}

two arm -- better resolution for already selected events
\begin{equation}
\label{eq:kin 2a}
	\begin{aligned}
		\theta_x^* &= {
				\sum v_i^2 \sum L_i x_i - \sum L_i v_i \sum v_i x_i
				\over
				\sum v_i^2 \sum L_i^2 - \sum L_i v_i \sum L_i v_i
			}\ ,\\
		\theta_y^* &= (\theta_y^{*L} + \theta_y^{*R})/2\ .\\
	\end{aligned}
\end{equation}

%------------------------------

{\bf Alignment}. The standard three-step procedure \cite{totem-ijmp} has been applied, further extended by steps improving the vertical alignment. They exploit the fact that the elastic scattering with its two anti-collinear protons can rely the alignment in the left and right arm (with an uncertainty of $20\un{\mu m}$). Furthermore, the horizontal RPs in right arm showed a hit distribution usable for vertical alignment in addition to the other technique based on the vertical RPs. Together, the two methods reduce the uncertainty to about $70\un{\mu m}$. The horizontal alignment and rotation about the beam have been determined with the standard approach, with uncertainties $30\un{\mu m}$ and $2\un{m rad}$, respectively. Propagating the uncertainties to the reconstructed angles, Eq.~(\ref{eq:kin 2a}), gives $0.28\un{\mu rad}$ ($0.19\un{\mu rad}$) for horizontal (vertical) angle. The error of the RP rotations would cause admixture of $\theta_y^*$ to $\theta_x^*$ with proportionality constant having standard deviation $0.005$.

% the observed hit inefficiencies -- possible asymmetries -- discuss ???
% mention final alignment check = 2D Gaussian fit of $\theta_x^*$ vs.~$\theta_y^*$ from both diagonals ??

%------------------------------
{\bf Optics}. The optics matching method \cite{totem-optics} has been applied. The residual uncertainty has a form of factors scaling the reconstructed scattering angles:
% double-arm; 
$0.34\un{\%}$ (horizontal) and $0.25\un{\%}$ (vertical), including the effects of magnet harmonics.



%------------------------------
{\bf Resolution}


%----------------------------------------------------------------------------------------------------
\subsection{Differential cross-section}

\todo{As for 90m:} 
For a given $t$ bin, the value of differential cross-section is evaluated by selecting and counting elastic events as follows
\begin{equation}
{\d\sigma\over \d t}(\hbox{bin}) =
	{\cal N}\, {\cal U}({\rm bin})\, {\cal B}\ 
	{\sum\limits_{t \in \hbox{bin}} {\cal A}(\theta_x^*, \theta_y^*)\, {\cal E}(\theta_y^*)\over \Delta t}\ ,
\end{equation}
where $\Delta t$ is the width of the bin, ${\cal N}$ is a normalisation factor and the other symbols stand for various correction factors:
 ${\cal U}$ for unfolding, ${\cal B}$ for background subtraction, ${\cal A}$ for acceptance correction and ${\cal E}$ for detection and reconstruction efficiency.


{\bf Tagging}. The cuts used to select the elastic events are summarized in Tab.~\ref{tab:cuts}, all are applied at $4\sigma$-level. Cuts 1 and 2 require the reconstructed-track collinearity between the left and right arm. Cut 3 ensures that the protons come from the same vertex (horizontally). Thanks to the very good resolution, the collinearity cuts are very strong and consequently the other possible cuts (cf.~\cite{epl101-el}) bring no significant improvement. The tagging efficiency is studied by applying the cuts also at $5\un{\sigma}$-level. This selection yields $0.3\un{\%}$ more events, almost uniformly in $|t|$ (at very low $|t|$ in increases to $\approx 1\un{\%}$ -- anyway well in systematics). This kind of inefficiency is irrelevant for this analysis.

\begin{table}
\caption{The elastic selection cuts. The superscripts R and L refer to the right and left arm. The $\alpha \theta_x^*$ term in cut 3 is intended to absorb possible effects of residual optics imperfections. The right-most column gives a typical RMS of the cut distribution.
}
\label{tab:cuts}
\begin{center}
\vskip-3mm
\begin{tabular}{ccc}\hline\hline
number & cut & RMS ($\equiv 1\sigma$)\cr\hline
1 & $\theta_x^{*\rm R} - \theta_x^{*\rm L}$				& $3.9\un{\mu rad}$	\cr
2 & $\theta_y^{*\rm R} - \theta_y^{*\rm L}$				& $1.0\un{\mu rad}$	\cr
3 & $x^{*\rm R} - x^{*\rm L} - \alpha \theta_x^*$		& $250\un{\mu m}$ 	\cr\hline\hline
\end{tabular}
\end{center}
\end{table}


%-------------------------
{\bf Background}.
% background = impurity of the cuts above
Method: plot a cut quantity ($\Delta \theta_{x,y}^*$) under various cut combinations. Separation of signal and background: central peak (signal) stays, while tails (background) drop. Residual after all cuts $\rightarrow$ interpolate to signal region $\rightarrow$ background negligible ($B /(B+S) < 0.1\un{\%}$, for both x and y cuts). Further test: $|t|$-distributions with one cut released -- additional events (background, $\approx 0.3\un{\%}$) distributed almost uniformly over $|t|$ (no peaking), except in very low $|t|$ -- small increase to $\approx 1.5\un{\%}$ (anyway, well in within systematics).


%-------------------------

{\bf Acceptance correction}. Beyond the usual acceptance limitations -- sensor edge (relevant for low $|\theta^*_y|$) and LHC apertures (high $|\theta_y^*|$) -- two additional ones have been identified. Both stem from the horizontal RPs interfering with the elastic protons, resulting in uncertain detection efficiency. Consequently, an additional acceptance restriction $-50 < \theta_x^* < 80\un{\mu rad}$ has been adopted to avoid the interference regions. Otherwise, the standard correction technique \cite{8tev-90m} has been applied, comprising a geometrical correction (exploiting the verified azimuthal symmetry) and a correction for smearing around the acceptance limits.


%-------------------------

{\bf Efficiency corrections} include corrections for inefficiencies from various sources: trigger inefficiency ${\cal I}_{\rm trig}$, reconstruction inefficiency ${\cal I}_{\rm det}$ and pile-up inefficiency ${\cal I}_{\rm PU}$ (RPs unable to resolve multiple tracks):

\begin{equation}
\label{efficiency}
	\begin{aligned}
		{\cal E}(\theta_y^*) &= {1\over 1 - {\cal I}_{\rm trig}} {1\over 1 - {\cal I}_{\rm det}(\theta_y^*)} {1\over 1 - {\cal I}_{\rm PU}}\ ,\\
		{\cal I}_{\rm det}(\theta_y^*) &= \sum\limits_{i\in \rm RPs} {\cal I}^i_{3/4}(\theta_y^*) + 2 {\cal I}_{2/4}\\
	\end{aligned}
\end{equation}

\> ``standard procedure'' for the standard contributions (ref. to previous publications?): ``3-out-of-4'' (uncorrelated 1-RP inefficiencies), ``shower in near'' (near-far correlated) and ``pile-up'' (coincidence with beam-halo or any other particle)

\> 3-out-of-4 results
\>> right arm: typical results (near $\approx 98\un{\%}$, far $\approx 96.5\un{\%}$ efficiency)
\>> left arm: efficiency in far RP unexpectedly low ($\approx 90\un{\%}$) -- due to showers in horizontal RPs (horizontal in left arm closer that in the right one) -- but experimentally determinable and thus fully correctable

\> pile-up results shown in Fig.~\ref{fig:overview}
\>> strong time-dependece: linear rise within the data-taking periods, decrease in beam cleanings


%-------------------------

{\bf Unsmearing} The standard per-bin technique has been applied \cite{8tev-90m}. Due to the very small beam divergence, the effect is negligible for all bins except the very low-$|t|$ ones where the rapid cross-section growth appears because of the Coulomb interaction (correction $< 2.5\un{\%}$). For the uncertainty estimate, the uncertainties of $\theta_x^*$ and $\theta_y^*$ resolutions as well as fit-model dependence have been considered, each contribution giving few per-mille for the lowest-$|t|$ point.

\iffalse
\> but time-dependent smearing sigma, determined from the variation of $\theta_y^{*R} - \theta_y^{*L}$
\>> mention the subtlety with $\sigma(\theta_x^*)$ ?? Cannot be measured directly. The only handle comes from the emittances (crude only). But (at least for low-$|t|$), the impact of $\theta_x^*$ smearing is small -- the value of $\theta^*$ is mostly made by $\theta_y^*$, therefore $\theta_x^*$ must be very small. Consequently, $\Delta t_x = 2 \theta_x^* \Delta \theta_x^* \approx 0$
\fi

%-------------------------
{\bf Normalisation}. Same integral between $|t| = 0.014$ and $0.2030\un{GeV^2}$ as DS1 from \cite{prl111}, where luminosity-independent calibration was applied. Leading uncertainty: $4.2\un{\%}$ from luminosity-independent method, transfer negligible ($0.5\un{\%}$).
% stat: 0.3%, syst: 0.3% (DS2 at 90m), 0.3% (1km)


%-------------------------

The {\bf binning} has been optimised with respect to two aspects. At low $|t|$, the bin size has been set to the triple of the resolution in $t$. At higher $|t|$, a fixed statistical fluctuation of $4\un{\%}$ has been targeted.

%-------------------------
{\bf Systematic uncertainties}

\> ``standard procedure'' (ref. to previous publications?) of uncertainty assessment

\> leading uncertainties: residual misalignment (very low-$|t|$), normalisation (flat)

\> compatible results obtained if data right after or right before beam cleaning are used



%----------------------------------------------------------------------------------------------------
\subsection{Final data merging}

\> merging datasets + diagonals

\> central values

\> statistical uncertainties

\> systematic uncertainties


\input data_table.tex


\begin{figure*}
\begin{center}
\includegraphics[width=18cm]{fig/t_dist_tabulation.pdf}
\vskip-3mm
\caption{Differential cross-section. \todo{shall we keep all 3?} \todo{describe!}}
\label{fig:dsdt}
\end{center}
\end{figure*}

\> present/comment Fig.~\ref{fig:dsdt}


