The TOTEM experiment at the CERN LHC has measured elastic proton-proton 
scattering at the centre-of-mass energy 
$\sqrt{s}=8\,$TeV and squared four-momentum transfers $|t|$ from $6\times10^{-4}\,\rm GeV^{2}$ to 0.2\,GeV$^{2}$.
Near the lower end of the $t$-interval, for
$|t| \lesssim 2\times10^{-3}\,\rm GeV^{2}$, the differential cross-section is 
sensitive to the 
interference between the hadronic and the electromagnetic scattering amplitudes.
This article presents the measured cross-section and the constraints it 
imposes on the functional forms of the interference term and of modulus and
phase of the hadronic elastic amplitude. The data exclude the traditionally 
applied Simplified West and Yennie interference formula requiring a constant 
phase and a purely exponential modulus of the hadronic amplitude. 
For non-exponential parametrisations of the hadronic 
modulus, on the other hand, the data are compatible with hadronic phase 
functions producing either centrally or peripherally dominated impact 
parameter distributions in elastic scattering events. In both cases, 
the $\rho$-parameter, i.e. the arctangent of the hadronic phase at $t = 0$,
was found to be $0.12 \pm 0.03$. The results for the total hadronic 
cross-section are $\sigma_{\rm tot} = (102.9 \pm 2.3)$\,mb and 
$(103.0 \pm 2.3)$\,mb for central and peripheral phase formulations, 
respectively. Both are consistent with previous TOTEM measurements.
