\section{Introduction}
%
Elastic scattering of protons is a process mediated by the strong and the electromagnetic interactions -- the weak interaction can be neglected since its carriers are heavy compared to the small momentum transfers, $|t|$, typical for elastic scattering. In this context, the strong interaction is traditionally called nuclear and the electromagnetic called Coulomb. In quantum-theory description, each of the interactions is described by a complex scattering amplitude, nuclear ${\cal A}^{\rm N}$ and Coulomb ${\cal A}^{\rm C}$. Moreover, the superposition 
of the two amplitudes generates an interference term in the combined interaction cross-section. 

The Coulomb amplitude can be calculated from QED (e.g.~Section 3.2 in \cite{block06}), using empirical electric and magnetic form factors of the proton (e.g.~\cite{puckett10}). It can be shown (e.g.~Section 1.3.1 in~\cite{jan_thesis}) that, at low $|t|$, the effect of both form factors can be described by a single function ${\cal F}$. 

Since the Coulomb amplitude is known, the Coulomb-nuclear interference (CNI) exposes the phase of the nuclear amplitude, which is necessary for a complete understanding of the interaction but not directly observable in the pure hadronic differential cross-section. The interference effect is most pronounced in the $t$-region where the two amplitudes have similar magnitudes, i.e. -- for typical LHC centre-of-mass energies of a few TeV -- near $t \sim 5 \times 10^{-4}\,\rm GeV^{2}$. Thus the experimental sensitivity to 
the nuclear phase, which for general nuclear amplitudes is a function of $t$, 
is limited to a region at very small $|t|$, making any conclusions on the 
functional form of the phase very difficult.
In the analyses of past experiments (see e.g.~\cite{} from ISR onwards) 
typically a strongly simplified interference formula was used. This so-called
Simplified West-Yennie (SWY) formula~\cite{wy68} is based on restrictive assumptions on the 
hadronic amplitude, implying in particular a purely exponential modulus and a 
constant phase for all $t$ (see the discussion in 
Section~\ref{sec:cni_framework}).
As result, traditionally the phase value $\arg {\cal A}^{\rm N} (t=0)$, or
equivalently, $\rho \equiv \cot \, \arg {\cal A}^{\rm N} (0) = \frac{{\cal R A}^{\rm N} (0)}{{\cal I A}^{\rm N} (0)}$ was quoted. 
Most of the interest in $\rho$ lies in its predictive power on total cross-sections at higher centre-of-mass energies via dispersion 
relations~\cite{dremin-dispersion}. 

The present article discusses the first measurement of elastic scattering in the
CNI region at the CERN LHC by the TOTEM experiment. 
The data have been collected at $\sqrt{s} = 8\,$TeV with a special beam optics 
($\beta^{*}=1000\,$m, see Section~\ref{sec:datataking}) and cover a $|t|$-interval
from $6\times10^{-4}\,\rm GeV^{2}$ to 0.2\,GeV$^{2}$, thus overlapping comfortably
with another data set taken at the same energy but with different beam optics
($\beta^{*}=90\,$m; 0.027\,GeV$^{2} < |t| < 0.2\,$GeV$^{2}$). The analysis of the
latter~\cite{8tev-90m} excludes a purely exponential hadronic elastic scattering
amplitude with more than $7\,\sigma$ confidence and thereby also excludes
one of the conditions for the applicability of the SWY formula.
Therefore the more general and complex Kundrat-Lokajicek (KL) interference 
formula~\cite{kl94} is applied in the present study, which offers much more freedom for the
choice of the theoretically unknown functional forms of the hadronic modulus 
and phase. Since the data cannot unambiguously determine all functional forms and their parameters, the results of this study, still representatively 
expressed in terms of $\rho$, become conditional to the choice of amplitude models. Finally, each of these choices will be interpreted in impact parameter space.

In order to optimally exploit the information of all available data, the two 
data sets are analysed in a combined way, providing complementary constraints:
The $\beta^{*}=90\,$m set with very high statistics at higher $|t|$ fixes the 
hadronic modulus, whereas the $\beta^{*}=1000\,$m set with its reach to very low
$|t|$ determines phase parameters.




%\todo{why phase interesting?} \todo{introduce: coulomb-interference formula}
%\> motivation for low-$|t|$ measurement
%\>> why it is important and interesting
%
%\> historical context, previous measurements
%
%\> explain structure: first 1000m measurement, then CNI analysis (where also 90m data will be used)
