\section{Summary and Outlook}

\> differential cross-section measurement summary
\>> first time at LHC: $|t|$ down to the Coulomb-interference region
\>>> RPs extremely close to beam (unprecedented)
\>>> novel collimation procedure


\> CNI analysis summary
\>> $N_b=1$, constant phase: excluded
\>>> SWY formula excluded, too
\>> $N_b=1$, peripheral phase: can fit data, but doubtful from other perspectives
\>> $N_b=3$, both constant and peripheral phase: good description of data $\Rightarrow$ centrality is not a necessity
\>> summary on presented $\rho$ values?

\> $\rho$
\>> at LHC: for the first time extracted via Coulomb-interference
\>> results from Section~\ref{sec:fit exp3} can be summarised
\begin{equation}
\label{eq:rho final}
\rho = 0.12 \pm 0.03
\end{equation}

\> $\sigma_{\rm tot}$ -- first LHC results where:
\>> rho determined in the same analysis (not from an external source)
\>> explicit separation of Coulomb and nuclear components -- conceptually more accurate than all previous works
\>> summary on presented $\sigma_{\rm tot}$ values?
\>>> extremely stable
\>>> little greater than if Coulomb neglected (expected as long as $\rho > 0$)

\> prospects for future
\>> how could things be done better: requirements from theory/experiment
\>>> need better distinction between the hadronic and CNI components
\>>> either from theory: narrow the range of possible parametrisations
\>>> either from experiment: extremely precise data (more statistics, lower $|t|$, ...)
\>> TOTEM plans, measurements foreseen
\>>> extremely high beta* (2500 m), what $|t|$ reach (also as a function of $s$)
\>>> wish: more statistics -- better discrimination power (N vs.~CNI component)
\>>> wish: lower $|t|$ reach -- helps in several issues in one go. If ``pure'' Coulomb measured, normalisation constrained -- uncertainty of si tot reduced. The rising edge of acceptance shifted, thus higher statistics in the CNI region. Similarly, the (leading) impact of the vertical-alignment uncertainty suppressed.
\>> other possible experiments: n+n, n+p scattering -- Coulomb reduced, better hint on the nuclear component
