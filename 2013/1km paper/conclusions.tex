\section{Summary and Outlook}
\label{sec:summary}
For the first time at LHC the differential cross-section of elastic proton-proton scattering has been measured at $|t|$-values down to the Coulomb-nuclear interference (CNI) region. This was made possible by a special beam optics, a novel collimation procedure and by moving the RPs to an unprecedented distance of only $3\,\sigma$ from the centre of the circulating beam.

To fit $\d\sigma/\d t$ in the CNI region, two interference formulae -- Simplified West and Yennie (SWY) and Kundr\' at-Lokaj\' i\v cek (KL) -- 
were explored in conjunction with different mathematical descriptions of the modulus and phase of the nuclear amplitude as a function of $t$. The nuclear modulus was parametrised as an exponential function with a polynomial of degree $N_b=1$ or $3$ in the exponent. These two alternatives allowed to test whether the nuclear modulus can be purely exponential or more flexibility is required. For the phase two options were considered, leading to different impact-parameter distributions of elastic scattering events: a constant phase implying a central behaviour, and another description favouring peripheral collisions. The following conclusions can be drawn.
\begin{itemize}
\item Purely exponential nuclear modulus ($N_b=1$), constant phase: excluded with more than $7\un{\sigma}$ confidence. Since this is the only combination compatible with the SWY formula, the data exclude the usage of the formula.
\item Purely exponential nuclear modulus ($N_b=1$), peripheral phase: the data do not exclude this option which, however, is disfavoured from other perspectives.
\item Non-exponential nuclear modulus ($N_b=3$): both constant and peripheral phases are well compatible with the data, therefore the central impact-parameter picture prevalent in phenomenological descriptions is not a necessity.
\end{itemize}

The $\rho$ parameter was for the first time at LHC extracted via the Coulomb-nuclear interference. In the preferred fits ($N_b=3$):
\begin{equation}
\label{eq:rho final}
\rho = 0.12 \pm 0.03\ .
\end{equation}

The new total cross-section determination is conceptually more accurate than in all previous LHC publications since the CNI effects are explicitly treated. Moreover, the value of $\rho$ comes from the same analysis, not from an external source, which underlines consistency. The $\sigma_{\rm tot}$ values are very well consistent among all non-excluded fits and compatible with the previous measurements. As expected, the new determination yields slightly greater values relative to previous results where the negative CNI was not taken into account. Also note that if the SWY formula with purely exponential hadronic modulus are used, the total cross-section is underestimated by about $1\un{mb}$.

It may be seen from the results in Section \ref{sec:coulomb} that the physical interpretation of elastic collisions depends on various additional assumptions in mathematical expression of corresponding physics models. It concerns both nuclear and Coulomb fit components. We would like to draw the reader's attention to at least one issue. The standard QED expression for the Coulomb cross-section, Eq.~(\ref{eq:coul cs}), tends to infinity for $t\to 0$ which can be seen as a consequence of large impact parameter values assumed in the initial state. However, in any experiment only small impact parameter values -- e.g. given by the beam dimensions -- are responsible for the measured data. Greater theoretical attention should be devoted to the precise description of the experimental initial states and to the modelling of the fundamental interactions.

For even stronger results in the future the key point is a better distinction between the nuclear and CNI cross-section components, which can be achieved from both theoretical and experimental sides. New theory developments may narrow down the range of allowed parametrisations of the nuclear modulus and phase or better constrain the induced CNI effects. The experimental improvements include increasing statistics and reducing the lower $|t|$ threshold. For the former, TOTEM has already upgraded the RP mechanics such that both vertical pots can be simultaneously placed very close to the beam. For the latter, TOTEM foresees an optics with extremely high $\beta^* \approx 2500\un{m}$ which would allow to reach the CNI region even at Run II energies. Moreover, recent experience with the $\beta^* = 90\un{m}$ optics at $\sqrt s = 13\un{TeV}$ shows that very low beam emittances can be achieved, thus possibly further reducing the RP distance from the beam.
