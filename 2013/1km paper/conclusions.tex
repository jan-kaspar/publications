\section{Summary and Outlook}
\label{sec:summary}
For the first time at LHC the differential cross-section of elastic proton-proton scattering has been measured at $|t|$-values down to the Coulomb-nuclear interference (CNI) region. This was made possible by a special beam optics, a novel collimation procedure and by moving the RPs to an unprecedented distance of only $3\,\sigma$ from the beam centre.

To fit $\d\sigma/\d t$ in the CNI region, two interference formulae -- Simplified West and Yennie (SWY) and Kundr\' at-Lokaj\' i\v cek (KL) -- 
were explored in conjunction with different mathematical descriptions of the modulus and phase of the nuclear amplitude as a function of $t$. The nuclear modulus was parametrised as an exponential function with a polynomial slope of degree 0 or 2 ($N_b=1$ or $3$ parameters). These two alternatives allowed to test whether the nuclear modulus can be purely exponential or more flexibility is required. For the phase two options were considered, leading to different impact-parameter distributions of elastic scattering events: a constant phase implying a central behaviour, and a description favouring peripheral collisions. The following conclusions can be drawn.
\begin{itemize}
\item Purely exponential nuclear modulus ($N_b=1$), constant phase: excluded with more than $7\un{\sigma}$ confidence. Since this is the only combination compatible with the SWY formula, its application is excluded, too.
\item Purely exponential nuclear modulus ($N_b=1$), peripheral phase: the data do not exclude this option which, however, may be disfavoured from other perspectives.
\item Non-exponential nuclear modulus ($N_b=3$): both constant and peripheral phases are well compatible with data, therefore the central impact-parameter picture is not a necessity.
\end{itemize}

The $\rho$ parameter was for the first time at LHC extracted via the Coulomb-nuclear interference. The preferred fits ($N_b=3$) can be summarised by
\begin{equation}
\label{eq:rho final}
\rho = 0.12 \pm 0.03\ .
\end{equation}

The new total cross-section determination is conceptually more accurate than in all previous LHC publications since the CNI effects are explicitly treated. Moreover, the value of $\rho$ comes from the same analysis, not an external source, which underlines consistency. The $\sigma_{\rm tot}$ values are very well consistent among all non-excluded fits and compatible with the previous measurements. As expected, the new determination yields slightly greater values relative to previous results where the negative CNI was not taken into account.

For even stronger results in future the key point is a better distinction between the nuclear and CNI cross-section components, which can be achieved from both theoretical and experimental sides. New theory developments may narrow down the range of allowed parametrisations of the nuclear modulus and phase or better constrain the induced CNI effects. The experimental improvements include increasing statistics and reducing the lower $|t|$ threshold. For the former, TOTEM has already upgraded the RP mechanics such that both vertical pots can be simultaneously placed very close to the beam. For the latter, TOTEM foresees an optics with extremely high $\beta^* \approx 2500\un{m}$ which would allow to reach the CNI region even at Run II energies. Moreover, recent experience with $\beta^* = 90\un{m}$ optics at $\sqrt s = 13\un{TeV}$ shows that very low beam emittances can be achieved, thus possibly reducing the RP distance from the beam.
