\section{Summary and Outlook}
\label{sec:summary}
For the first time at LHC the differential cross-section of elastic proton-proton scattering has been measured at $|t|$-values down to the Coulomb-nuclear interference (CNI) region. This was made possible by a special beam optics, a novel collimation procedure and by moving the RPs to an unprecedented distance of only 
$3\,\sigma$ from the beam centre.

%\> differential cross-section measurement summary
%\>> first time at LHC: $|t|$ down to the Coulomb-interference region
%\>>> RPs extremely close to beam (unprecedented)
%\>>> novel collimation procedure
To fit $\d\sigma/\d t$ in the CNI region, two interference formulae --
Simplified West and Yennie (SWY), and Kundr\' at-Lokaj\' i\v cek (KL) -- 
were explored in 
conjunction with different mathematical descriptions of the modulus and phase 
of the hadronic amplitude as a function of $t$. The hadronic modulus was parameterised as an exponential function with a polynomial slope of degree 0 to 2 ($N_b=1$ to 3 parameters).
For the phase two options were considered, leading to different impact 
parameter distributions of elastic scattering events: a constant phase implying a preferentially central behaviour, and a description favouring peripheral collisions.
The following conclusions can be drawn:
\begin{itemize}
\item A constant phase : excluded ...

And so on ... bla ... bla 

SWY formula excluded, too
\item $N_b=1$, peripheral phase: can fit data, but doubtful from other perspectives
\item $N_b=3$, both constant and peripheral phase: good description of data $\Rightarrow$ centrality is not a necessity
\end{itemize}

\> $\rho$
\>> at LHC: for the first time extracted via Coulomb-interference
\>> results from Section~\ref{sec:fit exp3} can be summarised
\begin{equation}
\label{eq:rho final}
\rho = 0.12 \pm 0.03
\end{equation}

\> $\sigma_{\rm tot}$ -- first LHC results where:
\>> rho determined in the same analysis (not from an external source)
\>> explicit separation of Coulomb and nuclear components -- conceptually more accurate than all previous works
\>> summary on presented $\sigma_{\rm tot}$ values?
\>>> extremely stable
\>>> little greater than if Coulomb neglected (expected as long as $\rho > 0$)

\> prospects for future
\>> how could things be done better: requirements from theory/experiment
\>>> need better distinction between the hadronic and CNI components
\>>> either from theory: narrow the range of possible parametrisations
\>>> either from experiment: extremely precise data (more statistics, lower $|t|$, ...)
\>> TOTEM plans, measurements foreseen
\>>> extremely high beta* (2500 m), what $|t|$ reach (also as a function of $s$)
\>>> wish: more statistics -- better discrimination power (N vs.~CNI component)
\>>> wish: lower $|t|$ reach -- helps in several issues in one go. If ``pure'' Coulomb measured, normalisation constrained -- uncertainty of si tot reduced. The rising edge of acceptance shifted, thus higher statistics in the CNI region. Similarly, the (leading) impact of the vertical-alignment uncertainty suppressed.
\>> other possible experiments: n+n, n+p scattering -- Coulomb reduced, better hint on the nuclear component
