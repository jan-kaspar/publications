\section{Summary and Outlook}

\> cross-section measurement summary
\>> first time at LHC: t down to the Coulomb-interference region
\>>> RPs extremely close to beam (unprecedented)
\>>> novel collimation procedure


\> CNI analysis summary
\>> generally summarise the feasibility of the task
\>> summarise the examples: numbers, uncertainties
\>>> dependence on assumptions: non-trivial (rho, B), limited (si tot)
\>>> any conclusion?
\>> emphasize -- first LHC results where
\>>> rho determined in the same analysis (not from an external source)
\>>> explicit separation of Coulomb and nuclear components -- conceptually more accurate than all previous works


\> prospects for future
\>> how could things be done better: requirements from theory/experiment
\>>> need better distinction between the hadronic and CNI components
\>>> either from theory: narrow the range of possible parametrisations
\>>> either from experiment: extremely precise data (more statistics, lower $|t|$, ...)
\>> TOTEM plans, measurements foreseen
\>>> extremely high beta* (2500 m), what t reach (also as a function of s)
\>> other possible experiments: n+n, n+p scattering -- Coulomb reduced, better hint on the nuclear component
