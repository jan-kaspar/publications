\documentclass[3p,onecolumn,12pt,times,longtitle]{elsarticle}

\usepackage{amsmath}
\usepackage{amssymb}
%\usepackage{graphicx}
\usepackage{color}
%\usepackage{mathptmx}
\usepackage{url}
\usepackage{fleqn}
\usepackage{enumitem}

% TODO: only for draft
\usepackage{lineno}

%----------------------------------------------------------------------------

\def\d{{\rm d}}
\def\un#1{\,{\rm #1}}
\def\ung#1{\quad[{\rm #1}]}
\def\unt#1{[{\rm #1}]}
\def\e{{\rm e}}
\def\I{{\rm i}}
\def\T{{\rm T}}
\def\vec#1{\mathbf{#1}}
\def\mat#1{\mathsf{#1}}
\def\etal{et al.}
\def\todo#1{{\color{red}TODO: #1}}
\def\TODO#1{{\color{red}TODO: #1}}

\setbox123\hbox{$0$}
\setbox124\hbox{$.$}
\def\S{\hbox to\wd123{\hss}}
\def\.{\hbox to\wd124{\hss}}

\def\Name#1{#1, }
\def\Review#1#2#3#4{{\it #1} {\bf #2} (#3) #4}

%-----------------------------------------------------------------------------

\journal{Nuclear Physics B}

\begin{document}

\begin{frontmatter}

\title{Measurement of Elastic pp Scattering at $\sqrt{\hbox{s}} = \hbox{8}$\,TeV in the Coulomb-Nuclear Interference Region -- Determination of the $\mathbf{\rho}$-Parameter and the Total Cross-Section}

\input authorlist_npb

\date{\today}

\begin{abstract}
\input abstract
\end{abstract}

\begin{keyword}
proton-proton interactions \sep elastic scattering \sep Coulomb-Nuclear Interference \sep total cross-section \sep rho parameter \sep TOTEM \sep LHC
\PACS 13.85.Dz % Elastic scattering
\PACS 13.85.Lg % Total cross sections
\PACS 13.40.Ks % Electromagnetic corrections to strong- and weak-interaction processes
\end{keyword}
\end{frontmatter}

%--------------------------------------------------

% TODO: only for draft
\linenumbers

\input introduction.tex

\input experimental_apparatus.tex

\input beam_optics.tex

\input data_taking.tex

\input differential_cross_section.tex

\input coulomb.tex

\input conclusions.tex

\section*{Acknowledgements}
\input acknowledgements.tex

%--------------------------------------------------

\begin{thebibliography}{99}
\input bibliography.tex
\end{thebibliography}


\end{document}
