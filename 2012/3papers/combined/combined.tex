\documentclass[doublecol]{../macros/epl2} 
% or \documentclass[page-classic]{epl2} for one column style

\title{Luminosity independent measurements of total, elastic and inelastic cross-sections at $\sqrt s = 7\,\rm TeV$}
%\shorttitle{Title} %Insert here a short version of the title if it exceeds 70 characters

\author{
The TOTEM Collaboration -- !! PRELIMINARY !!\\
G.~Antchev\thanks{INRNE-BAS, Institute for Nuclear Research and Nuclear Energy, Bulgarian Academy of Sciences, Sofia, Bulgaria.}\addtocounter{footnote}{-1}
%\addtocounter{footnote}
\and P.~Aspell\inst{8}
\and I.~Atanassov\inst{8}\hspace{-0.15cm}\footnotemark
\and V.~Avati\inst{8}
\and J.~Baechler\inst{8}
\and V.~Berardi\inst{5b,5a}
\and M.~Berretti\inst{7b}
\and E.~Bossini\inst{7b}
\and M.~Bozzo\inst{6b,6a}
\and P.~Brogi\inst{7b}
\and E.~Br\"{u}cken\inst{3a,3b}
\and A.~Buzzo\inst{6a}
\and F.~S.~Cafagna\inst{5a}
\and M.~Calicchio\inst{5b,5a}
\and M.~G.~Catanesi\inst{5a}
\and C.~Covault\inst{9}
\and M.~Csan\'{a}d\inst{4}
\and T.~Cs\"{o}rg\H{o}\inst{4}
\and M.~Deile\inst{8}
 \and K.~Eggert\inst{9}
 \and V.~Eremin\thanks{Ioffe Physical - Technical Institute of Russian Academy of Sciences.}
 \and R.~Ferretti\inst{6a,6b}
 \and F.~Ferro\inst{6a}
 \and A. Fiergolski\thanks{Warsaw University of Technology, Poland.}
 \and F.~Garcia\inst{3a}
 \and S.~Giani\inst{8}
 \and V.~Greco\inst{7b,8}
 \and L.~Grzanka\inst{8}\hspace{-0.15cm}\thanks{Institute of Nuclear Physics, Polish Academy of Science, Cracow, Poland.}\addtocounter{footnote}{-2}
 \and J.~Heino\inst{3a}
 \and T.~Hilden\inst{3a,3b}
 \and M.~R.~Intonti\inst{5a}
%\and M.~Janda\inst{1b}
 \and J.~Ka\v{s}par\inst{1a,8}
 \and J.~Kopal\inst{1a,8}
 \and V.~Kundr\'{a}t\inst{1a}
 \and K.~Kurvinen\inst{3a}
 \and S.~Lami\inst{7a}
 \and G.~Latino\inst{7b}
 \and R.~Lauhakangas\inst{3a}
 \and  T.~Leszko\footnotemark
 %\thanks{Warsaw University of Technology, Poland}
 \and E.~Lippmaa\inst{2}
 \and M.~Lokaj\'{\i}\v{c}ek\inst{1a}
 \and M.~Lo~Vetere\inst{6b,6a}
 \and F.~Lucas~Rodr\'{i}guez\inst{8}
 \and M.~Macr\'{\i}\inst{6a}
 \and L.~Magaletti\inst{5b,5a}
%\and G.~Magazz\`{u}\inst{7a}
 \and T.~M\"aki\inst{3a}
 \and A.~Mercadante\inst{5b,5a}
%\and M.~Meucci\inst7b
 \and N.~Minafra\inst{8} 
 \and S.~Minutoli\inst{6a}\addtocounter{footnote}{1}
 \and F.~Nemes\inst{4}\hspace{-0.15cm}\thanks{Department of Atomic Physics, ELTE University, Hungary.}
 \and H.~Niewiadomski\inst{8}
%\and E.~Noschis\inst{8}
%\and T.~Nov\'{a}k\inst{4}\thanks{KRF,  Gy\"{o}ngy\"{o}s, Hungary}
 \and E.~Oliveri\inst{7b}
 \and F.~Oljemark\inst{3a,3b}
 \and R.~Orava\inst{3a,3b}
 \and M.~Oriunno\inst{8}\hspace{-0.15cm}\thanks{SLAC National Accelerator Laboratory, Stanford CA, USA.}
 \and K.~\"{O}sterberg\inst{3a,3b}
 \and P.~Palazzi\inst{7b}
 \and J.~Proch\'{a}zka\inst{1a}
 \and M.~Quinto\inst{5a}
 \and E.~Radermacher\inst{8}
 \and E.~Radicioni\inst{5a}
 \and F.~Ravotti\inst{8}
 \and E.~Robutti\inst{6a}
 \and L.~Ropelewski\inst{8}
 \and G.~Ruggiero\inst{8}
 \and H.~Saarikko\inst{3a,3b}
% \and G.~Sanguinetti\inst{7a}
 \and A.~Santroni\inst{6b,6a}
 \and A.~Scribano\inst{7b}
 \and W.~Snoeys\inst{8}
 \and J.~Sziklai\inst{4}
 \and C.~Taylor\inst{9}
 \and N.~Turini\inst{7b}
 \and V.~Vacek\inst{1b}
 \and M.~Vitek\inst{1b}
 \and J.~Welti\inst{3a,3b}
 \and J.~Whitmore\inst{10}
 }          %ends author list
\shortauthor{The TOTEM Collaboration (G.~Antchev \etal)}
%\vspace{0.5cm}
\institute{
\inst{1a}{Institute of Physics of the Academy of Sciences of the Czech Republic, Praha, Czech Republic.}\\
\inst{1b}{Czech Technical University, Praha, Czech Republic.}\\
\inst{2} {National Institute of Chemical Physics and Biophysics NICPB, Tallinn, Estonia.}\\
\inst{3a}{Helsinki Institute of Physics, Finland.}\\
\inst{3b}{Department of Physics, University of Helsinki, Finland.}\\
\inst{4} {MTA Wigner Research Center, RMKI, Budapest, Hungary.}\\
\inst{5a}{INFN Sezione di Bari, Italy.}\\
\inst{5b}{Dipartimento Interateneo di Fisica di Bari, Italy.}\\
\inst{6a}{Sezione INFN, Genova, Italy.}\\
\inst{6b}{Universit\`{a} degli Studi di Genova, Italy.}\\
\inst{7a}{INFN Sezione di Pisa, Italy.}\\
\inst{7b}{Universit\`{a} degli Studi di Siena and Gruppo Collegato INFN di Siena, Italy.}\\
\inst{8} {CERN, Geneva, Switzerland.}\\
\inst{9} {Case Western Reserve University, Dept. of Physics, Cleveland, OH, USA.}\\
\inst{10}{Penn State University, Dept. of Physics, University Park, PA, USA.}\\
}


\pacs{13.60.Hb}{Total and inclusive cross sections (including deep-inelastic processes)}


\abstract{%
!! THIS IS JUST A PLACEHOLDER !!\\
TOTEM has measured the differential cross-section for elastic proton-proton scattering
at the LHC energy of √
s = 7 TeV analysing data from a short run with dedicated large-β∗ optics.
A single exponential fit with a slope B = (20.1 0.2stat 0.3syst ) GeV−2 describes the range of the
± ±four-momentum transfer squared from 0.02 to 0.33 GeV2. After the extrapolation to = 0,
|t| |t|a total elastic scattering cross-section of (24.8 0.2stat 1.2syst) mb was obtained. Applying
± ±the optical theorem and using the luminosity measurement from CMS, a total proton-proton
cross-section of (98.3 0.2stat 2.8syst) mb was deduced which is in good agreement with the
± ±expectation from the overall fit of previously measured data over a large range of center-of-mass
energies. From the total and elastic pp cross-section measurements, an inelastic pp cross-section
of (73.5 0.6stat +1.8
 syst) mb was inferred.
± −1.3
}


\def\d{{\rm d}}
\def\un#1{\,{\rm #1}}
\def\ung#1{\quad[{\rm #1}]}
\def\unt#1{[{\rm #1}]}
\def\e{{\rm e}}

\setbox123\hbox{$0$}
\setbox124\hbox{$.$}
\def\S{\hbox to\wd123{\hss}}
\def\.{\hbox to\wd124{\hss}}


\begin{document}

\maketitle

%--------------------------------------------------
\section{Introduction}

TODO: explain context

This paper: explanation of how different complementary 

\section{Elastic \& optical point}

\begin{equation}
\label{eq:si tot 1}
\sigma_{\rm tot}^2 = {16\pi\, (\hbar c)^2\over 1 + \rho^2}\, \left. \d\sigma_{\rm el}\over\d t\right|_0
\end{equation}

$\sigma_{\rm tot} = 98.6\un{mb} \pm 2.3\%$, $\sigma_{\rm inel} = 73.2\un{mb} \pm 1.7\%$

\section{Sum of elastic and inelastic}

\begin{equation}
\label{eq:si tot 2}
\sigma_{\rm tot} = \sigma_{\rm el} + \sigma_{\rm inel}
\end{equation}

$\rho$-independent: $\sigma_{\rm tot} = 99.1\un{mb} \pm 4.4\%$

$\rho$ and $\cal L$-independent: $\sigma_{\rm el} / \sigma_{\rm inel} = 0.345 \pm 2.6\%$, $\sigma_{\rm el} / \sigma_{\rm tot} = 0.257 \pm 2.0\%$



\section{Lumi independent total, el. and inelast. cross-sections}

\begin{equation}
\label{eq:si tot 3}
\sigma_{\rm tot} = {16\pi\, (\hbar c)^2\over 1 + \rho^2}\, {\d N_{\rm el}/\d t|_0 \over N_{\rm el} + N_{\rm inel} }
\end{equation}

$\sigma_{\rm tot} = 98.0\un{mb} \pm 2.5\%$

Using the $\sigma_{\rm el} / \sigma_{\rm tot}$ ratio from preceding paragraph:
$\sigma_{\rm el} = 25.1\un{mb} \pm 4.3\%$ and
$\sigma_{\rm inel} = 72.9\un{mb} \pm 2.0\%$



%\section{Total cross section combined}
%
%Arithmetic average of $\sigma_{\rm tot}$ from Eqs.~\ref{eq:si tot 1}, \ref{eq:si tot 2} and \ref{eq:si tot 3}:
%$\sigma_{\rm tot} = 98.6\un{mb} \pm 2.3\%$




\section{Luminosity determination}

\subsection{October}

\begin{equation}
\label{eq:lumi oct}
{\cal L} = {1 + \rho^2 \over 16\pi\, (\hbar c)^2}\, { (N_{\rm el} + N_{\rm inel})^2 \over \d N_{\rm el}/\d t|_0 }
\end{equation}


${\cal L} = 83.7 \un{\mu b^{-1}} \pm 3.8\%$

\subsection{June}
Using intercept:
\begin{equation}
\label{eq:lumi jun 1}
{\cal L} =  {16\pi\, (\hbar c)^2\over 1 + \rho^2}\, \left. \d N_{\rm el}\over\d t\right|_0\, {1\over \sigma_{\rm tot}^2}
\end{equation}
taking $\sigma_{\rm tot}$ from Eq.~\ref{eq:si tot 3} yields ${\cal L} = 1.66 \un{\mu b^{-1}} \pm 5.2\%$.

Using elastic cross section:
\begin{equation}
\label{eq:lumi jun 1}
{\cal L} = {N_{\rm el}\over \sigma_{\rm el}}
\end{equation}
taking $\sigma_{\rm tot}$ from below Eq.~\ref{eq:si tot 3} yields ${\cal L} = 1.65 \un{\mu b^{-1}} \pm 4.5\%$.



\section{Rho determination}

From
\begin{equation}
\label{eq:rho}
\rho^2 = 16\pi\ (\hbar c)^2\ {\cal L}\ {\d N_{\rm el}/\d t|_0\over (N_{\rm el} + N_{\rm inel})^2} - 1
\end{equation}
we obtained $\rho^2 = 0.0092 \pm 0.0558$ (mean and standard deviation). This estimate can not be translated into terms of $\rho$ in a straight-forward manner -- an important part of the $\rho^2$ distribution extends to negative values, where square root is not defined. Instead, we calculated that at $95\%$ confidence level $\rho^2 < 0.10$. This upper bound can be expressed as $\rho < 0.32$. Alternatively, one can pursue the Bayes' approach to estimate $|\rho|$. Taking a uniform prior $|\rho|$ distribution yields a posterior distribution with mean $0.145$ and standard deviation $0.091$.



\section{Cosmic ray relevance}



	
\section{Conclusions}

\begin{figure}
\onefigure{fig_el/sigma_el_to_sigma_tot.pdf}
\vskip-5mm
\caption{(Color online) The ratio of the elastic to total cross-section as a function of the scattering energy $\sqrt s$. TODO: fit ??}
\label{fig:sigma rat}
\end{figure}

TODO: $\sigma_{\rm el} / \sigma_{\rm inel}$ -- indicates the ``shape'' of the proton; need interpretation for the ratio growing with $\sqrt s$






%--------------------------------------------------
\acknowledgments
Acknowledgements -- Insert here the text.


%--------------------------------------------------
\begin{thebibliography}{0}

\bibitem{epl96}
    %First measurements of the total proton-proton cross section at the LHC energy of $\sqrt s =7\,\rm TeV$ CERN-PH-EP-2011-158
	\Name{Antchev G.~{\it et al.}~(TOTEM Collaboration)}
	\REVIEW{Europhys.~Lett.}{96}{2011}{21002}

\bibitem{epl95}
    %Proton-proton elastic scattering at the LHC energy of \sqrt{s} = 7 TeV, Europhys. Lett. 95 (2011) 41001,CERN-PH-EP-2011-101 
	\Name{Antchev G.~{\it et al.}~(TOTEM Collaboration)}
	\REVIEW{Europhys.~Lett.}{95}{2011}{41001}

\bibitem{jinst}
    %The TOTEM Experiment at the CERN Large Hadron Collider, JINST 3 S08007, 2008
	\Name{Anelli G.~{\it et al.}~(TOTEM Collaboration)}
	\REVIEW{JINST}{3}{2008}{S08007}

\bibitem{lafferty94}
 	%``Where to stick your data points: The treatment of measurements within wide bins,''
	\Name{Lafferty G.~D.~and Wyatt T.~R.}
	\REVIEW{Nucl.\ Instrum.\ Meth.}{A 355}{1995}{541}

\bibitem{compete} 
	\Name{J.~R.~Cudell {\it et al.} (COMPETE Collaboration)}
	\REVIEW{Phys.\ Rev.\ Lett.}{89}{2002}{201801}

\bibitem{P1} 
	\Name{Antchev G.~{\it et al.}~(TOTEM Collaboration)}
	\REVIEW{Europhys.~Lett.}{TODO}{2012}{TODO}

\bibitem{P2} 
	\Name{Antchev G.~{\it et al.}~(TOTEM Collaboration)}
	\REVIEW{Europhys.~Lett.}{TODO}{2012}{TODO}

\end{thebibliography}

\end{document}
