\input base
\input biblio
\input colors
\input slides



\SetBackground{fig/bg3.jpg}
%\def\FgColor{\cmykBlack}\edef\TitColor{\cmykBlack}

\def\Bref#1{Ref\hbox{. }\bref{#1}}
\def\comment#1{}

\def\fg{\cmyk{\FgColor}}
\def\fgt{\cmyk{\TitColor}}
\def\blue{\rgb{\0.5 0.5 0.9}}
\def\green{\rgb{0.5 0.9 0.5}}
\def\red{\rgb{0.9 0.5 0.5}}
\def\yellow{\rgb{0.9 0.9 0}}

\def\date{June 7, 2010}


\newpage %-------------------------------------------------------------------------------------------
%\centerline{\fg 3rd preliminary version}
\hbox{}\vfil
\title{\bPxx First Results of the TOTEM Roman Pots}
\vskip2\baselineskip
\bgroup
		\obeylines\leftskip0pt plus1fil
		{\bPxv Jan Ka\v spar}
		CERN and Institute of Physics of the AS CR
		on behalf of the {\bf TOTEM Collaboration}
		\vskip2\baselineskip
		Vertex 2010
		Loch Lomond, \date
\egroup

\vskip2\baselineskip

\line{\hss
	\fig[,3cm]{fig/cern_logo_white.pdf}\hss
	\fig[,3cm]{fig/totemLogo_white.pdf}\hss
	\fig[,3cm]{fig/logo-FZU-male_white.pdf}\hss
}
\footline={}

\newpage %-------------------------------------------------------------------------------------------
\title{Outline}

\vfil
\vskip-1cm

\bgroup
\def\tit#1{\vskip1cm{\bPxii #1}}
\obeylines
\leftskip0pt plus 1fil
\tit{The TOTEM experiment}
physics programme
detector requirements
the Roman Pot system

\tit{Edgeless Silicon Detectors}
{\bPxii with Current Terminating Structure}
concept and characteristics
performance studies

\tit{Roman Pot Production Tests}

\tit{First LHC results} 
\egroup

\vfil

\newpage %-------------------------------------------------------------------------------------------
\title{TOTEM Experiment -- Detector Requirements}

\vfil

\bgroup
\itident6mm
\iitident8mm

\line{\vbox{\hsize6cm
{\bf TOTEM physics programme}
\> total pp cross-section
\> elastic scattering
\> hard and soft diffraction}%
\raise6mm\hbox{$\left.\vrule width0pt height8mm\right\rbrace$}
\raise2mm\vbox{\hsize8cm
\centerline{detection of very forward protons}
\centerline{$\Downarrow$}
\centerline{special detectors needed: Roman Pots}
}\hss}

\hbox{}\vfil

{\bf TOTEM Roman Pots}
\> stations at $147\un{m}$ and $220\un{m}$ on both sides of IP
\> movable beam-pipe insertions
\>> as close to the beam as safe ($\approx 10\sigma$)
\>> retracted when beam instable
\egroup

\vfil

\vskip2\baselineskip
\centerline{\fig*[15cm]{fig/RP_stations_original.pdf}}

\vfil
\vfil

\newpage %-------------------------------------------------------------------------------------------
\title{The Roman Pot System}

\line{\hskip-6.1mm$\bullet$ each station has 2 units\hss}

\bgroup
\parindent=0pt
\line{\hss
\vbox{\hsize5.6cm\> each unit has top, bottom and\hfil\break horizontal pot\par\fig[,38mm]{fig/rp_unit.jpg}}\hskip1mm\hfil
\vbox{\hsize5cm\> each pot contains a detector\hfil\break package\par\fig[,38mm]{fig/rp.jpg}}\hskip1mm\hfil
\vbox{\hsize4.2cm\> each detector package\hfil\break comprises 10 Si detectors\par\fig[,38mm]{fig/rp_package.jpg}}\hss
}

\vskip2mm

\line{\hss
\fig*[,44mm]{fig/overlap.pdf}\hfil
\fig[,44mm]{fig/hybrid2.jpg}\hfil
\fig[,44mm]{fig/reference_frame.png}\hss
}
\egroup

\> overlap crucial for alignment


\newpage %-------------------------------------------------------------------------------------------
\title{The Requirements for Sensors}

\centerline{\fig[15cm]{asy/protonTransport.pdf}}

\vfil

\centerline{maximization of acceptance for low scattering angles}
\centerline{$\Downarrow$}
\centerline{detectors as close to the beam as safe}
\centerline{$\Downarrow$}
\centerline{narrow insensitive edge needed}

\newpage %-------------------------------------------------------------------------------------------
\title{Edgeless Silicon Detectors with CTS}
\unskip
\centerline{(\SmallerFonts CTS = Current Terminating Structure)}

\line{\hss\fig[,42mm]{fig/cts_scheme.png}\hfil\fig[,42mm]{fig/silicon_explained.png}\hskip1mm\hss}

\vskip2mm

\itskip=2pt
\line{\hss\vbox{\hsize8cm\rightskip=0pt plus 1fil
\> TOTEM design
\> very high resistivity Si n-type $\langle 111\rangle$
\> $300\un{\mu m}$ thick, $V_{\rm dep}=20\un{V}$
\> standard planar technology fabrication / dicing with diamond saw
\> single sided detector, 512 microstrips (pitch $66\un{\mu m}$)
\> pitch adapter on detector (VFAT / APV25 compatible)
\> strips at $45^{\circ}$ from the sensitive edge
\> AC coupled (punch-through)
\> only $\approx 50 \un{\mu m}$ of dead area (standard technology $0.5\div 1\un{mm}$)
}\hfil\raise1cm\hbox{\fig*[,45mm]{fig/iv_characteristic.pdf}}\hss}

\newpage %-------------------------------------------------------------------------------------------
\title{Silicon Detector Prototype Tests}
\unskip
\centerline{\SmallerFonts (details will be soon published in paper ''Reconstruction Performance of Silicon Detectors with Reduced Edge}
\vskip-2pt
\centerline{\SmallerFonts Insensitive Volume Manufactured in CTS and 3D-Planar Technologies'' by the TOTEM collaboration)}

\itskip1pt
\> muon beam, analogue readout with APV25
\> alignment: metrology + software $\rightarrow$ $1\un{\mu m}$ precision
\> tracking uncertainty $< 10\un{\mu m}$

\vfil

\title{Edge Efficiency}
\unskip
\> efficiency curves for two of the test detectors

\line{\hss\fig*[,5.5cm]{fig/edge1.pdf}\fig*[,5.5cm]{fig/edge2.pdf}\hskip2mm\hss}

\> $w$ coordinate perpedicular to the edge
\> red line represents the physical edge
\> $10\percent \div 90\percent$ rise (region with blue boundary) found in range $30\div 40\un{\mu m}$
\> plateau at about $95\percent$ because of unbonded/noisy strips

\newpage %-------------------------------------------------------------------------------------------
\title{Signal to Noise}

\line{\hss\vtop{\hsize7cm
\> signal-to-noise ratio histogram
\vskip\baselineskip

\fig*[8cm]{fig/signal_to_noise_100.pdf}

\>> good noise/signal separation
}\hskip2mm
\vtop{\hsize6.4cm
\> signal-to-noise as a function of distance from the edge ($w$)

\fig*[7cm]{fig/signal_to_noise.pdf}

\>> the ''dead'' area is partially efficient
}
\hss}

\newpage %-------------------------------------------------------------------------------------------
\title{Charge Sharing and Cluster Size}

\line{\hss\vtop{\hsize7cm
\> ratio $Q_{\rm main\ strip}/Q_{\rm cluster\ total}$
\vskip1mm
\fig*[,6.2cm]{fig/charge_sharing.pdf}
\>> hits close to a strip center: most charge collected by the strip
\vskip2mm
\>> hits between strips: reduced ratio, i.e.\break increased charge sharing
}\hskip2mm
%
\vtop{\hsize7.3cm
\> mean cluster size\strut
\vskip1mm

\fig*[,6.2cm]{fig/cluster_size.pdf}
\>> hits close to a strip center: mostly one-strip clusters 
\vskip2mm
\>> hits between strips: larger clusters
}
\hss}


\newpage %-------------------------------------------------------------------------------------------
\title{Track Reconstruction}

\line{\hss\vtop{\hsize7cm
\> distribution of residuals in a test detector\hfil\break(residual = recon. hit - reference track) 
\vskip1mm
\fig*[,5.5cm]{fig/test_residuals.pdf}
\>> std. deviation close to the expected value $66\un{\mu m}/\sqrt{12} \approx 19\un{\mu m}$
}\hskip2mm
\vtop{\hsize8cm
\> mean residuals as a function of distance from the edge ($w$)
\vskip1mm
\fig*[,5.5cm]{fig/reconstruction_bias.pdf}
\>> small bias ($< 40\un{\mu m}$) close to the edge, fully correctable
}
\hss}



\newpage %-------------------------------------------------------------------------------------------
\title{Irradiation Tests}

\line{\hskip-5mm\fig*[,9cm]{fig/irradiation.pdf}\hskip3mm\raise6.6cm\vbox{\hsize7cm
\> detectors irradiated with $24\un{GeV}$ protons
\> no efficiency decrease for irradiation below $10^{13}\un{p\cdot cm^{-2}}$
\> higher bias needed to reach full efficiency for irradiations above $10^{13}\un{p\cdot cm^{-2}}$
%\> $10^{14}\un{p\cdot cm^{-2}}$ currently considered as\break irradiation limit
%\vskip\baselineskip
%\> detectors functional to $\approx 1\un{fb^{-1}}$
}\hss}



\newpage %-------------------------------------------------------------------------------------------
\title{Final Electronics System}

\line{
\vbox{\hsize8.5cm
\> custom-developed front-end chip: VFAT2
\>> digital
\>> trigger capability
\>> 128 channels (4 chips per sensor)
\>> 256 clock cycles memory
\>> used for all TOTEM sub-detectors (also T1 and T2)

\> and much more (backup slide)

\vskip1.5cm

\fig*[7cm]{fig/vfat.jpg}

}
\fig*[,10cm]{fig/hybrid.jpg}
\hss}



\newpage %-------------------------------------------------------------------------------------------
\title{Production Tests 2007-2009}

\> every detector package tested
\> with final electronics
\> with muon beam/with cosmic rays

\vskip2cm

\> threshold scan with pion beam

\iitskip=2pt
\line{\hss\fig*[9cm]{fig/threshold_scan.pdf}\hfil\raise20mm\vbox{\hsize6cm
\>> noise stops at $\approx 3\,000\un{e}$
\>> plateau up to $\approx15\,000\un{e}$
\>> efficiency reduced to about half at $22\,000\un{e}$
\>> with noise maximum around $1\,000\un{e}$ $\Rightarrow$ signal-to-noise between $20$ and $25$
}\hss}



\newpage %-------------------------------------------------------------------------------------------
\title{Internal Alignment}

\> comparison of internal alignment results for one detector package

\line{\hss\fig[14.5cm]{asy/alignment_comparison.pdf}\hss}



\newpage %-------------------------------------------------------------------------------------------
\title{Cooling}

\> cooling fully functional

\line{\hss\fig*[15.5cm]{fig/temperature.jpg}\hss}



\newpage %-------------------------------------------------------------------------------------------
\title{First LHC Data, December 2009}

\itskip=2pt
\> pot edge at about $4.5\si$ from the beam
\vskip\itskip

\line{\hss\fig*[15.5cm]{fig/firstData.pdf}\hss}


\newpage %-------------------------------------------------------------------------------------------
\title{Recent LHC Data, April 2010}

\> showers seen in retracted position (movement commissioning still in progress)

\line{\hss\fig[14.3cm]{fig/screenshot.png}\hss}


\newpage %-------------------------------------------------------------------------------------------
\title{Summary \& Outlook}

\> both stations at $220\un{m}$ fully equipped and installed
\>> $99\percent$ of channels working
\>> first data taken and analyzed (commissioning)
\>> ready to take physics data
\>> Roman Pot position calibration -- precondition for routine RP insertions

\vskip\baselineskip

\> detectors for stations at $147\un{m}$ in production/under test

\newpage %-------------------------------------------------------------------------------------------
\hbox{}\vfil
\title{Backup slides}



\newpage %-------------------------------------------------------------------------------------------
\title{Silicon Detector Prototype Tests}
\unskip
\centerline{\SmallerFonts (details will be soon published in paper ''Reconstruction Performance of Silicon Detectors with Reduced Edge}
\vskip-2pt
\centerline{\SmallerFonts Insensitive Volume Manufactured in CTS and 3D-Planar Technologies'' by the TOTEM collaboration)}

\> muon beam, $\approx 4\cdot10^{5}$ events, analogue readout with APV25

\> 2 reference packages (track definition), 2 test packages

\line{\hss\fig*[,4cm]{fig/test_geometry.pdf}\hss\fig[,4cm]{fig/test_photo.png}\hss}


\vskip\baselineskip
\> careful alignment
\comment{development of the alignment strategy}
\>> metrology
\>> software alignment (shift in read-out direction, rotation around beam, edge position)
\>> eventually $1\un{\mu m}$ precision

\> track interpolation in test packages: uncertainty $< 10\un{\mu m}$



\newpage %-------------------------------------------------------------------------------------------
\title{Final Electronics System}

\centerline{\fig*[15cm]{fig/electronics.pdf}}


\vfil\eject\bye
