\SpecialChapter{Conclusion}
\def\CaptionPrefix{C.}
\eqn=0\tabn=0\fign=0

In the first chapter we have presented a number of predictions of the best known hadronic models of elastic $\rm pp$ scattering for two \abb{LHC} energies $\sqrt s = 7$ and $14\un{TeV}$. A comparison to the first \abb{LHC} data at $\sqrt s = 7\un{TeV}$ will be shown at the end of this conclusion chapter. Then, in \Sc{el coulomb} we have discussed the interference between the electromagnetic (Coulomb) and strong (hadronic) interactions. For the electromagnetic scattering, we have first proposed an effective form-factor that combines the electric and magnetic form factors of the proton. We have expressed our doubt that the commonly used \abb{OPE} approximation may not be accurate for momentum transfers $|t|$ above $1\un{GeV^2}$. We outlined several approaches to the Coulomb-hadronic interference. We have found the derivation of \WaY{} formula inconsistent. We have identified a complication in the derivation of \KaL{} formula which, however, has practically no numerical consequences. All the discussed formulae were derived in $\O{\al}$ approximations. We have compared their predictions to a new eikonal calculation to all orders of $\al$. When the form factors have been omitted, we have found a significant differences. Generally, we have found the formula of \KaL{} the best tool available for experimental data analyses. But still, for momentum transfers above $1\un{GeV^2}$ we expect sizable corrections due to multi-photon exchange effects that are not included in the present eikonal description.

In \Sc{ttm tcs} we have presented the luminosity-independent method of total cross-section measurement and suggested two methods of the extrapolation to $t=0\un{GeV^2}$. We have calculated total cross-section uncertainty estimates $1\div2\percent$ for the high-$\be^*$ optics and $5\percent$ for the medium-$\be^*$ optics. There are several reasons to regard these values as very preliminary, most notably the nominal (expected) optics and beam parameters have been used. Recently \abb{TOTEM} has made its first measurements with the $90\un{m}$ optics (see \bref{totem11-2}) and it seems that the emittance can easily be reduced to about a half and that the horizontal scattering angle can be reconstructed with a reasonable precision. Consequently, one could exploit the same extrapolation technique as for the high-$\be^*$ optics. This brings us to an optimistic conclusion that for the medium-$\be^*$ optics one may expect a better precision than the above quoted uncertainty.

In the third chapter, we have described the \abb{RP} simulation and reconstruction software. The modules developed by the author of this thesis have been then discussed in detail.

When the alignment procedure from \Sc{al} was applied to the 2010 \abb{LHC} runs, the relative misalignments among the sensors of one unit were reduced to about $1\un{\mu m}$ of transverse shift and $0.1\un{mrad}$ of rotation about the beam axis. The alignment of units with respect to the beam was achieved with an uncertainty of about $5\un{\mu m}$ in the horizontal direction and $15\un{\mu m}$ in the vertical (for the run used for the elastic cross-section determination in \Sc{felm}).

In the fifth chapter, we have presented \abb{TOTEM}'s first measurement of the $\rm pp$ differential cross-section at $\sqrt s = 7\un{TeV}$. We have compared the result to the model predictions from \Sc{el pred}. While there is relatively good match for $|t| \ls 0.5\un{GeV^2}$, none of the predictions is compatible with \abb{TOTEM}'s measurement for higher $|t|$ values.
