\input base
\input colors
\input biblio
\input book

\input references.tex

\pdfcatalog{/PageMode /UseOutlines}

\let\BiggerFonts\SetFontSizesXII
\let\NormalFonts\SetFontSizesX
\let\SmallerFonts\SetFontSizesVIII

\NormalFonts

\ParIndent=5mm
\itskip=0mm

\Reftrue
\Toctrue

\def\FootText{(VERSION 1)}


%----------------------------------------------------------------------------------------------------

\def\pmt#1{{\tt #1}}

\def\FC{F^{\rm C}}
\def\FH{F^{\rm H}}
\def\FCH{F^{\rm C+H}}

\def\KL{Kundr\' at-Lokaj\char237\char232 ek}
\def\KaL{Kundr\' at and Lokaj\char237\char232 ek}
\def\WY{West-Yennie}
\def\WaY{West and Yennie}

%----------------------------------------------------------------------------------------------------

\font\fchapter 		= pplb8z at 14.4pt

\def\chapterBase#1#2{%
	\advance\nchapter1
	\nsection=0
	\nsubsection=0
	\nssubsection=0
	\eqn=0
	\tabn=0
	\fign=0
	%
	\edef\currentChapterNumber{\the\nchapter}%
	\edef\currentChapterName{Chapter \currentChapterNumber\ #2}%
	\edef\currentPartNumber{\currentChapterNumber}%
	\edef\currentPartName{\currentChapterNumber\ #2}%
	%
	\TOCwrite{\TOCline}{0}{\currentChapterNumber}{#2}{\the\pageno}%
	%
	\pdfdest name {sn:\currentChapterNumber} xyz
	\pdfoutline goto name {sn:\currentChapterNumber} count \GetOutlineCount{\currentChapterNumber} {\currentPartName}%
	%
	\vskip1\baselineskip
	\vbox to0pt{}
	\hbox{%
		\vtop{%
			\noindent\baselineskip30pt
			\fchapter\currentChapterNumber.\ #2
		}%
		\ifShowLabels
			\rlap{\cmyk{\labelColor}\quad#1\cmyk{\cmykBlack}}%
		\fi
	}%
	\vskip1\baselineskip
	\mark{}%
	\parindent = 0pt
	\everypar={\parindent=\ParIndent \everypar={}}%
}

\def\SpecialChapter#1{%
	\vskip1\baselineskip
	\edef\currentChapterName{#1}\edef\currentPartName{}%
	\TOCwrite{\TOCline}{0}{}{\currentChapterName}{\the\pageno}%
	\pdfdest name {s:\currentChapterName} xyz
	\pdfoutline goto name {s:\currentChapterName} count 0 {\currentChapterName}%
	\hbox{%
		\vtop{%
			\noindent
			\fchapter\currentChapterName
		}%
	}%
	\vskip1\baselineskip
	\mark{}%
}

\def\InsertToc{%
	\vskip3cm%
	\def\currentChapterName{Contents} \def\currentPartName{}%
	\pdfdest name {TOC} xyz
	\pdfoutline goto name {TOC} count 0 {\currentChapterName}%
	\hbox{%
		\vtop{%
			\noindent
			\fchapter Contents
		}%
	}%
	\vskip1\baselineskip
	\input\jobname.toc
}

%----------------------------------------------------------------------------------------------------
% title in English

\pdfdest name {EnTitlePage} xyz
\pdfoutline goto name {EnTitlePage} count 0 {Title page in English}%


\centerline{\fPbxiv Charles University in Prague}
\centerline{\fPbxiv Faculty of Mathematics and Physics}
\vskip2	cm
\centerline{\fPbxx Summary of Doctoral Thesis}
\vskip1cm
\centerline{\IncludeSizedGraphics{5cm}{fig/mffLogo.pdf}}
%\vskip5cm
\vskip3cm
\centerline{\fPbxiv Jan Ka\v spar}
\vskip1cm
\centerline{\fPbxx Elastic scattering at the LHC}
\centerline{\fPbxx\FootText}
\vskip2cm
\centerline{\fPbxiv Institute of Particle and Nuclear Physics}
\ialign to\hsize{\tabskip 0pt plus1fil\strut #&\hfil\fPbxiv #\ \tabskip0pt&\fPbxiv #\hfil\tabskip 0pt plus1fil\cr
&Supervisor:		& RNDr. Vojt\v ech Kundr\' at, DrSc.\cr
%&					& Mario Deile, PhD\cr
&Study Programme:	& Physics\cr
&Specialization:	& Subnuclear Physics\cr
}
\vskip2cm
\centerline{\fPbxiv Prague 2011}

\vfil\eject

%----------------------------------------------------------------------------------------------------
% title in Czech

\bgroup
\input utf8-csf

\pdfdest name {CzTitlePage} xyz
\pdfoutline goto name {CzTitlePage} count 0 {Title page in Czech}%

\centerline{\fPbxiv Univerzita Karlova v Praze}
\centerline{\fPbxiv Matematicko-fyzikální fakulta}
\vskip2	cm
\centerline{\fPbxx Autoreferát doktorské disertační práce}
\vskip1cm
\centerline{\IncludeSizedGraphics{5cm}{fig/mffLogo.pdf}}
%\vskip5cm
\vskip3cm
\centerline{\fPbxiv Jan Kašpar}
\vskip1cm
\centerline{\fPbxx Elastický rozptyl na LHC}
\centerline{\fPbxx\FootText}
\vskip2cm
\centerline{\fPbxiv Ústav částicové a jaderné fyzkiky}
\ialign to\hsize{\tabskip 0pt plus1fil\strut #&\hfil\fPbxiv #\ \tabskip0pt&\fPbxiv #\hfil\tabskip 0pt plus1fil\cr
&Supervisor:		& RNDr.~Vojtěch Kundrát, DrSc.\cr
%&					& Mario Deile, PhD\cr
&Study Programme:	& Fyzika\cr
&Specialization:	& Subjaderná fyzika\cr
}
\vskip2cm
\centerline{\fPbxiv Praha 2011}

\egroup

\vfil\eject

%----------------------------------------------------------------------------------------------------
% announcement in Czech

\bgroup
\input utf8-csf

\parindent0pt

\pdfdest name {AnnouncementPage} xyz
\pdfoutline goto name {AnnouncementPage} count 0 {Announcement page}%

\TODO{Tato disertační práce byla vypracována v rámci interního doktorandského studia, které uchazeč absolvoval na Matematicko-fyzikální fakultě Univerzity Karlovy v Praze v letech 2005-2011}.

\vfil

\halign{\hbox to 5.4cm{\strut#\hss\ }&\vtop{\advance\hsize-5.4cm\noindent\strut#\strut}\cr
Doktorand:			& Jan Kašpar\cr
&\cr
Školitel:			& RNDr.~Vojtěch Kundrát, DrSc.\cr
&\cr
Školící pracoviště:	& Fyzikální ústav Akademie věd, v.~v.~i. \cr
					& Na Slovance 1999/2, 182 21 Praha 8\cr
&\cr
Oponenti:			& jméno\cr
					& katedra\cr
					& adresa\cr
&\cr
					& jméno\cr
					& katedra\cr
					& adresa\cr
}

\vfil

Autoreferát byl rozeslán dne ...

\vfil

Obhajoba disertační práce se koná dne ... v ... hodin před komisí pro obhajoby disertačních prací v oboru F9 na Ústavu částicové a jaderné fyziky, V Holešovičkách 2, 180 00 Praha 8, v místnosti ...

\vfil

\halign{\hbox to 5.4cm{\strut#\hss\ }&\vtop{\advance\hsize-5.4cm\noindent\strut#\strut}\cr
Předseda oborové rady oboru F9:	& prof.~Ing.~Josef Žáček, DrSc.\cr
								& Ústav částicové a jaderné fyziky\cr
								& V Holešovičkách 2, 180 00 Praha 8\cr
}

\egroup

\vfil\eject

%----------------------------------------------------------------------------------------------------

\InsertToc
\vfill\eject

%----------------------------------------------------------------------------------------------------

\BeginText
\footline={\hss\the\pageno\hss}
\def\makeheadline{}

\SpecialChapter{Introduction}

The elastic scattering of protons is the simplest $\rm pp$ interaction possible, however, it still presents a challenge for theory to describe it. The complication may be seen in the fact that the coupling constant of quantum chromodynamics (QCD) becomes too large for low energy scales (low momentum transfer is characteristic for the elastic scattering of protons). Consequently, one can not apply the simple perturbation calculations like in quantum electrodynamics (QED) for example. Instead of describing the elastic scattering form first principles, many model descriptions have been developed. These more or less QCD-motivated models are often build on Regge theory and/or eikonal formalism grounds. We will discuss several of these models and present their  cross-section predictions for the LHC energies. We will show that the predictions differ considerably, up to several orders of magnitude for higher $|t|$ values (see \Fg{el mod dsdt large}). This wide spread reflects the limited capabilities of our present theoretical description of elastic scattering. It is clear, therefore, that more measurements are needed in order to guide the theoretical research in the right direction.
% dispersion relations

Let us also remark that the elastic scattering -- a process, where the protons undergo just a glancing collision and stay intact -- is intimately connected with proton's structure. Thus by studying the elastic scattering, one can learn about the structure of proton (see for example the model of Islam et al.).

The forward elastic scattering is connected with the total cross-section by the optical theorem. It thus not surprising to find quite large theoretical uncertainty in total cross-section predictions for the LHC. As we will show in \Fg{sigma tot}, the spread of predictions may be as large as from $85$ to $110\un{mb}$ for the energy of $14\un{TeV}$. This is partially due to a wide class of plausible energy dependences of the total cross-section (ranging from $s^{\al}$ to $\log^2 s$, see e.g.~Sec.~7.1 in \bref{barone}) and partially due to large uncertainties of the cosmic-ray data and the conflicting Tevatron measurements. With these data, one can hardly favor any of the proposed theoretical descriptions over another. Again, it is evident that new and precise data are needed (in fact, TOTEM has recently made its first measurement of the total cross-section, as indicated in \Fg{sigma tot}).
% Froissart?

In the previous paragraphs we have shown the need for new measurements of the elastic and total cross-sections in the high-energy regions. For that purpose, the TOTEM experiment has been built at the LHC accelerator at CERN. The TOTEM experiment is dedicated to forward hadronic phenomena. It will measure (or have already made first measurements of) the total cross-section, elastic scattering differential cross-section in a wide kinematic range and a large spectrum of diffractive processes.


\TODO{Very short plan of thesis - from abstract}

Comparisons of the model predictions from \Sc{el} to the measurement results from \Sc{felm} will be shown in the conclusion chapter.


%----------------------------------------------------------------------------------------------------
\chapter[el]{Elastic scattering of protons}

In this thesis, we will devote the first chapter to a theoretical discussion of proton-proton elastic scattering. For this process, only the strong and electromagnetic interactions are relevant. We will first focus on the former, traditionally called hadronic interaction. We will review some of the best known hadronic models (\Sc{el models}) and present their predictions for two LHC-relevant energies $7$ and $14\un{TeV}$ (\Sc{el pred}). Then, in \Sc{el coulomb} we will discuss the interference between the electromagnetic and strong interaction. We will summarize the most common approaches and review the relations among them. We will present a new eikonal calculation to all orders of $\al$, the fine structure constant. We will conclude this section by making numerical comparisons between various interference formulae for the center-of-mass energy $7\un{TeV}$.

\fig{fig/pdf/el_mod_dsdt_large.pdf}{el mod dsdt large}{Differential cross-section as predicted by the four models, in the $|t|$ range from $0$ to $20\un{GeV^2}$.}

\fig{fig/pdf/el_diagrams.pdf}{el diagrams}{Some of the Feynman diagrams contributing to proton-proton scattering. A: one photon exchange (OPE) diagram. B: a representation of complete hadronic scattering, that is the sum of all diagrams without photon exchange. C and D: the lowest order diagram contributing to the Coulomb-hadronic interference phase.}

\Fg{el ff comparison} compares the aforementioned form factors. We observe very little difference among Kelly's, Arrington's and Puckett's form factors. On the other hand, for $|t| \gs 0.1\un{GeV^2}$ there is a significant difference between the effective (solid lines) and electric (dashed lines) form factors. The same holds for the difference between Hofstadter's effective form factor and the dipole function $F_{\rm d}$ (dotted line).

\fig{fig/pdf/el_diagrams2.pdf}{el diagrams2}{A: The type of diagrams that is included in the eikonal calculation of the complete Coulomb amplitude. B -- E: the lowest order sub-diagrams missing in the eikonal calculation. The double line in E represents an excited state of a proton.}

\fig{fig/pdf/el_cic_diff_Psi_sum.pdf}{el cic diff Psi sum}{}

\fig{fig/pdf/el_cic_diff_Z_sum.pdf}{el cic diff Z sum}{}

\fig{fig/pdf/el_mod_Z.pdf}{el mod Z}{The importance of the interference term (see \Eq{el Z}) as a function of $t$. Calculated with CKL formula and Puckett form factor.}

In the first chapter we have presented a number of predictions of the best known hadronic models of elastic $\rm pp$ scattering for two LHC energies $\sqrt s = 7$ and $14\un{TeV}$. A comparison to the first LHC data at $\sqrt s = 7\un{TeV}$ will be shown at the end of this conclusion chapter. Then, in \Sc{el coulomb} we have discussed the interference between the electromagnetic (Coulomb) and strong (hadronic) interactions. For the electromagnetic scattering, we have first proposed an effective form-factor that combines the electric and magnetic form factors of the proton. We have expressed our doubt that the commonly used OPE approximation may not be accurate for momentum transfers $|t|$ above $1\un{GeV^2}$. We outlined several approaches to the Coulomb-hadronic interference. We have found the derivation of \WaY{} formula inconsistent. We have identified a complication in the derivation of \KaL{} formula which, however, has practically no numerical consequences. All the discussed formulae were derived in $\O{\al}$ approximations. We have compared their predictions to a new eikonal calculation to all orders of $\al$. When the form factors have been omitted, we have found a significant differences. Generally, we have found the formula of \KaL{} the best tool available for experimental data analyses. But still, for momentum transfers above $1\un{GeV^2}$ we expect sizable corrections due to multi-photon exchange effects that are not included in the present eikonal description.


%----------------------------------------------------------------------------------------------------
\chapter[ttm]{The TOTEM experiment}

The TOTEM experiment \bref{totem04,totem08,totem10} is dedicated to measurement of forward hadronic phenomena at the LHC. The tree pillars of its physics programme are: elastic scattering measurement in a wide $t$-range, total cross-section measurement and a study of soft and hard diffraction.

\TODO{shorten - move to introduction}
We presented several theoretical/phenomenological descriptions of the elastic scattering in \Sc{el}. Looking at the model predictions in \Fg{el mod dsdt large}, one can see differences of several orders of magnitude. TOTEM plans to measure the differential cross-section in the $|t|$-range from about $10^{-3}\un{GeV^2}$ to more than $10\un{GeV^2}$. This large coverage can be achieved by exploiting several running scenarios. The discriminative power of such measurement can already be seen in the first TOTEM results, see \Fg{ttm mod cmp dsdt} in the conclusion chapter.

As much as the model predictions differ for the elastic cross-section, they differ for the total-cross section. From Tab.~2 in \bref{kklp11} one can read a variation between $95$ and $110\un{mb}$ for the energy of $14\un{TeV}$. Extrapolations from lower energy measurements yield even higher uncertainty (see \bref{cudell02}): from about $85\un{mb}$ to $120\un{mb}$. TOTEM plans to measure the total cross-section with the luminosity independent method. This method takes as an input the elastic and inelastic event rates and thus is closely related to the other two items of the physics programme. TOTEM aims at the final precision about $1\percent$, which would provide a decisive answer to many total-cross-related hypothesis.

TOTEM's physics programme sets special requirements for the detector apparatus. In particular large rapidity coverage - to detect most fragments from inelastic collisions and excellent acceptance for outgoing elastic and diffractive protons. To accomplish this task, TOTEM comprises three sub-detectors: the \em{telescopes T1 and T2} and a system of \em{Roman Pots} (RPs). The role of the telescopes is to detect the products of inelastic collisions. They are able to reconstruct many tracks per event and extrapolate them to the interaction point (IP), which is essential for background rejection. The role of the RP system is to detect the surviving protons, scattered to very low angles. That is why the RPs are installed far from the IP (scattered protons can gain larger elongations from the beam) and can be approached very close to the beam (to detect protons with small elongations, that is with small scattering angles). For safety reasons, no detectors are allowed to be inside the beam-pipe before the beam is stable. This means that the RPs must be movable beam-pipe insertions, retracted to a safe position during beam fill and adjustments and approached close to the beam for measurements. All three sub-detectors are trigger capable which is crucial for the total cross-section measurement. All detectors are installed symmetrically on both sides of IP5 shared with the CMS experiment.

%\fig{fig/pdf/ttm_det_overview.pdf}{ttm det overview}{An overview of the TOTEM detectors. All detectors are installed symmetrically left (sector 45) and right (sector 56) of the interaction point 5 (IP5). Only the \rhs~is shown in the figure. Bottom part: the structure of the LHC around IP5. The vertical boxes represent quadrupole (Q) and dipole (D) magnets. D1 is the separation magnet -- splitting each beam to its proper beam-line, see the \rhs~of the figure. The RP stations at $147$ and $220\un{m}$ are shown by red dots (each dot represents a unit). Upper box: zoomed region (see the guiding dashed lines) around the IP5 -- the telescopes (red) T1 and T2 embedded into the CMS detector (gray). The T1 telescopes are installed inside the CMS end-caps, the T2 telescopes on the outer side of the CMS HF calorimeters.}

The forward protons, before being registered by the RP detectors, pass through the lattice of the LHC magnets, and thus the observed hit pattern depends on the accelerator settings (beam optics). In this way, the optics defines the acceptance and the resolution of the proton kinematics reconstruction. Besides the optics, the beam collision parameters (such as luminosity) can be optimized for certain physics measurements. TOTEM plans to use the following running scenarios.

\> The \em{high-$\be^*$} scenario with $\be^* = 1535\un{m}$ optics and luminosity ${\cal L} \approx 10^{28} \un{cm^{-2}s^{-1}}$ is dedicated to the measurement of the total cross-section and low-$|t|$ elastic scattering. In this scenario, the RPs have an excellent angular resolution $0.3\un{mrad}$. The elastic cross-section will be measured down to $|t| \approx 2\cdot10^{-3}\un{GeV^2}$.

\> The \em{medium-$\be^*$} scenario with $\be^* = 90\un{m}$ optics and luminosity ${\cal L} \approx 10^{28} \hbox{ -- } 10^{29}\un{cm^{-2}s^{-1}}$ combines the properties of high and low $\be^*$ scenarios allowing for measurements of the total cross-section, medium-$|t|$ elastic scattering and all diffractive processes. The vertical angular resolution is about $2.4\un{\mu rad}$, the elastic cross-section can be measured down to $|t| \approx 4\cdot 10^{-2}\un{GeV^2}$.

\> The \em{low-$\be^*$} scenarios with $\be^*$ between $0.5$ and $3.5\un{m}$ were primarily designed for high luminosities ${\cal L} \approx 10^{32} \hbox{ -- } 10^{33}\un{cm^{-2}s^{-1}}$. They allow for diffractive studies and large-$|t|$ elastic scattering measurements. The angular resolutions is about $16\un{\mu rad}$. The lowest measurable elastic $|t|$ depends on the actual $\be^*$ value. For the standard 2010 optics ($\be^* = 3.5\un{m}$ and $\sqrt s = 7\un{TeV}$), the elastic cross-section was determined down to $|t| \approx 0.4\un{GeV^2}$.

\vskip1mm
As mentioned above, TOTEM plans to measure the total cross-section by the luminosity independent method. We implicitly mean the cross-section due to the hadronic interaction only (it would become infinite if the electromagnetic interaction was included). The method can be summarized by:
$$
	\si_{\rm tot} = {1\over 1+\rh^2} {\d N_{\rm el}/\d t|_{t=0}\over N_{\rm el} + N_{\rm inel}}\ .
$$
It relates the total cross-section $\si_{\rm tot}$ to the forward elastic rate $\d N_{\rm el}/\d t|_{t=0}$, integral elastic rate $N_{\rm el}$, integral inelastic rate $N_{\rm inel}$ and the ratio of the real to imaginary part of the forward elastic amplitude $\rh$.

The inelastic rate $N_{\rm inel}$ will be measured with the telescopes T1 and T2.
%The value of $\rh$ may be determined by analyzing the elastic differential cross-section in the Coulomb-interference region (if acceptance permits), or it can be taken from an external source (such as the COMPETE fits \bref{cudell02}). Since the expected value of $\rh$ is about $0.14$ and since it enters the $\si_{\rm tot}$ expression as $1 + \rh^2$, the influence of any $\rh$ errors is small.
The differential elastic rate $\d N_{\rm el}/\d t$ at $t=0\un{GeV^2}$ can not be measured directly, it can only be extrapolated from the measured low-$|t|$ data. There are two optics that extend to sufficiently low $|t|$ values: $\be^* = 1535$ and $90\un{m}$. This extrapolation procedure is important also for the integral elastic rate $N_{\rm el}$ measurement as it allows to correct for the events missed due to the finite acceptance.

For the extrapolation, one needs an adequate parameterization of the $t$ distribution. Based on the theoretical models, we showed that the low-$|t|$ hadronic amplitude can well be described by $A \exp[B(t) + \i P(t)]$, where $A^2$ is proportional to the searched forward elastic rate and $B$ and $P$ are polynomials of low degrees. However the measured $t$ distribution is affected by the electromagnetic interaction and by the limited acceptance and resolution of the detectors. These two effects must be taken into account in the extrapolation procedure. We showed that the Coulomb-hadronic interference is relevant only for the $\be^*=1535\un{m}$ optics and the smearing effects are sizable only for the $90\un{m}$ optics. Both interference and smearing effects are rather small (few percent) and are treated as perturbations -- they are unfolded in an iterative manner.

To study the extrapolation procedure, we made a simple MC simulation corresponding to about a day of data-taking. We considered two optics as two extreme cases: $\be^* = 1535\un{m}$ (both transverse components of $t$ can be determined) and $90\un{m}$ (only the vertical component $t_y$ can be measured). For each case a different strategy was proposed.

For the $1535\un{m}$ optics, the measured $t$-distribution was simply fitted with the suggested parameterization. The Coulomb-interference effects were separated in a few iterations. We tried to optimize the fit parameters: lever-arm (lower and upper bounds), binning and the moduli of the $B$ and $P$ polynomials. This was done in a ``model independent way'' -- we used five different models, but only optimized only the extrapolation deviation. An example of the outcome is shown in \Fg{ext results sum} left.

For the $90\un{m}$ optics, we first converted the suggested $t$ parameterization into a $t_y$ parameterization. Then we performed a similar optimization as for the other optics. An example result is show in \Fg{ext results sum} right. There, no smearing correction was applied, leading to an overall shift of about $-2\%$.

\fig{fig/pdf/ext_results_sum.pdf}{ext results sum}{The extrapolation deviation of the cross-section at $t=0\un{GeV^2}$, as a function of the fit lower bound $|t|_{\rm low}$. The deviation is calculated as (extrapolated - original) / original, where e.g.~original denotes the original (true) value of $\d\si^{\rm H}/\d t|_{0}$.}

The error propagation leads to total cross-section measurement uncertainties $1-2\%$ (for $1535\un{m}$ optics) and about $5\%$ (for $90\un{m}$ optics). There are several reasons to regard these values as very preliminary, most notably the expected optics and beam parameters were used. Recently TOTEM has made its first measurements with the $90\un{m}$ optics (see \bref{totem11-2}) and it seems that the emittance can easily be reduced to about a half and that the horizontal scattering angle can be reconstructed with a reasonable precision. Consequently, one could exploit the same extrapolation technique as for the high-$\be^*$ optics. This brings us to an optimistic conclusion that for the medium-$\be^*$ optics one may expect a better precision than the above quoted uncertainty.



%----------------------------------------------------------------------------------------------------
\chapter[sr]{Roman Pot simulation and reconstruction software}

In order to develop and tune the reconstruction algorithms (techniques where the scattered proton kinematics is deduced from the RP measurements), it is essential to understand the processes that scattered protons undergo until they are detected by the RP detectors. For this purpose computer Monte Carlo (MC) simulation provides a very valuable tool. Consequently, the simulation and reconstruction modules form a significant part of the TOTEM RP software. The contributions by this thesis' author (outlined below) mostly complement the work described in \bref{hubert}.

\> {\bf Elegent} (ELastic Event GENeraTor) is MC generator of elastic $\rm pp$ and $\rm\bar pp$ collisions with $\d\si/\d t$ distributions given by the models discussed in \Sc{el}. It supports four modes of treating the Coulomb-hadronic interference: entirely neglected, according to full and simplified \WY{} formula and according to the (corrected) \KL{} formula.

\> The aim of the {\bf beam smearing} module is to account for the beam smearing effects that are usually not considered by event generators. These smearing effects can be divided into three classes: \em{angular smearing} (the particles within a bunch are not all parallel), \em{energy smearing} (there is an energy fluctuation within a bunch) and \em{vertex smearing} (the bunches have non-zero dimensions, thus the interactions are distributed in space). 

\> The {\bf fast simulation} modules provide a less detailed, but faster alternative to the Geant4-based simulation. They are primarily intended for statistical studies of the track-based alignment.

\> The aim of the {\bf pattern recognition} is to match sensor hits with particle tracks. This implicitly includes suppressing noise hits that are not associated with any track. Our algorithm exploits an optimization of the Hough transform and can be used with tracks with arbitrary angles (advantageous for track-based alignment applications).

\> The {\bf elastic-event reconstruction} module provides a light-weight alternative to the inelastic proton reconstruction (see Sec.~8.2 in \bref{hubert}). The primary aim of this study (performed during the simulation phase of the experiment) was to quantify the capabilities of the RP system with various optics. An outcome example (elastic $t$ resolutions) can be seen in \Fg{elr res t sum}.

\fig{fig/pdf/elr_res_t_sum.pdf}{elr res t sum}{$t$ resolution in elastic-scattering reconstruction for two optics: $\be^* = 1535\un{m}$ (left) and $90\un{m}$ (right).}

\> The {\bf data quality monitor} (TotemDQM) is an interactive program that can visualize the results of all the reconstruction steps. It has primarily been designed for (quasi)-online control of the data being acquired. But it can also be used for event scanning (visual inspection of the events' details) or for software tuning and debugging.



%----------------------------------------------------------------------------------------------------
\chapter[al]{Alignment of Roman Pots}

An accurate detector alignment is of major importance for the TOTEM experiment in order to deliver precise measurements. Among the sub-detectors of TOTEM, the alignment of the Roman Pots (RPs) presents the biggest challenge since they are movable and, in principle, they can be at a different position for every run. The impact of misalignment is most pronounced for the high-$\be^*$ optics, where the beam smearing and resolution effects are very low.

We presented a RP alignment procedure that comprises the three steps below.
\bitm
\itm First the RPs are approached to the desired position as precisely as possible, then their position with respect to the beam is verified with dedicated devices such as the linear voltage differential transformers (LVDTs). The LVDTs are regularly calibrated against the LHC collimators (\em{collimation alignment}). We found that each LVDT had to be treated as an independent device with its own offset from a global reference frame.
%
\itm A lot of information on the relative alignment between the RP sensors can be obtained by studying the reconstructed tracks. The underlying idea of this \em{track-based alignment} is that sensor misalignments give rise to residuals (hit distances from track fits). Thanks to the overlap between the vertical and horizontal pots, one can determine the relative positions of all RP sensors within each station.
%
\itm However, not all misalignments induce residuals (e.g.~a global shift or rotation), thus these misalignment modes are inaccessible to the track-based alignment. To overcome this limitation, one may exploit use known symmetries of certain physics processes -- \em{profile alignment}. From this point of view, elastic scattering (\em{elastic alignment}) looks extremely useful. Not only it posses the full azimuthal symmetry, also its two anti-collinear protons can be used for aligning the opposite arms.
\eitm

\vskip1mm
By studying the impact of misalignment to the RP detector measurements we found that the track-based alignment is only sensitive (within the expected uncertainties) to transverse shifts and rotations about the longitudinal (beam) axis. In order to extract the misalignment information from track measurements, we pursued the approach of Millepede \bref{millepede} -- the simultaneous fit of the track and misalignment parameters.

We discussed, quite generally, what misalignment modes are inaccessible (\em{singular modes}) to the track-based alignment. For the RP system these modes include: global and linearly progressive shift in both transverse directions (4 modes) and global rotations of $U$ and $V$ detector subsets (2 modes). Moreover we identified the \em{weak modes} of the track-based alignment -- the modes that can be resolved with a large uncertainty only. These modes emerge since the LHC track sample is very well collimated. For the RP system the weak modes include: linearly progressive rotations of $U$ and $V$ detector subsets.

We presented a set of constraints that fix the 8 singular and weak modes and thus enable to solve the track-based alignment task. Most of these inaccessible modes are addressed by other methods presented in the thesis.

We discussed both statistical and systematic errors of the track-based alignment. The statistical error is a caused by the finite resolution of the silicon sensors. This error can be propagated to the alignment parameters, thus every alignment result is complemented with its statistical uncertainty. The systematic errors may come from the approximations we made in building the track-based algorithm. We argued that these errors would be negligible.

In order to validate the track-based alignment we made a number of (fast) MC simulations. An example is show in \Fg{al stat final}. For both alignment classes (shifts and rotations), the systematic error is compatible with zero (within the uncertainty). The uncertainty estimate falls as $1/N_{\rm tracks}$ with increasing number of tracks. The right-hand side column shows that the uncertainty estimate slightly overshoots the statistical error.

\fig{fig/pdf/al_stat_final.pdf}{al stat final}{A statistical study of the track-based alignment.
%method with the final constraints. The simulations have been performed with a realistic track slope spread $\si_a = 0.1\un{mrad}$.
Each curve corresponds to the third sensor of a RP, see the legend. The semi-transparent areas represent $1\si$ error bands. The vertical dotted lines mark a typical number of tracks in LHC runs.
%(20 repetitions), geometry 2.7x3.3, misalignment rotz4, gauss6,8, 0.1mrad.
}

Eventually, the track-based alignment was applied to the LHC data, see an example in \Fg{al comp det per unit}. The pattern of these points reflects the two-fold origin of the misalignments -- the misalignment of detector packages (DPs) within stations (dashed lines) and the misalignment of sensors within DPs (the residuals from the dashed fit lines). Since every DP contains 5 $U$ and 5 $V$ sensors, one can extract both transverse shift components and all three rotations. This is to be compared to a single sensor for which only one shift component (in its read-out direction) and only the rotation about the beam axis can be determined. Coming back to \Fg{al comp det per unit}, notice the stability of the results -- all nine point sets (representing data taking campaigns from August to October) are overlapping. 

\fig{fig/pdf/al_comp_det_per_unit.pdf}{al comp det per unit}{Track-based alignment applied to the 2010 LHC runs (showing results for 56-220-near unit). Each color corresponds to one data set (9 in total). The dashed lines represent the DP misalignments, the residuals (distances point-dashed line) show the internal (within DP) misalignments.
%The internal rotation is the rotation of a sensor with respect to its DP (i.e.~the full rotation minus $\De\rh^{\rm DP}_z$).
%overlap=f
}

The typical precision of the track-based alignment for the 2010 LHC runs was about $1\un{\mu m}$ for the transverse shifts and about $0.1\un{mrad}$ for the rotations about the beam axis.

The singular modes related to the transverse shifts include the horizontal and vertical, global and linearly progressive shifts. These can be alternatively expressed as the horizontal and vertical shifts of the near and far unit of each station. These shifts are inaccessible to the track-based alignment, but can be determined by locating the beam with respect to the displaced detectors. For that purpose, one may exploit hit-pattern symmetries. The nominal optics were conceived with zero vertical dispersion, which assures the vertical symmetry of hits about the beam center. Furthermore the elastic-scattering hit pattern was expected to be symmetric about the vertical axis crossing the beam center. In reality -- see \Fg{al prof hits} -- the picture is more complicated. Due to the optics imperfections the vertical dispersion is non-zero (the magenta line has a slope) and the axis of the elastic hits (green) is tilted. Since there are two vertical RPs (top and bottom) in each unit, the green fit can still be used for alignment purposes. We split the hit distribution into horizontal slices and determined the peak of the horizontal distribution in each of them (the elastic peak dominates the asymmetric low-$\xi$ diffraction). In principle the same can be done for the magenta line fits, however, since there is only one horizontal RP per unit, the fit must be extrapolated to the beam location and the final precision in uninteresting for alignment applications.

\fig{fig/pdf/al_prof_hits.pdf}{al prof hits}{Typical hit distributions at the $220\un{m}$ stations (scoring planes at $\pm 217\un{m}$). Data from 21 September. The colorful lines represent fits of low-$\xi$ (green) and large-$\xi$ (magenta) hits.
% analysis with vsym2 geometry
}

One can make a stronger use of the hit symmetries if a sample of elastic hits is selected, see an illustration in \Fg{al el plots sum}. The vertical beam position (dash-dotted line) is determined in the right-hand plot. The bottom-pot (blue) distribution (after having removed the part affected by the acceptance) is flipped (green) and moved up and down until a match with the top-pot (red) distribution is found. In this way one obtains the center of the vertical hit distribution -- the vertical beam position. The horizontal beam position is determined in the left plot by the intersection of the green fit with the dash-dotted line (the vertical beam position). For the data from 29-30 October (the data used for the first elastic scattering measurement described in \Sc{felm}) this alignment has uncertainty of about $5\un{\mu m}$ (horizontally) and $15\un{\mu m}$ (vertically).

\fig{fig/pdf/al_el_plots_sum.pdf}{al el plots sum}{RP alignment with elastic scattering. Left: hit map in a unit, Right: the vertical profiles from the top (red) and bottom (blue) RP of that unit.}

%----------------------------------------------------------------------------------------------------
\chapter[felm]{The first elastic scattering measurement at the LHC}

This chapter describes the first elastic scattering measurement at the LHC, made by TOTEM \bref{totem11}. In particular we focused on the contributions by this thesis author -- the background estimation and subtraction and the unfolding of resolution effects.

This analysis is based on the data from 29-30 October 2010, when the RPs were approached to $7\si$ from the beam and the LHC was running at the standard 2010 optics with $\be^* = 3.5\un{m}$. Only the $220\un{m}$ stations were used. The horizontal pots were inserted only at the end of the run, just for alignment reasons. Most of the running time they were retracted not to disturb the elastic proton measurement. The event sample used for the analysis corresponds to an integrated luminosity of $6.1\un{nb^{-1}}$. The analysis procedure is sketched in \Fg{felm scheme} described below.

\fig{fig/pdf/felm_scheme.pdf}{felm scheme}{The analysis work-flow, from raw data to the differential cross-section of the elastic scattering.}

\> The {\bf alignment} step refers to the RP alignment technique described in \Sc{al}. The alignment with elastic events was not applied, in order not to create any bias for the reconstruction. Instead a sample of low-$\xi$ protons was used for the horizontal alignment.

\> {\bf Optics tuning}. Various measurable quantities have been used to fine-tune the optics by optimizing the rotations and strengths of the LHC magnets within their nominal uncertainties.

\> In the {\bf reconstruction} step, the projections of the scattering angle at IP are deduced from the RP measurements. For each projection a different strategy is used, in order to minimize the impact of the uncertainty of the optical functions. For the vertical (horizontal) projection, the scattering angle is determined from vertical hit positions (horizontal angles between the near and far units).

\> The elastic {\bf event selection} cuts require two anti-collinear low-$\xi$ tracks.

\> The {\bf final vertical alignment} addresses the residual vertical misalignments by determining the shift of the vertical scattering angle distribution.

\> {\bf Background subtraction}. Here we calculate and subtract the number and the distribution of the non-elastic events that passed the event-selection cuts.

\> By the {\bf acceptance correction} one corrects for the finite detector size.

\> The {\bf unsmearing} is a procedure that accounts for the finite resolution of the RPs detectors and the beam smearing effects.

\> In the {\bf normalization} step, the event-count histograms $\d N/\d t$ are transformed in the differential cross-section histograms $\d\si/\d t$. For that purpose, the luminosity and various efficiency factors (trigger, detection, reconstruction, etc.) are used.

\> {\bf Merging diagonals}. The data from the two ``diagonals'' (from left-bottom to right-top and from left-top to right-bottom) are merged.

\vskip1mm

In total there are six event-selection cuts. All of them are applied on the $3\si$ level, therefore we expect a very good efficiency. On the other there may be a number of non-elastic events that pass these cuts -- this is what we call background. The study of the background was split into three phases: the integral (number of background events), the distribution of these events and the contribution to the elastic rate $\d N/\d t$.

The background integral was determined by relaxing one of the six cuts and analyzing the distribution of the cut quantity. We exploited two-Gaussian fits (signal + background) as well as one-Gaussian fits (background only) through the distribution tails. The result was that the background formed $(8\pm 1)\percent$ of the sum (signal + background, i.e.~all events that passed the cuts).

Regarding the distribution we showed that it is reasonable to expect the left and right-arm protons to be almost independent. We suggested a one-arm background parameterization $\exp(b_x t_x + b_y t_y) / \sqrt{t_x t_y}$ (the background protons are reconstructed as elastic and thus assigned $t_x$ and $t_y$ values only) and showed that this parameterization describes the background well. Extending this parameterization into the unobserved region leads naturally to an acceptance correction of the background.

Each background event passing the selection cuts is reconstructed as elastic and is assigned a value of $\bar t$ (by averaging left and right-arm scattering angles). Knowing the integral and the distribution, it is then straight-forward to calculate the contribution of the background to the elastic differential rate (distribution of $\bar t$ provided that the selection criteria are satisfied). The result is shown in \Fg{felm background cmp sum} -- our background calculation (red) is compared to the sum (elastic signal + background, black). It is evident that the most critical point of background subtraction is the diffractive dip.


\bmfig
\fig{fig/pdf/felm_background_cmp_sum.pdf}{felm background cmp sum}{[7cm]A background (red) to sum (black) comparison after the acceptance correction. The red dashed lines show the uncertainty of our background determination.}
%
\fig{fig/pdf/felm_unfolding_m1m2_cmp.pdf}{felm unfolding m1m2 cmp}{[7cm]A comparison of the fit-based and bin-based unsmearing methods.
%(for diagonal 45 top -- 56 bottom and with $\si(\De\th) = 12\un{\mu rad}$).
}
\emfig

The leading smearing effects include the beam divergence and the finite resolution of the silicon sensors. For both horizontal and vertical projections, the beam divergence is dominant, giving rise to an angular smearing (after averaging of the left and right-arm measurements) of $(12.5 \pm 0.5)\un{\mu rad}$. The smearing acts independently in each projection, however, we showed that for the analyzed scattering angles (above $170\un{\mu rad}$), the smearing can well be described a simplified model, where the smearing acts directly in the full scattering angle. In other words, the measured angular distribution is a convolution of the true distribution with a Gaussian reflecting the smearing. We used two methods to determine the true angular distribution: a fit-based and a bin-based.

In the fit-based method, the measured distribution is fitted with a convenient parameterization -- a parameterization allowing for an easy smearing deconvolution. We chose a sum of three Gaussians. Moreover, we added a fourth term governing the low-angle behavior (thus mostly outside acceptance) which is compatible with the expected differential cross-section behavior $\exp(bt)$ for low $|t|$. With this fit, we calculated the smearing correction (the ratio of the true over the measured angular distribution) -- see the red curve in \Fg{felm unfolding m1m2 cmp}.

In the bin-based method, the trend of the smearing effects is established by performing several additional smearing steps. The trend is non-linear, that is why we made three extra smearing steps. This allows to use a cubic (the simplest non-linear) back-extrapolation. The extrapolation is applied bin-per-bin and yields the smearing correction which turns the measured angular distribution to the true one (see the blue histogram in \Fg{felm unfolding m1m2 cmp}).

The fit-based and bin-based methods are compared in \Fg{felm unfolding m1m2 cmp}. Despite the large fluctuation of the bin-based result, a very good match is evident. The discrepancy at low $\th$ values is practically irrelevant because it falls below the analysis cut (drawn as the dashed line). Generally speaking we trust more the fit-based result, we consider the bin-based method rather as a reassuring cross-check.

%----------------------------------------------------------------------------------------------------
\SpecialChapter{Conclusion}

In the previous chapter, we have presented TOTEM's first measurement of the $\rm pp$ differential cross-section at $\sqrt s = 7\un{TeV}$. The final result, compared to the predictions of the models from \Sc{el} and the model of Jenkovszky et al.~\bref{jenkovszky11}, is shown in \Fg{ttm mod cmp dsdt}. The discriminative power of TOTEM's measurement is evident. In the region below the diffractive dip, $|t| \ls 0.5\un{GeV^2}$, there is a relatively good agreement among the models themselves and the measurement. But for momentum transfers $|t|$ between $0.5$ and $1\un{GeV^2}$ the models exhibit rather large spread in their predictions (one and half orders). Moreover the measurement of TOTEM does not favour any the presented models. The measurement falls into a gap between two bands of predictions (one band is formed by the models of Islam et al., Petrov et al.~with two pomerons and Jenkovszky et al., the other band is formed by the models of Block et al., Bourrely et al.~and Petrov et al.~with three pomerons).

\fig{fig/pdf/ttm_mod_cmp_dsdt.pdf}{ttm mod cmp dsdt}{Comparison of model predictions and TOTEM's first measurement of $\rm pp$ differential cross-section at $\sqrt s = 7\un{TeV}$.}

%----------------------------------------------------------------------------------------------------
\SpecialChapter{Bibliography}
\input references.tex
\PrintReferences

\EndText

\end
