\input base
\input utf8-csf

\SetFontSizesX

\halign{\hbox to 4cm{\strut#\hss\ }&\vtop{\advance\hsize-4cm\noindent\strut#\strut}\cr
Název práce: 			& Elastický rozptyl na LHC\cr
Autor:					& Jan Kašpar\cr
Katedra:				& Ústav částicové a jaderné fyziky\cr
Vedoucí doktorské práce:& RNDr.~Vojtěch Kundrát, DrSc., Fyzikální ústav AV ČR, v.~v.~i.\cr
%						& Mario Deile, PhD, CERN, Ženeva\cr
Abstrakt:				&
Elastický rozptyl protonů, přestože proces zdánlivě jednoduchý, představuje pro teorii stále velkou výzvu. V této práci se zabýváme elastickým rozptylem z teoretického i experimentálního hlediska. V teoretické části shrneme několik modelů a jejich předpovědi pro LHC. V diskuzi věnované interferenci mezi Coulombickou a hadronovou interakcí představíme nový eikonálový výpočet do všech řádů v konstantě jemné struktury $\al$. V experimentální části práce popíšeme experiment TOTEM, který je mimo jiné zasvěcen měření elastického rozptylu protonů na LHC. Toto měření je realizováno především detektory v římských hrncích (ŘH). To jsou pohyblivé vakuové nádoby zasouvající se do urychlovačové trubice stovky metrů od interakčního bodu. Díky tomu jsou ŘH schopny detekovat částice rozptýlené do velmi malých úhlů. V práci taktéž diskutujeme některé aspekty simulačního a rekonstrukčního softwaru pro ŘH. Velký prostor je věnován alignmentu ŘH, tj.~určení pozic senzorů v ŘH navzájem i vůči svazku. V závěru práce popíšeme první analýzu elastického rozptylu naměřeného experimentem TOTEM na LHC. Výsledný diferenciální účinný průřez je srovnán s modelovými předpověďmi.
\cr
Klíčová slova:			& elastický rozptyl protonů, Coulomb-hadronová interference, TOTEM, římské hrnce, alignment\cr
}

\bye
