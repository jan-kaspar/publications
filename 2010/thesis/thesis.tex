\input biblio
\input book

\input references.tex

\pdfcatalog{/PageMode /UseOutlines}

\let\BiggerFonts\SetFontSizesXII
\let\NormalFonts\SetFontSizesX
\let\SmallerFonts\SetFontSizesVIII

\NormalFonts

\ParIndent=5mm

\Reftrue
\Toctrue

%\iffalse
%	% to show labels
%	\ShowLabelstrue \advance\hoffset-1.5cm
%\else
%	\def\FootText{(VERSION 15)}
%\fi

\ShowLabelstrue

% this leads to 2cm margins at top, bottom and outer edges, 3cm margins at the inner edge
\pageamp=5mm

% for display version
\def\linkColor{\cmykBlue}

% for print version
%\def\linkColor{\cmykBlack}

\def\pmt#1{{\tt #1}}


%----------------------------------------------------------------------------------------------------
%----------------------------------------------------------------------------------------------------

\pdfdest name {TitlePage} xyz
\pdfoutline goto name {TitlePage} count 0 {Title page}%


\centerline{\fPbxiv Charles University in Prague}
\centerline{\fPbxiv Faculty of Mathematics and Physics}
\vskip2	cm
\centerline{\fPbxx Doctoral Thesis}
\vskip1cm
\centerline{\IncludeSizedGraphics{5cm}{fig/logo_mff.pdf}}
%\vskip5cm
\vskip3cm
\centerline{\fPbxiv Jan Ka\v spar}
\vskip1cm
\centerline{\fPbxx Elastic scattering at the LHC}
\centerline{\fPbxx\FootText}
\vskip2cm
\centerline{\fPbxiv Institute of Particle and Nuclear Physics}
\ialign to\hsize{\tabskip 0pt plus1fil\strut #&\hfil\fPbxiv #\ \tabskip0pt&\fPbxiv #\hfil\tabskip 0pt plus1fil\cr
&Supervisor:		& RNDr.~Vojt\v ech Kundr\' at, DrSc.\cr
%&					& Mario Deile, PhD\cr
&Study Programme:	& Physics\cr
&Specialization:	& Subnuclear Physics\cr
}
\vskip2cm
\centerline{\fPbxiv Prague 2011}

\vfil\eject
\forceoddpage

%----------------------------------------------------------------------------------------------------

\pdfdest name {Acknowledgements} xyz
\pdfoutline goto name {Acknowledgements} count 0 {Acknowledgements}%

\line{\hss To my family -- for their admirable support}
\bgroup
\input utf8-csf
\line{\hss Mé rodině -- za jejich skvělou podporu}

\vfil

First of all, I would like to thank my advisors Vojtěch Kundrát (for the theoretical part) and Mario Deile (for the experimental part). I am grateful for their knowledge and expertise -- I have learnt a lot from them. I am grateful for their time that they never hesitated to spend with me, my questions and my problems. I am grateful for their friendly and motivating leadership. Last but not least, I am grateful for their having read the thesis manuscript and the handful of useful comments and suggestions.

I would like thank to the whole team of the TOTEM experiment. Without them it would not have been possible to write this thesis. Let me thank in particular to the members of the software group for a constructive collaboration. Many thanks to Leszek Grzanka for the software repository maintenance. 

I am grateful to Karsten Eggert, Kenneth \"Osterberg, Simone Giani, Valentina Avati, Hubert Niewiadomski and Munir Islam for many interesting and stimulating discussions. They gave me a lot. I am indebted particularly to Valentina and Simone for reading this thesis and giving me a plenty of valuable hints.

I would like to acknowledge the CERN Doctoral Student Programme and the CERN TOTEM group for supporting my research and this thesis.

Finally, let me express my profound gratitude to my family, especially to my wife Pavla and my parents Marie and Jaroslav, for their endless support and 
encouragement. It has meant a lot for me. Thanks!

\vskip3\baselineskip

Závěrem bych rád vyjádřil svou vděčnost mé rodině, především mé ženě Pavle a mým rodičům Marii a Jaroslavovi, za jejich báječnou podporu slovem i skutkem. Znamená to pro mě hodně. Děkuju!


\egroup

\eject
\forceoddpage

%----------------------------------------------------------------------------------------------------

\pdfdest name {Declaration} xyz
\pdfoutline goto name {Declaration} count 0 {Declaration}%

I declare that I carried out this doctoral thesis independently, and only with the cited sources, literature and other professional sources.

I understand that my work relates to the rights and obligations under the Act No.~121/2000 Coll., the Copyright Act, as amended, in particular the fact that the Charles University in Prague has the right to conclude a license agreement on the use of this work as a school work pursuant to Section 60 paragraph 1 of the Copyright Act.

\vskip2\baselineskip

%In Geneva, 25 November, 2011 (submitted version)
%In Geneva, 8 December, 2011 (1st corrected version)
In Geneva, 8 February, 2012 (2nd corrected version)

\vskip2\baselineskip
\line{\hss Jan Ka\v spar}

\vfil\eject
\forceoddpage

%----------------------------------------------------------------------------------------------------

\hbox{}%

\pdfdest name {AbstractPage} xyz
\pdfoutline goto name {AbstractPage} count 0 {Abstract}%

\vfil

\bgroup
\input utf8-csf
\halign{\hbox to 4cm{\strut#\hss\ }&\vtop{\advance\hsize-4cm\noindent\strut#\strut}\cr
Název práce: 			& Elastický rozptyl na LHC\cr
Autor:					& Jan Kašpar\cr
Katedra:				& Ústav částicové a jaderné fyziky\cr
Vedoucí doktorské práce:& RNDr.~Vojtěch Kundrát, DrSc., Fyzikální ústav AV ČR, v.~v.~i.\cr
%						& Mario Deile, PhD, CERN, Ženeva\cr
Abstrakt:				&
Elastický rozptyl protonů, přestože proces zdánlivě jednoduchý, představuje pro teorii stále velkou výzvu. V této práci se zabýváme elastickým rozptylem z teoretického i experimentálního hlediska. V teoretické části shrneme několik modelů a jejich předpovědi pro LHC. V diskuzi věnované interferenci mezi Coulombickou a hadronovou interakcí představíme nový eikonálový výpočet do všech řádů v konstantě jemné struktury $\al$. V experimentální části práce popíšeme experiment TOTEM, který je mimo jiné zasvěcen měření elastického rozptylu protonů na LHC. Toto měření je realizováno především detektory v římských hrncích (ŘH). To jsou pohyblivé vakuové nádoby zasouvající se do urychlovačové trubice stovky metrů od interakčního bodu. Díky tomu jsou ŘH schopny detekovat částice rozptýlené do velmi malých úhlů. V práci taktéž diskutujeme některé aspekty simulačního a rekonstrukčního softwaru pro ŘH. Velký prostor je věnován alignmentu ŘH, tj.~určení pozic senzorů v ŘH navzájem i vůči svazku. V závěru práce popíšeme první analýzu elastického rozptylu naměřeného experimentem TOTEM na LHC. Výsledný diferenciální účinný průřez je srovnán s modelovými předpověďmi.
\cr
Klíčová slova:			& elastický rozptyl protonů, Coulomb-hadronová interference, TOTEM, římské hrnce, alignment\cr
}
\egroup

\vfil

\halign{\hbox to 4cm{\strut#\hss\ }&\vtop{\advance\hsize-4cm\noindent\strut#\strut}\cr
Title:		& Elastic scattering at the LHC\cr
Author:		& Jan Ka\v spar\cr
Departement:& Institute of Particle and Nuclear Physics\cr
Supervisor of the thesis:& RNDr.~Vojt\v ech Kundr\' at, DrSc., Institute of Physics AS CR\cr
%			& Mario Deile, PhD, CERN, Geneva\cr
%Abstract:	& [abstract of 80-200 words in English]\cr
Abstract:	& 
The seemingly simple elastic scattering of protons still presents a challenge for the theory. In this thesis we discuss the elastic scattering from theoretical as well as experimental point of view. In the theory part, we present several models and their predictions for the LHC. We also discuss the Coulomb-hadronic interference, where we present a new eikonal calculation to all orders of $\al$, the fine-structure constant. In the experimental part we introduce the TOTEM experiment which is dedicated, among other subjects, to the measurement of the elastic scattering at the LHC. This measurement is performed primarily with the Roman Pot (RP) detectors -- movable beam-pipe insertions hundreds of meters from the interaction point, that can detect protons scattered to very small angles. We discuss some aspects of the RP simulation and reconstruction software. A central point is devoted to the techniques of RP alignment -- determining the RP sensor positions relative to each other and to the beam. At the end we present the analysis of TOTEM's first elastic scattering measurement at the LHC. The resulting differential cross section is compared to model predictions.\cr
%Keywords:	& [ 3-5 keywords in English]\cr
Keywords:	& elastic scattering of protons, Coulomb-hadronic interference, TOTEM, Roman Pots, alignment\cr
}

\vfil
\eject\forceoddpage

%----------------------------------------------------------------------------------------------------

\InsertToc

\eject\forceoddpage

%----------------------------------------------------------------------------------------------------

\BeginText

\input introduction.tex
\input elastic.tex
\input totem.tex
\input simu_reco.tex
\input alignment.tex
\input first_el_measurement.tex
\input conclusion.tex

%----------------------------------------------------------------------------------------------------

\SpecialChapter{Bibliography}
\PrintReferences

%----------------------------------------------------------------------------------------------------

\SpecialChapter{List of Tables}
\PrintListOfTables

%----------------------------------------------------------------------------------------------------

\SpecialChapter{List of Abbreviations and Symbols}

\noindent The following table summarizes the typographical conventions used in this thesis.
\vskip1mm

\centerline{\vbox{\halign{\bstrut\qquad #\hfil&\qquad #\hfil&\qquad #\hfil\qquad\cr\ln
structure			& example				& font/symbol\cr\ln
scalar				& $s$ 					& italics\cr
vector				& $\vec v$				& bold\cr
element of vector	& $v_i$					& italics\cr
dot product			& $\vec v\cdot \vec w$	& middle dot\cr
cross product		& $\vec v\times \vec w$	& cross\cr
matrix				& $\mat M$				& slanted\cr
state				& $\ket\ps$				& ket\cr
operator element	& $\mel{\ph}{\op A}{\ps}$ & bra-ket, operator with hat\cr
software parameter  & \pmt{parameter}		& typewriter\cr\ln
}}}

\vskip2\baselineskip

\noindent The list of abbreviations used in thesis.

\ListAbb{ALICE}{A Large Ion Collider Experiment}{An LHC experiment.}
\ListAbb{ALFA}{Absolute Luminosity For Atlas}{The very forward tracker system of the ATLAS detector.}
\ListAbb{ATLAS}{A Toroidal LHC Apparatus}{An LHC experiment.}
\ListAbb{BFKL}{Balitsky-Fadin-Kuraev-Lipatov}{Resumation of gluon ladders leading to a ``hard pomeron'', see e.g.~Ch.~8 in \bref{barone}.}
\ListAbb{BLM}{Beam Loss Monitor}{A device to detect beam losses.}
\ListAbb{BPM}{Beam Position Monitor}{A device to determine the position of the beam.}
\ListAbb{CDF}{Cumulative Distribution Function}{}
\ListAbb{CE}{Complete Eikonal}{An eikonal calculation carried out to all orders of the coupling constant.}
\ListAbb{CERN}{Conseil Europ\' eenne pour la Recherche Nucl\' eaire}{European Organization for Nuclear Research in Geneva.}
\ListAbb{CKL}{corrected \KL}{Refers to the Coulomb-interference formula \Eq{el phase CKL}.}
\ListAbb{CM}{Center Of Mass}{Used to refer to the center-of-mass reference frame.}
\ListAbb{CMS}{the Compact Muon Solenoid}{An LHC experiment.}
\ListAbb{CSC}{Cathode Strip Chamber}{A type of gaseous particle detectors, see \Sc{ttm det}.}
\ListAbb{CTS}{Current Terminating Structure}{A ring structure along the active of edge of the TOTEM silicon sensors, see \Sc{ttm det}.}
\ListAbb{DAQ}{Data AcQusition}{A system to collect data from detectors and save them in a computer storage.}
\ListAbb{DIGI}{DIGItal}{A format of data as received from the VFAT chips, i.e.~one boolean value (hit/no hit) per detector channel.}
\ListAbb{DP}{Detector Package}{A package of ten hybrids with silicon detectors, see \Sc{ttm det}.}
\ListAbb{DPE}{Double Pomeron Exchange}{A class of diffractive processes, see \Fg{ttm processes}.}
\ListAbb{DQM}{Data Quality Monitor}{A computer program to inspect the quality of data being acquired.}
\ListAbb{GEM}{Gas Electron Multiplier}{A type of gaseous particle detectors, see \Sc{ttm det}.}
\ListAbb{H8}{}{A test hall at CERN, where TOTEM detectors were/are tested.}
\ListAbb{HP}{Hard Pomeron}{A version of the model of Islam et al., see Sec.~\sref{el models}.}
\ListAbb{IP}{Interaction Point}{The point where the two beams collide.}
\ListAbb{IR}{InfraRed}{Generally used to refer to a low-energy region.}
\ListAbb{KL}{\KL}{Refers to the Coulomb-interference formula Eq.~\ref{el phase KL}.}
\ListAbb{LAB}{LABoratory frame}{A reference frame bound to the laboratory, e.g.~the accelerator}
\ListAbb{LS}{the method of Least Squares}{}
\ListAbb{MC}{Monte Carlo}{A technique of random number generation, integral calculations or physics simulations.}
\ListAbb{LHC}{Large Hadron Collider}{The world's largest and most powerful collider.}
\ListAbb{LHCb}{Large Hadron Collider beauty}{An LHC experiment.}
\ListAbb{LHCf}{Large Hadron Collider forward}{An LHC experiment.}
\ListAbb{LxG}{Low-$x$ gluons}{A version of the model of Islam et al., see Sec.~\sref{el models}.}
\ListAbb{LVDT}{Linear Voltage Differential Transformer}{A device that determines the positions of TOTEM Roman Pots.}
\ListAbb{OPE}{One Photon Exchange}{An approximation in QED.}
\ListAbb{QED}{Quantum ElectroDynamics}{}
\ListAbb{QFT}{Quantum Field Theory}{}
\ListAbb{QCD}{Quantum ChromoDynamics}{}
\ListAbb{QM}{Quantum Mechanics}{}
\ListAbb{RECO}{RECOnstruced}{A format of data consisting of coordinate specifications of every reconstructed hit cluster.}
\ListAbb{RMS}{Root Mean Square}{}
\ListAbb{RP}{Roman Pot}{A a pot-shaped movable beam-pipe insertion with detectors. A system of Roman Pots is one sub-detector of the TOTEM experiment, see \Sc{ttm det}.}
\ListAbb{SD}{Single Diffraction}{A class of diffractive processes, see \Fg{ttm processes}.}
\ListAbb{SWY}{Simplified West-Yennie}{Refers to the Coulomb-interference formula Eq.~\ref{el phase SWY}.}
\ListAbb{T1}{Telescope 1}{A sub-detector of the TOTEM experiment, see \Sc{ttm det}.}
\ListAbb{T2}{Telescope 2}{A sub-detector of the TOTEM experiment, see \Sc{ttm det}.}
%\ListAbb{TDR}{Technical Design Report}{}
\ListAbb{TOTEM}{TOTal and Elastic Measurements}{An LHC experiment devoted to the forward hadronic phenomena, see \Sc{ttm}.}
\ListAbb{TPE}{Two Photon Exchange}{An approximation in QED.}
\ListAbb{VFAT}{Very Forward Atlas and Totem}{The front-end chip used to read out the TOTEM's detectors.}
\ListAbb{WY}{West-Yennie}{Refers to the Coulomb-interference formula Eq.~\ref{el phase WY}.}

%\PrintAbbreviations

\vfil\eject\forceoddpage

\EndText

\end
