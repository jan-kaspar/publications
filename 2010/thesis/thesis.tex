\input base
\input colors
\input biblio
\input book

\input references.tex

\pdfcatalog{/PageMode /UseOutlines}

\let\BiggerFonts\SetFontSizesXII
\let\NormalFonts\SetFontSizesX
\let\SmallerFonts\SetFontSizesVIII

\NormalFonts

\ParIndent=5mm

\Reftrue
\Toctrue

\iftrue
	% to show labels
	\ShowLabelstrue \advance\hoffset-1.5cm
\else
	\def\FootText{(VERSION 11)}
\fi

\def\pmt#1{{\tt #1}}



%----------------------------------------------------------------------------------------------------
%----------------------------------------------------------------------------------------------------

\pdfdest name {TitlePage} xyz
\pdfoutline goto name {TitlePage} count 0 {Title page}%


\centerline{\fPbxiv Charles University in Prague}
\centerline{\fPbxiv Faculty of Mathematics and Physics}
\vskip2	cm
\centerline{\fPbxx DOCTORAL THESIS}
\vskip1cm
\centerline{\IncludeSizedGraphics{5cm}{fig/mffLogo.pdf}}
%\vskip5cm
\vskip3cm
\centerline{\fPbxiv Jan Ka\v spar}
\vskip1cm
\centerline{\fPbxx Elastic scattering at the LHC}
\centerline{\fPbxx\FootText}
\vskip2cm
\centerline{\fPbxiv Institute of Particle and Nuclear Physics}
\ialign to\hsize{\tabskip 0pt plus1fil\strut #&\hfil\fPbxiv #\ \tabskip0pt&\fPbxiv #\hfil\tabskip 0pt plus1fil\cr
&Supervisors:		& RNDr. Vojt\v ech Kundr\' at, DrSc.\cr
&					& Mario Deile, PhD\cr
&Study Programme:	& Physics\cr
&Specialization:	& Subnuclear Physics\cr
}
\vskip2cm
\centerline{\fPbxiv Prague 2011}


\vfil\eject

%----------------------------------------------------------------------------------------------------

\line{\hss To my family -- for their amazing support}
\bgroup
\input utf8-csf
\line{\hss Mé rodině -- za jejich svkělou podporu}
\egroup

\vfil

The thanks...

\eject

%----------------------------------------------------------------------------------------------------

\hbox{}\vfil

I declare that I carried out this doctoral thesis independently, and only with the cited sources, literature and other professional sources.

I understand that my work relates to the rights and obligations under the Act No.~121/2000 Coll., the Copyright Act, as amended, in particular the fact that the Charles University in Prague has the right to conclude a license agreement on the use of this work as a school work pursuant to Section 60 paragraph 1 of the Copyright Act.

\vskip2\baselineskip

In Prague, date

\vskip2\baselineskip
\eject

%----------------------------------------------------------------------------------------------------

\hbox{}%

\pdfdest name {AbstractPage} xyz
\pdfoutline goto name {AbstractPage} count 0 {Abstract}%

\vfil

\bgroup
\input utf8-csf
\halign{\hbox to 5.3cm{\strut#\hss}&\vbox{\advance\hsize-5.3cm\noindent\ #}\cr
Název práce: 			& Elastický rozptyl na LHC\cr
Autor:					& Jan Kašpar\cr
Katedra:				& Ústav částicové a jaderné fyziky\cr
Vedoucí doktorské práce:& RNDr.~Vojtěch Kundrát, DrSc., Fyzikální ústav AV ČR, v.~v.~i.\cr
						& Mario Deile, PhD, CERN, Ženeva\cr
Abstrakt:				& [abstract of 80-200 words in Czech]\cr
Klíčová slova:			& elastický rozptyl protonů, TOTEM, římské hrnce, alignment\cr
}
\egroup

\vfil

\halign{\hbox to 5.4cm{\strut#\hss\ }&\vtop{\advance\hsize-5.4cm\noindent#}\cr
Title:		& Elastic scattering at the LHC\cr
Author:		& Jan Ka\v spar\cr
Departement:& Institute of Particle and Nuclear Physics\cr
Supervisors of the doctoral thesis:& RNDr.~Vojt\v ech Kundr\' at, DrSc., Institute of Physics AS \v CR\cr
			& Mario Deile, PhD, CERN, Geneva\cr
%Abstract:	& [abstract of 80-200 words in English]\cr
Abstract:	& 
The elastic scattering of protons is the simplest $\rm pp$ interaction possible, however, it still presents a challenge for the theory. In this thesis we discuss the elastic scattering from theoretical as well as experimental point of view. In the theory part, we present several models and their predictions for the LHC. We also discuss the Coulomb-hadronic interference, where we present a new eikonal calculation to all orders of $\al$, the fine-structure constant. In the experimental part we introduce the TOTEM experiment which is dedicated, among others, to the measurement of the elastic scattering at the LHC. This measurement is performed primarily with Roman Pot (RP) detectors -- movable beam-pipe insertions hundreds of meters from the interaction point, that can detect protons scattered to very small angles. We discuss some aspects of the RP simulation and reconstruction software. A great detail is devoted to the technique of RP alignment -- a method of determination of the relative positions among the RP sensors and their position with respect to the beam. The thesis terminates with a presentation of the analysis of the first elastic scattering measurement at the LHC, made by TOTEM.

\cr
%Keywords:	& [ 3-5 keywords in English]\cr
Keywords:	& elastic scattering of protons, TOTEM, Roman Pots, alignment\cr
}

\vfil
\eject\forceoddpage

%----------------------------------------------------------------------------------------------------

\InsertToc

\eject\forceoddpage

%----------------------------------------------------------------------------------------------------

\BeginText

\input introduction.tex
\input elastic.tex
\input totem.tex
\input simu_reco.tex
\input alignment.tex
\input first_el_measurement.tex
\input conclusion.tex

%----------------------------------------------------------------------------------------------------

\SpecialChapter{Bibliography}
\PrintReferences

%----------------------------------------------------------------------------------------------------

\SpecialChapter{List of Tables}
\PrintListOfTables

%----------------------------------------------------------------------------------------------------

\SpecialChapter{List of Abbreviations}

\DefAbb{ALFA}{}{}
\DefAbb{ATLAS}{}{}
\DefAbb{BFKL}{Balitsky-Fadin-Kuraev-Lipatov}{Resumation of gluon ladders leading to a ``hard pomeron'', see e.g. Ch.~8 in [\abref{barone}].}
\DefAbb{BLM}{Beam Loss Monitor}{}
\DefAbb{BPM}{Beam Position Monitor}{}
\DefAbb{CERN}{Conseil Europ\' eenne pour la Recherche Nucl\' eaire}{European Organization for Nuclear Research in Geneva.}
\DefAbb{CKL}{corrected Kundr\' at-Lokaj\' i\v cek}{Refers to the Coulomb-interference formula Eq.~\ref{el phase CKL}.}
\DefAbb{CMS}{Center Of Mass reference system}{}
\DefAbb{HP}{Hard Pomeron}{A version of the model of Islam et al., see Sec.~\sref{el models}.}
\DefAbb{IR}{InfraRed}{}
\DefAbb{KL}{Kundr\' at-Lokaj\' i\v cek}{Refers to the Coulomb-interference formula Eq.~\ref{el phase KL}.}
\DefAbb{LHC}{Large Hadron Collider}{The largest collider in the world.}
\DefAbb{LxG}{Low-$x$ gluons}{A version of the model of Islam et al., see Sec.~\sref{el models}.}
\DefAbb{LVDT}{Linear Voltage Differential Transformator}{A device that determines positions of TOTEM Roman Pots.}
\DefAbb{OPE}{One Photon Exchange}{An approximation in QED.}
\DefAbb{QED}{Quantum ElectroDynamics}{}
\DefAbb{QFT}{Quantum Field Theory}{}
\DefAbb{QCD}{Quantum ChromoDynamics}{}
\DefAbb{QM}{Quantum Mechanics}{}
\DefAbb{RMS}{Root Mean Square}{}
\DefAbb{RP}{Roman Pot}{A a pot-shaped movable beam-pipe insertion, see Sec.~\sref{ttm det}.}
\DefAbb{SWY}{Simplified West-Yennie}{Refers to the Coulomb-interference formula Eq.~\ref{el phase SWY}.}
\DefAbb{TDR}{Technical Design Report}{}
\DefAbb{TOTEM}{TOTal and Elastic Measurements}{An LHC experiment focused on the forward hadronic phenomena.}
\DefAbb{TPE}{Two Photon Exchange}{An approximation in QED.}
\DefAbb{WY}{West-Yennie}{Refers to the Coulomb-interference formula Eq.~\ref{el phase WY}.}

\PrintAbbreviations

\EndText

\end
