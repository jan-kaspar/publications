\chapter[ttm]{The TOTEM experiment}

% Approach: why, what, how

The \abb{TOTEM} experiment \bref{totem04,totem08,totem10} is dedicated to measurement of forward hadronic phenomena at the \abb{LHC}. The tree pillars of its physics programme are:
\> elastic scattering measurement in a wide $t$-range,
\> total cross-section measurement and
\> a study of soft and hard diffraction.
\vskip\itskip

We have discussed several theoretical/phenomenological descriptions of the elastic scattering in \Sc{el}. Looking at the model predictions in \Fg{el mod dsdt large}, one can see differences of several orders of magnitude. \abb{TOTEM} plans to measure the differential cross-section in the $|t|$-range from about $10^{-3}\un{GeV^2}$ (the domain of Coulomb-hadronic interference) to more than $10\un{GeV^2}$. This large coverage can be achieved by exploiting several running scenarios -- see the end of \Sc{rp measurement}. The discriminative power of such a measurement can already be seen in the first \abb{TOTEM} results, see \Fg{ttm mod cmp dsdt} in the conclusion chapter.

\fig{fig/pdf/ttm_sigma_tot.pdf}{sigma tot}{The total cross-section as a function of the centre-of-mass energy. The green ($\rm\bar pp$) and blue ($\rm pp$) hollow dots represent the data from PDG \bref{pdg10}. For some of the points we have marked the source experiments (the vertical order of the labels respects the vertical order of the points). The red dot is a recent result from \abb{TOTEM} \bref{totem11-2}. This new measurement is not included in the COMPETE fits \bref{cudell02}. The solid black line represents their best fit, the dashed lines show the total error band from all models considered. The magenta cross represents the targeted precision of the \abb{TOTEM} experiment.}

As much as the model predictions differ for the elastic cross-section, they differ for the total-cross section. From Tab.~2 in \bref{kklp11} one can read a variation between $95$ and $110\un{mb}$ for the energy of $14\un{TeV}$. Extrapolations from lower energy measurements yield an even higher uncertainty: from about $85\un{mb}$ to $120\un{mb}$ (see the dashed lines in \Fg{sigma tot}). \abb{TOTEM} plans to measure the total cross-section with the luminosity independent method, see \Sc{ttm tcs} for details. This method takes as an input the elastic and inelastic event rates and thus is closely related to the other two items of the physics programme. \abb{TOTEM} aims at a final precision of about $1\percent$ (see the magenta cross in \Fg{sigma tot}), which would provide a decisive answer to many total cross-section related hypotheses.

\Fg{ttm processes} shows typical event topologies for certain diffractive process classes that \abb{TOTEM} plans to measure. Looking at the cross-section estimates, one can see that these processes constitute a large part of the total cross-section. A characteristic property of these processes is the presence of scattering products in very forward directions, alternatively described by high pseudorapidities
\eqref{\et = - \log \tan {\th\over 2}\ .}{pseudorapidity}

\fig{fig/pdf/ttm_processes.pdf}{ttm processes}{Diffractive process classes with a cross-section estimate for the center-of-mass energy $\sqrt s = 14\un{TeV}$. For each class there are two types of diagrams (using the same color code). Left: a Feynman-like diagram showing the nature of the process, with Pomeron exchange (the double lines) as an effective description of the diffraction phenomena. Single external lines denote protons, the triple outgoing lines represent proton dissociation. Right: a sample hit map in the pseudorapidity ($\et$) vs.~azimuthal angle ($\ph$) space.}


\abb{TOTEM}'s physics programme thus sets special requirements for the detector apparatus. In particular large rapidity coverage - to detect most fragments from inelastic collisions and excellent acceptance for outgoing elastic and diffractive protons. To accomplish this task, \abb{TOTEM} comprises three sub-detectors: the tracking \em{telescopes \abb{T1} and \abb{T2}} and a system of \em{Roman Pots} (\abb{RP}s). The role of the telescopes is to detect the products of inelastic collisions. They are able to reconstruct many tracks per event and extrapolate them to the interaction point (\abb{IP}), which is essential for background rejection. The telescopes were designed to withstand the high radiation load expected at their location. The role of the \abb{RP} system is to detect the surviving protons. As indicated in \Fg{ttm processes}, these are scattered to very small angles. That is why the \abb{RP}s are installed far from the \abb{IP} (scattered protons can gain larger deviations from the beam) and can be approached very close to the beam (to detect protons with small deviations, that is with small scattering angles). For safety reasons, no detectors are allowed to be inside the beam-pipe before the beam is stable. This means that the \abb{RP}s must be movable beam-pipe insertions, retracted to a safe position during beam injection and adjustments and approached close to the beam for measurements. All three sub-detectors are trigger capable which is crucial for the total cross-section measurement.



%\TODO{add the phi vs.~eta diagram to explain the forwardness?}
%\fig{fig/pdf/ttm_acceptance_overview.pdf}{ttm acceptance overview}{acceptance overview}




\section[ttm det]{Detector apparatus}

\Fg{ttm det overview} shows an overview of the detector apparatus of \abb{TOTEM}. All detectors are placed symmetrically about the interaction point 5 which is shared with the \abb{CMS} experiment.
%See \Fg{lhc} (in the introduction chapter) for the geographical location of \abb{IP}5 and the orientation of the sectors 45 and 56.

\fig{fig/pdf/ttm_det_overview.pdf}{ttm det overview}{An overview of the \abb{TOTEM} detectors. All detectors are installed symmetrically left (sector 45) and right (sector 56) of the interaction point 5 (\abb{IP}5). Only the right arm (56) is shown in the figure. Bottom part: the structure of the \abb{LHC} around \abb{IP}5. The vertical boxes represent quadrupole (Q) and dipole (D) magnets.
%D1 is the separation magnet -- splitting each beam to its proper beam-line, see the \rhs~of the figure.
The \abb{RP} stations at $147$ and $220\un{m}$ are shown by red dots (each dot represents a unit). Upper box: zoomed region (see the guiding dashed lines) around the \abb{IP}5 -- the telescopes (red) \abb{T1} and \abb{T2} embedded into the \abb{CMS} detector (gray). The \abb{T1} telescopes are installed inside the \abb{CMS} end-caps, the \abb{T2} telescopes after the \abb{CMS} HF calorimeters.
}

\block{Telescopes T1}

The \abb{T1} telescopes are installed inside the \abb{CMS} end-caps, they extend in the forward areas from $7.5$ to $10.5\un{m}$ on both sides of the interaction point. They cover pseudorapidity regions $3.1 < \et < 4.7$.

There is one telescope (see \Fg{t1 installed} on each side of \abb{IP}5). Each telescope consists of two quarters (see \Fg{t1 quarter}). Each quarter is formed by 5 planes equally spaced in $z$ (along the beam pipe). Each plane includes 3 trapezoidal Cathode Strip Chamber (\abb{CSC}) detectors, each covering about $60^\circ$ in azimuth. There is $3^\circ$ rotation and overlap between the adjacent planes.

\bmfig
\fig[,5cm]{fig/external/t1_installed.jpg}{t1 installed}{[7.5cm]Installation of a \abb{T1} telescope into a \abb{CMS} end-cap. The telescope is opened -- split into two quarters left and right of the beam pipe.}
\fig[,5cm]{fig/external/t1_quarter.jpg}{t1 quarter}{[6.6cm]A \abb{T1} quarter contains 5 planes, each of them consist of 3 \abb{CSC} detectors.}
\emfig

The \abb{CSC} detectors are gaseous detectors (multi-wire proportional chambers) with both cathode plates segmented into strips. These strips and the anode wires are oriented at $60^\circ$ with respect to each other. Thus there are 3 read-out coordinates, which enables reconstruction of events with multiple tracks. The cathode strips are $4.5\un{mm}$ wide and placed with $5\un{mm}$ pitch. The anode wires, used also for triggering, have a diameter of $30\un{\mu m}$ and are installed with $3\un{mm}$ pitch. The gas gap is as wide as $10\un{mm}$ filled with $\rm Ar/CO_2/CF_4$ mixture. The size of the detectors is growing with increasing distance from the \abb{IP}, reaching the maximum of $\approx 1\un{m} \times 0.68\un{m}$. The spatial resolution of one \abb{CSC} detector is about $1\un{mm}$. For further details we refer the reader to Sec.~5.2 in \bref{totem08}.



\block{Telescopes T2}

The \abb{T2} telescopes are installed in the forward shielding of \abb{CMS} behind the HF and before the Castor calorimeter. They are centered at about $13.5\un{m}$ from the interaction point. The \abb{T2} telescopes cover the pseudorapidity region $5.3 < \et < 6.5$.

There is one telescope (see \Fg{t2 installed}) on each side of \abb{IP}5. Each telescope consists of two quarters. As shown in \Fg{t2 quarter}, each quarter includes 10 semicircular planes assembled back-to-back in 5 pairs. The distance between two adjacent pairs is $91\un{mm}$. Each plane is equipped with a Gas Electron Multiplier (\abb{GEM}) detector. Each detector has an azimuthal coverage of $192^\circ$, therefore there is an overlap between the two quarters within each telescope.

The \abb{GEM} detectors are gaseous detectors where the electron multiplication is realized by three perforated Cu-clad polyimide foils, separated by $2\un{mm}$. The inner radius of the perforation holes is $65\un{\mu m}$, their density is roughly $50/\rm mm^2$. With an electric field about $3.6 \un{kV/cm}$ between the \abb{GEM} foils, the overall gas amplification is roughly $8000$. The \abb{GEM} detectors have double read-out: strips (radial coordinate) and pads (coarse radial and azimuthal). The strip radii range from $43$ to $144\un{mm}$, their width and spacing is $80$ and $400\un{\mu m}$ respectively. The pads, used for triggering, are of sizes from $2\times 2\un{mm^2}$ to $7\times 7\un{mm^2}$. The spatial resolution is about $0.15\un{mm}$ in the radial direction and $0.8^\circ$ in the azimuthal one. Further details can be found in Sec.~5.3 in \bref{totem08}.

% gas detector, radiation hard, high rate, good spatial, and timing resolution


\bmfig
\fig[,5cm]{fig/external/t2_installed.jpg}{t2 installed}{[7.5cm]A \abb{T2} telescope installed in the shielding of a \abb{CMS} HF calorimeter.}
\fig[,5cm]{fig/external/t2_quarter.jpg}{t2 quarter}{[7.4cm]A \abb{T2} quarter consists of 5 planes with 2 back-to-back mounted \abb{GEM} detectors.}
\emfig


\block{The Roman Pot system}

The system of Roman Pots (\abb{RP}s) consists of four stations as shown in \Fg{ttm det overview}. Each station (see \Fg{rp station}) is formed by two units. Each unit includes horizontal, top and bottom \abb{RP} and a beam position monitor (\abb{BPM}), see \Fg{rp unit}. A Roman Pot itself is shown in \Fg{rp pot}. In order to minimize the material budget in the path of detectable protons, the \abb{RP} walls are replaced by the thin window around the sensors. Its front face is $500\un{\mu m}$ thick, the face exposed to the beam only $150\un{\mu m}$. Inside each \abb{RP}, there is a detector package (\abb{DP}) of 5 back-to-back mounted pairs of hybrids, see \Fg{rp package}. A hybrid is an electronic board carrying an edgeless silicon sensor and four read-out (\abb{VFAT}) chips, as shown in \Fg{rp hybrid}. Each \abb{RP} station is followed by a series of beam loss monitors (\abb{BLM}), the role of which is to monitor the beam losses induced by the \abb{RP} operations. One of the nearest \abb{BLM}s is shown in \Fg{rp blm bpm}. Further details on the \abb{RP} system can be found in Ch.~4 in \bref{totem08}.

\bmfig
\fig{fig/pdf/rp_station.pdf}{rp station}{[7.3cm]A \abb{RP} station installed in the tunnel. The green frames show the two units (only top and horizontal pots are visible)}
\fig{fig/pdf/rp_unit.pdf}{rp unit}{[7.3cm]A \abb{RP} unit with horizontal, top and bottom \abb{RP} and a \abb{BPM}.}
\emfig

\bmfig
\fig{fig/pdf/rp_pot.pdf}{rp pot}{[6.8cm]A Roman Pot with a thin window around the silicon sensors.}
\fig{fig/pdf/rp_package.pdf}{rp package}{[7cm]A package of 10 detector hybrids mounted on a \abb{RP}.}
\emfig

\bmfig
\fig{fig/pdf/rp_hybrid.pdf}{rp hybrid}{[7cm]A detail of a hybrid zoomed on the silicon detector and four \abb{VFAT} chips \bref{vfat}.}
\hskip5mm
\fig{fig/pdf/rp_blm_bpm.pdf}{rp blm bpm}{[7cm]A \abb{BPM} and a \abb{BLM} at a \abb{RP} station. A piece of a top \abb{RP} can be seen in the top-left corner. The beam pipes can be seen in the bottom-right corner (outgoing beam top, incoming bottom).}
\emfig

In order to minimize the distance between the \abb{RP} edge facing the beam and the sensitive detector volumes, the \abb{RP}s are equipped with edgeless silicon detectors with current terminating structure (\abb{CTS}) \bref{ruggiero07}. This structure consists two rings. The current terminating ring collects the current generated in the highly damaged region at the cut edge (see \Fg{rp hybrid}), avoiding its diffusion into the sensitive volume. The clean-up ring prevents any further diffusion of this edge current into the sensitive area. With this technology the insensitive edge is reduced to about $50\un{\mu m}$.

The silicon sensors are single-sided $\rm p^+$-$\rm n$ micro-strip detectors with $512$ strips at a pitch $P = 66\un{\mu m}$, processed on very high resistivity
%($> 10\un{k\Om\ cm}$)
$n$-type silicon wafers. The strips are oriented at $45^\circ$ with respect to the edge facing the beam (the cut edge). The biasing voltage is of the order of $100\un{V}$. For better performance, the sensors are cooled down to a temperature about $-20^\circ\rm C$. The $512$ strip are read out by four \abb{VFAT} chips (see \bref{vfat}), that is $128$ channels each. For more details on the silicon sensors, we refer the reader to Sec.~4.3 in \bref{totem08}.

In order to refer to any component of the \abb{RP} system a naming scheme was established in \bref{totem-rp-naming}. Since we will use it in what follows, we summarize the scheme in \Tb{rp naming scheme}. For example, the 2nd plane in the near horizontal \abb{RP} in the $220\un{m}$ station in the right arm can be referred as 56-220-near-hor-2.

\tab[\bstrut\qquad#\quad\hfil&\quad#\quad\hfil&\quad#\qquad\hfil\cr]{rp naming scheme}{The \abb{RP} naming scheme. Far and near is to be understood with respect to the interaction point. Plane 1 is the nearest one to the \abb{IP}.}{\ln
sector (arm)	& 45 (left), 56 (right)		\cr
station			& 147, 220		\cr
unit			& far, near		\cr
pot				& top, bot = bottom, hor = horizontal	\cr
plane (sensor)	& 1 to 10		\cr
vfat			& 1 to 4		\cr
strip			& 0 to 127		\cr
\ln
}

In the following chapters we will make important considerations based on the proportions of the \abb{RP} system components. That is why we review the important dimensions now. The sensors have the shape of a square (edge about $3.6\un{cm}$) with one corner cut off (the length of the cut edge is about $2.2\un{cm}$). The spacing between the sensors in a package is summarized in \Tb{ttm det package}. The total width of the package is about $4.1\un{cm}$. \Tb{ttm rp station} summarizes the \abb{RP} distances from the interaction point. One can see that the length of the $220\un{m}$ ($147\un{m}$) station is about $5.4\un{m}$ ($1.7\un{m}$).

\htab{ttm det package}{The longitudinal positions (along the beam axis) of the silicon sensors in a detector package, in millimeters and relative to the center of the \abb{RP}.}{\bln
\hbox{sensor/plane}	& 1 & 2 & 3 & 4 & 5 & 6 & 7 & 8 & 9 & 10 \cr\bln
\hbox{position}		& -20.3 & -15.7 & -11.3 & -6.7 & -2.3 & 2.3 & 6.7 & 11.3 & 15.7 & 20.3 \cr\bln
}

\htab{ttm rp station}{The distances (in meters) of the \abb{RP} centers from the \abb{IP}5.}{
\omit&\multispan6\bhrulefill\cr
\omit&\multispan3\strut\bvrule\hfil near unit\hfil&\multispan3\vrule\hfil far unit\hfil\cr
\omit&\multispan6\bhrulefill\cr
\omit&\omit\bvrule\strut\hfil\hbox{top}\hfil & \hbox{bottom} &\hbox{horizontal} &\hbox{horizontal} &\hbox{top} &\hbox{bottom} \cr\bln
\hbox{147 m station} & 148.784 & 148.784 & 149.234 & 150.026 & 150.476 & 150.476 \cr\ln
\hbox{220 m station} & 214.628 & 214.628 & 215.078 & 219.550 & 220.000 & 220.000 \cr\bln
}

\Fg{rp station 3d} shows the geometry of the \abb{RP} stations. It defines the $xyz$ coordinate system which is bound to the station. The $z$ axis overlaps with the beam axis, the horizontal $x$ axis points outside the \abb{LHC} ring and the $y$ axis points up (against gravity). We have also defined $U$ and $V$ directions, which are obtained by rotating the $x$ and $y$ directions by $45^\circ$. The $U$ and $V$ directions are useful since they describe the read-out and/or strip directions of the silicon sensors. Besides these two coordinate systems, the figure shows the $uv$ reference frames. Each of them is bound to a sensor. Its origin overlaps with the center of the sensor, the $u$ axis is parallel to the strips, the $v$ axis is perpendicular to them and hence gives the \em{read-out direction}. Due to the back-to-back sensor assembly, the odd and even planes have the $u$ and $v$ axes swapped. A similar situation takes place for the sensors' center offsets (red arrows, about $0.5\un{mm}$) from the package axes (dotted lines).

Later on, we will call \em{$V$ sensors} all the sensors which have their (nominal -- neglecting misalignment) read-out directions parallel or anti-parallel to the $V$ direction. Similarly for $U$ sensors.

\fig[14cm]{fig/pdf/rp_station_3d.pdf}{rp station 3d}{The geometry of \abb{RP} stations (only the first two planes/sensors of the 56-220-near unit are drawn). The proportions have been modified for graphical reasons, they are not to scale. The blue line marks the beam axis. The red arrows mark the offsets of sensors' centers from the package axes (slightly inclined dotted lines). The $v$ vectors give the read-out directions. The green lines show the distances between sensors' cut edges and the beam axis.
}

Since the position of the \abb{RP}s is critical for all \abb{RP} operations, every \abb{RP} module is equipped with two independent devices to determine its position: a motor-step counter and a linear voltage differential transformer (\abb{LVDT}) \bref{totem-lvdt}. These devices are calibrated to give the distance between the face exposed to the beam of the thin window (see \Fg{rp pot}) and the beam axis. If the gap between the thin window and the sensors' edges is known, one can calculate the distances from the beam to the sensors' cut edges (green lines in \Fg{rp station 3d}). These are the quantities relevant for the track reconstruction.

%\> RP approach -- typical $10\si$, define beam sigma!



\section[rp measurement]{Proton Measurement with Roman Pot detectors}

\fig{fig/pdf/ttm_proton_transport.pdf}{ttm proton transport}{A scheme of the proton transport from the interaction point to a \abb{RP} station. $y_{\rm N}$ and $y_{\rm F}$ are the hit points in the near and far units of the station.}


Imagine a collision which takes place at a \em{vertex}
\eqref{(x^*, y^*, s^*)^\T}{ttm ip vertex}
and where a proton is emitted, as sketched in \Fg{ttm proton transport}. One may relate the momentum $p^*$ of the outgoing proton to the nominal beam momentum
\eqref{p^* = p_{\rm nom} (1 - \xi^*)\ ,}{ttm xi}
where $\xi^*$ is the relative \em{momentum loss}. The direction of the proton is described by the \em{scattering angle} $\th^*$ and the \em{azimuthal angle} $\ph^*$:
\eqref{\vec p^* = p^* \pmatrix{\sin\th^* \cos\ph^*\cr \sin\th^*\sin\ph^* \cr \cos\th^*}
%=  p \pmatrix{\th_x\cr \th_y\cr \sqrt{1 - \th_x^2 - \th_y^2}}
\ .}{ttm mom comp}
The coordinate system $(x, y, s)$ is illustrated in \Fg{ttm proton transport}, the $s$ axis represents the beam axis.

The protons of our interest are scattered to low angles and therefore stay close to the outgoing beam as they enter the \abb{LHC} magnet lattice. When the outgoing proton reaches the region of a \abb{RP} station, it has momentum $p$, direction described by angles $\th$ and $\ph$ and may hit the detectors at points $y_{\rm N}$ and $y_{\rm F}$ just as shown in \Fg{ttm proton transport}.

We will use the notation that quantities related to the \abb{IP} will be denoted with a star superscript, those without will be related to \abb{RP} stations. The energy of the protons is conserved during the \abb{LHC} transport from the \abb{IP} to \abb{RP} stations, therefore $p = p^*$ and $\xi = \xi^*$. Thus we will not make distinction between their \abb{IP} and \abb{RP} values.

It will be useful to define the \em{projections of the scattering angle} (both for \abb{IP} and \abb{RP}-station quantities)
\eqref{\th_x = \th \cos\ph\ ,\qquad \th_y = \th \sin\ph\ .}{ttm th x y}
Similarly, one can define the projections of the momentum transfer $t$:
\eqref{t_x = t \cos^2\ph^*\ ,\qquad t_y = t \sin^2\ph^*\ .}{ttm t x y}
This definition follows from the low-scattering-angle approximation in \Eq{el t}, which we will commonly use later on (the scattering angles of interest are at most of the order $10^{-3}\un{rad}$, see e.g.~\Fg{el mod dsdt large}). The definitions \Eq{ttm th x y,ttm t x y} yield relations
\eqref{t = t_x + t_y = - p^2 \th^{*2} = - p^2 (\th_x^{*2} + \th_y^{*2})\ .}{ttm t th}

Since the magnetic field within the \abb{RP} stations is negligible, protons follow a straight trajectory there. The relation between \abb{RP} sensor measurements and the parameters of the trajectory is therefore simple (for details see \Sc{al psi}). What is much more interesting is the proton transport from the \abb{IP} to the \abb{RP} stations. We will discuss it in the rest of this section.

The transverse motion of protons in the \abb{LHC} can be described by a solution of the Hill's equation (see e.g.~Sec.~3.3 in \bref{wilson} or Ch.~4 in \bref{dolezal}):
\eqref{x(s) = \sqrt{\be(s)\, \ep}\ \cos\big(\phi(s) + \phi_0\big)\ .}{ttm hill sol}
We have written the solution for the horizontal direction only, for the vertical it is analogous. The solution has a form of oscillations with the amplitude given by the \em{emittance} $\ep$ (defining the size and divergence of the beam) and the \em{betatron function} $\be(s)$ (reflecting the layout and strengths of the \abb{LHC} magnets). The \em{phase} of the oscillations is given by $\phi(s)$ which is related to $\be(s)$ by $\d\phi/\d s = 1/\be(s)$. The emittance is usually expressed in terms of the \em{normalized emittance} $\ep_{\rm N} = \be\ga \ep$, where $\be$ and $\ga$ are the relativistic factors, see \Eq{sm lorentz}. The advantage of the normalized emittance is that it does not change during beam acceleration, see Sec.~4.1 in~\bref{wilson}.

From \Eq{ttm hill sol} one can read off the amplitude of the oscillations, which effectively determines the beam width:
\eqref{\si(s) = \sqrt{\be(s)\ep}\ .}{ttm beam sigma}
This quantity is often called a \em{beam sigma}. The beam size is of a special interest at the interaction point -- it characterizes the vertex distribution. Denoting the value of the $\be(s)$ function at the interaction point as $\be^*$, one can write
\eqref{\si_{x^*} = \sqrt{\be^* \ep}\ .}{ttm vertex sm}
By differentiating \Eq{ttm hill sol} one can obtain the $s$-evolution of proton angles $\th_x = \d x/\d s$. Again, one can extract the amplitude of its oscillations, that is a measure of the beam angular divergence. Evaluating it at the \abb{IP} yields
\eqref{\si_{\th^*} = \sqrt{\ep\over\be^*}\ .}{ttm beam div}

The solution \Eq{ttm hill sol} was derived for protons of the nominal momentum $p_{\rm nom}$. If the actual proton momentum differs by $\De p = \xi\, p_{\rm nom}$, one has to add a term $D(s) \xi$ to the solution, see Sec.~5.2 in~\bref{wilson}. Then, for the proton transport from the \abb{IP} to a position $s$ one can write (see Sec.~5.2.4 in~\bref{wilson})
\eqref{x(s) =\ L_x(s, \ldots)\, \th_x^* + v_x(s,\ldots)\, x^* + D_x(s,\ldots)\,\xi}{ttm lin par}
and equivalently for the vertical direction. In the previous relation we have introduced the \em{effective length} $L$, the \em{magnification} $v$ and the \em{dispersion} $D$. We will refer to them as \em{optical functions}. These functions depend on the distance $s$ from the \abb{IP}, but may also depend on the variables $\th_x^*$, $x^*$ and $\xi$. For the optics of practical interest, however, $\xi$ is the only relevant dependence.

Strictly speaking, the form of \Eq{ttm lin par} is oversimplified. Due to higher order corrections, $x$ and $y$ projections are not strictly independent and consequently one should include also the dependence on $\th_y^*$ and $y^*$. For the foreseen optics these terms are negligible. However in reality, due to magnet misalignments (mostly their rotations), the coupling between $x$ and $y$ projections may become important, see e.g.~\Fg{al el plots yx}.

In practice the optical functions are calculated with dedicated computer programs such as MAD-X \bref{mad-x}. Since they might be too slow for large Monte Carlo (\abb{MC}) simulations, \abb{TOTEM}'s software (see \Sc{sr}) includes a module that calculates a polynomial parameterization of a MAD-X output, for details see Sec.~6.4 in \bref{hubert}.

For typical \abb{LHC} beams one may expect a beam-momentum offset
\eqref{\bar\xi = \O{10^{-3}}}{ttm energy off}
and the proton-to-proton momentum fluctuation
\eqref{\si_\xi = \O{10^{-4}}\ .}{ttm energy fl}

Since the optics \Eq{ttm lin par} relates the kinematics of a proton at the \abb{IP} and the \abb{RP} stations, it defines what protons can be detected with the (finite-size) \abb{RP} detectors. Formally, one could think of an \em{acceptance} function $A(\th_x^*, x^*, \th_y^*, y^*, \xi)$ which gives $1$ if such a proton can be detected and $0$ otherwise. One can also take an average over the quantities that are not of interest. This may lead, for example, to the acceptance $A(t, \xi)$, which gives the fraction of protons with momentum transfer $t$ and momentum loss $\xi$ that can be detected. Similarly one may define the \em{elastic acceptance} $A(t)$ as the detected ratio of the elastic events with momentum transfer $t$. Several examples of elastic acceptances for different optics are shown in \Fg{ttm elastic acceptance}.

%In an elastic measurement analysis (see e.g.~\Sc{felm}) one needs to correct for the limited acceptance. It is clear that when the acceptance is too low, the \em{acceptance correction} is large and 

\fig{fig/pdf/ttm_elastic_acceptance.pdf}{ttm elastic acceptance}{The elastic acceptances for three optics types. Calculated for $\sqrt{s} = 14\un{TeV}$ and with \abb{RP} sensors at $10\si + 0.5\un{mm}$. The horizontal line marks the points $|t_{30}|$ where the acceptances reach $30\percent$.}

The rich physics programme of \abb{TOTEM} requires several running scenarios, including several optics. We will now review the scenarios as they were originally foreseen for the center-of-mass energy $\sqrt s = 14\un{TeV}$. The main characteristics stay the same for the reduced energy of $7\un{TeV}$ -- see for example the optics used for the first elastic scattering measurement, presented in \Sc{felm}. We will classify the scenarios by the beta-function values at the interaction point $\be^*$.

\> The \em{high-$\be^*$} scenario is dedicated to the measurement of the total cross-section and low-$|t|$ elastic scattering. The optics with $\be^* = 1535\un{m}$ provides high effective lengths: $L_x$ up to $110\un{m}$ and $L_y$ up to $270\un{m}$ (both in the $220\un{m}$ stations). This ensures a good $t$-resolution (see \Sc{elr 1535}). To improve it further, this scenario counts on a reduced normalized emittance of $\ep_{\rm N}=1\un{\mu m\ rad}$. This leads to a very low beam divergence at the \abb{IP} $\si_{\th^*} = 0.3\un{\mu rad}$. The expected beam sizes (i.e.~the beam sigmas) at the $220\un{m}$ stations are $\si_x = 0.03\un{mm}$ and $\si_y = 0.085\un{mm}$. The beam size, quite generally, determines the approach of the \abb{RP}s to the beam -- a typical approach is about $10\si$ in each direction. The resulting elastic acceptance is drawn in red in \Fg{ttm elastic acceptance}. The acceptance reaches $30\percent$ at $|t_{30}| = 2\cdot10^{-3}\un{GeV^2}$. This scenario will use a rather low luminosity ${\cal L} \ls 10^{28} \un{cm^{-2}s^{-1}}$.

\> The \em{medium-$\be^*$} scenario is based on an universal optics with $\be^* = 90\un{m}$. It combines the properties of high and low $\be^*$ scenarios allowing for measurements of the total cross-section, medium-$|t|$ elastic scattering and all diffractive processes. In the $220\un{m}$ stations the horizontal effective length $L_x$ is rather small, below $2.9\un{m}$, while in the vertical direction $L_y$ goes up to $260\un{m}$, which is similar to the high-$\be^*$ optics. Moreover, $L_x$ practically vanishes in the far units, which simplifies the distinction between elastic and inelastic protons (the horizontal elongation can only come from proton's momentum loss, cf.~\Eq{ttm lin par}). The standard normalized emittance $\ep_{\rm N} = 3.75\un{\mu m\ rad}$ was planned to be used (nowadays the \abb{LHC} reaches emittance values as low as $2\un{\mu m\ rad}$), leading to a beam divergence of about $2.4\un{\mu rad}$. The expected beam sizes at the $220\un{m}$ stations are $\si_x = 0.4\un{mm}$ and $\si_y = 0.6\un{mm}$. In other words, the pots will have to stay much further from the beam center than in the case of the high-$\be^*$ optics. Consequently, the elastic acceptance (blue histogram in \Fg{ttm elastic acceptance}) starts at much higher $|t|$ -- it reaches $30\percent$ only at $4\cdot 10^{-2}\un{GeV^2}$. This scenario counts on a luminosity of ${\cal L} \ls 10^{28} \un{cm^{-2}s^{-1}}$.

\> The \em{low-$\be^*$} scenarios ($\be^*$ between $0.5$ and $3.5\un{m}$) were designed for diffractive studies and large-$|t|$ elastic scattering measurements. They are optimized for high luminosities ${\cal L} \approx 10^{32} \hbox{ -- } 10^{33}\un{cm^{-2}s^{-1}}$ and not for forward proton detection -- the effective lengths at $220\un{m}$ stations do not exceed $L_x = 2\un{m}$ and $L_y = 20\un{m}$. With the standard normalized emittance $\ep_{\rm N} = 3.75\un{\mu m\ rad}$ the beam divergence is rather large: $\si_{\th^*} \approx 16\un{\mu rad}$. The beam sizes at the $220\un{m}$ are about $\si_x = 0.1\un{mm}$ and $\si_y = 0.3\un{mm}$. The elastic acceptance (green histogram in \Fg{ttm elastic acceptance}) reaches $30\percent$ at $|t_{30}| = 2.5\un{GeV^2}$.

\fig{fig/pdf/ttm_hit_distribution.pdf}{ttm hit distribution}{Simulated distributions of elastic-scattering protons in the 56-220-far unit, each color corresponds to an optics type. The gray ellipses show $10$ beam sigma contours. The \abb{RP} sensors are placed at $10\si + 0.5\un{mm}$.}

\Fg{ttm hit distribution} illustrates the properties of the three types of optics. The horizontal (vertical) spread of hits is determined by the horizontal effective length $L_x$ (the vertical $L_y$). That is why the $\be^* = 90$ and $2\un{m}$ optics yield very narrow elastic scattering hit distributions. The gray ellipses show the $10\si$ beam contours, thereby controlling the positions of the \abb{RP}s.




\section[ttm tcs]{The total cross-section}

One of the main items on the physics programme of \abb{TOTEM} is the measurement of the total cross-section. This implicitly means the total cross-section due to the strong (hadronic) interaction only. If the electromagnetic interaction were included, the total cross-section would be infinite (consequence of the long range of the electromagnetic force).

\abb{TOTEM} plans to use the luminosity-independent method. This method avoids using any luminosity measurement, which is usually subject to a large uncertainty. Instead, the method exploits the optical theorem. If one takes its ``spin-less'' approximation \Eq{el spinless si} and combines it with the definition of luminosity $N = {\cal L} \si$ (relating an event rate $N$ to the corresponding cross-section $\si$) and with the decomposition of the total cross-section to the elastic and inelastic part $\si_{\rm tot} = \si_{\rm el} + \si_{\rm inel}$, one obtains:
\eqref{
	\si_{\rm tot} = {1\over 1+\rh^2} {\d N_{\rm el}/\d t|_{t=0}\over N_{\rm el} + N_{\rm inel}}
	\ ,\qquad
	{\cal L} = (1+\rh^2) {(N_{\rm el} + N_{\rm inel})^2\over \d N_{\rm el}/\d t|_{t=0} }\ .
}{ttm li meth}
Here, $\rh = \Re F(0) / \Im F(0)$ (cf.~\Eq{el rh}), i.e.~it gives the ratio of the real to the imaginary part of the elastic scattering amplitude $F$ at $t = 0\un{GeV^2}$.

The approximation \Eq{el spinless si} is rather an expectation than a rigorously proved statement, thus some skepticism is due. A.~Martin questioned the negligibility of the spin effects in \bref{martin85}. He compared the $\si_{\rm tot}$ measurements from the ISR and $\rm SP\bar PS$, where several methods (including the luminosity-independent one) had been used. In both cases, he found all methods to give compatible results. However, the large luminosity uncertainty at the $\rm SP\bar PS$ still allows for non-negligible spin effects. It is thus difficult to draw any conclusion, especially in the view of extrapolating it to the \abb{LHC} energies. Therefore it will be important to use several $\si_{\rm tot}$-determination methods and compare the results also at the \abb{LHC}.

Coming back to the luminosity-independent method, \Eq{ttm li meth}, one can identify four inputs: $\d N_{\rm el}/\d t|_{t=0}$, $N_{\rm el}$, $N_{\rm inel}$ and $\rh$. The inelastic rate $N_{\rm inel}$ will be measured with the telescopes \abb{T1} and \abb{T2}, for details see Sec.~6.3 in \bref{totem08}. The value of $\rh$ may be determined by analyzing the elastic differential cross-section in the Coulomb-interference region (if acceptance permits), or it can be taken from an external source (such as the COMPETE fits \bref{cudell02}). Since the expected value of $\rh$ is about $0.14$ and since it enters the $\si_{\rm tot}$ expression as $1 + \rh^2$, the influence of any $\rh$ errors is small.

The differential elastic rate $\d N_{\rm el}/\d t$ at $t=0\un{GeV^2}$ can not be measured directly, it can only be extrapolated from the measured low-$|t|$ data. As discussed in \Sc{rp measurement}, there are two optics that extend to sufficiently low $|t|$ values: $\be^* = 1535$ and $90\un{m}$ (see the acceptance curves in \Fg{ttm elastic acceptance}). This extrapolation procedure is important also for the integral elastic rate $N_{\rm el}$ measurement as it allows to correct for the events missed due to the finite acceptance.

In the rest of this section, we will discuss the extrapolation techniques and show the results for the nominal $\be^* = 1535$ and $90\un{m}$ optics at $\sqrt s = 14\un{TeV}$.

Looking at the low-$|t|$ elastic cross-section predictions in \Fg{el mod dsdt narrow}, one can see an ``almost'' exponential decrease. This observation can be better quantified by plotting the exponential slope \Eq{el B}, see \Fg{ext B}. The slope is not strictly constant, but for most models (except the one of Islam et al.) its variation is not large up to $|t| \approx 0.25\un{GeV^2}$. Moreover, the shape of the curves suggests that they can be well approximated by a polynomial of a low degree (such as a parabola). Thus the modulus of the hadronic amplitude can be parameterized as:
\eqref{|\FH(t)| = \e^{M(t)}\ ,\qquad M(t) \hbox{ a polynomial.}}{ext M par}

\fig{fig/pdf/ext_B.pdf}{ext B}{The exponential slope as predicted by the models discussed in \Sc{el models}. The vertical dashed lines mark the acceptance limits ($|t_{30}|$ values) for each of the considered optics (see \Sc{rp measurement}).}

\bmfig
\fig{fig/pdf/ext_C.pdf}{ext C}{[7.2cm]The relative difference between cross-sections with and without the Coulomb interaction (included according to the \KL{} formula \Eq{el phase CKL}). The dashed lines mark the acceptance limits $t_{30}$.}
\fig{fig/pdf/ext_smearing.pdf}{ext smearing}{[7cm]The relative difference between cross-sections after and before the resolution smearing. The expected angular resolutions are shown in \Fg{elr 1535 dth} ($1535\un{m}$ optics) and \Fg{elr 90 dth} ($90\un{m}$ optics).}
\emfig

The parameterization \Eq{ext M par} is related to the hadronic amplitude $\FH$. However, the measured $t$-distribution relates to the cross-section due to both hadronic and Coulomb interactions: $\d\si^{\rm C+H}/\d t$ (see \Sc{el coulomb}). Moreover the experimental $t$-distribution is affected by the limited acceptance and detector resolution. These two effects need to be taken into account in the extrapolation procedure. Looking at the \Fg{ext C}, one may conclude that while the effect of the Coulomb interaction is important for the $1535\un{m}$ optics, it is practically negligible for the $90\un{m}$ optics (staying below $|t| \approx 0.25\un{GeV^2}$). It is the other way round for the smearing effects -- negligible for the $1535\un{m}$ and relevant for the $90\un{m}$ optics, see \Fg{ext smearing}.

\def\OutlineLabel{Extrapolation for beta* = 1535 m optics}
\def\TOCLabel{Extrapolation for $\be^* = 1535$ m optics}
\subsection{Extrapolation for $\be^* = 1535\un{m}$ optics}

Since the Coulomb-interaction effects are relevant for this optics and since these effects act on the level of amplitude (not cross-section, see \Sc{el coulomb}), we need to extend the parameterization \Eq{ext M par} to include also a phase component. \Fg{el mod phase} suggests that, for $|t| \ls 0.25\un{GeV^2}$, a polynomial parameterization is adequate again. Thus, the hadronic amplitude can be parameterized as:
\eqref{\FH(t) = \e^{M(t)} \e^{\i P(t)}\ ,\qquad M(t), P(t) \hbox{ polynomials.}}{ext MP par}

In the rest of this section we will present the main results of an old extrapolation study, made even before the \abb{TOTEM} offline software (see \Sc{sr}) existed. That is why we used only a simplified \abb{MC} simulation -- comparable with the fast simulation presented in \Sc{fast simu}, including a simulation of the beam smearing effects similar to \Sc{beam smearing}. The size of the \abb{MC} sample was chosen to correspond to about a day of data-taking with a realistic luminosity (see \Sc{rp measurement}).

The fitting procedure was based on the parameterization \Eq{ext MP par}. In principle this should be inserted into the Coulomb-interference formulae \Eq{el F CH decomp H lin ff,el phase CKL} and cross-section definition \Eq{el spinless si} in order to obtain a full fitting model. However, pursuing this approach would be too time consuming, mainly due to the numerical integrations in \Eq{el phase CKL}. Instead, we tried to separate the Coulomb effects in several iterations. At the beginning we fitted the cross-section data directly with the parameterization \Eq{ext MP par} (thus entirely neglecting the Coulomb interaction). With the results, we evaluated the Coulomb effects via \Eq{el F CH decomp H lin ff,el phase CKL}. These effects were consequently subtracted from the input cross-section data, thus effectively extracting the purely-hadronic cross-section. The subtracted data were then used as the starting point of the second iteration.

% corrections by iterations: data = U0 (U is uncorrected, C is corrected), guess C0 (e.g.~by taking C0 = U0), then iterate: Ci --> Ui --> corr(i) = Ci/Ui --> C(i+1) = U0 * corr(i) and again unless |C(i+1) - C(i)| < threshold

We made a number of fits, trying to find such fit settings that minimize the deviation of the extrapolated cross-section at $t=0\un{GeV^2}$ from its original value. The parameters to optimize included: the lever-arm (the lower $|t|_{\rm low}$ and upper bound $|t|_{\rm up}$ of the fit), the binning of the input data, the degrees of modulus $M(t)$ and phase $P(t)$ polynomials and the number of Coulomb-correction iterations. We aimed at performing this optimization in a ``model independent way'' -- we used five model variants from \Sc{el models} and tried to minimize the mean extrapolated-original deviation.

\fig{fig/pdf/ext_results_1535.pdf}{ext results 1535}{The extrapolation deviation of the cross-section at $t=0\un{GeV^2}$, as a function of the fit lower bound $|t|_{\rm low}$. The deviation is calculated as (extrapolated - original) / original, where e.g.~original denotes the original (true) value of $\d\si^{\rm H}/\d t|_{0}$.}

\Fg{ext results 1535} shows probably the most interesting result of the study -- the dependence of the extrapolation deviation on the lower bound of the fit. This extrapolation was performed with a cubic modulus polynomial $M(t)$, constant phase polynomial $P(t)$, one Coulomb-correction iteration and the upper bound $|t|_{\rm up} = 0.054\un{GeV^2}$. For most of the models, the deviation lies in a band $\pm 0.2\percent$. The increasing deviation of the model of Islam et al.~might be explained by its exponential slope behavior, see \Fg{ext B}. It is deviating most from a constant value and the polynomial approximation is less accurate than for the other models.

%\>> discuss the phase: the one percent result might be too good




\def\OutlineLabel{Extrapolation for beta* = 90 m optics}
\def\TOCLabel{Extrapolation for $\be^* = 90$ m optics}
\subsection{Extrapolation for $\be^* = 90\un{m}$ optics}

For the $\be^* = 90\un{m}$, the situation is quite different. The acceptance only starts at higher $|t|$ values, where the Coulomb effects can be neglected (see \Fg{ext C}). This means that it is not necessary to include a phase part of the hadronic amplitude $\FH$ as in \Eq{ext MP par} and one can use the parameterization \Eq{ext M par}. 

Another difference compared to the $1535\un{m}$ optics is the much more pronounced beam smearing, its effects are quantified in \Fg{ext smearing}. The dominant error contribution is coming from the horizontal scattering angle $\th_x^*$ reconstruction, see \Fg{elr 90 dth}.

In principle, for the extrapolation one may use a very similar method as for the $1535\un{m}$ optics. One would just replace the Coulomb-correction iterations with smearing-correction iterations. However, we tried another approach. An approach that might be used even if $t_x$ could not be measured at all. In such a case, one can only work with a ``vertical'' cross-section $\d\si/\d t_y$. This can be transformed into the full cross-section $\d\si/\d t$ by exploiting the azimuthal symmetry (thus $\d\si/\d t_x = \d\si/\d t_y$) and the relation $t = t_x + t_y$ (see \Eq{ttm t th}):
\eqref{{\d\si\over\d t}(t) \propto \int\limits_t^0 \d u\, {\d\si\over\d t_y}(u)\, {\d\si\over\d t_y}(t - u)\ .}{ext dsdty to dsdt}
However, since low $|t_y|$ information is missing (out of acceptance), an extrapolation step would be needed just for this transformation. It therefore turns out better to transform the $t$-parameterization \Eq{ext M par} in a $t_y$-parameterization and fit it directly through $t_y$ data. The transform follows from the relation between $t$ and $t_y$ \Eq{ttm t x y}. Again, assuming the azimuthal symmetry ($\ph$ is uniformly distributed), one can write
\eqref{{\d\si\over\d t_y}(t_y) = {2\over\pi} \int\limits_{0}^{\pi/2} {\d\ph\over\sin^2\ph}\ \ {\d\si\over\d t}\!\left(t_y\over\sin^2\ph\right)\ .}{ext dsdt dsdty}
This transform can now by applied to the parameterization \Eq{ext M par}. Moreover, considering that the quadratic and higher terms of the polynomial $M(t)$ do not carry essential contributions (see \Fg{ext B}), one can approximate:
\eqref{{\d\si\over\d t} = e^{a\, +\, bt\, +\, ct^2\, +\, \ldots} \quad\Rightarrow\quad {\d\si\over\d t_y}(t_y) \approx {1\over\sqrt{\pi}}\, {e^{a\, +\, bt_y\, +\, ct_y^2\, +\, \ldots}\over\sqrt{|b\,t_y|}}\ .}{ext dsdt dsdty app}

To test this extrapolation approach, we used the same (simplified) \abb{MC} simulation as for the $1535\un{m}$ optics. The main results are shown in \Fg{ext results 90}. We fitted the $t_y$ distribution with the parameterization \Eq{ext dsdt dsdty app} to extract the $a$ parameter, which determines the cross-section at $t=0\un{GeV^2}$. The fits were carried out on the intervals from $|t_y|_{\rm low}$ (horizontal axis) to $|t_y|_{\rm up} = 0.25\un{GeV^2}$. No attempt was made to correct for the smearing effects, which results in the overall offset $-2\percent$. But there is no principal problem to include such a correction in future. Most models fall in a band $\pm 1\percent$ around the offset $-2\percent$. The model of Islam et al.~deviates slightly more. Looking back to \Fg{ext B} one can see the exponential slope $B(t)$ being more curved for the model of Islam than for the others. Therefore the approximation \Eq{ext dsdt dsdty app} is less accurate for this model.

\fig{fig/pdf/ext_results_90.pdf}{ext results 90}{The extrapolation deviation as a function of the fit lower bound $|t_y|_{\rm low}$. The deviation is calculated as (extrapolated - original) / original, where e.g.~original denotes the original (true) value of $\d\si^{\rm H}/\d t|_{0}$.}



\subsection{Summary}

The uncertainty estimates of the luminosity-independent method can be obtained by error propagation through \Eq{ttm li meth}. Taking the worst cases of the extrapolations from \Fg{ext results 1535,ext results 90} and the other parameter uncertainties from Sec.~6.3 in \bref{totem08} yields \Tb{ext error summary}.

\htab{ext error summary}{Uncertainty estimates for the luminosity-independent method.}{
\multispan2&\multispan2\bhrulefill\cr
\multispan2	&\omit\bvrule\hfil\bstrut\tskip $\be^* = 90\un{m}$\hfil\tskip & \be^* = 1535\un{m} \cr\bln
\d N_{\rm el}/\d t|_0	& \vcenter{\hsize9cm\noindent\bstrut Extrapolation of elastic rate to $t=0\un{GeV^2}$} & 4\percent & 0.2 \percent \cr\ln
N_{\rm el}	& \vcenter{\hsize9cm\noindent\bstrut Total elastic rate (correlated with extrapolation)} & 2\percent & 0.1\percent \cr\ln
N_{\rm inel}	& \vcenter{\hsize9cm\noindent\bstrut Total inelastic rate (error dominated by single diffractive trigger losses)} & 1\percent & 0.8\percent \cr\ln
\rh & \vcenter{\hsize9cm\noindent\bstrut Error contribution from $(1 + \rh^2)$ (using full COMPETE \bref{cudell02} error band)\phantom{g}} &\multispan2\vrule\hfil $1.2\percent$\hfil \cr\bln
\omit&\omit\bvrule\bstrut\tskip Total for $\si_{\rm tot}$\hfil& 5\percent & 1\div2\percent \cr
\omit&\multispan3\hrulefill\cr
\omit&\omit\bvrule\bstrut\tskip  Total for ${\cal L}$\hfil & 7\percent & 2\percent \cr
\omit&\multispan3\bhrulefill\cr
}

Since we have used the nominal (expected) optics and beam parameters, these results are of a rather preliminary nature. For instance, the \abb{LHC} has recently been running with normalized emittances as low as $2 \un{\mu m\ rad}$, which is about a half of the expected value. Also, it turned out that $t_x$ can be reconstructed reasonably well for the $\be^* = 90\un{m}$ optics too (see \bref{totem11-2}). Thus, one could use the same extrapolation method as for the high-$\be^*$ optics. To conclude, the uncertainty for the $90\un{m}$ optics was found to be significantly lower than anticipated.
