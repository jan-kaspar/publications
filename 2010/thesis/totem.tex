\iffalse
TOTEM, RP, T1, T2, IP, BPM, BLM
\fi

\chapter[ttm]{The TOTEM experiment}

MOTIVATION: The forward hadronic physics is one of the least explored and understood areas of the particle physics.

WHAT: The physics programme of the TOTEM experiment includes:
\> elastic scattering measurement in a wide $t$-range,
\> total cross-section measurement and
\> a study of soft and hard diffraction.

\TODO{add the phi vs. eta diagram to explain the forwardness?}

HOW (detectors): This challenging program brings special requirements for the detector apparatus. In particular large rapidity coverage - to detect most fragments from inelastic collisions and excellent acceptance for outgoing elastic and diffractive protons. To accomplish this task, TOTEM comprises three subdetectors: the telescopes T1 and T2 to detect products of inelastic collisions and a system of Roman Pots (RP) for surviving proton detection.

\TODO{acceptance overview plot? Our acceptance starts where CMS stops}

\> brief description of T1 and T2 with a few pictures
\>> reference to more detailed information: \bref{totem04,totem08,totem10}

\> describe the RP system
\>> motivation: very small angles -- need move close to the beam and detectors far from the IP
\>> break down with pictures, few info about the sensors, define hybrid
\>> introduce (with pictures BPMs, BLMs, LVDTs)
\>> define the standard naming: \bref{totem-rp-naming}

\tab[\strut\quad#\quad\hfil&\quad#\quad\hfil&\quad#\quad\hfil\cr]{rp naming schemes}{The RP naming scheme.}{\ln
arm		& 45 (=left), 56 (=right)		\cr
station	& 147, 220		\cr
unit	& far, near		\cr
pot		& top, bot = bottom, hor = horizontal	\cr
plane	& 1 to 10		\cr
vfat	& 1 to 4		\cr
strip	& 0 to 127		\cr
\ln
}

\>> give idea about the dimensions (for reference reasons later), + spacing between sensors (total stack size), detector pitch, detector orientation

\htab{ttm rp station}{The $s$ positions of the RP centers, in meters from the IP5.}{
\omit&\multispan6\bhrulefill\cr
\omit&\multispan3\strut\bvrule\hfil near unit\hfil&\multispan3\vrule\hfil far unit\hfil\cr
\omit&\multispan6\bhrulefill\cr
\omit&\omit\bvrule\strut\hfil\hbox{top}\hfil & \hbox{bottom} &\hbox{horizontal} &\hbox{horizontal} &\hbox{top} &\hbox{bottom} \cr\bln
\hbox{147 m station} & 148.784 & 148.784 & 149.234 & 150.026 & 150.476 & 150.476 \cr\ln
\hbox{220 m station} & 214.628 & 214.628 & 215.078 & 219.550 & 220.000 & 220.000 \cr\bln
}

\htab{ttm det package}{The positions of silicon sensors in a detector package (relative to the center of a RP). In milimeters. Sensor 1 is the nearest to the IP5.}{\bln
\hbox{sensor}	& 1 & 2 & 3 & 4 & 5 & 6 & 7 & 8 & 9 & 10 \cr\bln
\hbox{position}	& -20.3 & -15.7 & -11.3 & -6.7 & -2.3 & 2.3 & 6.7 & 11.3 & 15.7 & 20.3 \cr\bln
}

In 56, top pots, V sensors shifted in x by +0.514mm, U sensors by -0.514mm.

\> define u, v (the system rotating with sensor) and U, V (45 deg. rotation of about s of the xys system), strips along u, read-out direction is v
\TODO{make sure this is used everywhere}

\> RP approach -- typical $10\si$, define beam sigma!


\iffalse
\bmfig
\fig[8cm]{fig/external/RP_stations_original.pdf}{s}{[7cm]RP stations}
\fig[6cm]{fig/external/rp_unit.jpg}{rp unit}{[7cm]RP unit}
\emfig

\bmfig
\fig[6cm]{fig/external/rp.jpg}{rp rp}{[7cm]A Roman Pot}
\fig[6cm]{fig/external/rp_package.jpg}{rp package}{[7cm]Detector package}
\emfig

\bmfig
\fig[6cm]{fig/external/hybrid2.jpg}{rp hybrid}{[7cm]A hybrid with a silicon detector and four VFAT chips.}
\fig[6cm]{fig/external/silicon_explained.png}{rp sensor}{[7cm]A detail of a silicon sensor.}
\emfig
\fi

\section[rp measurement]{Proton Measurement with Roman Pot detectors}

\fig{fig/pdf/ttm_proton_transport.pdf}{ttm proton transport}{A scheme of the proton transport from the interaction point to a RP station. $y_{\rm N}$ and $y_{\rm F}$ are the hit points in the near and far units of the station.}


Imagine a collision which takes place at a \em{vertex}
\eqref{(x^*, y^*, s^*)^\T}{ttm ip vertex}
and where a proton is created, as sketched in \Fg{ttm proton transport}. One may relate the momentum $p$ of the outgoing proton to the nominal beam momentum
\eqref{p^* = p_{\rm nom} (1 - \xi^*)\ ,}{ttm xi}
where $\xi^*$ is the \em{momentum loss}. The direction of the proton is described by the \em{scattering angle} $\th^*$ and the \em{azimuthal angle} $\ph^*$:
\eqref{\vec p^* = p^* \pmatrix{\sin\th^* \cos\ph^*\cr \sin\th^*\sin\ph^* \cr \cos\th^*}
%=  p \pmatrix{\th_x\cr \th_y\cr \sqrt{1 - \th_x^2 - \th_y^2}}
\ .}{ttm mom comp}
The coordinate system $(x, y, s)$ is illustrated in \Fg{ttm proton transport}, the $s$ axis represents the beam axis.

The protons of our interest are scattered to low angles and therefore they stay close the outgoing beam and they enter the LHC magnet lattice. When the outgoing proton reaches the region of a RP station, it has momentum $p$, direction described by angles $\th$ and $\ph$ and may hit the detectors at points $x_{\rm N}$ and $x_{\rm F}$ just as shown in \Fg{ttm proton transport}.

We will use the notation that quantities related to the IP will be denoted with a star superscript, those without will be related to RP stations. Energy of the protons shall be conserved during the LHC transport from the IP to RP stations, therefore $p = p^*$ and $\xi = \si^*$. Thus we will not make distinction between their IP and RP values.

It will be useful to define the \em{projections of the scattering angle} (both for IP and RP-station quantities)
\eqref{\th_x = \th \cos\ph\ ,\qquad \th_y = \th \sin\ph\ .}{ttm th x y}
Similarly, one can define the projections of the momentum transfer $t$:
\eqref{t_x = t \cos^2\ph^*\ ,\qquad t_y = t \sin^2\ph^*\ ,}{ttm t x y}
This definition follows, in fact, from the low-scattering-angle approximation in \Eq{el t}, which we will commonly use in what follows (the scattering angles of interest are at most of the order $10^{-3}\un{rad}$, see e.g. \Fg{el mod dsdt large}). The definitions \Eq{ttm th x y,ttm t x y} yield relations
\eqref{t = t_x + t_y = - p^2 \th^{*2} = - p^2 (\th_x^{*2} + \th_y^{*2})\ .}{ttm t th}

Since the magnetic field within the RP stations is negligible, protons follow a straight trajectory there. The relation between RP sensor measurements and the parameters of the trajectory is, therefore simple (for details see \Sc{al psi}). What is much more interesting is the proton transport from the IP to the RP stations. We will discuss it in the rest of this section.

\> analytical description
\>> the solution to Hill's equation, define beta function, emittance, phase advance (+ref)
\>> alternative: the matrix equations (ref to Hubert, +); define optics, optical functions: effective length, magnification, dispersion

\> numerical tools; MAD-X

\> optics parameterizations
\>> linearized
\eqref{\eqnarray{
x(s) &=&\ L_x(s, \ldots)\, \th_x^* + v_x(s,\ldots)\, x^* + D_x(s,\ldots)\cr
y(s) &=&\ L_y(s, \ldots)\, \th_y^* + v_y(s,\ldots)\, y^* + D_y(s,\ldots)\cr
}}{to lin par}
\>> polynomial, ref to Hubert

\> so far one proton only, in reality bunches: non-zero spatial, momentum and angular spread --> beam smearing
\>> beam divergence

\eqref{\si_{\th^*} = }{to beam div}

\>> vertex smearing

\eqref{\si_{x^*} = }{to vertex sm}

\>> energy fluctuation and offset

\eqref{\si_\xi = }{to energy fl}

\eqref{\bar\xi = }{to energy off}

\> the concept of acceptance
\>> acceptance of for a single proton, function A(t, phi, xi), plus averaging over parameters up to A(t)
\>> elastic acceptance - combined acceptance of two protons that we would be able to see an elastic event

\> the tree optics classes - high beta, medium beta, low beta
\>> classify by the optical functions (it controls what the optics is good for): Lx, Ly, D
\>> associated quantities - emittance + resulting beam divergence, vertex smearing
\>> nominal detector distance = 10 si bd, this sets minimal t: $t_{30}$ ($2\cdot10^{-3}$ for 1535, $4\cdot10^{-2}$ for 90 and $2.5$ for 2)
\>> goal of the optics (this sets the requirements for the luminosity)
\>> luminosity

\iffalse
\itskip0pt
\indent 1) {\fgt\bf high $\be^*$\fg}
\> $\be^* = 1535\un{m}$
\> ${\cal L} \approx 10^{28} \div 10^{29}\un{cm^{-2}s^{-1}}$
\> elastic resolution: $\si(\th_x) \approx 0.23\un{\mu rad}$, $\si(\th_y) \approx 0.22\un{\mu rad}$
\> vertical sensors at $1.35\un{mm}$ from beam center

\vfil
2) {\fgt\bf medium $\be^*$\fg}
\> $\be^* = 90\un{m}$
\> ${\cal L} \approx 10^{30}\un{cm^{-2}s^{-1}}$
\> elastic resolution: $\si(\th_x) \approx 5\un{\mu rad}$ (low effective length), $\si(\th_y) \approx 1.7\un{\mu rad}$
\> vertical sensors at $6.4\un{mm}$ from beam center

\vfil
3) {\fgt\bf low $\be^*$\fg}

\> $\be^* = 0.5\div 2\un{m}$ (early running: $p = 5\un{TeV}$, $\be^* \sim 3\un{m}$)
\> ${\cal L} \approx 10^{33}\un{cm^{-2}s^{-1}}$
\> elastic resolution: $\si(\th_x) \approx 16\un{\mu rad}$, $\si(\th_y) \approx 12\un{\mu rad}$
\> vertical sensors at $3.3\un{mm}$ from beam center

\vfil
\> sensors at $10\si + 0.5\un{mm}$ from beam center
\> resolution (usually) limited by beam divergence
\fi

\bmfig
\fig{fig/pdf/elastic_acceptance.pdf}{elastic acceptance}{$\sqrt{s} = 14\un{TeV}$, the dashed horizontal line gives the $t_{30}$}
\fig{fig/pdf/hit_distribution.pdf}{hit distribution}{Distribution of elastic hits for the three types of optics.}
\emfig







\iffalse

\section{The total cross-section}

\> tot cs due to the hadronic interaction only, the one for the em int. is infinite

\> Luminosity independent method
\>> discuss the two amplitudes (spin averaged and spin preserving)
\>> 4 inputs: el cs at 0, rho at 0, N el and N inel
\>> N inel by T1 and T2, not discussed here, refere
\> rho at 0 (ref to \Eq{el rh}): Coulomb-interference or an external input

\eqref{
	\si_{\rm tot} = {1\over 1+\rh^2} {\d N/\d t|_{t=0}\over N_{\rm el} + N_{\rm inel}}
	\ ,\qquad
	{\cal L} = (1+\rh^2) {(N_{\rm el} + N_{\rm inel})^2\over \d N/\d t|_{t=0} }
}{ttm li meth}

The differential rate at $t=0\un{GeV^2}$ can not be measured directly, it needs to be extrapolated from measured low $|t|$ data. As shown in \Fg{elastic acceptance}, there are two optics that extend to sufficiently low $|t|$ values: $\be^* = 1535$ and $90\un{m}$. This extrapolation procedure is important also for the (integral) elastic rate $N_{\rm el}$ measurement as it allows to correct for the events missed due to the finite acceptance.


\>> B looks ``parabolic'' -- polynomial fit to B from data

\iffalse
islam_bfkl, 9E-01, 3E-01, 2E-02, 1E-03, 3E-04
 islam_cgc, 9E-01, 3E-01, 2E-02, 1E-03, 3E-04
      ppp2, 6E-02, 4E-03, 1E-04, 3E-06, 1E-06
      ppp3, 8E-01, 2E-01, 4E-03, 4E-03, 2E-03
       bsw, 5E-01, 2E-01, 1E-03, 2E-04, 2E-04
        bh, 2E+00, 3E-01, 7E-03, 3E-03, 2E-03
      mean, 8E-01, 2E-01, 8E-03, 1E-03, 8E-04
\fi

\fig{fig/pdf/ext_B.pdf}{ext B}{[7cm]bla}

\> what one measures is $h(t')$ -- includes smearing and acceptance and both interactions (hadronic + em)
\>> one needs to correct measured to hadronic (+unsmeared, acc. corr.)
\>> for 1535 smearing is not a problem (need a justification), but Coulomb can go large (below few$\cdot 10^{-3}\un{GeV^2}$)
\>> for 90 (in the range of its acceptance) Coulomb effect is small, but smearing is an issue - mostly because t x resolution

\bmfig
\fig{fig/pdf/ext_C.pdf}{ext C}{[7cm]bla}
\fig{fig/pdf/ext_smearing.pdf}{ext smearing}{[7cm](smeared - orig) / orig, smearing sigma as from \Sc{elr}}
\emfig

\> corrections by iterations: data = U0 (U is uncorrected, C is corrected), guess C0 (e.g. by taking C0 = U0), then iterate:
Ci --> Ui --> corr(i) = Ci/Ui --> C(i+1) = U0 * corr(i) and again unless |C(i+1) - C(i)| < threshold

\> a very old extrapolation study, unpublished, before OfflineSW existed
\>> no G4 simulation, however simulation of beam smearing as in \Sc{beam smearing} and simulation of detector resolution similar to the one in \bref{fast simu}
\>> proton transport with a parameterized optics \TODO{ref}
\>> realistic statistics (1 day of data-taking at a realistic luminosity)

\> generally optimized: lever-arm (fit from ... to ...), the binning, the degree of modulus and phase polynomial, number of coulomb and anti-smearing correction iterations

\> try to do it in a model independent way -- used the 6 models as presented in \Sc{el}, optimizing for the mean discrepancy

\> here we show the most interesting results, the dependence on a hypothetical low $|t|$ bound of the fit

\> 1535: cubic fit for modulus, constant in phase, one coulomb iteration, upper bound 0.054
\>> discuss the phase: the one percent result might be too good

\fig{fig/pdf/ext_results.pdf}{ext results}{delta on the level of cross-section at t=0: (extrapolated - orig) / orig}

\> two options for 90 m
\>> heavy unsmearing
\>> or use t y dist only (preferred)

\> 90m: the t y fit procedure
\>> no unsmearing done -- the -2 percent shift
\>> ty fit on 0.04 to 0.25

\> propagation of the errors, the standard table

\htab{ext error summary}{bla}{
	&	& \be^* = 90\un{m} & 1535\un{m} \cr\bln
\d N/\d t|_0	& \hbox{Extrapolation of elastic rate to } t = 0 & 4\percent & 0.2 \percent \cr\ln
N_{\rm el}	& \hbox{Total elastic rate (correlated with extrapolation)} & 2\percent & 0.1\percent \cr\ln
N_{\rm inel}	& \vbox{\hsize7cm\noindent\strut Total inelastic rate (error dominated by Single Diffractive trigger losses)} & 1\percent & 0.8\percent \cr\ln
\rh\equiv\Re A(t)/\Im A(t)|_{t = 0} & \vcenter{\hsize7cm\noindent\strut Error contribution from $(1 + \rh^2)$ (using full COMPETE error band)\phantom{g}} &\multispan2\vrule\hfil $1.2\percent$\hfil \cr\bln
&\hbox{Total for } \si_{\rm tot} & 5\percent & 1\div2\percent \cr\bln
&\hbox{Total for } {\cal L} & 7\percent & 2\percent \cr\bln
}

\fi
