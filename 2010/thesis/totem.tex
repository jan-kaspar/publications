\iffalse
TOTEM, RP, T1, T2, IP, BPM, BLM
\fi

\chapter[ttm]{The TOTEM experiment}

MOTIVATION: The forward hadronic physics is one of the least explored and understood areas of the particle physics.

WHAT: The physics programme of the TOTEM experiment includes:
\> elastic scattering measurement in a wide $t$-range,
\> total cross-section measurement and
\> a study of soft and hard diffraction.

\> define SD and DPE

\TODO{add the phi vs. eta diagram to explain the forwardness?}

HOW (detectors): This challenging program brings special requirements for the detector apparatus. In particular large rapidity coverage - to detect most fragments from inelastic collisions and excellent acceptance for outgoing elastic and diffractive protons. To accomplish this task, TOTEM comprises three subdetectors: the telescopes T1 and T2 to detect products of inelastic collisions and a system of Roman Pots (RP) for surviving proton detection.

\TODO{acceptance overview plot? Our acceptance starts where CMS stops}

\section[ttm det]{Detector apparatus}

\> brief description of T1 and T2 with a few pictures
\>> reference to more detailed information: \bref{totem04,totem08,totem10}

\> describe the RP system
\>> motivation: very small angles -- need move close to the beam and detectors far from the IP
\>> break down with pictures, few info about the sensors, define hybrid
\>> introduce (with pictures BPMs, BLMs, LVDTs)
\>> define the standard naming: \bref{totem-rp-naming}

\tab[\strut\quad#\quad\hfil&\quad#\quad\hfil&\quad#\quad\hfil\cr]{rp naming schemes}{The RP naming scheme.}{\ln
arm		& 45 (=left), 56 (=right)		\cr
station	& 147, 220		\cr
unit	& far, near		\cr
pot		& top, bot = bottom, hor = horizontal	\cr
plane	& 1 to 10		\cr
vfat	& 1 to 4		\cr
strip	& 0 to 127		\cr
\ln
}

\>> give idea about the dimensions (for reference reasons later), + spacing between sensors (total stack size), detector pitch, detector orientation

\htab{ttm rp station}{The $s$ positions of the RP centers, in meters from the IP5.}{
\omit&\multispan6\bhrulefill\cr
\omit&\multispan3\strut\bvrule\hfil near unit\hfil&\multispan3\vrule\hfil far unit\hfil\cr
\omit&\multispan6\bhrulefill\cr
\omit&\omit\bvrule\strut\hfil\hbox{top}\hfil & \hbox{bottom} &\hbox{horizontal} &\hbox{horizontal} &\hbox{top} &\hbox{bottom} \cr\bln
\hbox{147 m station} & 148.784 & 148.784 & 149.234 & 150.026 & 150.476 & 150.476 \cr\ln
\hbox{220 m station} & 214.628 & 214.628 & 215.078 & 219.550 & 220.000 & 220.000 \cr\bln
}

\htab{ttm det package}{The positions of silicon sensors in a detector package (relative to the center of a RP). In milimeters. Sensor 1 is the nearest to the IP5.}{\bln
\hbox{sensor}	& 1 & 2 & 3 & 4 & 5 & 6 & 7 & 8 & 9 & 10 \cr\bln
\hbox{position}	& -20.3 & -15.7 & -11.3 & -6.7 & -2.3 & 2.3 & 6.7 & 11.3 & 15.7 & 20.3 \cr\bln
}

In 56, top pots, V sensors shifted in x by +0.514mm, U sensors by -0.514mm.

\> define u, v (the system rotating with sensor) and U, V (45 deg. rotation of about s of the xys system), strips along u, read-out direction is v
\TODO{make sure this is used everywhere}

\> RP approach -- typical $10\si$, define beam sigma!


\iffalse
\bmfig
\fig[8cm]{fig/external/RP_stations_original.pdf}{s}{[7cm]RP stations}
\fig[6cm]{fig/external/rp_unit.jpg}{rp unit}{[7cm]RP unit}
\emfig

\bmfig
\fig[6cm]{fig/external/rp.jpg}{rp rp}{[7cm]A Roman Pot}
\fig[6cm]{fig/external/rp_package.jpg}{rp package}{[7cm]Detector package}
\emfig

\bmfig
\fig[6cm]{fig/external/hybrid2.jpg}{rp hybrid}{[7cm]A hybrid with a silicon detector and four VFAT chips.}
\fig[6cm]{fig/external/silicon_explained.png}{rp sensor}{[7cm]A detail of a silicon sensor.}
\emfig
\fi

\section[rp measurement]{Proton Measurement with Roman Pot detectors}

\fig{fig/pdf/ttm_proton_transport.pdf}{ttm proton transport}{A scheme of the proton transport from the interaction point to a RP station. $y_{\rm N}$ and $y_{\rm F}$ are the hit points in the near and far units of the station.}


Imagine a collision which takes place at a \em{vertex}
\eqref{(x^*, y^*, s^*)^\T}{ttm ip vertex}
and where a proton is created, as sketched in \Fg{ttm proton transport}. One may relate the momentum $p$ of the outgoing proton to the nominal beam momentum
\eqref{p^* = p_{\rm nom} (1 - \xi^*)\ ,}{ttm xi}
where $\xi^*$ is the \em{momentum loss}. The direction of the proton is described by the \em{scattering angle} $\th^*$ and the \em{azimuthal angle} $\ph^*$:
\eqref{\vec p^* = p^* \pmatrix{\sin\th^* \cos\ph^*\cr \sin\th^*\sin\ph^* \cr \cos\th^*}
%=  p \pmatrix{\th_x\cr \th_y\cr \sqrt{1 - \th_x^2 - \th_y^2}}
\ .}{ttm mom comp}
The coordinate system $(x, y, s)$ is illustrated in \Fg{ttm proton transport}, the $s$ axis represents the beam axis.

The protons of our interest are scattered to low angles and therefore they stay close the outgoing beam and they enter the LHC magnet lattice. When the outgoing proton reaches the region of a RP station, it has momentum $p$, direction described by angles $\th$ and $\ph$ and may hit the detectors at points $x_{\rm N}$ and $x_{\rm F}$ just as shown in \Fg{ttm proton transport}.

We will use the notation that quantities related to the IP will be denoted with a star superscript, those without will be related to RP stations. Energy of the protons shall be conserved during the LHC transport from the IP to RP stations, therefore $p = p^*$ and $\xi = \si^*$. Thus we will not make distinction between their IP and RP values.

It will be useful to define the \em{projections of the scattering angle} (both for IP and RP-station quantities)
\eqref{\th_x = \th \cos\ph\ ,\qquad \th_y = \th \sin\ph\ .}{ttm th x y}
Similarly, one can define the projections of the momentum transfer $t$:
\eqref{t_x = t \cos^2\ph^*\ ,\qquad t_y = t \sin^2\ph^*\ ,}{ttm t x y}
This definition follows, in fact, from the low-scattering-angle approximation in \Eq{el t}, which we will commonly use in what follows (the scattering angles of interest are at most of the order $10^{-3}\un{rad}$, see e.g. \Fg{el mod dsdt large}). The definitions \Eq{ttm th x y,ttm t x y} yield relations
\eqref{t = t_x + t_y = - p^2 \th^{*2} = - p^2 (\th_x^{*2} + \th_y^{*2})\ .}{ttm t th}

Since the magnetic field within the RP stations is negligible, protons follow a straight trajectory there. The relation between RP sensor measurements and the parameters of the trajectory is, therefore simple (for details see \Sc{al psi}). What is much more interesting is the proton transport from the IP to the RP stations. We will discuss it in the rest of this section.

The transverse motion of protons in the LHC can be described by a solution of the Hill's equation (see e.g.~Sc.~3.3 in \bref{wilson} or Ch.~4 in \bref{dolezal}):
\eqref{x(s) = \sqrt{\be(s) \ep}\ \cos\big(\phi(s) + \phi_0\big)\ .}{ttm hill sol}
We have written the solution for the horizontal direction only, for the vertical it is analogous. The solution has a form of oscillations with the amplitude given by the emittance $\ep$ (a measure of the beam size and divergence\footnote{%
A quantity which is used more often is the normalized emittance $\ep_{\rm N} = \be\ga \ep$. $\be$ and $\ga$ are the relativistic factors, see \Eq{sm lorentz}. The advantage of the normalized emittance is that it does not change during beam acceleration, see Sc.~4.1 in~\bref{wilson}.
}) and the $\be(s)$ function (reflects the layout and strengths of the LHC magnets). The phase of the oscillations is given by $\phi(s)$ which is related to $\be(s)$ by $\d\phi/\d s = 1/\be(s)$.

From \Eq{ttm hill sol} one can read off the amplitude of the oscillations, which effectively determines the beam size:
\eqref{\si(s) = \sqrt{\be(s)\ep}\ .}{ttm beam sigma}
This quantity is often called a \em{beam sigma}. The beam size is of a special interest at the interaction point -- it controls the vertex distribution. Denoting $\be^*$ the value of the $\be(s)$ function at the interaction point, one can write
\eqref{\si_{x^*} = \sqrt{\be^* \ep}\ .}{ttm vertex sm}
By differentiating \Eq{ttm hill sol} one can obtain the evolution proton angles $\th_x = \d x/\d s$. Again, one can extract the amplitude of its oscillations, that is a measure of the beam angular divergence. Evaluating it at the IP yields
\eqref{\si_{\th^*} = \sqrt{\ep\over\be^*}\ .}{ttm beam div}

The solution \Eq{ttm hill sol} was derived for protons of the nominal momentum $p_{\rm nom}$. If the actual proton momentum differs by $\De p = \xi p_{\rm nom}$, one has to add a term $D(s) \xi$ to the solution, see Sc.~5.2 in~\bref{wilson}. Then, for the proton transport from the IP to a position $s$ one can write (see Sc.~5.2.4 in~\bref{wilson})
\eqref{x(s) =\ L_x(s, \ldots)\, \th_x^* + v_x(s,\ldots)\, x^* + D_x(s,\ldots)\,\xi}{ttm lin par}
and equivalently for the vertical direction. In the previous relation we have introduced the \em{effective length} $L$, the \em{magnification} $v$ and the \em{dispersion} $D$, we will refer to them as \em{optical functions}. These functions depend on the distance $s$ from the IP, but may also depend on the variables $\th_x^*$, $x^*$ and $\xi$. For the optics of practical interest, however, the dependence on $\xi$ is the only relevant one.

Strictly speaking, the form of \Eq{ttm lin par} is oversimplified. Due to higher order corrections, $x$ and $y$ projections are not strictly independent and consequently one should include also dependence on $\th_y^*$ and $y^*$. For the nominal optics these terms are negligible. However in reality, due to magnet misalignments (mostly their rotations), the binding between $x$ and $y$ projections may become important, see e.g.~\Fg{al el plots yx}.

In practice the optical functions are calculated with dedicated computer programs such as MAD-X \bref{mad-x}. Since they might be to slow for large Monte Carlo simulations, TOTEM's software (see \Sc{sr}) includes a module that calculates a polynomial parameterization of a MAD-X output, for details see Sc.~6.4 in \bref{hubert}.

For standard LHC runs one may expect relative error of beam momentum measurement
\eqref{\bar\xi = \O{10^{-3}}}{ttm energy off}
and the proton-to-proton momentum fluctuation
\eqref{\si_\xi = \O{10^{-4}}\ .}{ttm energy fl}

Since the optics \Eq{ttm lin par} relates the kinematics of a proton at the IP and the RP stations, it defines what protons can be detected with the (finite-size) RP detectors. Formally, one could think of an \em{acceptance} function $A(\th_x^*, x^*, \th_y^*, y^*, \xi)$ which gives $1$ if such a proton can be detected and $0$ otherwise. One can also take an average over the quantities that are not of interest. This may lead to, for example to acceptance $A(t, \xi)$, which gives the ratio protons with momentum transfer $t$ and momentum loss $\xi$ that can be detected. Similarly one may define the \em{elastic acceptance} $A(t)$ as the detected ratio of the elastic events with momentum transfer $t$. Several examples of the elastic acceptance are shown in \Fg{ttm elastic acceptance}.

%In an elastic measurement analysis (see e.g.~\Sc{felm}) one needs to correct for the limited acceptance. It is clear that when the acceptance is too low, the \em{acceptance correction} is large and 

\fig{fig/pdf/ttm_elastic_acceptance.pdf}{ttm elastic acceptance}{The elastic acceptances for the three optics types. Calculated for $\sqrt{s} = 14\un{TeV}$ and with RP sensors at $10\si + 0.5\un{mm}$. The horizontal line marks the points $|t_{30}|$ where the acceptances reach $30\percent$.}

The rich physics programme of TOTEM requires several running scenarios, including several optics. We will now review the foreseen scenarios, classified by the beta-function at the interaction point $\be^*$. Certain parameters have been calculated assuming the CMS energy of $\sqrt s = 14\un{TeV}$.

\> The \em{high $\be^*$} scenario is dedicated to the measurement of the total cross-section and low-$|t|$ elastic scattering. The optics with $\be^* = 1535\un{m}$ provides high effective lengths: $L_x$ up to $110\un{m}$ and $L_y$ up to $270\un{m}$ (both in the $220\un{m}$ stations). This ensures a good $t$-resolution (see \Sc{elr 1535}). To improve it further, this scenario counts with a reduced normalized emittance of $\ep_N=1\un{\mu m\ rad}$. This leads to a very low beam divergence at the IP $\si_{\th^*} = 0.3\un{\mu rad}$. The expected beam sizes (i.e. the beam sigmas) at the $220\un{m}$ stations are $\si_x = 0.03\un{mm}$ and $\si_y = 0.085\un{mm}$. The beam size, quite generally, determines the approach of the RPs to the beam -- a typical approach is about $10\si$ in each direction. The resulting elastic acceptance is drawn in red in \Fg{ttm elastic acceptance}. The acceptance reaches $30\percent$ at $|t_{30}| = 2\cdot10^{-3}\un{GeV^2}$. This scenario will use a rather low luminosity ${\cal L} \approx 10^{29} \hbox{ -- } 10^{30}\un{cm^{-2}s^{-1}}$.

\> The \em{medium $\be^*$} scenario is based on a very universal optics with $\be^* = 90\un{m}$. It combines the properties of high and low $\be^*$ scenarios allowing for measurements of the total cross-section, medium-$|t|$ elastic scattering and certain diffractive processes. In the $220\un{m}$ stations the horizontal effective length $L_x$ is rather small, below $2.9\un{m}$, while in the vertical direction $L_y$ goes up to $260\un{m}$, which is similar to the high-$\be^*$ optics. The standard normalized emittance $\ep_{\rm N} = 3.75\un{\mu m\ rad}$ is planned to be used, leading to a beam divergence of about $2.4\un{\mu rad}$. The expected beam sizes at the $220\un{m}$ stations are $\si_x = 0.4\un{mm}$ and $\si_y = 0.59\un{mm}$. In other words, the pots will have to stay much further from the beam center than in the case of the high-$\be^*$ optics. Consequently, the elastic acceptance (blue histogram in \Fg{ttm elastic acceptance}) starts much later -- it reaches $30\percent$ only at $4\cdot 10^{-2}\un{GeV^2}$. This scenario counts with a luminosity of ${\cal L} \approx 10^{30}\un{cm^{-2}s^{-1}}$.

\> The \em{low $\be^*$} runs will be used for large-$\xi$ diffractive studies and large-$|t|$ elastic scattering measurements (the data analyzed in \Sc{felm} were obtained with this type of optics). These scenarios with $\be^*$ between $0.5$ and $3\un{m}$ are optimized for high luminosities ${\cal L} \approx 10^{32} \hbox{ -- } 10^{33}\un{cm^{-2}s^{-1}}$ and not for forward proton detection -- the effective length at the $220\un{m}$ stations do not exceed $L_x = 2\un{m}$ and $20\un{m}$. With the standard normalized emittance $\ep_{\rm N} = 3.75\un{\mu m\ rad}$ the beam divergence is rather large: $\si_{\th^*} \approx 16\un{\mu rad}$. The beam sizes at the $220\un{m}$ are about $\si_x = 0.1\un{mm}$ and $\si_y = 0.3\un{mm}$. The elastic acceptance (green histogram in \Fg{ttm elastic acceptance}) reaches $30\percent$ at $|t_{30}| = 2.5\un{GeV^2}$.

\vskip\itskip

\Fg{ttm hit distribution} illustrates the properties of the three types of optics. The horizontal (vertical) spread of hits is controlled by the horizontal effective length $L_x$ (the vertical $L_y$). That is why the $\be^* = 90$ and $2\un{m}$ optics yield very narrow elastic scattering hit distributions. The gray ellipses show $10\si$ contours, thereby controlling the positions of the RPs.


\fig{fig/pdf/ttm_hit_distribution.pdf}{ttm hit distribution}{Simulated distributions of elastic-scattering protons in the 56-220-far unit, each color corresponds to an optics type. The gray ellipses show $10$ beam sigma contours. The RP sensors are place at $10\si + 0.5\un{mm}$.}








\section{The total cross-section}

\> tot cs due to the hadronic interaction only, the one for the em int. is infinite

\> Luminosity independent method
\>> discuss the two amplitudes (spin averaged and spin preserving)
\>> 4 inputs: el cs at 0, rho at 0, N el and N inel
\>> N inel by T1 and T2, not discussed here, refere
\> rho at 0 (ref to \Eq{el rh}): Coulomb-interference or an external input

\eqref{
	\si_{\rm tot} = {1\over 1+\rh^2} {\d N/\d t|_{t=0}\over N_{\rm el} + N_{\rm inel}}
	\ ,\qquad
	{\cal L} = (1+\rh^2) {(N_{\rm el} + N_{\rm inel})^2\over \d N/\d t|_{t=0} }
}{ttm li meth}

The differential rate at $t=0\un{GeV^2}$ can not be measured directly, it needs to be extrapolated from measured low $|t|$ data. As shown in \Fg{ttm elastic acceptance}, there are two optics that extend to sufficiently low $|t|$ values: $\be^* = 1535$ and $90\un{m}$. This extrapolation procedure is important also for the (integral) elastic rate $N_{\rm el}$ measurement as it allows to correct for the events missed due to the finite acceptance.


\>> B looks ``parabolic'' -- polynomial fit to B from data

\iffalse
islam_bfkl, 9E-01, 3E-01, 2E-02, 1E-03, 3E-04
 islam_cgc, 9E-01, 3E-01, 2E-02, 1E-03, 3E-04
      ppp2, 6E-02, 4E-03, 1E-04, 3E-06, 1E-06
      ppp3, 8E-01, 2E-01, 4E-03, 4E-03, 2E-03
       bsw, 5E-01, 2E-01, 1E-03, 2E-04, 2E-04
        bh, 2E+00, 3E-01, 7E-03, 3E-03, 2E-03
      mean, 8E-01, 2E-01, 8E-03, 1E-03, 8E-04
\fi

\fig{fig/pdf/ext_B.pdf}{ext B}{[7cm]bla}

\> what one measures is $h(t')$ -- includes smearing and acceptance and both interactions (hadronic + em)
\>> one needs to correct measured to hadronic (+unsmeared, acc. corr.)
\>> for 1535 smearing is not a problem (need a justification), but Coulomb can go large (below few$\cdot 10^{-3}\un{GeV^2}$)
\>> for 90 (in the range of its acceptance) Coulomb effect is small, but smearing is an issue - mostly because t x resolution

\bmfig
\fig{fig/pdf/ext_C.pdf}{ext C}{[7cm]bla}
\fig{fig/pdf/ext_smearing.pdf}{ext smearing}{[7cm](smeared - orig) / orig, smearing sigma as from \Sc{elr}}
\emfig

\> corrections by iterations: data = U0 (U is uncorrected, C is corrected), guess C0 (e.g. by taking C0 = U0), then iterate:
Ci --> Ui --> corr(i) = Ci/Ui --> C(i+1) = U0 * corr(i) and again unless |C(i+1) - C(i)| < threshold

\> a very old extrapolation study, unpublished, before OfflineSW existed
\>> no G4 simulation, however simulation of beam smearing as in \Sc{beam smearing} and simulation of detector resolution similar to the one in \Sc{fast simu}
\>> proton transport with a parameterized optics \TODO{ref}
\>> realistic statistics (1 day of data-taking at a realistic luminosity)

\> generally optimized: lever-arm (fit from ... to ...), the binning, the degree of modulus and phase polynomial, number of coulomb and anti-smearing correction iterations

\> try to do it in a model independent way -- used the 6 models as presented in \Sc{el}, optimizing for the mean discrepancy

\> here we show the most interesting results, the dependence on a hypothetical low $|t|$ bound of the fit

\> 1535: cubic fit for modulus, constant in phase, one coulomb iteration, upper bound 0.054
\>> discuss the phase: the one percent result might be too good

\fig{fig/pdf/ext_results.pdf}{ext results}{delta on the level of cross-section at t=0: (extrapolated - orig) / orig}

\> two options for 90 m
\>> heavy unsmearing
\>> or use t y dist only (preferred)

\> 90m: the t y fit procedure
\>> no unsmearing done -- the -2 percent shift
\>> ty fit on 0.04 to 0.25

\> propagation of the errors, the standard table

\htab{ext error summary}{bla}{
	&	& \be^* = 90\un{m} & 1535\un{m} \cr\bln
\d N/\d t|_0	& \hbox{Extrapolation of elastic rate to } t = 0 & 4\percent & 0.2 \percent \cr\ln
N_{\rm el}	& \hbox{Total elastic rate (correlated with extrapolation)} & 2\percent & 0.1\percent \cr\ln
N_{\rm inel}	& \vbox{\hsize7cm\noindent\strut Total inelastic rate (error dominated by Single Diffractive trigger losses)} & 1\percent & 0.8\percent \cr\ln
\rh\equiv\Re A(t)/\Im A(t)|_{t = 0} & \vcenter{\hsize7cm\noindent\strut Error contribution from $(1 + \rh^2)$ (using full COMPETE error band)\phantom{g}} &\multispan2\vrule\hfil $1.2\percent$\hfil \cr\bln
&\hbox{Total for } \si_{\rm tot} & 5\percent & 1\div2\percent \cr\bln
&\hbox{Total for } {\cal L} & 7\percent & 2\percent \cr\bln
}
