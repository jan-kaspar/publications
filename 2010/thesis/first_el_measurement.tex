\chapter[felm]{The first elastic scattering measurement at the LHC}

This chapter is devoted to the first measurement of the elastic scattering differential cross-section at the \abb{LHC}. This achievement, published in \bref{totem11}, is a result of efforts of the entire \abb{TOTEM} collaboration. After a brief overview of the analysis procedure, we will focus on the contributions by the author of this thesis. A comparison to model predictions will be shown at the end of the chapter (\Sc{felm cmp}).

This analysis is based on the data from 29-30 October, when the \abb{LHC} was running at the standard 2010 optics with $\be^* = 3.5\un{m}$ and $\sqrt s = 7\un{TeV}$. Only the $220\un{m}$ stations were used. The horizontal pots were inserted only at the end of the run, just for alignment reasons. Most of the running time they were retracted not to disturb the elastic proton measurement (done with vertical pots only, see e.g.~\Fg{ttm hit distribution} right). The vertical effective lengths in the $220\un{m}$ stations lied in the range $L_y = (22 \hbox{ -- } 23)\un{m}$. The horizontal scattering angles were reconstructed from the track angles between near and far units (see \Eq{felm xy reco}). The proportionality factors $\d L_{x}/\d z$ had values of about $0.31$. A dedicated run with four low-luminosity bunches ($7\cdot10^{10}$ protons per bunch) was used. The normalized emittance was about $2\un{\mu m\ rad}$, which yielded the beam divergence $\si_{\th^*} \approx 17\un{\mu rad}$. For this special run the \abb{RP}s were approached as close as $7\si$ from the beam. This allowed us to measure the elastic differential cross-section down to $|t| = 0.36\un{GeV^2}$. The event sample used for the analysis (with horizontal \abb{RP}s retracted) corresponds to an integrated luminosity of $6.1\un{nb^{-1}}$.

\fig{fig/pdf/felm_scheme.pdf}{felm scheme}{The analysis work-flow, from raw data to the differential cross-section of the elastic scattering.}

\Fg{felm scheme} reviews the analysis steps, each of them is briefly described below. But before going into details, let us explain the terminology used. The sector 45 (56) is called left (right) arm of the \abb{TOTEM} experiment. The elastic events consist of two protons with practically opposite directions (up to beam divergence). Thus from detector point of view, they split into two groups -- with top \abb{RP}s active in the sector 45 (so called \em{diagonal} 45 top -- 56 bottom) or with top \abb{RP}s active in the sector 56 (diagonal 45 bottom -- 56 top).

\> The {\bf alignment} has been performed almost as described in \Sc{al}. We have used the \abb{LVDT} calibration from the latest collimation alignment (see \Tb{al lvdt off}), then the track-based alignment method (with the final constraints) has been applied (see \Sc{al lhc res}). These have been followed by the profile alignment (\Sc{al prof}), but no elastic alignment (\Sc{al elast}) has  been performed. The reason has been to refrain from biasing the event selection in any way. However, an alternative of method 1 from \Sc{al elast} has been employed. For that a sample of low-$\xi$ protons has been used. In this way we have obtained the horizontal alignment with a reasonable precision (below $10\un{\mu m}$), but the vertical misalignment has remained of the order of $100\un{\mu m}$ (see \Eq{al window to dp uncer}) -- this will be addressed in the step ``final vertical alignment''.

\> {\bf Optics tuning}. Various measurable quantities (including the elastic peak tilts, see \Tb{al el yx}) have been used to fine-tune the optics by optimizing the rotations and strengths of the \abb{LHC} magnets within their nominal uncertainties.

\> In the {\bf reconstruction} step, the projections of the scattering angle at \abb{IP} (see \Eq{ttm th x y}) are deduced from the \abb{RP} measurements. For each projection a different strategy is used, in order to minimize the impact of the uncertainty of the optical functions. For the vertical projection ($y$), the phase advance $\phi+\phi_0$ in \Eq{ttm hill sol} is close to $0$, thereby maximizing the amplitude of $y$ and minimizing the impact of the phase advance uncertainties. Thus the scattering angle can be determined directly from $y$ (cf.~\Eq{ttm lin par}): $\th_y^* = y/L_y$. In contrary, for the horizontal projection ($x$), the phase advance is close $\pi/2$. Consequently, it is better to use the angle $\th_x = \d x/\d s$, where the cosine term is replaced by a sine. With the given phase advance, the sine is close to its maximum, moreover the impact of phase advance variations is minimized. The angle $\th_x$ can be reconstructed within each station as $(x_{\rm F} - x_{\rm N}) / d$, where $d$ denotes the distance between near and far \abb{RP}s (see \Tb{ttm rp station}). To summarize, the scattering angle projections have been reconstructed as follows:
\eqref{\eqnarray{
	\hat\th_x^* = {\th_x^{\rm L} + \th_x^{\rm R}\over 2}\ ,\qquad
		& \th_x^{\rm L} = {1\over {\d L^{45}_x\over \d z}} {x^{45}_{\rm F} - x^{45}_{\rm N}\over d}\ ,\qquad
		& \th_x^{\rm R} = {1\over {\d L^{56}_x\over \d z}} {x^{56}_{\rm F} - x^{56}_{\rm N}\over d}\cr
	\hat\th_y^* = {\th_y^{\rm L} + \th_y^{\rm R}\over 2}\ ,\qquad
		& \th_y^{\rm L} = {1\over 2} \left( {y^{45}_{\rm F}\over L^{45}_{y, \rm F} } + {y^{45}_{\rm N}\over L^{45}_{y, \rm N} } \right)\ ,\qquad
		& \th_y^{\rm R} = {1\over 2} \left( {y^{56}_{\rm F}\over L^{56}_{y, \rm F} } + {y^{56}_{\rm N}\over L^{56}_{y, \rm N} } \right)\ ,\cr
}}{felm xy reco}
where for example $y^{56}_{\rm F}$ is the $y$ coordinate of a hit in the 56-220-far unit and $L^{56}_{y, \rm F}$ is the corresponding effective length. The factors $\d L/\d z$ are known optical functions, see the end of \Sc{rp measurement}.

\par\parindent\itindent\indent\hang
Let us comment on the vertex terms (see \Eq{ttm lin par}) that might seem to be omitted in our reconstruction formulae \Eq{felm xy reco}. The typical values of the magnification $v_{x, y}$ were about $-3.5$, the vertex smearing (see \Eq{ttm beam sigma}) was about $20\un{\mu m}$. Thus the vertex terms lead to a fluctuation with spread about $70\un{\mu m}$. However, the form \Eq{felm xy reco} is such that most of the vertex terms cancel. The differences between the magnifications left and right and near and far are on the level of $0.1$. Consequently the net effect of the vertex terms to the reconstruction is of the order $0.1 \cdot 20\un{\mu m} = 2\un{\mu m}$, which can be treated as small (random) perturbation.

\> The elastic {\bf event selection} requires a track in each of the four pots of a diagonal. Moreover the cuts summarized in \Fg{felm cuts} are imposed.

\> {\bf Final vertical alignment}. After the initial alignment (the first step in \Fg{felm scheme}), the vertical alignment with respect to the beam has been rather unsatisfactory, causing a shift $\De\th_y^*$ in the reconstructed vertical scattering angle $\hat \th_y^*$. Since the distribution of true vertical angles $\th_y^*$ is symmetric about zero, the reconstructed vertical angles $\hat\th_y^*$ are symmetric about $\De\th_y^*$. Due to the limited acceptance, only the tails of the $\hat\th_y^*$ distribution are available (each tail from one diagonal). This situation is very similar to fitting $y$ distributions in method 2 in \Sc{al elast}. Thus also this method (see \Fg{al el shift test}) could be reused.

\> {\bf Background subtraction}. The event-selection cuts (\Fg{felm cuts}) have been applied on the $3\si$ level, thus one may expect a very good efficiency. However, there might have been a handful of non-elastic events that have passed the cuts (i.e.~background). The background study presented in \Sc{felm bckg} is an updated version of the one published in \bref{totem11}. In the old study, the background was subtracted before the acceptance correction, as shown in \Fg{felm scheme}. This approach assumes the same acceptance corrections for signal (elastic scattering) and background. In the updated version, the background acceptance correction is performed separately, and the background is subtracted only afterwards. Within \Fg{felm scheme}, it could be described by swapping the background and acceptance correction steps.

\> The {\bf acceptance correction}, as described in \Sc{rp measurement}, is a procedure to correct for the finite size of \abb{RP} detectors (in other words to correct for some elastic/background events not being detected.)

\> The {\bf unsmearing} is a procedure that accounts for the finite resolution of the \abb{RP}s detectors and the beam smearing effects, for details see \Sc{felm unfold}.

\> In the {\bf normalization} step, the event-count histograms $\d N/\d t$ are transformed in the differential cross-section histograms $\d\si/\d t$. For that purpose, the luminosity and various efficiency factors (trigger, detection, reconstruction, etc.) are used.

\> {\bf Merging diagonals}. The results from both diagonals are merged and the final differential cross-section histogram is produced.





\section[felm bckg]{Background}

\fig{fig/pdf/felm_cuts.pdf}{felm cuts}{The selection cuts (for the diagonal 45 bottom -- 56 top). The color scale follows logarithmically the bin contents in the order: white, blue, green, red and black.}

The elastic-event-selection cuts for the final analysis have evolved compared to those used in \Sc{al elast}. At the end six cuts have been used, as shown in \Fg{felm cuts}. They are qualitatively similar to those presented in \Fg{al el selection}. The cut condition could be written as
\eqref{|q_i| < 3 \si_i\ ,}{felm cut}
where the \em{cut quantity} $q_i$ gives the distance from the ``expectation'' line which is drawn dotted in \Fg{felm cuts}. $\si_i$ is an experimentally determined standard deviation of the elastic-event distribution projected into the cut quantity.

Cuts 1 and 2 represent collinearity requirements, that is the left and right-arm protons are required to have opposite directions. The rest of the cuts implement low-$\xi$ requirements -- for such particles there is a strong correlation between the hit position and the angle in a station.


As \Eq{felm cut} suggests, all cuts have been applied on the $3\si$-level. Therefore we expect them to be of a very high efficiency. That is the number of true elastic events that fail the selection cuts is negligible. On the other hand, we expect a number of non-elastic ($\equiv$ background) events to pass the cuts -- these events will be studied in this section.

The task can be split into three parts: the \em{integral} (the number of background events which have passed the selection), the \em{distribution} (in $t_x$ and $t_y$) and the \em{contribution to the elastic $t$-distribution}.

But before the analysis, let us mention what background contributions we expect. From the physics processes these are mainly double pomeron exchange (\abb{DPE}) events or the pile-up of two single diffraction (\abb{SD}) events. With the luminosity per bunch collision of about $1.4\cdot10^{25}\un{cm^{-2}}$ and the cross-sections $6.8\un{mb}$ (for \abb{SD} from Pythia) and $1.3\un{mb}$ (for \abb{DPE} from Phojet) one obtains per-bunch event probabilities $0.014$ (\abb{DPE}) and $0.008$ (twice \abb{SD} with a proton in each arm). Thus both contributions are roughly on the same level of importance. The output of the mentioned \abb{MC} generators can be, in a first approximation, parameterized as follows:
\eqref{
{\d\si^{\rm SD}\over\d\xi\, \d t } \sim {\e^{b t}\over \xi^{1+\ep}}\ ,\qquad
{\d\si^{\rm DPE}\over\d\xi_{\rm L} \d t_{\rm L} \d\xi_{\rm R} \d t_{\rm R} } \sim {\e^{b t_{\rm L}}\over \xi^{1+\ep}_{\rm L}} {\e^{b t_{\rm R}}\over \xi_{\rm R}^{1+\ep}}
\ .}{felm sd dpe cs}

\block{The integral}

If $h_{\rm B}(q_1, \ldots , q_6)$ was the distribution of the background events in the six cut quantities, then the number of background events passing the selection cuts would be
\eqref{N_{\rm B} = \int\limits_{-3\si_1}^{+3\si_1} \d q_1\ \ldots\!\! \int\limits_{-3\si_6}^{+3\si_6} \d q_6\  h_{\rm B}(q_1, \ldots , q_6)\ .}{felm bckg int}
Some information on the background distribution $h_{\rm B}$ can be obtained by relaxing one of the cuts and analyzing the distribution of the corresponding cut quantity. In the region $|q_i| < 3\si_i$ the distribution is dominated by the elastic events ($\equiv$ signal), but in the tails $|q_i|>3\si_i$ one can sense the background. For example for the first cut quantity, the tails should be described by (one-cut-quantity distribution)
\eqref{h_1(q_1) = \int\limits_{-3\si_2}^{+3\si_2} \d q_2\ \ldots\!\! \int\limits_{-3\si_6}^{+3\si_6} \d q_6\  h_{\rm B}(q_1, \ldots , q_6)\ .}{felm bckg q1 dist}
Provided that one can reconstruct the $h_1(q_1)$ from tails, the background integral can be calculated
\eqref{N_{\rm B} = \int\limits_{-3\si_1}^{+3\si_1} \d q_1\ h_1(q_1)\ .}{felm bckg int 2}
Such a procedure can be repeated for each cut quantity, in principle yielding six independent determinations of the integral $N_{\rm B}$.

\fig{fig/pdf/felm_background_int_dg_fit.pdf}{felm background int dg fit}{An example of the one-cut-quantity distributions (diagonal 45 bottom -- 56 top). The blue curve represents a double-Gaussian fit, with the red curve being its background part.}

Two examples of the one-cut-quantity distributions are shown in \Fg{felm background int dg fit}. It turns out reasonable to describe the signal+background data as a superposition of two Gaussians.
%\TODO{1: $\ch^2/\hbox{ndf} = 1.6$, 4: $3.1$}). Although the $\ch^2/\hbox{ndf} \approx$ in both shown cases
There is an important difference between the two plots. For cut 1 (left), there are enough background events on both sides of the signal peak. Thus, the background fit is trustworthy. In contrary, for cut 4 (right), there is nearly no background in the right tail -- the fit is doubtful in this case. Unfortunately this the case for most of the cuts. It is related to a simple observation that every cut has a different discrimination power. From this point of view, cut 1 is the best one and that is why there is enough background at both tails, allowing for a reasonable fit.

Besides the double-Gauss fits we have tried single-Gauss fits through the tail data only. Eventually, we have calculated a weighted mean (the weight reflecting our confidence in a given result), yielding the values in \Tb{felm bckg int result}.

\tab{felm bckg int result}{The numbers of background events that passed the selection cuts.}{
\omit&\multispan2\bhrulefill\cr
\omit&\omit\bvrule\bstrut\tskip 45 bottom -- 56 top\tskip&\hbox{45 top -- 56 bottom}\cr\bln
N_{\rm B} & 2500 \pm 300 & 2700 \pm 300\cr\bln
}

\block{The distribution}

We will assume that the left and right protons are independent in the background events. First of all, it is a reasonable assumption. For pile-up background (\abb{SD}, beam-gas) it is evident. For \abb{DPE} described by the cross-section \Eq{felm sd dpe cs} it holds too. Had we enough background data, we could test this hypothesis, however we will see that the statistics is not rich. Taking the assumption, we can factorize the two-proton distribution $h_{\rm 2p}$ into one-proton distributions $h_{\rm 1p}$:
\eqref{h_{\rm 2p}(t_x^{\rm L}, t_y^{\rm L}, t_x^{\rm R}, t_y^{\rm R}) = h_{\rm 1p}(t_x^{\rm L}, t_y^{\rm L})\ h_{\rm 1p}(t_x^{\rm R}, t_y^{\rm R})\ .}{felm bckg dist fact}

We consider these distributions as functions of $t$'s only since we are investigating the background to elastic scattering. Elastic protons have $\xi\equiv 0$ by definition, therefore only $t_x$ and $t_y$ are reconstructed. The same applies to inelastic protons that pass the elastic-event selection as background events. Their $\xi$ distribution is absorbed to an effective $t_x$ and $t_y$ distribution.

Moreover, on the level of one station $t_x$ ($t_y$) is reconstructed from $x$ ($y$) hit positions at the station, thus one could think of $t_x$ and $t_y$ as an alternative hit position description. In particular, the acceptance function expressed in $t_x$ and $t_y$ is as simple as in $x$ and $y$ -- it is merely $\Th(|t_y| > |t_y|_{\rm min})$. \footnote{%
We will use the symbol $\Th$ for a function that has value 1 if the condition in parentheses is fulfilled and is 0 otherwise.
}

\bmfig
\fig{fig/pdf/felm_background_dist_txty.pdf}{felm background dist txty}{[7cm]The distribution of background $h_{\rm O}$ (diagonal 45 bottom -- 56 top, left arm). The color scale follows logarithmically the bin contents in the order: white, blue, green, red and black.}
\fig{fig/pdf/felm_background_dist_tx.pdf}{felm background dist tx}{[7cm]The observed $t_x$-distribution (left arm, diagonal 45 bottom -- 56 top) with an $\exp(b t_x)/\sqrt{-t_x}$ fit (red curve).}
\emfig

The background distribution observed in one arm (the left one for instance) can then be expressed as
\eqref{\eqnarray{
h_{\rm O}(t_x^{\rm L}, t_y^{\rm L}) & = &
\int \d t_x^{\rm R}\, \d t_y^{\rm R}\ h_{\rm 2p}(t_x^{\rm L}, t_y^{\rm L}, t_x^{\rm R}, t_y^{\rm R})
	\ \Th(|t_y^{\rm L}| > |t_y^{\rm L}|_{\rm min})
	\ \Th(|t_y^{\rm R}| > |t_y^{\rm R}|_{\rm min})
\cr
&\propto &\ h_{\rm 1p}(t_x^{\rm L}, t_y^{\rm L})
	\ \Th(|t_y^{\rm L}| > |t_y^{\rm L}|_{\rm min})
\ .\cr
}}{felm bckg dist obs}
An example of such distribution is shown in \Fg{felm background dist txty}. To build the histogram, we have used only the events that have failed cut 1 ($\th_x$ collinearity), which anyway provides most background events. One can clearly see the acceptance cut around $|t_y| = 0.3\un{GeV^2}$ (see also \Tb{felm bckg results}), the effect of the last $\Th$ function in \Eq{felm bckg dist obs}. The distribution seems to decrease exponentially with $t = t_x + t_y$, something one would naively expect from the \abb{SD} and \abb{DPE} cross-sections \Eq{felm sd dpe cs}. However, if $t$ was distributed exponentially $\exp(b t)$, the corresponding distribution of $t_x$ and $t_y$ would be $\exp(b t_x + b t_y)/\sqrt{t_x t_y}$
%(the square-root factor comes from the Jacobian)
. Moreover, we reconstruct the background protons as elastic: the effect of non-zero $\xi$ is absorbed into the observed $t_x$ vs.~$t_y$ distribution. Since the dispersion acts mostly in the horizontal direction, we may expect the $t_x$-distribution to be altered. This consideration has led us to try the suggested parameterization with $b$'s independent for $x$ and $y$:
\eqref{h_{\rm 1p}(t_x, t_y) \propto {\e^{b_x t_x}\over \sqrt{-t_x}} {\e^{b_y t_y}\over \sqrt{-t_y}}\ .}{felm dist param}

As a first test, one can compare the $t_x$-distribution with the $x$ part of the suggested parameterization. \Fg{felm background dist tx} shows an excellent match. Then, we have performed a careful fit of the $t_x$ vs.~$t_y$ data. For that we have cut off the part of the histogram which is full of empty bins and we have removed the closest bins to the acceptance cut. The resulting bin selection is shown as the red dashed triangle in \Fg{felm background dist txty}. The results of the fit are summarized in \Tb{felm fit}, note the very good $\ch^2$ (with about 120 bins contributing). Whereas the slopes are clearly different between the $x$ and $y$ projection, their values are consistent within all four fits. This has led us to the conclusion that the background distribution is the same left and right and for both diagonals. Further on, we will use the mean slopes (last row of the table). The error of the mean has been calculated as the standard deviation of the 4 measurements divided by $\sqrt{n = 4}$.

\tab[\bvrule\hfil\ #\ \hfil&\bvrule\bstrut\hfil\ #\ \hfil&\vrule\hfil\ #\ \hfil&\vrule\hfil\ #\ \hfil&\vrule\hfil\ #\ \hfil&\bvrule\hfil\ #\ \hfil\bvrule\cr]{felm fit}{The result of a fit of parameterization \Eq{felm dist param} through $t_x$ vs.~$t_y$ data.}{%
\omit&\multispan4\bhrulefill&\omit\cr
\omit&\multispan2\bstrut\bvrule\hfil 45 bottom -- 56 top\hss&\multispan2\vrule\hss\hbox{45 top -- 56 bottom}\hss&\omit\bvrule\hss\cr
\omit&\multispan4\hrulefill&\omit\bhrulefill\cr
\omit&left&right&left&right&mean\cr\bln
$b_x\ ({\rm GeV^{-2}})$ & $3.3 \pm 0.2$ & $2.1 \pm 0.2$ & $2.5 \pm 0.2$ & $2.7 \pm 0.2$ & $2.7 \pm 0.3$\cr\ln
$b_y\ ({\rm GeV^{-2}})$ & $5.4 \pm 0.2$ & $4.9 \pm 0.2$ & $4.7 \pm 0.2$ & $4.5 \pm 0.2$ & $4.9\pm 0.2$\cr
\multispan5\hrulefill&\omit\bhrulefill\cr
$\ch^2/\hbox{ndf}$ & 1.5 & 0.8& 1.1 &\omit\vrule\hfil 0.9\hfil\bvrule & \omit\cr
\multispan5\bhrulefill&\omit\cr
}

\block{The background contribution to the elastic $t$-distribution}

Our fit \Eq{felm dist param} determines the two-proton background distribution \Eq{felm bckg dist fact} up to a normalization constant. But this is fixed by the known integral of the background events, see \Tb{felm bckg int result}. This integral gives the number of observed (i.e.~within acceptance) background events that satisfy the selection cuts. It relates to the background distributions as:
\eqref{\eqnarray{
N_{\rm B} & = & \int \d t_x^{\rm L}\, \d t_y^{\rm L}\, \d t_x^{\rm R}\, \d t_y^{\rm R}\ h_{\rm 2p}(t_x^{\rm L}, t_y^{\rm L}, t_x^{\rm R}, t_y^{\rm R})
	\ \Th(\hbox{acceptance})\ \Th(\hbox{selection})\ ,\cr
%\Th(\hbox{acceptance}) & = &\ \Th(|t_y^{\rm L}| > |t_y^{\rm L}|_{\rm min})\ \Th(|t_y^{\rm R}| > |t_y^{\rm R}|_{\rm min})\cr
%\ \Th(\hbox{selection}) & = &\ \Th(|\sqrt{t_x^{\rm R}} - \sqrt{t_x^{\rm L}}| < ) \cr
}}{felm bckg dist norm cond}
where $\Th(\hbox{acceptance})$ requires the $t$ values to be such that the event falls into the acceptance (cf.~\Eq{felm bckg dist obs}). Similarly, $\Th(\hbox{selection})$ implements the first two ($\th_x$ and $\th_y$ collinearity) selection requirements.

When a background event passes the selection cuts, it is assigned an ``elastic'' $t$-value (cf.~\Eq{felm xy reco}):
\eqref{\bar t(t_x^{\rm L}, t_y^{\rm L}, t_x^{\rm R}, t_y^{\rm R})
= \left( \sqrt{t_x^{\rm L}} + \sqrt{t_x^{\rm R}}\over 2 \right)^2 + \left( \sqrt{t_y^{\rm L}} + \sqrt{t_y^{\rm R}}\over 2 \right)^2\ .}{felm bckg t el}
Thus, the background contribution to the elastic $t$-distribution is given by the distribution of $\bar t$, indeed within the selection cuts:
\eqref{{\d N_{\rm B, el}\over \d t}(t) = \int
	\d t_x^{\rm L}\, \d t_y^{\rm L}\, \d t_x^{\rm R}\, \d t_y^{\rm R}
	\ h_{\rm 2p}(t_x^{\rm L}, t_y^{\rm L}, t_x^{\rm R}, t_y^{\rm R})
	\ \de\!\left(t - \bar t(t_x^{\rm L}, t_y^{\rm L}, t_x^{\rm R}, t_y^{\rm R}) \right)
	\Th(\hbox{selection})\ .
}{felm bckg el t dist}
That corresponds to the background contribution ``after the acceptance correction''. For the background, the acceptance correction has been achieved by simply extending the $h_{\rm 2p}$ distribution to the non-observed region (outside acceptance). One can easily modify \Eq{felm bckg el t dist} in order to obtain the background contribution to the elastic $\d N/\d t$ without the acceptance correction -- it is enough to multiply the integrand with $\Th(\hbox{acceptance})$.

We have evaluated the integral \Eq{felm bckg el t dist} with the aid of \abb{MC} techniques (see the blue histograms in \Fg{felm background cmp}). We have found that the resulting distributions can well be parameterized as:
\eqref{{\d N_{\rm B, el}\over \d t} = a \e^{b t + c t^2}}{felm bckg el t dist param}
in the parts not affected by the acceptance (see the red solid lines in \Fg{felm background cmp}). In fact, we have made a number of \abb{MC} simulations. We have taken the mean $h_{\rm 1p}$ parameters from \Tb{felm fit} and varied them within their uncertainties. We have done the same for the background integral $N_{\rm B}$ from \Tb{felm bckg int result}. This has led to the $\d N_{\rm B,el}/\d t$ uncertainty bands which are drawn as red dashed lines in \Fg{felm background cmp}. The mean fit parameters $a$, $b$ and $c$ with their uncertainties are summarized in \Tb{felm bckg results}.

\fig{fig/pdf/felm_background_cmp.pdf}{felm background cmp}{Background-signal comparisons for the diagonal 45 bottom - 56 top. Left: before, Right: after the acceptance correction. The blue histograms are the $\d N_{\rm B, el}/\d t$ simulations with mean $h_{1p}$ and $N_{\rm B}$ parameters. The red lines represent their fits with the parameterization \Eq{felm bckg el t dist param}. The dashed red lines delimit $1\si$ error bands of the background (red) curves. The black histograms correspond to signal+background, i.e.~to all events that have passed the selection cuts.}


\tab{felm bckg results}{The background contribution to the elastic $t$-distribution (after the acceptance correction) parameterized as \Eq{felm bckg el t dist param}.}{
\omit&\multispan2\bhrulefill\cr
\omit&\omit\bvrule\bstrut\tskip 45 bottom -- 56 top\tskip&\hbox{45 top -- 56 bottom}\cr\bln
|t_{y}^{\rm L}|_{\rm min}\ \rm (GeV^{-2}) & 0.315 \pm 0.01 & 0.335 \pm 0.01\cr\ln
|t_{y}^{\rm R}|_{\rm min}\ \rm (GeV^{-2}) & 0.315 \pm 0.01 & 0.265 \pm 0.01\cr\bln
a\ \rm (GeV^{-2}) & 2.6 \pm 0.6 & 2.4 \pm 0.6 \cr\ln
b\ \rm (GeV^{-2}) & 6.9 \pm 0.6 & 6.9 \pm 0.6 \cr\ln
c\ \rm (GeV^{-4}) & 0.4 \pm 0.1 & 0.4 \pm 0.1 \cr\bln
}

Taking the data from \Fg{felm background cmp} right, one can subtract the background (red) from signal+background (black), leading to an acceptance-corrected $t$-distribution of elastic-scattering only. This distribution then enters the next step -- the unfolding.


\section[felm unfold]{Unfolding}

The scattering angles $\th_x'$ and $\th_y'$ \footnote{
In this section we will always mean the physics scattering angles (i.e.~at the interaction point, see \Sc{rp measurement}). To meet our notation from \Sc{rp measurement} we should use star superscripts $^*$, however this would lead to expressions graphically too heavy and thus we will drop them in this section. Moreover, we will use our traditional distinction between measured/reconstructed (with prime) and true/original (without primes) quantities.
} are reconstructed by taking the average over the two one-arm measurements (cf.~\Eq{felm xy reco}):
\eqref{\th_x' \equiv {\th_x^{\rm L} + \th_x^{\rm R}\over 2} = \th_x + \De\th_x}{felm unf th err}
and equivalently for the $y$ projection. The left and right angle measurements are subject to an error, our estimation of which is summarized by \Eq{elr th err proj sig}. We expect two main contributions: due to the beam-divergence ($\si_{\rm B} \approx 17\un{\mu rad}$) and due to the finite resolution of the silicon sensors (%
% Lx ~ 1.68 m, Ly ~ 21.45
$\si_{\rm R} \approx 0.4\un{\mu rad}$ for $y$ and $5.7\un{\mu rad}$ for $x$ projection). Because of the arithmetic mean, the standard deviation of $\De\th_x$ is reduced by a factor of $\sqrt 2$. We expect therefore $\si(\De\th_x) \approx 12.5\un{\mu rad}$. In fact, it can be measured by examining the distribution of $(\th_x^{\rm L} - \th_x^{\rm R})/2$. It yields
\eqref{\si(\De\th_x) \simeq \si(\De\th_y) = (12.5\pm 0.5)\un{\mu rad}\ .}{felm unf sigma proj}

Suppose that $h(\th)$ is the true distribution of the (non-smeared) scattering angle $\th$
%($h(\th)$ proportional to $\d\si/\d\th$ cross-section)
. Then, after the acceptance correction etc., we would reconstruct a distribution $h'(\th')$ smeared due to the finite angular resolution. These distributions can be related by an integral transformation (see e.g.~Sec.~16.6.1 in \bref{riley10} or Sec.~1.8.2 in \bref{becvar}):
\eqref{h'(\th') = \int {\d\ph\over 2\pi} \int \d\De\th_x\ E(\De\th_x) \int \d\De\th_y\ E(\De\th_y)
	\ \left| \d\th\over\d\th' \right|\ h(\th)\ ,
}{felm unf sm model}
where $E$ represents the angular-measurement-error distribution. $\th \equiv \th(\th', \ph, \De\th_x, \De\th_y)$ is the non-smeared value of scattering angle corresponding to the smeared angle $\th'$, (non-smeared) azimuthal angle $\ph$ and angular measurement errors $\De\th_x$ and $\De\th_y$. It can be obtained by solving this equation:
\eqref{\th'^2 = (\th\cos\ph + \De\th_x)^2 + (\th\sin\ph + \De\th_y)^2\ .}{felm unf th eq}

If we compare the measurement error size \Eq{felm unf sigma proj} with the lowest $\th$ in our analysis $\approx 170\un{\mu rad}$ (see the dashed line in \Fg{felm unfolding m1 fit}), we see that we may safely work in a limit $\si(\De\th_{x, y}) \ll \th$. In this approximation, \Eq{felm unf th eq} can be solved easily:
\eqref{\th = \th' - (\De\th_x \cos\ph + \De\th_y \sin\ph)\ .}{felm unf th sol}
Inserting it to \Eq{felm unf sm model}, the Jacobian $|\d\th/\d\th'|$ simplifies to 1. We can well describe the angular-measurement-error distributions $E$ by Gaussians with sigma given by \Eq{felm unf sigma proj}. If we expand $h(\th)$ around $\th = \th'$ we obtain
\eqref{h'(\th') = h(\th') + h^{(2)}(\th') {\si^2\over 2} + \ldots \ ,}{felm unf sm model exp}
where $h^{(2)}$ represents the second derivative of $h(\th)$. The same first two terms would be obtained in a smearing model
\eqref{h'(\th') = \int \d\De\th\ E(\De\th)\ h(\th' - \De\th)  \ .}{felm unf sm model sim}
That is a Gaussian smearing that acts directly on $\th$, moreover with the sigma numerically equal to \Eq{felm unf sigma proj}. This is what one expects from the error propagation from $\th_x$ and $\th_y$ to $\th$, see the comment below \Eq{elr th err sig}.

The simplified smearing model \Eq{felm unf sm model sim} is technically advantageous. The smearing applies in $\th$ (which is directly related to $t$, the quantity of interest), it does not mix the $x$ and $y$ components. We will numerically verify its validity for our data later on. For the time being, we can state that it will provide a good approximation to the full model \Eq{felm unf sm model} whenever the $\th$-distribution $h(\th)$ can be locally (in region $\th \pm \hbox{few times }\De\th_{x, y}$) well described by a cubic function (terms with odd derivatives drop from expansions of the type \Eq{felm unf sm model exp}).

We will now describe two unsmearing methods: a fit-based and a bin-based method. Each of them has advantages and disadvantages. The bin-based method suffers from the bin-content fluctuations which are further reinforced by the unsmearing procedure. A fit may, by its nature, provides a smoothing of the data, hence removes the troublesome fluctuations. However, there is a certain arbitrariness of the data description within their errors. It is thus reassuring to find both methods giving compatible results, as we will show at the end.


\block{The fit-based method}

The idea of this method is to describe the data with a function for which the unsmearing (reverse transform \Eq{felm unf sm model sim}) can be done analytically. Moreover, the function shall provide a realistic extrapolation to the low-$\th$ region, where we miss acceptance. In that region we expect the cross-section to be well approximated by $\d\si/\d t \propto \exp(bt)$. This translates into the $\th$-distribution as $h(\th) \propto \th \exp(-bp^2 \th^2)$.

We have parameterized the data ($\th'$ histogram) by
\eqref{h'(\th') = H(\th') + \sum_{i = 1}^3 G_i(\th'),
	\qquad H(\th') = a\, \th' \e^{- {\th'^2\over 2 T^2}},
	\qquad G_i(\th') = a_i\, \e^{- {(\th' - \mu_i) ^2\over 2 T_i^2}},
}{felm unf fm param}
with $a$, $\mu$ and $T$'s being the free parameters (11 in total). The $H$ function controls the low-$\th'$ behavior. The fit is plotted in \Fg{felm unfolding m1 fit}. The dashed vertical line represents our analysis cut - we do not trust the data below this point, because of too large acceptance corrections (which are subject to too large uncertainties). We show the data points below the analysis cut here in order to demonstrate that the ``extrapolation'' function $H$ is not unreasonable.

\bmfig
\fig{fig/pdf/felm_unfolding_m1_fit.pdf}{felm unfolding m1 fit}{[7cm]The reconstructed $\th$-distribution (diagonal 45 top -- 56 bottom) with a fit according to the parameterization \Eq{felm unf fm param}.}
\fig{fig/pdf/felm_unfolding_m1_correction.pdf}{felm unfolding m1 correction}{[7cm]The smearing correction for diagonal 45 top -- 56 bottom and two representative values of $\si(\De\th)$.}
\emfig

The unsmearing (the reverse of the transform \Eq{felm unf sm model sim}) keeps the form of the parameterization \Eq{felm unf fm param} unchanged, only the parameters receive modifications:
\eqref{\eqnarray{
H(\th') \to H(\th)& :\quad & T^2 \to T^2 - \si^2,\quad a \to a \left( T^2\over T^2 - \si^2 \right)^{3\over 2} \cr
G_i(\th') \to G_i(\th)& :\quad & T_i^2 \to T_i^2 - \si^2,\quad a_i \to a_i \left( T_i^2\over T_i^2 - \si^2 \right)^{1\over 2},\quad \mu_i \to \mu_i\ ,\cr
}}{felm unf fm unsmear}
where $\si$ is given by \Eq{felm unf sigma proj}.

\Fg{felm unfolding m1 correction} show the resulting smearing correction
\eqref{\hbox{smearing correction} = {\hbox{unsmeared distribution} \over \hbox{measured distribution}}\ .}{felm unf fm sm corr}
The two curves correspond to two values of $\si(\De\th)$, representing the uncertainty of its determination, see \Eq{felm unf sigma proj}. The difference between the blue and red curves thus evaluates the systematic error of this unsmearing method. The statistical error, drawn as the light-red band, has been calculated by propagating the uncertainties of the 11 fit parameters. We have adjusted the fit such that, in our opinion, the resulting error band reasonably reflects the actual uncertainty. As we have anticipated in the introduction, a fit always brings in a certain ambiguity in data description. Different parametrizations, different fit-procedure settings, etc.~can lead to a number of results. Even though the fits might be equally good (similar value of $\ch^2$), they might be quite different. We had this in mind when calculating the light-red error band.

The ultimate uncertainty has been calculated as the total (maximal) error band for $\si(\De\th) = 12$ and $13\un{\mu rad}$ for both diagonals.

Before we move to the second method, let us verify the validity of the simplified smearing model \Eq{felm unf sm model sim}. Let us take the unsmeared distribution and re-apply the smearing according to the full model \Eq{felm unf sm model}. We have performed the integrations with the help of a \abb{MC} simulation. The result is shown in \Fg{felm unfolding m1 test}, one can see a perfect match.

\fig{fig/pdf/felm_unfolding_m1_test.pdf}{felm unfolding m1 test}{Our unsmearing result smeared again with the full model \Eq{felm unf sm model}, compared to the input data.}

\block{The bin-based method}

The idea of this method is to ``learn'' the trend of the smearing corrections by applying several additional (extra) smearing steps, as depicted in \Fg{felm unfolding m2 scheme}. The ultimate correction (from measured to unsmeared distributions) can then be extrapolated from corrections 1 to 3.

\fig{fig/pdf/felm_unfolding_m2_scheme.pdf}{felm unfolding m2 scheme}{The scheme of the bin-based method. The arrows pointing right represent smearing, the ones pointing left represent unsmearing.}

Before applying any smearing step, we should make sure that the input distribution contains enough data. For example, if we like to examine the smeared distribution above a given value $\th_{\rm c}$, the input distribution (before any smearing step) should extend to at least $\th_{\rm c} -3\si(\De\th)$. Therefore, if the distribution after 3 extra smearing steps should be meaningful above the analysis cut $\th_c = 170\un{\mu rad}$, the measured distribution should be defined down to $(170 - 9\cdot 12.5) \un{\mu rad}\approx 60 \un{\mu rad}$. It is therefore necessary to extrapolate the measured distribution down to that value. To achieve this, we have used the $H$ parameterization (see \Eq{felm unf fm param}), just as in the fit-based method.

The extrapolated measured distribution is shown in \Fg{felm unfolding m2 addsm}, together with the distributions after the extra smearing steps which have been calculated with a \abb{MC} method according to the model \Eq{felm unf sm model sim}. One can see that the effect of smearing is not linear. That is why we have chosen to perform three extra smearing steps -- it allows for a quadratic (the simplest non-linear) extrapolation.

The correction functions (compatible with the definition \Eq{felm unf fm sm corr}) are displayed in \Fg{felm unfolding m2 corrections}. The ultimate correction has been obtained by a per-bin quadratic extrapolation of the corrections 1 to 3. It is clear that bin fluctuations get amplified by the extrapolation, thus the fluctuations of the black curve are the most pronounced.

\bmfig
\fig{fig/pdf/felm_unfolding_m2_addsm.pdf}{felm unfolding m2 addsm}{[7cm]The extrapolated measured $\th$-distribution (diagonal 45 top -- 56 bottom) and the distributions after extra smearing steps ($\si(\De\th) = 12\un{\mu rad}$ used).}
\fig{fig/pdf/felm_unfolding_m2_corrections.pdf}{felm unfolding m2 corrections}{[7cm]The smearing correction functions as defined in \Fg{felm unfolding m2 scheme}.}
\emfig

\block{Method comparison}

The fit-based and bin-based methods are compared in \Fg{felm unfolding m1m2 cmp}. Despite the large fluctuation of the bin-based result, a very good match is evident. The discrepancy at low $\th$ values is practically irrelevant because it falls below the analysis cut (drawn as the dashed line). We have only used the fit-based result (and its uncertainty) for the cross-section calculation. Nevertheless, we consider the bin-based result as a reassuring cross-check.

\fig{fig/pdf/felm_unfolding_m1m2_cmp.pdf}{felm unfolding m1m2 cmp}{A comparison of the fit-based and bin-based method (for diagonal 45 top -- 56 bottom and with $\si(\De\th) = 12\un{\mu rad}$).}




\section[felm cmp]{Comparison to models}

The final cross-section distribution compared to the predictions of the models from \Sc{el models} and the model of Jenkovszky et al.~\bref{jenkovszky11} is shown in \Fg{ttm mod cmp dsdt}. The discriminative power of \abb{TOTEM}'s measurement is evident. In the region below the diffractive dip, $|t| \ls 0.5\un{GeV^2}$, there is a relatively good agreement among the models themselves and the measurement. But for momentum transfers $|t|$ between $0.5$ and $1\un{GeV^2}$ the models exhibit a rather large spread in their predictions (one and half orders). Moreover the measurement of \abb{TOTEM} does not favour any of the presented models. The measurement falls into a gap between two bands of predictions (one band is formed by the models of Islam et al., Petrov et al.~with two pomerons and Jenkovszky et al., the other band is formed by the models of Block et al., Bourrely et al.~and Petrov et al.~with three pomerons).

\fig{fig/pdf/ttm_mod_cmp_dsdt.pdf}{ttm mod cmp dsdt}{Comparison of model predictions and \abb{TOTEM}'s first measurement of $\rm pp$ differential cross-section at $\sqrt s = 7\un{TeV}$.}

We have extracted the exponential slope $B$ (see \Eq{el B}) from \abb{TOTEM}'s cross-section measurement (thick black histogram in \Fg{ttm mod cmp dsdt}). We have split the histogram into groups of six consecutive bins. Each of these groups was fitted with an exponential, resulting in a point in in \Fg{ttm mod cmp B}. The horizontal error bars indicate the span of the six-bin groups, the vertical error bars reflect the fit uncertainty. The discriminative power of this plot is smaller ($B$ accounts only for the ``shape'' and not the normalization of the cross-section), but still one can conclude that the models of Jenkovszky et al.~and of Petrov et al.~(the two-pomeron version) are disfavored by \abb{TOTEM}'s data.

\fig{fig/pdf/ttm_mod_cmp_B.pdf}{ttm mod cmp B}{Comparison of the exponential slope $B$ from model predictions and \abb{TOTEM}'s first measurement of $\rm pp$ differential cross-section at $\sqrt s = 7\un{TeV}$.}
