\input slides
\input utf8-csf

%----------------------------------------------------------------------------------------------------

\makecom{toc indent 0}{0mm}
\makecom{toc indent 1}{7mm}
\makecom{toc indent 2}{14mm}

\def\TOCline#1#2#3#4{%
	\edef\Indent{\csname toc indent #1\endcsname}%
	\hbox{%
		\hbox to\Indent{\hss}%
		\vtop{\hsize7cm\advance\hsize-\Indent\noindent\cmyk{#4}\strut#2\ #3\strut}%
	}%
}

\def\FigLabel{\SmallerFonts\it}

\def\Question#1{\vbox{\vskip2mm\it ``#1''\vskip2mm\hrule}}

\def\cTi{\cmyk{\TitColor}}
\def\cFg{\cmyk{\FgColor}}

%----------------------------------------------------------------------------------------------------

\StdBackground

%----------------------------------------------------------------------------------------------------

\footline={}
\hbox{}
\vfil
\title{Elastic Scattering at the LHC}
\vfil
\centerline{Jan Kašpar}
\vfil
\centerline{doctoral thesis defense, 10 April 2012}


\newpage%------------------------------------------------------------------------------------------
\title{(Reduced) Table of Contents}

\line{\offinterlineskip\hss\vtop{%
%\TOCline{0}{}{Introduction}{\FgColor}
\TOCline{0}{1}{Elastic scattering of protons}{\FgColor}
\TOCline{1}{1.1-2}{Hadronic models and Their Predictions for the LHC}{\FgColor}
\TOCline{1}{1.3}{Coulomb interference}{\FgColor}
\TOCline{0}{2}{The TOTEM experiment}{\FgColor}
\TOCline{1}{2.1}{The Roman Pot system}{\FgColor}
\TOCline{1}{2.2}{Proton Measurement with Roman Pot detectors}{\FgColor}
%\TOCline{1}{2.3}{The total cross section}{\FgColor}
\TOCline{0}{3}{Roman Pot simulation and reconstruction software}{\FgColor}
\TOCline{1}{3.1}{Elegent}{\FgColor}
\TOCline{1}{3.2}{Beam smearing}{\FgColor}
\TOCline{1}{3.3}{Fast simulation}{\FgColor}
\TOCline{1}{3.4}{Pattern recognition}{\FgColor}
\TOCline{1}{3.5}{Reconstruction of elastic events}{\FgColor}
%\TOCline{1}{3.x}{Data Quality Monitor}{\FgColor}
}\hskip 5mm plus1fil\vtop{%
\TOCline{0}{4}{Alignment of Roman Pots}{\FgColor}
\TOCline{1}{4.1}{Collimation alignment}{\FgColor}
\TOCline{2}{4.2.1-2}{Track--based alignment}{\FgColor}
\TOCline{2}{4.2.3}{Singular and weak misalignment modes}{\FgColor}
\TOCline{2}{4.2.4}{Imposed constraints}{\FgColor}
\TOCline{2}{4.2.6}{Monte-Carlo tests}{\FgColor}
\TOCline{2}{4.2.8}{Alignment of detector packages}{\FgColor}
\TOCline{2}{4.2.10}{LHC Alignment Results}{\FgColor}
\TOCline{2}{4.2.9}{Internal alignment results}{\FgColor}
\TOCline{1}{4.3}{Profile methods}{\FgColor}
\TOCline{1}{4.4}{Elastic Alignment}{\FgColor}
\TOCline{1}{4.5}{Summary}{\FgColor}
\TOCline{0}{5}{The first elastic scattering measurement at the LHC}{\FgColor}
\TOCline{1}{5.1}{Background}{\FgColor}
\TOCline{1}{5.2}{Unfolding}{\FgColor}
\TOCline{1}{5.3}{Results and comparison to models}{\FgColor}
}\hss}

%\newpage%------------------------------------------------------------------------------------------
%\title{Introduction}
%\> motivation
%\> the need of hadronic models


\newpage%------------------------------------------------------------------------------------------
\title{Chapter 1. Elastic Scattering of Protons}
\vskip-3mm
\subtitle{Kinematics}

\> process: p($1$) + p($2$) $\longrightarrow$ p($1'$) + p($2'$)

\line{\hss\fig*{fig/pdf/el_scat_scheme.pdf}\hss}

\> scattering angle $\th$
\> azimuthal angle $\ph$

\> projections of scattering angle
\>> horizontal $\th_x = \th \cos\ph$
\>> vertical $\th_y = \th \sin\ph$

\> proton momentum $p$

\> four-momentum transfer squared (and its components)
$$t = -2p^2 (1 - \cos\th) \simeq -(p\th)^2\ \,\qquad t_x = t \cos^2\ph\ ,\qquad t_y = \cos^2\ph$$

\newpage%------------------------------------------------------------------------------------------
\title{Chapter 1. Elastic Scattering of Protons}
\vskip-3mm
\subtitle{Dynamics}

\> only two forces relevant

{
\itindent7mm
\iitindent10mm

\vfil
\noindent \cmyk{\TitColor}1) Electromagnetic interaction\cmyk{\FgColor}

\line{\raise8mm\vbox{\hsize10cm
\> QED can be used
\> effective form factor: combines electric and magnetic 
\>> magnetic non-negligible above $|t|\gs 0.1\un{GeV^2}$
\> one-photon-exchange approximation is insufficient above $|t| \gs 1\un{GeV^2}$
\>> need to account for diagrams with proton excitations
}\hskip3mm plus1fil\fig*[,4.5cm]{fig/pdf/el_ff_comparison_slides.pdf}}
%\centerline{\fig*[,1cm]{fig/pdf/el_diagrams2.pdf}}

\vfil
\noindent \cmyk{\TitColor}2) Strong interaction\cmyk{\FgColor}

\> QCD cannot be applied directly
\> phenomenological models needed

\vfil
\noindent \cmyk{\TitColor}1+2) Interference between electromagnetic and strong interaction effects\cmyk{\FgColor}

}

\newpage%------------------------------------------------------------------------------------------
\context{1. Elastic Scattering of Protons}
\title{1.1-2 Hadronic Models}


\> typical model constructions
\>> Regge theory
\>> eikonal description

\> model references

\>> Islam et al. (Int. J. Mod. Phys. A21: 1-42, 2006)
\>> Petrov et al. (Eur. Phys. J. C28: 525-533, 2003)
\>> Bourrely et al. (Eur. Phys. J. C28: 97-105, 2003)
\>> Block et al. (Phys. Rev. D60: 054024, 1999)
\>> Jenkovszky et al. (arXiv:1105.1202, 2011)

\> predictions for (some) LHC energies:

\centerline{\fig*[,5cm]{fig/pdf/el_mod_dsdt_large.pdf}}

\> implemented within Elegent MC generator (chapter 3)


\newpage%------------------------------------------------------------------------------------------
\context{1. Elastic Scattering of Protons}
\title{1.3 Coulomb Interference}



\centerline{\cmyk{\TitColor}\bf Perturbative QFT\cmyk{\FgColor}}

\centerline{\fig*[8cm]{fig/pdf/el_diagrams.pdf}}

\> {\bf West-Yennie (WY) formula}
\>> exponentiation of infrared divergences
$$F^{\rm C+H}(t) = F^{\rm C}(t) + F^{\rm H}(t) \, \e^{\i \al \Ps(t)}$$
\>> relative Coulomb-hadronic phase $\Ps$ finite (and real by definition)
\>> number of approximations: on-shell contribution to diagram C only, ...
\>> consistency problem: $\Ps$ in final formula may become complex

\> {\bf simplified West-Yennie (SWY) formula}
\>> assumes constant exponential slope (ruled out by measurements)

\vfil
\centerline{\cmyk{\TitColor}\bf Eikonal description\cmyk{\FgColor}}

\> additivity of eikonals
$$
	F^{\rm C+H}(t) = {s\over 2\i}\int\limits_0^\infty b\,\d b\,J_0(b\sqrt{-t})\,\left( \e^{2\i\de^{\rm C+H}(b)} - 1 \right)
	\ ,\qquad
	\de^{\rm C+H}(b) = \de^{\rm C}(b) + \de^{\rm H}(b)
$$

\> {\bf Cahn}, {\bf Kundrát-Lokajíček} formulae: $\O{\al}$ approximations for $F^{\rm C+H}$
\> {\bf this thesis}: also direct calculation (all orders of $\al$, but not all relevant diagrams)

\newpage%------------------------------------------------------------------------------------------
\context{1. Elastic Scattering of Protons}
\title{1.3 Coulomb Interference}

\> most interference formulae have form:
$$F^{\rm C+H}(t) = F^{\rm C}(t) + F^{\rm H}(t) \, \e^{\i \al \Ps(t)}$$

\vfil
\centerline{\bf Quantity $\Ps$ in different approaches}
\vskip3mm

\centerline{\fig*[12cm]{fig/pdf/el_cic_diff_Psi_slides.pdf}}

\> SWY only reliable for very low $|t|$
\> non-negligible differences between CKL and ``this thesis'' calculations
\> $\Ps$ real only in SWY calculation

\newpage%------------------------------------------------------------------------------------------
\context{1. Elastic Scattering of Protons}
\title{1.3 Coulomb Interference}

\vskip-3mm

%\> effect of $\Ps$ in cross section?

\centerline{\bf Importance of the interference term}

$$Z(t) = {|F^{\rm C+H}(t)|^2 - |F^{\rm C}(t)|^2 - |F^{\rm H}(t)|^2\over |F^{\rm C+H}(t)|^2}$$

\vskip2mm
{
\iitskip4pt
\line{\fig*[4.5cm]{fig/pdf/el_cic_diff_Z_slides.pdf}\hss\raise8mm\vbox{\hsize10cm
\> comparison between interference formulae
\>> SWY reliable only for small $|t|$
\>> differences between CKL and ``this thesis'' less pronounced (only $1\%$ difference at the peak at $|t| \approx 0.5\un{GeV^2}$)
\>> suppression understood -- interference can only be important if $|F^{\rm C}| \approx |F^{\rm H}|$:
$$\d\si/\d t \propto |F^{\rm C} + F^{\rm H} \e^{\i \al \Psi(t)}|^2$$
}}

\line{\fig*[4.5cm]{fig/pdf/el_mod_Z_slides.pdf}\hss\raise30mm\vtop{\hsize10cm
\> comparison between hadronic models -- interference important in 
\>> ``Coulomb-interference region'' ($|t|\approx 10^{-3}\un{GeV^2}$)
\>> minima of hadronic models
}}
}




\newpage%------------------------------------------------------------------------------------------
\title{Chapter 2: The TOTEM Experiment}

\itskip0pt

\centerline{\bf\cTi physics programme\cFg}
\vskip1mm
\> elastic scattering measurement in a wide $t$-range
\> total cross section measurement
\> a study of soft and hard diffraction

\vfil
\vskip-5mm
$$\Downarrow$$
\vfil

\centerline{\bf\cTi requirements\cFg}
\vskip1mm
\> detect most fragments from inelastic collisions
\> excellent acceptance for scattered forward protons

\vfil
\vskip-3mm
$$\Downarrow$$
\vfil

\centerline{\bf\cTi detector apparatus\cFg}
\vskip1mm
% SAY: symmetric left (sector 45) -- right (sector 56)

\line{%
	\hss
	\fig*[,5.5cm]{fig/pdf/lhc.pdf}%
	\hskip0mm
	\fig*[,5.5cm]{fig/pdf/ttm_det_overview.pdf}%
	\hss
}

\newpage%------------------------------------------------------------------------------------------
\context{2. The TOTEM Experiment}
\title{2.1 The Roman Pot System}

\line{\hss\fig*[15.9cm]{fig/external/lhc_layout.pdf}\hss}

\vfil

\bgroup
\advance\hsize5mm
\line{\hskip-5mm\hss\offinterlineskip
	\vbox{\hbox{\FigLabel\strut 2 units in each station}\hbox{\fig*[,3.5cm]{fig/pdf/rp_station.pdf}}}%
	\hss
	\vbox{\hbox{\FigLabel\strut 3 RPs in each unit}\hbox{\fig*[,3.5cm]{fig/pdf/rp_unit.pdf}}}%
	\hss
	\vbox{\hbox{\FigLabel\strut Roman Pot (RP)}\hbox{\fig*[,3.5cm]{fig/pdf/rp_pot.pdf}}}%
	\hss
}

\vfil

\line{\hskip-5mm\hss\offinterlineskip
	\vbox{\hbox{\FigLabel\strut detector package (DP) in each RP}\hbox{\fig*[,3.5cm]{fig/pdf/rp_package.pdf}}}%
	\hss
	\vbox{\hbox{\FigLabel\strut ``edge-less'' silicon sensor}\hbox{\fig*[,3.5cm]{fig/pdf/rp_hybrid.pdf}}}%
	\hss
	\vbox{\hbox{\FigLabel\strut strip orientations within a DP}\hbox{\fig*[,3.5cm]{fig/external/strip_orientation.pdf}}}%
	\hss
}
\egroup

\vfil

\> DP = stack of $5U$ + $5V$ back-to-back mounted ``edge-less'' Si sensors with strip pitch $66\un{\mu m}$


\newpage%------------------------------------------------------------------------------------------
\context{2. The TOTEM Experiment}
\title{2.2 Proton Measurement with Roman RPs}

\line{\hss\fig*[11cm]{fig/pdf/ttm_proton_transport.pdf}\hss}

\line{\hss\vtop{\hsize7.5cm
\> proton transport ($\equiv$ optics) in \em{general}

\line{\hss\fig*[7cm]{fig/external/proton_transport.png}\hss}
}\hskip3mm\vtop{\hsize7.5cm
\> proton transport for \em{elastic} protons
$$x(s) \simeq L_x(s)\, \th_x^* + v_x(s)\, x^*$$
$$y(s) \simeq L_y(s)\, \th_y^* + v_y(s)\, y^*$$
\> optical functions: effective lengths $L$, magnifications $v$
}\hss}

\vfil

\> optics defines what can be seen and how it looks like
\>> sample of elastic events under three TOTEM-relevant optics scenarios

\line{\hss\fig*[,4cm]{fig/pdf/ttm_hit_distribution.pdf}\hss}

% comment: sometimes very low $L_x$ -> application for horizontal alignment


\newpage%------------------------------------------------------------------------------------------
\title{Chapter 3: Roman Pot Simulation and Reconstruction Software}

% mention: importance

\line{\hss\fig*[,10.5cm]{fig/pdf/sr_sw_structure_slides.pdf}\hss}


\newpage%------------------------------------------------------------------------------------------
\context{3. RP Software}
\title{3.1 Elegent}

\line{\hss\fig[8.5cm]{fig/pdf/sr_sw_structure_slides_eg.pdf}\hskip3mm\raise8.2cm\vtop{\hsize6.8cm
\> ELastic Event GENeraTor
\> generates random elastic scattering events according to a chosen phenomenological model (chapter 1)
\> output in standard format (HepMC)
}\hss}

\newpage%------------------------------------------------------------------------------------------
\context{3. RP Software}
\title{3.2 Beam Smearing}

\line{\hss\fig[8.5cm]{fig/pdf/sr_sw_structure_slides_bs.pdf}\hskip3mm\raise8.2cm\vtop{\hsize6.8cm
\> accounts for beam-smearing effects
\>> vertex smearing
\>> angular smearing
\>> energy smearing

\bls

\line{%
	\hss
	\fig*[7cm]{fig/pdf/smearing_vertex.pdf}%
	\hss
}

\bls
\bls

\line{%
	\hss
	\fig*[7cm]{fig/pdf/smearing_angular_energy.pdf}%
	\hss
}
}\hss}


\newpage%------------------------------------------------------------------------------------------
\context{3. RP Software}
\title{3.3 Fast Simulation}

\line{\hss\fig[8.5cm]{fig/pdf/sr_sw_structure_slides_fs.pdf}\hskip3mm\raise8.2cm\vtop{\hsize6.8cm
\> alternative to time-expensive Geant4\\ simulation
\> RP hits calculated
\>> by direct use of optics parameterization
\>> and a simple strip-discretization\\ algorithm
\> used for alignment MC simulations
% argue that it is all OK for that purpose
}\hss}


\newpage%------------------------------------------------------------------------------------------
\context{3. RP Software}
\title{3.4 Pattern Recognition}

\line{\hss\fig[8.5cm]{fig/pdf/sr_sw_structure_slides_pr.pdf}\hskip3mm\raise8.2cm\vtop{\hsize6.8cm
\> recognizes linear-track patterns within RP hits
\> independently in $U$ and $V$ projections
\> employs an optimized Hough-transform\\ algorithm 

\vskip\baselineskip

\line{%
	\offinterlineskip
	\hss
	\vbox{%
		\hbox{\FigLabel hit $(x, y)$ transformed into line $b = -xa + y$}%
		\fig*[7cm]{fig/pdf/sr_pattern_reco.pdf}%
	}%
	\hss
}

\bls

\line{%
	\offinterlineskip
	\hss
	\vbox{%
		\hbox{\FigLabel recognition results for two events}%
		\fig*[7cm]{fig/pdf/sr_pattern_reco_ex.pdf}%
	}%
	\hss
}
}\hss}


\newpage%------------------------------------------------------------------------------------------
\context{3. RP Software}
\title{3.5 Reconstruction of Elastic Events}
\itskip0pt

\line{\hss\fig[8.5cm]{fig/pdf/sr_sw_structure_slides.pdf}\hskip3mm\raise8.2cm\vtop{\hsize6.8cm

\> elastic reconstruction performance study

\> test of entire chain

\> outcome example -- expected $t$-resolution
\bls

\line{\offinterlineskip
	\hss
	\vbox{\FigLabel\hbox{$\be ^* = 1535\un{m}$}\hbox{\fig*[5.2cm]{fig/pdf/elr_1535_res_t.pdf}}}%
	\hss
}

\bls

\line{\offinterlineskip
	\hss
	\vbox{\FigLabel\hbox{$\be^* = 90\un{m}$}\hbox{\fig*[5.2cm]{fig/pdf/elr_90_res_t.pdf}}}%
	\hss
}

}\hss}


% 90m: fit doesn't describe the data


\newpage%------------------------------------------------------------------------------------------
\title{Chapter 4: Alignment of Roman Pots}

%\line{\hss\vbox{\htab*{
%\omit&\multispan2\bhrulefill\cr
%\omit& \be^* = 1535\un{m},\ \sqrt s = 14\un{TeV} & \be^* = 3.5\un{m},\ \sqrt s = 7\un{TeV}\cr\bln
%L_y							& 270\un{m}			& 21\un{m} \cr\ln
%\th_{\rm min}				& 6\un{\mu rad}		& 160\un{\mu rad} \cr\bln
%\De y						&\multispan2\bvrule\hfil $100\un{\mu m}$\hfil \cr\ln
%\De\th_y \equiv \De y/L_y	& 0.4\un{\mu rad}	& 5\un{\mu rad} \cr\ln
%\De\th_y / \th_{\rm min}	& 7\%	& 3\%	\cr\bln
%}}\hss}

% what is specific to RP alignment

\> RPs movable devices
\>> alignment for every run
\>> RP position depends on beam conditions

\vfil

\> RPs detect very small angles (down to few $\mu\rm rad$)\\
$\Rightarrow$ tracks at very small distances from the beam\\
$\Rightarrow$ very accurate alignment needed

\vfil

\line{\raise8mm\vbox{\hsize11cm
\> 2 levels of alignment
\>> alignment of RPs (passive material): beam-based alignment
\>> alignment of sensors (active material): track-based alignment 
}\hskip3mm plus1fil
\fig*[4cm]{fig/pdf/al_rp_scheme.pdf}}

\vfil

\> ultimate goal: alignment of sensors wrt.\ beam

\vfil

\vskip-5mm
\line{\raise8mm\vbox{\hsize11cm
\> several complementary methods
\>> collimation alignment: rough alignment of RPs wrt.~beam
\>> track-based alignment: relative alignment of sensors of one station
\>> alignment with elastic events: fine alignment of sensors wrt.~beam
}\hskip3mm plus1fil
\fig*[4cm]{fig/pdf/detector_overlap.pdf}}


\newpage%------------------------------------------------------------------------------------------
\context{Chapter 4: Alignment of Roman Pots}
\title{4.1 Collimation Alignment}

\vskip-3mm
\itskip0pt

\> prior to any data-taking
\> alignment of RPs (passive material)

\line{\hss\fig*[,4cm]{fig/pdf/al_collim_scheme.pdf}\hss}
\line{\fig*[,5.5cm]{fig/pdf/al_collim_ex.pdf}\hss\vbox{\hsize6cm
\> precision
\>> alignment of RPs: $\approx 20\un{\mu m}$ (step)
\>> alignment of sensors: $\approx 100\un{\mu m}$ only (poorly defined placement of sensors in within a RP)
}\hss}

%\> beam scraped by collimator(s) at a given distance
%\> RPs approached until ``touch'' is signalized by beam-loss monitors downstream
%\> beam scraped symmetrically around beam center\\
%$\Rightarrow$ vertical RPs aligned wrt.~beam axis\\
%$\Rightarrow$ calibration of the position-measurement offsets
%
%\line{\hss\fig*[,4cm]{fig/pdf/al_collim.pdf}\hss}
%
%\> horizontal pots (one jaw only): need to rely on the calculation of beam $\si$ (can be verified/improved later)




\newpage%------------------------------------------------------------------------------------------
\context{Chapter 4. Alignment of Roman Pots}
\title{4.2.1-2 Track--based Alignment}

\centerline{\bf\cTi sensor misalignments\cFg}
\vskip1mm

\line{\raise1cm\vbox{\hsize7cm
\> generally 3 shifts and 3 rotations
$$\Downarrow$$
\> only 2 misalignments relevant
\>> shift in read-out direction $\De s$
\>> rotation about beam axis $\De\rh$

}\hskip0mm\fig*[,4cm]{fig/pdf/al_sensor_misalignments.pdf}\hss}

\vfil

\centerline{\bf\cTi track based alignment method\cFg}
\vskip1mm

\line{\hss\fig*[,3.9cm]{fig/pdf/al_proton_sensor_interaction_slides.pdf}\hskip3mm\vbox{\hsize7.3cm
\> a sensor's measurement outcome:
$$m \simeq \underbrace{\vec d\cdot(\vec a z + \vec b - \vec c)}_{\hbox{track}} \underbrace{\hskip2mm - \hskip2mm \De s\hskip2mm + \hskip2mm\ldots \cdot \De\rh}_{\hbox{effect of misalignments}}$$

\> simultaneous fit of track ($\vec a$, $\vec b$) and misalignment parameters ($\De s$, $\De\rh$)
$$\downarrow$$
\> misalignment parameters determined by\break a ``residual analysis''
}\hss}



\newpage%------------------------------------------------------------------------------------------
\context{4.2. Track--based Alignment}
\title{4.2.3 Singular Modes}
\vskip-3mm
\itskip0pt

\> \hbox{\vtop{\hsize8cm\noindent misalignments determined by a ``residual analysis''}\hskip5mm $:$\hskip5mm $\vec v_{\rm residuals} = \mat S\, \vec v_{\rm misalignments}$}
\> \hbox{\vtop{\hsize8cm\noindent some misalignment modes don't yield residuals}\hskip5mm $:$\hskip5mm\hbox{$\mat S$ is singular}}
\vskip-3mm
$$\Downarrow$$
\> \hbox{\vtop{\hsize8cm\noindent singular modes}\hskip5mm $:$\hskip5mm\hbox{modes with zero eigenvalue}}

\vfil

\> number of singular modes depends on the geometry and track distribution

\line{\hss\fig*[,4cm]{fig/pdf/al_eigenvalues_theta_slides.pdf}\vbox{\hsize9cm
\noindent $\leftarrow$ regular modes
\vskip9mm
\noindent $\leftarrow$ weak modes (transition from singular to regular)
\vskip5mm
\noindent $\leftarrow$ singular modes
\vskip9mm
}\hss}


\> singular modes in realistic case ($U$ and $V$ read-out directions, very parallel tracks)
\>> unit shifts in $x$ and $y$
\>> unit rotation about beam axis

%{\SmallerFonts\line{\hss\vbox{\htab*{
%\omit&\multispan{4}\bhrulefill\cr
%\omit			&\multispan2\bvrule\strut\hfil two read-out groups\hfil &\multispan2\strut\vrule\hfil three and more read-out groups\hfil\cr
%\omit\bhrulefill&\multispan{4}\hrulefill\cr
%						& \hbox{non-parallel tracks} & \hbox{parallel tracks} & \hbox{non-parallel tracks} & \hbox{parallel tracks} \cr\bln
%\hbox{read-out shifts}	&\multispan4\bvrule\hfil {\bf 4}: $x$ and $y$ global and linearly progressive shifts\hfil\cr\ln
%%
%&\hbox{{\bf 2}: gl. rot.}  &\hbox{{\bf 4}: gl. and l.p. rots.} &\hbox{{\bf 1}: gl. rot.} &\hbox{{\bf 2}: gl. and l.p. rot.} \cr
%\omit\vbox to 0pt{\vss\hbox{ rotations about $z$ }\vss}&\multispan4\cr
%& \hbox{for $U$ and $V$ indep.} & \hbox{for $U$ and $V$ indep.}&&\cr\ln
%%
%& \hbox{{\bf 4}: gl. and l.p.} &  & \hbox{{\bf 2}: gl. and l.p.}  & \cr
%\hbox{shifts in }z	& \hbox{shifts in }z & N_{\rm sensors} & \hbox{shift in }z& N_{\rm sensors} \cr
%& \hbox{for $U$ and $V$ indep.} &&&\cr\bln
%}}\hss}}

\vfil
\title{4.2.4 Imposed Constraints}
\vskip-3mm

\> singular modes can be regularized by imposing additional constraints (e.g.~mean sensor shift within a station is zero)

\> actual values of singular modes are determined at later stages of alignment (profile, elastic)

\newpage%------------------------------------------------------------------------------------------
\context{4.2. Track--based Alignment}
\title{4.2.6 Monte-Carlo Tests}

\centerline{\bf statistical test of alignment algorithm}
\centerline{(one color per each RP of a station)}

\line{\hss\fig*[,8.2cm]{fig/pdf/al_stat_final_slides.pdf}\hss}

\> systematic error compatible with zero
\> estimated uncertainty falls with $1/\sqrt{N_{\rm tracks}}$

\newpage%------------------------------------------------------------------------------------------
\context{4.2. Track--based Alignment}
\title{4.2.10 LHC Alignment Results}
\vskip-3mm

\line{\hss\vbox{\offinterlineskip\hbox{\FigLabel alignment results for one station -- 10 LHC runs}\fig*[8cm]{fig/pdf/al_comp_det_per_unit.pdf}}\hskip3mm\raise1mm\vbox{\hsize7.0cm
\> can determine all 3 DP rotations (order of $1\un{mrad}$)
\> shifts in within the package compatible with our expectation (order of $20\un{\mu m}$)
\> sensor rotations about the beam axis share a common component (DP rotation, order of $1\un{mrad}$)
\bls
\> note the result stability (10 LHC data-sets overlaid)
}}

\vfil

\title{4.2.8 Alignment of Detector Packages}
\vskip-3mm

\> rotation of detector package $\longrightarrow$ \em{rotation} of each sensor and \em{shift} of each sensor
\>> shift linearly dependent on sensor's longitudinal ($z$) position

\line{%
	\hss
	\fig*[,3cm]{fig/pdf/rp_package.pdf}
	\hss
	\fig*[,3cm]{fig/pdf/al_rp_misalignment.pdf}
	\hss
}




\newpage%------------------------------------------------------------------------------------------
\context{4.2. Track--based Alignment}
\title{4.2.9 Internal Alignment Results}

\centerline{\bf alignment of sensors within one RP}

\> comparison of results from
\>> optical measurement in lab
\>> track-based alignment applied to beam-test and cosmic data (H8)
\>> track-based alignment applied to LHC data

\line{\hss\fig*[,6.5cm]{fig/pdf/al_comp_det_per_pot_dp2_ext2.pdf}\hss}

\> good agreement, differences due to
\>> optical measurement insensitive to shifts among planes
\>> different conditions (mainly temperature) for beam-test and LHC data


%\newpage%------------------------------------------------------------------------------------------
%\title{4.2.10 LHC Results}
%
%\line{\hss\fig*[7cm]{fig/pdf/al_comp_rp_all_rot.pdf}\hss}
%
%\> add to TOC?

\newpage%------------------------------------------------------------------------------------------
\context{Chapter 4: Alignment of Roman Pots}
\title{4.3 Profile Methods}

\> account for singular modes of track-based alignment: unit $x, y$ shifts and rotation about $z$
\> alignment wrt.~beam: determining shifts $\equiv$ determining beam position

\> expected event symmetries
\>> up-down symmetry for all processes (zero vertical dispersion) $\Rightarrow$ vertical alignment
\>> left-right symmetry for elastic scattering (zero momentum loss) $\Rightarrow$ horizontal alignment

\vfil
\centerline{\bf observed hit distribution}
\vskip3mm
\line{\hss\fig*[6cm]{fig/pdf/al_prof_hits_slides.pdf}\hskip3mm\raise5mm\vbox{\hsize9cm
\> elastic axis (green)
\>> tilted due to $x$--$y$ coupling in optics
\>> still can be well fitted $\Rightarrow$ horizontal alignment with\break $\approx 10\un{\mu m}$ uncertainty

\bls

\> diffractive proton distribution (around violet line)
\>> influenced by non-zero vertical dispersion
\>> fitting difficult (profiles non-linear)
\>> vertical alignment needs extrapolation $\Rightarrow$ unsatisfactory precision
}}

%\line{\hss\fig*[4cm]{fig/pdf/al_prof_x_dists.pdf}\hss}


\newpage%------------------------------------------------------------------------------------------
\context{Chapter 4: Alignment of Roman Pots}
\title{4.4 Elastic Alignment}

\centerline{\bf determination of beam position (per unit) with elastic events}
\vskip2mm

\line{\hss\fig*[,4.7cm]{fig/pdf/al_el_plots_sum_slides.pdf}\hss}

\line{\vtop{\hsize4.5cm
\centerline{\bf elastic event selection}
}\hfil\vtop{\hsize4.9cm
\centerline{\bf vertical alignment}

\> remove data affected by acceptance effects (gray band)
\> bottom-RP distribution (blue)\\ flipped (black)
\> shifted up-down to match the top-RP distribution (red)
\> best match found (green)
}\hfil\vtop{\hsize4.9cm
\centerline{\bf horizontal alignment}
\> $xy$ distribution fitted (violet)
\> interpolated to vertical beam position
}}

%\> check/refinement of relative near-far vertical alignment
%\line{\hss\fig*[,4cm]{fig/pdf/al_el_plots_dyy_one.pdf}\hss}

%\newpage%------------------------------------------------------------------------------------------
%\title{4.4 Elastic Alignment II}
%
%\> remove?
%
%\line{\hss\fig*[,4cm]{fig/pdf/al_el_plots_ylyr.pdf}\hss}

\newpage%------------------------------------------------------------------------------------------
\context{Chapter 4: Alignment of Roman Pots}
\title{4.5 Alignment Summary}

\line{\hss\AddBckg[1mm]{\offinterlineskip\vbox{\cmyk{\cmykBlack}\halign{\bstrut\bvrule\ #\hfil\ &\bstrut\bvrule\ \hfil#\hfil\ &\vrule\bstrut\hfil\ #\ \hfil\bvrule\cr
\omit&\multispan2\bhrulefill\cr
\omit& internal alignment & among detector packages \cr\bln
misalignment & \vbox{\hbox{\bstrut shifts $\sim 20\un{\mu m}$}\hbox{rotations $\sim 1\un{mrad}$}} & \vbox{\hbox{\bstrut shifts $< 1\un{mm}$}\hbox{rotations $\sim \hbox{few }\un{mrad}$}} \cr
&&$\Downarrow$\cr
uncertainty after collimation alignment &  & \vbox{\hbox{\bstrut shifts $\sim 100\un{\mu m}$}\hbox{rotations $\sim \hbox{few}\un{mrad}$}}\cr\bln
\multispan3\vbox to3mm{\vss}\cr
\omit&\omit\hfil$\Downarrow$\hfil&\omit\cr
\multispan3\vbox to3mm{\vss}\cr
\omit&\multispan2\bhrulefill\cr
\omit& relative alignment & global alignment wrt.~beam \cr
\omit& (regular modes) & (singular modes) \cr\bln
uncertainty after track-based alignment & \vbox{\hbox{\bstrut shifts $\ls 1\un{\mu m}$}\hbox{rotations $\ls 0.1\un{mrad}$}} & \vbox{\hbox{\bstrut shifts $\sim 100\un{\mu m}$}\hbox{rotations $\sim \hbox{few}\un{mrad}$}} \cr
&&$\Downarrow$\cr
uncertainty after elastic alignment &  & \vbox{\hbox{\bstrut hor.~shifts $\ls 5\un{\mu m}$}\hbox{\bstrut vert.~shifts $\ls 20\un{\mu m}$}\hbox{\bstrut rotations $\ls \hbox{few}\un{mrad}$}} \cr\bln
}}}\hss}

\newpage%------------------------------------------------------------------------------------------
\title{Chapter 5: The First Elastic Scattering Measurement at the LHC}

\line{\hss\fig*[,4.5cm]{fig/pdf/felm_scheme_slides.pdf}\hss}

\vfil

\> angular reconstruction (optimized robustness against optics perturbations):

$$\eqnarray{
	\hat\th_x^* = {\th_x^{\rm L} + \th_x^{\rm R}\over 2}\ ,\qquad
		& \th_x^{\rm L} = {1\over {\d L^{45}_x\over \d z}} {x^{45}_{\rm F} - x^{45}_{\rm N}\over d}\cr
	\hat\th_y^* = {\th_y^{\rm L} + \th_y^{\rm R}\over 2}\ ,\qquad
		& \th_y^{\rm L} = {1\over 2} \left( {y^{45}_{\rm F}\over L^{45}_{y, \rm F} } + {y^{45}_{\rm N}\over L^{45}_{y, \rm N} } \right)\cr
}$$

\vfil

\> event selection: 6 cuts
\>> left-right collinearity ($x$ and $y$ projections)
\>> low momentum loss: correlation between local (RP) angle and hit position (per arm per projection)


\newpage%------------------------------------------------------------------------------------------
\context{Chapter 5: The First Elastic Scattering Measurement at the LHC}
\title{5.1 Background}

\vskip-4mm
\centerline{(everything that passes selection cuts but is not elastic)}
\vskip1mm

\line{\hss\vbox{%
	\fig*[5cm]{fig/pdf/felm_background_int_dg_fit_cut1.pdf}
	\fig*[5cm]{fig/pdf/felm_background_before.pdf}
	}\hskip3mm\raise17mm\vbox{\hsize10.5cm
	
\> background integral
\>> relax one of the cuts: e.g. $\th_x$ collinearity
\>> plot $\th_x^{*R} - \th_x^{*L}$
\>> fit signal (blue) in central (green) region
\>> fit the tails for background (red)
\>> number of background events: integral of red curve over the selection (green) region

\bls
\bls
\bls

\> background distribution
%\>> reasonable assumption: dominant contribution is left-right independent
\>> parameterize the distribution of background events on the cut tails
\>> interpolate it to the cut selection region (green)

\bls
\bls
\bls

\> background contribution to elastic distribution
\>> impose collinearity cuts on the background fit

}\hss}


\newpage%------------------------------------------------------------------------------------------
\context{Chapter 5: The First Elastic Scattering Measurement at the LHC}
\title{5.2 Unfolding}

\vskip-3mm

\line{\hss\raise15mm\vbox{\hsize10cm
\> accounts for the finite resolution of the RP detector system
\>> dominant contribution comes from beam divergence
\bls
\bls
\> method 1: analytic unsmearing
\>> data fitted with sum of 3 Gaussians and $\th\cdot$Gaussian
\>> analytic unsmearing simply by modifying parameters
}\hskip3mm\fig*[5cm]{fig/pdf/felm_unfolding_m1_fit_slides.pdf}\hss}



\line{\hss\raise15mm\vbox{\hsize10cm
\> method 2: learning the trend of corrections + back extrapolation

\line{\hss\fig*[9.7cm]{fig/pdf/felm_unfolding_m2_scheme.pdf}\hss}

\bls

\> results (multiplicative correction factor)
\>> both methods in agreement
\>> right plot indicates the value of the correction and its uncertainty
}\hskip3mm\fig*[5cm]{fig/pdf/felm_unfolding_m1m2_cmp.pdf}\hss}


\newpage%------------------------------------------------------------------------------------------
\context{Chapter 5: The First Elastic Scattering Measurement at the LHC}
\title{5.3 Results and Comparison to Models}

\> TOTEM's elastic $\d\si/\d t$ measurement at $\sqrt s = 7\un{TeV}$ compared to model predictions\\
(an updated plot version)

\line{\hss\fig*[15cm]{fig/pdf/ttm_mod_cmp_dsdt.pdf}\hss}


%----------------------------------------------------------------------------------------------------
%----------------------------------------------------------------------------------------------------
%----------------------------------------------------------------------------------------------------

\newpage%------------------------------------------------------------------------------------------
\hbox{}\vfil
\title{Backup Slides}

%\newpage%------------------------------------------------------------------------------------------
%\title{1.3 Coulomb Interference -- Results III}
%
%\> comparison between
%\>> SWY formula (often used, but based on inconsistent assumptions)
%\>> CKL formula (TODO)
%
%$$R(t) = {|F^{\rm C+H}_{\rm CKL}(t)|^2 - |F^{\rm C+H}_{\rm SWY}(t)|^2 \over |F^{\rm C+H}_{\rm CKL}(t)|^2}$$
%
%\centerline{\fig*[,4cm]{fig/pdf/el_mod_R.pdf}}
%
%\> at low $|t|$ discrepancy of few per-mile
%\> discrepancy pronounced at dips of hadronic amplitude

\newpage%------------------------------------------------------------------------------------------
\title{2.3 The Total Cross Section}

%\line{\hss\fig*[,3cm]{fig/pdf/ttm_sigma_tot.pdf}\hss}

% emphasis: very old study, now we know that we can do better

% mention 16\pi

\> elastic cross section related to total (hadronic) cross section
$$
	\si_{\rm tot} = {16\pi\over 1+\rh^2} {\d N_{\rm el}/\d t |_0 \over N_{\rm el} + N_{\rm inel}}\ ,\qquad
	\si_{\rm tot}^2 = {16\pi\over 1+\rh^2} \left. {\d\si_{\rm el}\over\d t} \right|_0
$$

\> elastic differential rate/cross section needs to be extrapolated to $t=0\un{GeV^2}$
\> models: guide for parameterization $F^{\rm H} = \e^{M(t)}\, \e^{\i P(t)}$

\> two cases considered (very old study)

\line{\hss\vbox{\hsize6cm
\centerline{$\mathbf{\be^*=1535\un{m}}$}
\> both $t_x$ and $t_y$ can be measured
\> CKL formula used

\line{\hss\fig*[,3.5cm]{fig/pdf/ext_results_1535.pdf}\hss}
}\hskip2mm plus 1fil\vbox{\hsize6cm
\centerline{$\mathbf{\be^*=90\un{m}}$}
\> only $t_y$ can be measured
\> phase completely inaccessible

\line{\hss\fig*[,3.5cm]{fig/pdf/ext_results_90.pdf}\hss}
}}

\newpage%------------------------------------------------------------------------------------------
\title{3. RP Software\hfill 3.5 Reconstruction of Elastic Events}

\> module to study elastic reconstruction performance (simulation phase of the experiment)

\> based on (elastic) proton transport parameterization:
$$x(s) = L_x(s)\,\th_x^* + v_x(s)\,x^*$$

\> optics functions ($L$ and $v$) assumed known at every RP $\Rightarrow$ LS fit for $\th_x^*$ and $x^*$\\
(in reality the optics is not precisely known, thus a more ``explorative'' approach must be used)

\> three steps
\>> hit selection (suppression of additional hits)
\>> track fitting (left only, right only, global)
\>> event selection (collinearity and vertex cuts between left and right fits)

\> one of the outcomes -- the $t$ resolution:

\line{\offinterlineskip
	\hss
	\vbox{\FigLabel\hbox{$\be ^* = 1535\un{m}$}\hbox{\fig*[,4cm]{fig/pdf/elr_1535_res_t.pdf}}}%
	\hss
	\vbox{\FigLabel\hbox{$\be^* = 90\un{m}$}\hbox{\fig*[,4cm]{fig/pdf/elr_90_res_t.pdf}}}%
	\hss
}

% 90m: fit doesn't describe the data

\newpage%------------------------------------------------------------------------------------------
\title{3.6 Data Quality Monitor}

\> interactive program to visualize results of the reconstruction steps

\> applications
\>> online control of data being acquired
\>> event scanning (visual inspection of event properties)
\>> reconstruction software tuning and debugging

\line{\hss\fig*[,7cm]{fig/external/dqm.png}\hss}


\bye
