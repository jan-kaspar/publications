\chapter{Introduction}

\> General introduction
\> {\bf motivation} for what I study.

\section{CERN, LHC}

\> bla bla


\section{Conventions}

\> typographical conventions

\tab[\strut\qquad #\hfil&\qquad #\hfil&\qquad #\hfil\qquad\cr]{typo conventions}{Typographical conventions.}{\ln
struture			& example				& font/symbol\cr\ln
scalar				& $s$ 					& italics\cr
vector				& $\vec v$				& bold\cr
element of vector	& $v_i$					& italics\cr
dot product			& $\vec v\cdot \vec w$	& middle dot\cr
matrix				& $\mat M$				& slanted\cr\ln
}


\section{Theoretical concepts}

\> def of cross-section, t, xi
\>> precise formulae

\> def of luminosity

One of the definitions of cross-section $\si$ relates it to the corresponding number of events $N$ per time
\eqref{{\d N\over \d t} = \si\,j\,n_T\.}{in cross-section}
$j$ denotes the flux of bombarding particles and $n_T$ is the number of target particles. This formula can be written in a more general form
\eqref{N = \si\,{\cal L}_{int},\qquad {\cal L}_{int} = \int\d t\,\d x\,\d y\,\d z\, j(x, y, z; t)\, \rh_T(x, y, z; t), }{in luminosity}
where ${\cal L}_{int}$ is the integrated luminosity in a given time interval and $\rh_T$ is the density of target particles.



\> optical theorem
\>> neglecting spins - several amplitudes, how to treat them in the optical theorem - Formanek

\> symmetries
\>> azimuthal symmetry? Question of Formanek
