\chapter{Introduction}

\section{Conventions}

\> typographical conventions

\tab[\strut\qquad #\hfil&\qquad #\hfil&\qquad #\hfil\qquad\cr]{typo conventions}{Typographical conventions.}{\ln
struture			& example				& font/symbol\cr\ln
scalar				& $s$ 					& italics\cr
vector				& $\vec v$				& bold\cr
element of vector	& $v_i$					& italics\cr
dot product			& $\vec v\cdot \vec w$	& middle dot\cr
matrix				& $\mat M$				& slanted\cr\ln
}

\> General introduction
\> {\bf motivation} for what I study.

\section{LHC}

\> LHC and proton trasport
\>> the matrix equations, ref to Hubert?
\>> more precise tools, MAD and ?
\>> parameterization, ref to Hubert
\>> linearized parameterization
\>> concept of optics

\>> smearing

\section{TOTEM}

\> Description of TOTEM detectors

\subsection{The Roman Pot system}

\fig[15cm]{fig/external/RP_stations_original.pdf}{rp stations}{RP stations}
\fig[10cm]{fig/external/rp_unit.jpg}{rp unit}{RP unit}
\fig[10cm]{fig/external/rp.jpg}{rp rp}{A Roman Pot}
\fig[10cm]{fig/external/rp_package.jpg}{rp package}{Detector package}
\fig[10cm]{fig/external/hybrid2.jpg}{rp hybrid}{A hybrid with a silicon detector and four VFAT chips.}
\fig[10cm]{fig/external/silicon_explained.png}{rp sensor}{A detail of a silicon sensor.}

\section{Theoretical concepts}

\> def of cross-section, t, xi
\>> precise formulae

\> optical theorem
\>> neglecting spins - several amplitudes, how to treat them in the optical theorem - Formanek

\> symmetries
\>> azimuthal symmetry? Question of Formanek
