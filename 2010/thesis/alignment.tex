\chapter[al]{Alignment of Roman Pots}

\TODO{!! ADD CITATIONS !!}

MOTIVATION: An accurate alignment is of major importance for the TOTEM experiment in order to deliver precise measurements. Among the subdetectors of TOTEM, the alignment of the RPs presents the biggest challenge since they are movable. The importance of alignment is most pronounced at the $\be^* = 1535\un{m}$ optics, where the beam divergence (the dominant smearing effect) is rather low and thus the impact of any misalignment has a large relevance. To give a feeling, a $100\un{\mu m}$ displacement of a vertical RP would lead to angular shift of about $0.4\un{\mu rad}$ (based on an effective length $L_y\approx 270\un{m}$, typical for this optics). This is to be compared to the spread of the beam divergence $0.3\un{\mu rad}$.

\TODO{experience from other experiments}

WHAT WE ALIGN, WRT WHAT:
What is important is the relative position between RP sensors (the other parts do not matter much) and the beam. Hence there are two players in the game -- RP positions and beam positions.

Coordinate system such that the beam is at x=y=0

\caption{Alignment procedure}

Three steps:

1) Approach the pots to the desired positions as precisely as possible; retrieve the position - LVDTs + their callibration by collimation alignment, later hope to use BPMs, \Sc{al collim}

2) track-based alignment, \Sc{al tb}

3) alignment with physics processes -- to correct the singular and weak modes of the TB al. -- profile \Sc{al prof} and elastic \Sc{al elast}

\iffalse
The procedure has two steps. First, move the pots to the desired position as precisely as possible. Then take data, analyze them and find what error has been done in the first step.

The first step relies mainly on two types devices -- RP motor control and Beam Position Monitors (BPMs). Both are calibrated to determine position of outer edge of the window or the beam with respect to the ideal beam-pipe center.

For the second step, a number of methods is available. Undoubtedly, one of the most powerful ones is the track-based alignment. It uses the tracks passing through the overlap between vertical and horizontal RPs and is, therefore, capable of determining the relative position between the pots. Its strength is underlined by the fact that it is based on a single assumption: tracks are straight lines. To determine the position of the beam, other methods must be used. One may profit from know symmetries of certain physics processes. Here, the analysis becomes delicate because in the observables the properties of the processes are mixed with the properties of the optics. This gives a certain superiority to elastic scattering. This process is often easy to separate and thus can provide a clean sample, as the first argument. As the second one, it comprises two protons exactly in the opposite direction and can thus be used for the alignment of the opposite arms of the experiment. Yet another advantage the protons have, by definition, zero momentum-loss and therefore, the dependence on the optics is largely reduced.
\fi

\caption{Expected misalignments}

Let's first asses what may go wrong and what are the corresponding misalignment estimates.

\> two components -- internal and DP
\>> make sure not to double define internal and DP (mis)alignments -- see \Sc{al rp fac}

\> \em{Internal misalignments} arise from finite precision of fixing the detectors within a package. The precision is estimated to be
\eqref{20\un{\mu m}\ .}{internal misalignment shift}
This also gives estimate about the rotation
\eqref{\De\rh \approx {2\cdot 20\un{\mu m}\over 4\un{cm}} = 1\un{mrad}\ .}{internal misalignment rotation}

\> \em{Detector package (DP) misalignments}
\>> errors in LVDT $~10\un{\mu m}$
\>> RP frame deformation ???
\>> uncertainty of the collimation alignment (the preceding step) \Eq{al window to dp uncer}, with the same uncertainty we put the RPs where we want them


\htab{al exp misal}{The expected misalignments.}{
\omit&\multispan2\bhrulefill\cr
\omit					&\omit\bvrule\tskip\bstrut sensors within a package\tskip	& \hbox{packages within a station} \cr
\omit					&\omit\bvrule\tskip\bstrut (internal misalignments)\tskip	& \hbox{(DP misalignments)} \cr\bln
\hbox{transverse shift}	& \hbox{about } 20\un{\mu m} 								& \hbox{few } 100\un{\mu m} \cr\ln
\hbox{shift in }z 		& \hbox{few } 100\un{\mu m}									& \hbox{below } 1\un{mm}\cr\ln
\hbox{rotations}		& \hbox{few }\rm mrad										& \hbox{few }\rm mrad \cr\bln
}

\section[al collim]{Collimation alignment}

The collimation alignment is a part of the RP commissioning procedure, where the RPs are aligned with respect to the collimators of the LHC (a similar procedure to the beam-based alignment of the collimators, see \bref{bracco2010}). Moreover it provides important data for the alignment of RPs themselves.

We recall (from \Sc{ttm}) that the position of each RP is monitored by a linear voltage differential transformator (LVDT). This device measures the distance of the thin window from the beam-pipe axis. The position of the beam, again with respect to the beam-pipe axis, is measured by the beam position monitors (BPMs). There are two of them at each station: at the upstream end (near BPM) and at the downstream end (far BPM). Each BPM measures both horizontal and vertical beam position. For the collimation alignment one more tool is used -- a series of beam loss monitors (BLMs) measures the radiation (from beam losses) at several locations downstream of each station.

At the beginning of the collimation alignment, the collimators scrape the beam at a given number $n$ of beam sigmas $\si_0$. This beam sigma stands for the standard deviation of the beam profile at the collimator location, see \Fg{al collim}. Due to the multi-turn effect, the beam is scraped symmetrically about the center of the beam (see e.g. section ``Beam based alignment'' in \bref{bracco10}). Thus after the scraping, the beam has sharp edges on both sides and consequently the beam width and center are well defined. Then, a RP is approached to the beam (see the top plot in \Fg{al collim ex}), until it touches its edge. At this moment, particles from the beam surface are scattered off the RP edge and give rise to peaks in the BLMs downstream (see \Fg{al collim ex} bottom). This is the moment where the RP (its thin window) is at the same beam-sigma-distance $n$ as the collimators. Denoting the beam sigma at the RP location $\si$ (as in \Fg{al collim}), the RP position is $n\si$. The beam widths $\si_0$ and $\si$ are related via the optics, see \Eq{ttm beam sigma}, and therefore the position of the beam can be calculated as
\eqref{n\si - t\ ,}{al vc m1}
where $t$ denotes the corresponding LVDT reading.

For the vertical beam position one need not rely on the optics (which always includes certain uncertainty). Suppose that the top pot has just touched the beam (as in \Fg{al collim ex}). Then the bottom pot can be approached until a touch is signalized by the BLMs. At the moment of the touch, the beam center is right in between the pots. Thus the beam center can be calculated from the LVDT readings only ($t$ for the top, $b$ for the bottom RP):
\eqref{{t + b\over 2}\ .}{al vc m2}
This procedure can be repeated few times (as shown in \Fg{al collim ex}) to improve the precision.

The uncertainty is given mostly by the RP-movement step size ($250\un{\mu m}$ in the figure, except the last one). Because of limited time resolution one cannot determine when exactly was the beam touched. One should thus aim at steps as small as possible. On the other hand, the smaller step, the less scraping and the lower peak in the BLM signal. If the step is too small, the induced beam loss can not be distinguished from the noise. We have observed a step of $20\un{\mu m}$ to be on the practical limit.

\fig{fig/pdf/al_collim_ex.pdf}{al collim ex}{An example of the collimation alignment (data for 56-220-near unit from 29 November 2009). The top and bottom pot movements (upper plot) induced signals in the BLMs (bottom plot). The correspondence between the movements and BLM peaks is visualized by the dotted lines. The time is given in minutes from the start of the test.}

Thus from \Fg{al collim ex} one can deduce the vertical beam position at $(313\pm 63)\un{\mu m}$ (assuming the beam was touched in the middle of the movement). This can be compared to the BPM reading $(-550\pm50)\un{\mu m}$: there is an evident discrepancy. The BPMs at that time were known to have an unknown offset in their readings and moreover to be rather sensitive to temperature and beam intensity. The latter leading to drifts as can be seen in \Fg{al collim bpm problem}. This has led us to consider the BPMs to be unreliable and not to use them for alignment purposes.

\fig{fig/pdf/al_collim_bpm_problem.pdf}{al collim bpm problem}{The readings from the BPMs in station 56-220 from 25 June 2010. Note the pronounced drifts.}

There was also an indication of a LVDT-related problem. As we have said, the vertical beam center can be determined by two methods: \Eq{al vc m1,al vc m2}. The results of these two methods did not agree. A valuable hint was then provided by the track-based alignment (see \Sc{al tb}) which determined the distances of the top and bottom RPs (or detector packages to be more precise). The track-based alignment clearly favored the optics-based method \Eq{al vc m1}.

All these methods can be reconciled by assuming that each LVDT scale has its own origin (offset), as shown in \Fg{al collim}. These offsets can easily be determined:
\eqref{o_t = n\si - t,\qquad o_b = -n\si - b\ ,}{al lvdt offsets}
where $t$ and $b$ are the top and bottom pot LVDT readings. Note that the offsets are given with respect to the beam center (not the beam-pipe center), which is, in fact, exactly what is needed. The offsets calculated after the collimation alignment on 21 September 2010 (see Tb.~1 in \bref{al collim}) are shown in \Tb{al lvdt off}. The uncertainty has two components -- from the step size (between $20$ and $50\un{\mu m}$) and from the calculation of the beam size $\si$, which is unfortunately difficult to estimate.

\fig{fig/pdf/al_collim.pdf}{al collim}{A sketch of the collimation alignment including the LVDT-scale origins (drawn as red dots). The beam is drawn in blue, its surface with solid and its center with dash-dotted line. The collimators scrape the beam at the distance of $n\si_0$, where $\si_0$ is the beam sigma at the collimator location. The RPs touch the beam at distance $n\si$, where $\si$ is the beam sigma at the RP location.}


\htab{al lvdt off}{The vertical LVDT offsets in millimeters, determined from the test on 21 September 2010. The signs follow $y$ axis in \Fg{al collim}.}{\bln
\hbox{unit} & o_t & o_b \cr\bln
\hbox{45-220-near} & +0.515 & +0.135\cr\ln
\hbox{45-220-far} & +0.389 & +0.111\cr\bln
\hbox{56-220-near} & -1.254 & -0.286\cr\ln
\hbox{56-220-far} & +0.262 & -0.162\cr\bln
}

Knowing the offsets, one can calculate the positions of the RPs (their thin windows) from the LVDT measurements. From that one can infer the positions of the silicon detectors and thereby obtain an initial geometry for the track-based alignment -- the second step in the alignment procedure. The key point here is to know the distance between the thin window and the detector package in each pot. Unfortunately, this distance is known only with an uncertainty
\eqref{\O{100\un{\mu m}}\ .}{al window to dp uncer}
This gives the order of misalignments that are present the initial geometry for the track-based alignment.

\section[al tb]{Track--based alignment}

\> Intro, partially the same as above: tracks should be straight lines, misalignments move the hits slightly off the track. They can be determined by an appropriate analysis of the residuals. \TODO{define residuals}

\> The basic structures in the track-based alignment are the detectors (sensors). Only these provide measurements and only these can be aligned... may be not that strong, one can get some idea about DP misalignments too

\subsection[al psi]{The relation between proton kinimatics and RP measurements}

Since the magnetic field within RP stations is negligible, protons follow a straight trajectory, as drawn in the side view of \Fg{al proton sensor interaction} (the same is shown in \Fg{ttm proton transport}, but with the third axis called $s$). The track can be described
\eqref{\pmatrix{x\cr y\cr z} = \pmatrix{a_x\cr a_y\cr 1} z + \pmatrix{b_x\cr b_y\cr 0}\ ,}{al local track}
where $a$ and $b$ parameters give the track slopes and intercepts.

When a proton enters the sensitive volume of a silicon sensor, it creates electron-hole pairs along its trajectory. This charge is collected by neighbouring strips, giving rise to measurable signal. While details of this charge sharing process can be found, for example, in Sec.~7.4~in \bref{hubert}, here, we will assume that just the nearest strips is active (drawn in green in the front view of \Fg{al proton sensor interaction}). It is a reasonable assumption for the TOTEM sensors. Then, the measurement outcome $m$ can be written
\eqref{m = v + \De m\ ,}{al measurement}
where $v$ gives the projection of the hit point in the $\vec v$ direction (i.e. the read-out direction). $\De m$ is the \em{resolution error}, that is the error made by rounding $v$ to the nearest strip position.

\fig{fig/pdf/al_proton_sensor_interaction.pdf}{al proton sensor interaction}{A scheme of a proton interaction with a sensor. Left (side view when $\rh_y = 0$): the thick black line represents a sensor, the blue line a proton track. The blue dot marks the point where the proton hits the sensor. It has the same meaning in the right (front view) figure. There, the thick blue line shows the hit point projection into the read-out ($\vec v$) direction. The strip drawn in green is the active strip, the resolution error is drawn in red.}

The value of $v$ can be related to the track parameters with the aid of a transformation from (global) coordinate system $xyz$ to (local) system $uvz'$. The transformation can be written as follows
\eqref{\pmatrix{u\cr v\cr z'} = \mat R \left[ \pmatrix{x\cr y\cr z}  - \pmatrix{c_x\cr c_y\cr c_z}  \right] \ .}{al global to local}
The $c$ parameters give the position of sensor's center, the rotation $\mat R$ can be parameterized with three angles $\rh_{x,y,z}$:
\eqref{\mat R =
	\pmatrix{
		\cos\rh_z  & \sin\rh_z & 0\cr
		-\sin\rh_z & \cos\rh_z & 0\cr
		0		   & 0         & 1\cr
	}
	\pmatrix{
		\cos\rh_y  & 0 & \sin\rh_y\cr
		0		   & 1 &          \cr
		-\sin\rh_y & 0 & \cos\rh_y\cr
	}
	\pmatrix{
		1 & 0		   & 0        \cr
		0 & \cos\rh_x  & \sin\rh_x\cr
		0 & -\sin\rh_x & \cos\rh_x\cr
	}\ .
}{al rotation parameterization}
This three shifts and three rotations reflect the position of a sensor in the global coordinate frame. In reality, despite our best efforts, the position can not be exactly the nominal one. It is therefore important to distinguish between \em{true/actual} (with primes) and \em{nominal/thought} (without primes) position parameters:
\eqref{\rh_i' = \rh_i + \De\rh_i\ ,\qquad c_i' = c_i + \De c_i\ ,}{al misalignments}
where $\De\rh_i$ and $\De c_i$ represent the \em{misalignment} or, on the other hand, \em{alignment corrections}.

The nominal sensors' positions (see \Sc{ttm}) are perpendicular to the $z$ axis, that is $\rh_x=\rh_y=0$. The order angular misalignments $\De\rh$ in summarized in \Tb{al exp misal}. Taking $5\un{mrad}$ as the order of these angles, one obtains
\eqref{\cos\rh_{x,y}' = 1 + \O{10^{-5}},\qquad \sin\rh_{x,y}' = \rh_{x,y}' + \O{2\cdot 10^{-8}}}{al small rotation approximation}
and thus it is a good approximation to take just the lowest terms in the Taylor expansion. A similar statement holds for $\rh_z'$ too, just the expansion shall be made at the nominal value $\rh_z$.

The fact that the track \Eq{al local track} hits a sensor can be expressed as $z' = 0$. Then it is straightforward to calculate the $v$ coordinate of the hit:
\eqref{v =
	\Bigg[
		\underbrace{\pmatrix{-\sin\rh_z\cr\cos\rh_z}}_{\vec d}
		+ \underbrace{\pmatrix{-\cos\rh_z\cr -\sin\rh_z}}_{\vec d_\perp} \De\rh_z
	\Bigg]^\T
	\Bigg[
		\underbrace{\pmatrix{a_x\cr a_y}}_{\vec a} (c_z + \De c_z)
		 + \underbrace{\pmatrix{b_x\cr b_y}}_{\vec b}
		 - \underbrace{\pmatrix{c_x\cr c_y}}_{\vec c}
		 - \underbrace{\pmatrix{\De c_x\cr \De c_y}}_{\De \vec c}
	\Bigg]^\T
	+ \O{10^{-7}\un{m}}\ .
}{al hit v}
In fact, we have displayed just the most important terms, the others are included in the $\cal O$ factor. The separation follows from the following order estimates: $c_z \sim 1\un{m}$ (distances between pots in a station), $(b - c)_{x, y} \sim 10^{-2}\un{m}$ (size of the sensors) and $a_{x, y} \sim 10^{-2}\un{rad}$ (the maximal angle that can be detected by near and far units simultaneously). The last estimate is rather an upper bound, see the typical track angles in \Tb{al lhc datasets}. Thus for realistic data the neglected terms would be rather $\O{10^{-8}\un{m}}$.

In \Eq{al hit v} we have defined the \em{unit} vector $\vec d$ which represents the (nominal/thought) \em{read-out direction}. The vector $\vec d_\perp$ is perpendicular to the read-out direction, see \Fg{al proton sensor interaction}. The terms if the left-hand bracket represent the true/actual read-out direction. The vectors $\vec a$, $\vec b$, $\vec c$ and $\De \vec c$ represent two-dimensional vectors of the track slope, the track intercept, sensor's center and the sensor's position misalignment.

The only rotation misalignment that has survived in \Eq{al hit v} is the one about the $z$ axis. Also, this is the only rotation which has non-zero nominal value. To simplify the notation, we will drop the $z$ subscripts in what follows:
$$\rh_z \longrightarrow \rh\ ,\qquad \De\rh_z \longrightarrow \De\rh\ .$$

So far we have considered one sensor and one event. But \Eq{al measurement,al hit v} can be written for any event and any sensor, for $i$-th sensor and $n$-th event they read
\eqref{
m_i^n = (\vec d_i + \De\rh_i\,\vec d_{\perp i})^\T \left[\vec a^n (z_i + \De z_i) + \vec b^n - (\vec c_i + \De\vec c_i)\right] + \De m_i^n + \O{10^{-7}\un{m}}\ ,
}{al meas i n}
where for brevity reasons we have made a replacement
$$c_{z,i} \longrightarrow z_i\ ,\qquad \De c_{z,i} \longrightarrow \De z_i\ .$$

In what follows, we will drop the last two terms on the \rhs, they will be treat as an error, see \Sc{al err}. The first term can be expanded
\eqref{\eqnarray{
\mu_i^n \equiv m_i^n + \vec d_i^\T \vec c_i = & &\cr
& +\vec d_i^\T (\vec a^n z_i + \vec b^n)&\qquad 10^{-2}\un{m}\cr
& -\vec d_i^\T \De\vec c_i&\qquad 10^{-5}\un{m}\cr
& +\vec d_i^\T \vec a^n \De z_i&\qquad 10^{-??}\un{m} \cr
& +\De \rh_i\ \vec d_{\perp i}^\T (\vec a^n z_i + \vec b^n - \vec c_i) &\qquad 10^{-5}\un{m} \cr
& +\De \rh_i\ \vec d_{\perp i}^\T (\vec a^n \De z_i - \De\vec c_i) &\qquad 10^{-8}\un{m}\ , \cr
}}{al effective measurement}
where we have introduced an \em{effective measurement}. It differs from the full one just by constant $\vec d^\T \vec c$ (projection of sensor's center to its read-out direction). From the order estimates (shown above on the right-hand side) it is clear that one may neglect the last term. This is an important result since the remaining terms are (at most) linear in the misalignment parameters $\De \vec c_i$, $\De z_i$ and $\De \rh_i$. Furthermore, one can split the contributions of a track measurement by a sensor at the nominal position and the effect of misalignments:
\eqref{\mu_i^n =
\underbrace{\vec d_i^\T (\vec a^n z_i + \vec b^n)}_{\hbox{track}}
+
\underbrace{\sum_j \ga_{j, i}^n\ \ch_{j, i}}_{\hbox{misalignment corrections}}
\ .}{al effective measurement 2}
In the above relation we have unified the notation for all misalignment quantities $\ch_j$, which influence the measurement via their coefficients $\ga_j$, see the summary in \Tb{al alignment quantities}

\htab{al alignment quantities}{Alignment quantity classes and their coefficients.}{\bln
j & \hbox{quantity} & \ch_{j, i}					& \ga_{j, i}^n\cr\bln
1 & \hbox{shift in read-out direction} & \De s_i 	& -1\cr\ln
2 & \hbox{shift in }z & \De z_i						& \vec d_i^\T \vec a^n  \cr\ln
3 & \hbox{rotation around}z & \De\rh_i				& \vec d_{\perp i}^\T (\vec a^n z_i + \vec b^n - \vec c_i)\cr\bln
}

From \Eq{al effective measurement} it is clear that the measurement is only sensitive to one component of the detector shift, in particular the \em{shift in the read-out direction}:
\eqref{\De s_i \equiv \vec d_i \cdot \De\vec c_i\ .}{al shr def}
Thus, this is the only shift component that can be determined by the track-based alignment on a sensor level. For a RP package both transverse shift components can be determined, see \Sc{al rp fac}.


\subsection[al sim fit]{Simultaneous fit of track and alignment parameters}

From \Eq{al effective measurement 2} we see that sensors' measurements depend on the track parameters $\vec a$ and $\vec b$ and misalignment parameters $\ch$. One might thus fit sensors' data from a sample of events with the parameterization \Eq{al effective measurement 2}. This would result in misalignment parameter estimates and, as a byproduct, estimates of the track parameters.

This idea was first implemented in Millepede (see \bref{millepede}). Here we use different notation as our work as been done independently.

In order to perform the fit, it is useful to switch into a matrix formalism. Let us put all effective measurements in $n$-th event into vector $\vec\mu^n$. Then \Eq{al effective measurement 2} can be recast into:
\eqref{\vec\mu^n =
	\underbrace{\mat\al^n \vec\ta^n}_{\hbox{track}}
	+
	\underbrace{\sum_j \mat\Ga_j^n \vec \ch_j}_{\hbox{misalignments}}
\ ,}{al vec mu}
where
\eqref{\vec\ta^n = (a_x^n, a_y^n, b_x^n, b_y^n)^\T}{al vec tau}
is the vector of the track parameters and 
\eqref{\mat \al^n = \pmatrix{
	\vdots		& \vdots		& \vdots	& \vdots \cr
	d_{ix} z_i	& d_{iy} z_i	& d_{ix}	& d_{iy} \cr
	\vdots		& \vdots		& \vdots	& \vdots \cr
}, \quad i\hbox{ goes through all sensors involved in }n\hbox{-th event.}}{al mat alpha}
The vector $\vec\ch_j$ groups all the misalignments of the $j$-th type for all sensors. The matrices $\mat\Ga_j^n$ contain the coefficients $\ga^n_{j,i}$ as introduced in \Eq{al effective measurement 2}.

Let us remark that starting with \Eq{al vec mu}, the superscripts are reserved for event numbers, while subscripts fro the misalignment quantity class (see \Tb{al alignment quantities}).

Now we may put the measurement from all events into one vector:
\eqref{\vec M = (\vec\mu^1, \vec\mu^2, \ldots)^\T}{al vec M}
and similarly for the track and misalignment parameters
\eqref{\vec P = (\vec\ta^1, \vec\ta^2, \ldots || \vec\ch_1, \vec\ch_2, \ldots)^\T\ .}{al vec P}
With this notation, relations of the type \Eq{al vec mu} can be written
\eqref{\vec M = \mat A \vec P\ ,\qquad
	\mat A = \pmatrix{
	\ddots & 		&		&\vrule	&\ddots	&				&\udots	\cr
		&\mat \al^n	&		&\vrule	&		& \mat\Ga_j^n	&	\cr
		&		& \ddots	&\vrule	&\udots	&				&\ddots	\cr
	}\ .
}{al effective measurement all}
The left-hand part of $\mat A$ matrix is block-diagonal (with matrices $\mat\al^n$ on the diagonal). The right-hand part is built from $\mat\Ga_j^n$ matrices, the index $n$ increases in top-down direction, the index $j$ in the left-right one.

This form of \Eq{al effective measurement all} is convenient for an application of the linear least squares (LS) method (see e.g.~Sec.~6.6 in \bref{barlow}). The method gives the following prescription for the estimate of the parameter vector $\vec P$:
\eqref{\hat\vec P = (\mat A^\T \mat V^{-1} \mat A)^{-1} \mat A^\T \mat V^{-1}\,\vec M\.}{al P estimate exact}
We use the hat in $\hat\vec P$ to emphasize that it is an estimate. The $\mat V$ matrix is the covariance matrix for the measurements $\vec M$. Since the measurements from different events are independent (as random variables), $\mat V$ has a block-diagonal structure
\eqref{\mat V = \pmatrix{
\ddots	&			&			\cr
		& \mat V^n	&			\cr
		&			& \ddots	\cr
}\ ,}{al mat V}
where a block $\mat V^n$ represents the covariance matrix of measurements from the $n$-th event (see \Sc{al err} for more details).

Unfortunately, \Eq{al P estimate exact} can not be applied directly. The problem comes from the fact that the $\ga^n_{j,i}$ coefficients (see \Tb{al alignment quantities}), which are contained in $\mat\Ga_j^n$ and $\mat A$ matrices, depend on the track parameters. And these are unknown at the fit time. Instead, one may adopt an \em{iterative} approach and use track parameter estimates obtained via a model where all misalignments are neglected. This is basically using $\mat \al^n$ as fit matrix for $\vec \mu^n$ data. Let us denote $\mat\Ga$ and $\mat A$ matrices obtained in this way with tildas:
\eqref{\vec\ta \rightarrow \hat\vec\ta: \quad \mat\Ga \rightarrow \tilde\mat\Ga, \mat A \rightarrow \tilde\mat A\ .}{al tau linearization}
This step brings in a necessity for several iterations, we will discuss them in \Sc{al err}.

Now we would like to extract the vector of misalignment estimates $\hat\vec\ch = (\hat\vec\ch_1, \hat\vec\ch_2, \ldots)^\T$ from the full parameter vector $\hat\vec P$ (see \Eq{al vec P}). It can be done by a straightforward evaluation of the \rhs{} of \Eq{al P estimate exact}. The inverse of
\eqref{\tilde\mat A^\T \mat V^{-1}\tilde\mat A =
\pmatrix{
\ddots 	& 											& 		&\vrule &\ddots & 														& \udots\cr
	 	& {\mat\al^n}^\T \mat V^{n^{-1}} \mat\al^n	& 		&\vrule & 		& {\mat\al^n}^\T \mat V^{n^{-1}}{\tilde\mat\Ga}_j^n 			& 		\cr
 		& 											&\ddots &\vrule &\udots & 														& \ddots\cr
\noalign{\hrule}
\ddots 	& 											&\udots	&\vrule &\ddots & 														& \udots\cr
	 	& {\tilde\mat\Ga_i}^{n^\T} \mat V^{n^{-1}} \mat\al^n& 		&\vrule & 		& \sum_n {\tilde\mat\Ga_i}^{n^\T} \mat V^{n^{-1}} {\tilde\mat\Ga}_j^n 	& \cr
\udots	& 											&\ddots &\vrule &\udots & 														& \ddots\cr
}
\ .}{al mat ATA}
can be obtained (let us assume momentarily that the inverse exists) with the aid of the following block matrix inverse rule (see e.g.~Sc.~4.3.2 in \bref{barnett}):
\eqref{\pmatrix{
	\cal A	&\strut\vrule	&\cal B	\cr
	\noalign{\hrule}
	\cal C	&\strut\vrule	&\cal D\cr
	}^{-1} = \pmatrix{
	\ldots							&\strut\vrule	&\ldots\cr
	\noalign{\hrule}
	-\mat S^{-1}{\cal C}{\cal A}^{-1}	&\strut\vrule	& \mat S^{-1}\cr
},\qquad \mat S = {\cal D} - {\cal C}{\cal A}^{-1}{\cal B}\ .}{al block inverse}
The matrix $\tilde\mat S$ can be written
\eqref{\tilde\mat S = \pmatrix{
	\ddots	& 																		& \udots\cr
			& \sum_n {\tilde\mat\Ga_i}^{n^\T} \mat\si^n {\tilde\mat\Ga}_j^n 	& \cr
	\udots	& 																		& \ddots\cr
}\ ,}{al mat S}
where we have used
\eqref{\mat\si^n = \mat V^{n^{-1}} - \mat V^{n^{-1}} \mat\al^n({\mat\al^n}^\T \mat V^{n^{-1}} \mat\al^n)^{-1} \mat {\mat\al^n}^\T V^{n^{-1}}\ .}{al sigma n}
The second bit needed for \Eq{al P estimate exact} is
\eqref{\tilde\mat A^\T \mat V^{-1}\,\vec M = \pmatrix{
	\vdots\cr
	{\mat\al^n}^\T \mat V^{n^{-1}} \,\vec \mu^n\cr
	\vdots\cr
	\ln
	\vdots\cr
	\sum_n{\tilde\mat\Ga_j}^{n^\T} \mat V^{n^{-1}} \, \vec \mu^n\cr
	\vdots\cr
}\ .}{al vec ATm}
Putting all together yields
\eqref{\tilde\mat S \hat\vec\ch \equiv
\tilde\mat S \pmatrix{
	\vdots\cr
	\hat\vec\ch_j\cr
	\vdots\cr
}
=\tilde\vec T\ ,\qquad
\tilde\vec T = \pmatrix{
	\vdots\cr
	\sum_n {\tilde\mat\Ga_j}^{n^\T} \mat\,\mat\si^n\,\vec \mu^n\cr
	\vdots\cr
}
\ .}{al fit equation}

In fact, the elements of $\mat\si^n\,\vec \mu^n$ vector are residuals for the $n$-track divided by the corresponding measurement uncertainty. We will call this ratio \em{normalized residuals} and denote $\vec r^n$. The \em{full residuals} will be denoted by $\vec R^n$:

\eqref{\mat\si^n\,\vec \mu^n \equiv \vec r^n = \mat V^{n^{-1}} \vec R^n\ .}{al vec R}

It is worth mentioning some properties of the $\mat\si^n$ matrix -- it is symmetric, singular\footnote{%
Each column of $\mat\al^n$ provides a non-trivial solution to $\mat\si^{n} \vec w = 0$ and thus $\mat\si^n$ is singular (see e.g.~theorem 1.5.3 in \bref{anton}).} and its product with $\mat V^n$ is idempotent
\eqref{
	\mat\si^{n^\T} = \mat\si^n\ ,\qquad
	\mat\si^{n} \mat\al^n = 0\ ,\qquad
	(\mat V^n \mat\si^n)^2 = \mat V^n \mat\si^n\ .
}{al prop sigma n}

Above, we have assumed that the $\mat S$ matrix is regular (and thus the matrix inversion in \Eq{al P estimate exact} can be performed). However, in \Sc{al sing modes} we will demonstrate the opposite -- this just reflects the fact that certain misalignment are inaccessible to the track-based alignment. Despite the deficiency, the result \Eq{al fit equation} is almost correct. An easy way to see it is to plug \Eq{al vec mu} to the definition of $\tilde\vec T$ in \Eq{al fit equation}, it yields
\eqref{\tilde\vec T =
\pmatrix{
	\vdots \cr
	\sum_n {\tilde\mat\Ga}_j^{n^\T} \mat\si^n \vec\mu^n\cr
	\vdots \cr
}
=
\pmatrix{
	\ddots & & \udots \cr
	 & \sum_n {\tilde\mat\Ga}_j^{n^\T} \mat\si^n \mat\Ga_i^n\cr
	\udots & & \ddots \cr
}
\vec\ch_i
=
\bar\mat S \vec\ch
}{al exact fit equation}

The only difference between $\tilde\mat S$ in \Eq{al fit equation} and $\bar\mat S$ in \Eq{al exact fit equation} is that $\tilde\mat S$ contains both $\mat\Ga$ matrices with tildes (see \Eq{al mat S}). This difference represents the error we make by the simplification step \Eq{al tau linearization}. Since the expected misalignments are small, the difference between $\mat\Ga$ and $\tilde\mat\Ga$ shall be small too and it is reasonable to expect that the solution $\hat\vec\ch$ of \Eq{al fit equation} would be close to the actual misalignments $\vec\ch$. The error can be reduced by taking several iterations, it will be discussed in \Sc{al err}.

Let us close this section with a simple example -- let us consider only the shifts in the read-out direction and a data sample where all sensors are active in all events. In such a case the coefficients $\ga^n = -1$ (cf. \Tb{al alignment quantities}) and thus matrices $\mat\Ga^n = -1$. Note that the $\ga^n$ coefficients are independent of the track parameters $\vec a^n$ and $\vec b^n$. Thus there is no need for the simplification \Eq{al tau linearization}, which effectively means $\tilde\mat S = \bar\mat S$ and there is no difference between the exact fit equation \Eq{al exact fit equation} and the simplified one \Eq{al fit equation}. They can be written (here the indeces $i,j$ identify the sensors):
\eqref{\tilde\mat S \hat\vec\ch = \tilde\vec T\ , \quad
	\tilde\mat S = \pmatrix{
		\ddots	&						&\udots	\cr
				& \sum_n\mat\si^n_{ij}	&		\cr
		\udots	&						&\ddots	\cr
	}\ ,\quad
	\hat\vec\ch = \pmatrix{
		\vdots\cr
		\De \hat s_j\cr
		\vdots\cr
	}\ ,\quad
	\tilde\vec T = \pmatrix{
		\vdots\cr
		\sum_n r^n_i\cr
		\vdots\cr
	}\ .
}{al fit eq ex}
Hence the $\tilde\vec T$ vector is given by the sum of the normalized residuals, the $\tilde\mat S$ embodies the relation between the residuals and the misalignment shifts $\De s_i$.

\TODO{Conclusion? Bridge to the next chapter}.

\subsection[al sing modes]{Singular and weak modes}

In the previous section we have derived the fit equation \Eq{al fit equation}. If the matrix $\mat S$ were regular, one could obtain the alignment parameters $\hat\vec\ch$ by means of matrix inversion. But let us look closer on the fit equation. Note that the sensors' measurements enter the \rhs{} in the form of residuals. That means that any misalignment mode that does not generate residuals can not be revealed by the track-based algorithm. These misalignment modes are those where the misalignments $\vec\ch_j$ can be compensated (keeping measurements $\vec\mu^n$ unchanged) by varying the track parameters $\vec\ta^n \to {\vec\ta '}^n$ (cf. \Eq{al vec mu}):
\eqref{\mat\al^n {\vec\ta}^n + \sum_j \mat\Ga_j^n \vec\ch_j = \vec\mu^n = \mat\al^n {\vec\ta'}^n\ .}{al tau chi equiv}
The $\mat\Ga$ matrices may depend on the track parameters (see \Tb{al alignment quantities}), writing that yields
\eqref{\sum_j \mat\Ga_j(\vec\ta^n)\, \vec\ch_j = \al^n\, (\underbrace{{\vec\ta'}^n - \vec\ta^n}_{\De\vec\ta^n})\ .}{al sm con}
If one can find a vector $\De\vec\ta^n$ for any set of track parameters $\vec\ta^n$, then from \Eq{al mat S} it follows
\eqref{
	\mat S \pmatrix{\vdots\cr \vec\ch_i\cr\vdots} = 
	\pmatrix{\vdots\cr \sum_n {\mat\Ga_i^n}^\T \mat\si^n \sum_j \mat\Ga_j^n\, \vec\ch_j \cr\vdots} =
	\pmatrix{\vdots\cr \sum_n {\mat\Ga_i^n}^\T \mat\si^n \mat \al^n\, \De\vec\ta^n \cr\vdots} = 0\ ,
}{al sm}
which is a direct consequence of \Eq{al prop sigma n}. This would mean that $\mat S$ is singular (see e.g.~theorem 1.5.3 in \bref{anton}), with $\vec\ch$ being an eigenvector corresponding to zero eigenvalue. This vector represents an misalignment mode that is not accessible to the track-based alignment. We will refer to such modes as \em{singular modes}.

Let us come back to \Eq{al sm con}. It is important is to find a $\De\vec\ta^n$ vector for any vector $\vec\ta^n$. No matter what the interpretation is for the latter one -- whether it is the vector of true or estimated track parameters, cf. \Eq{al tau linearization}. This is exactly what makes the difference between $\mat\Ga$ and $\tilde\mat\Ga$ and consequently among $\mat S$, $\bar\mat S$ and $\tilde\mat S$. And that is why the singular modes found in \Eq{al sm} are also singular modes for $\bar\mat S$ and $\tilde\mat S$ matrices.

Later, we will find useful to use the definition of the $\mat\al$ matrix \Eq{al mat alpha} and rewrite the \rhs{} of \Eq{al sm con} as
\eqref{\mat\al^n\, \De\vec\ta^n = \pmatrix{
	\vdots\cr
	\vec d_i \cdot (\De\vec a^n z_i + \De\vec b^n)\cr
	\vdots\cr
}, \quad i\hbox{ goes through all sensors involved in }n\hbox{-th event.}}{al sm track par}
The vectors $\De\vec a^n$ and $\De\vec b^n$ are merely an alternative description of the track-parameter variation $\De\vec\ta^n$, cf. \Eq{al vec tau}.

In the rest of this section we will identify all singular modes that may appear in the alignment of the RP sensors. We will first focus on the singular modes that arise from the interplay between the geometry of the detector apparatus and the geometry of the tracks. Afterwards we will turn our attention to the singular modes provoked by special (pathological) track distributions.

\caption{Geometrical singular modes}

In this part we will discuss the singular modes corresponding to each alignment quantity class, one by one. But prior to that, let us remark one characteristics of the RP sensors -- all of their nominal read-out directions are parallel (or anti-parallel) to either $U$ or $V$ axes (see \Sc{ttm}). This could be written as $\vec\d_i \in \{\pm U, \pm V\}$. We will consider a bit more general case, where the sensors, classified by their read-out directions, split into several \em{read-out groups}:
\eqref{\vec d_i \in \lbrace \pm\vec\de_1, \pm\vec\de_2, \ldots \rbrace \ ,
	%\qquad |\vec\de_i \cdot \vec\de_j| \neq 1\hbox{ for } i \neq j\ .
}{al read-out groups}
where the (non-collinear) vectors represent the read-out directions of each group. 

For \em{shifts in the read-out direction}, the condition \Eq{al sm con} reads (using \Eq{al sm track par})
\eqref{
	\De s_i \equiv
	\vec d_i \cdot \De\vec c_i =
	- \vec d_i \cdot ( \De\vec a^n z_i + \De\vec b^n )\ .
}{al sm shr sol}

Since the \lhs{} does not depend on the event number $n$, neither must the \rhs{}. That is why the ${\De\vec a}=(\De a_x, \De a_y)^\T$ and ${\De\vec b=(\De b_x, \De b_y)^\T}$ vectors do not need to carry a superscript. These two vectors define the structure of the singular modes -- they correspond to \em{global} ($\De\vec b$) and \em{linearly-progressive} ($\De\vec a\, z_i$) shifts in $x$ and $y$. By global shift we mean a shift that is common for all sensors. The size of a linearly-progressive shift is proportional to the $z$-position of a sensor. There are four free parameters, thereby four singular modes. The situation is illustrated in \Fg{al sing modes shr}.

\fig{fig/pdf/al_sing_modes_shr.pdf}{al sing modes shr}{An illustration of the singular modes \Eq{al sm shr sol}: a front view on a sensor package. Each sensor is represented by a dot (its center $\vec c$) and an arrow (its read-out direction $\vec d$), cf.~the front view in \Fg{al proton sensor interaction}. The hollow dots mark the nominal positions, the solid ones represent the misalignment mode. For simplicity we have considered a simplified package of six sensors only (see the numbers), moreover with an equal spacing $\De z$. The $U$ (red) and $V$ (blue) sensors are regularly alternating, the first one is at $z = 0$.}

\em{Shifts in $z$}. Writing the condition \Eq{al sm con} for each read-out group yields
\eqref{\vec\de_g \cdot \vec a^n \De z_i = \vec\de_g \cdot (
	\De\vec a^n\, z_i + \De\vec b^n
)\ ,\qquad\hbox{for sensors }i\hbox{ from group }g\ .}{al sm shz con grp}
Since this shall hold for all $\vec a^n$, the ratio $\vec\de_g\cdot\De\vec a^n / \vec\de_g\cdot\vec a^n$ must be $n$-independent. On the other hand, it can be different for every group. Formally written (the treatment of the $\De\vec b^n$ term is identical):
\eqref{
{\vec\de_g\cdot\De\vec a^n \over \vec\de_g\cdot\vec a^n} = \al_g\ ,\qquad
{\vec\de_g\cdot\De\vec b^n \over \vec\de_g\cdot\vec a^n} = \be_g\ .
}{al sm shz ratio}
The above relation determines what the $\De\vec a^n$ vector should be for a given track slope vector $\vec a^n$ and a set of $\al_g$ parameters. In fact, there is one such a relation per one read-out group. This should be compared to the rank of the $\De\vec a^n$ vector -- it is two-dimensional. That means that for one group only, that there is an infinite number of solutions ($\De\vec a^n$ vectors that fulfil \Eq{al sm shz ratio}). For two groups, there is one and unique solution. From \Eq{al sm shz con grp} one can read off the corresponding singular modes:
\eqref{\De z_i = \al_g z_i + \be_g\ ,\qquad\hbox{for sensors }i\hbox{ from group }g\ ,}{al sm shz sol 2g}
For more than two groups there is generally no solution (it is an over-constrained problem). However, if the $\al_g$ parameters are equal for all the read-out groups, the equation has a solution: $\De\vec a^n = \al \vec a^n$. It corresponds to singular modes:
\eqref{\De z_i = \al z_i + \be\ .}{al sm shz sol 3g}
In fact, these are just special cases of the singular modes \Eq{al sm shz sol 2g}, therefore they are present for any type of geometry (any number of read-out groups). Both types of the singular modes are illustrated in \Fg{al sing modes shz}, for a summary see \Tb{al sing mode overview}.


\fig{fig/pdf/al_sing_modes_shz.pdf}{al sing modes shz}{An illustration of the singular modes \Eq{al sm shz sol 2g,al sm shz sol 3g}: a top view on a sensor package, cf.~\Fg{al proton sensor interaction}. Each sensor is represented by a dot (its center $\vec c$) and an arrow (the $x$ projection of its read-out direction $\vec d$). The hollow dots mark the nominal positions, the solid ones represent the misalignment mode. For simplicity we have considered a simplified package of six sensors only (see the numbers), moreover with an equal spacing $\De z$. The first sensor is at $z = 0$.
Left: the case with two read-out groups, as in the nominal geometry the $U$ (red) and $V$ (blue) sensors are alternating regularly.
Right: a hypothetical situation with a third read-out group added (sensor four drawn in green).
}

\em{Rotations around $z$}. When applying the condition \Eq{al sm con} one shall keep both rotation and read-out-direction shift misalignments. It will turn out that they can not be separated. Written for each read-out group the condition reads:
\eqref{
	\De\rh_i\, \vec \de_{\perp_g} \cdot (\vec a^n z_i + \vec b^n - \vec c_i) =
	\vec\de_g \cdot (
		\De\vec a^n\, z_i +
		\De\vec b^n +
		\De\vec c_i
	)\ .
}{al sm rotz con}
In any realistic (non-pathological) geometry the vectors $(z_1, z_2, \ldots)^\T$, $(c_{x,1}, c_{x,2}, \ldots)^\T$, $(c_{y,1}, c_{y,2},\ldots)^\T$ and $(1, 1, \ldots)^\T$ are linearly independent. This means that the above condition can only be fulfilled if the coefficients to these vectors are equal on both sides of the equation:
\eqref{
	\De\rh_i\, \vec\de_{\perp_g} \cdot \vec a^n = \vec\de_g \cdot \De\vec a^n\ ,\qquad
	\De\rh_i\, \vec\de_{\perp_g} \cdot \vec b^n = \vec\de_g \cdot \De\vec b^n\ ,\qquad
	- \De\rh_i\, \vec\de_{\perp_g} \cdot \vec c_i = \vec\de_g \cdot \De\vec c_i\ .
}{al sm rotz con sep}
The first two relations require $\De\rh_i$ to be constant within every read-out group: $\De\rh_i \rightarrow \De\rh_g$. Moreover they are very similar to \Eq{al sm shz ratio} and so is the interpretation. For one group only there exists infinitely many solutions (vectors $\De\vec a^n$ and $\De\vec b^n$ that obey \Eq{al sm rotz con sep}). For two groups, there is exactly one solution. For more than two groups, the equation is, generally over-constrained and with no solution. The only exception occurs if $\De\rh_g$ is the same for all groups, then the equation is solved by $\De\vec a^n = \De\rh (-a^n_y, a^n_x)^\T$, etc.

The third relation in \Eq{al sm rotz con sep} gives the read-out-direction shift accompanying the rotation singular modes:
\eqref{\De s_i \equiv \vec d_i \cdot \De\vec c_i = - \De\rh_i\, \de s_i\ ,\qquad \de s_i = \vec d_{\perp_i} \cdot \vec c_i\ .}{al sm rotz shr}
The reason for this shift is that the singular mode is a rotation about the $z$ axis, but the alignment rotations are performed about sensors' centers, displaced by $\vec c_i$ from the $z$ axis.

The singular modes can then be summarized
\eqref{
\pmatrix{
	\De\rh_i \cr
	\De s_i
	}
= \De\rh_g \pmatrix{
	1\cr
	- \de s_i
}
\ .}{al sm rotz sol}
For three and more groups, $\De\rh$ is a constant, thus there is just one singular mode. For two groups, the values of $\De\rh_g$ are independent for each group, therefore there are two singular modes. The situation is illustrated in \Fg{al sing modes rotz}, a summary can be found in \Tb{al sing mode overview}.

\fig{fig/pdf/al_sing_modes_rotz.pdf}{al sing modes rotz}{An illustration of the singular modes \Eq{al sm rotz sol}: a front view on a sensor package. Each sensor is represented by a dot (its center $\vec c$) and an arrow (its read-out direction $\vec d$), cf.~the front view in \Fg{al proton sensor interaction}. The hollow dots and dashed arrows/lines show the nominal positions and read-out directions, the full dots and solid arrows represent the misalignment mode. For simplicity we have considered a simplified package of six sensors only (see the numbers), note that often they lie on top of each other in the drawing. Left: the case with two read-out groups, like in the nominal geometry, the $U$ (red) and $V$ (blue) sensors are alternating regularly. The $\De\vec c$ vector shows the accompanying read-out-direction shift \Eq{al sm rotz shr}. Right: a hypothetical situation with a third read-out group (sensor 4 drawn in green).
}

To illustrate the structure of the singular modes we have made a number of MC studies. We have considered a series of geometries, starting with the nominal one and adding rotation misalignments to all sensors. These mis-rotations have been randomly generated according to a normal distribution with a variable variance. Going from zero variance to positive values represents a smooth transition from a geometry with two read-out groups to geometries with more groups. For certain alignment modes this means a transition from singular to non-singular state, see \Fg{al eig rho}.

A convenient way to visualize the structure of the $\tilde\mat S$ matrix is to plot its eigenspectrum. The singular modes are represented by zero eigenvalues, the modes accessible to the track-based alignment by non-zero values. In fact the eigenvalues have an important meaning for the alignment application. To unveil it, let us rewrite the fit equation \Eq{al fit equation} in ``eigen coordinates'' -- coordinates relative to an orthonormal set of $\tilde\mat S$ eigenvectors. Denoting $\mat Q$ a matrix containing this set as columns, one can write (see e.g.~Eq.(6.85) in \bref{barnett}):
\eqref{
	\mat D\, \hat\vec\ch^{\rm E} = \tilde\vec T^{\rm E}\ ,\qquad
	\hat\vec\ch^{\rm E} = \mat Q^\T \hat\vec\ch\ ,\qquad
	\tilde\vec T^{\rm E} = \mat Q^\T \tilde\vec T\ ,
}{al fit eq eigen}
where $\mat D$ is a diagonal matrix containing the eigenvalues $\la$ of $\tilde\mat S$. The $\hat\vec\ch^{\rm E}$ and $\tilde\vec T^{\rm E}$ vectors are the vectors $\hat\vec\ch$ and $\tilde\vec T$ expressed in the ``eigen coordinates''. Since $\mat D$ is diagonal, the fit equation has a simple form, written for the $m$-th alignment mode:
\eqref{\hat\ch^{\rm E}_m = {\tilde T^{\rm E}_m\over \la_m} \longrightarrow {\tilde T^{\rm E}_m + \De\tilde T^{\rm E}_m \over \la_m}\ .}{al fit eq eigen one}
If $\la_m = 0$, the equation cannot be solved -- the mode $m$ is a singular mode. However one may expect troubles even if $\la_m$ is non-zero but small. One should keep in mind that the $\tilde T^{\rm E}$ is built from measurements, see \Eq{al fit equation}, and as such it is subject to an experimental error. Formally, this is expressed by adding the $\De\tilde T^{\rm E}_m$ term on the \rhs{} above. If $\la_m$ is small, the ratio $\De\tilde T^{\rm E}_m/\la_m$ may grow large, in other words $\hat\ch^{\rm E}_m$ can only be determined with a large uncertainty. In that sense, the eigenvalue $\la_m$ controls the determination power of the alignment mode $m$. If the eigenvalue is small, the determination is weak -- that is why we will call these modes \em{weak modes} in what follows. For a more complete discussion of the uncertainties, see \Sc{al err}.

Whereas the singular and weak modes are very different from the theoretical point of view, they look the same for computer programs. Any numerical calculation of eigenvalues has a limited accuracy (we have used ROOT matrix libraries \bref{root matrix} at double precision). Therefore the eigenvalues corresponding to singular modes are found small but non-zero. In order to distinguish between regular and singular (or very weak) modes one thus needs to set a finite \em{singular limit}. \Fg{al eig rho} shows that a value of $10^{-9}$ is reasonable.

It reasonable to expect that the eigenvalues of $\tilde\mat S$ would be proportional to the number of events $N_{\rm events}$. To have results event-number independent, it makes sense to define \em{normalized eigenvalues}:
\eqref{\la_{\rm N} = \hbox{eigenvalue of }\tilde\mat S/N_{\rm events}\ ,}{al S norm eig val}
which are used in \Fg{al eig rho}.


\fig{fig/pdf/al_eigenvalues_rho.pdf}{al eig rho}{\TODO{correct eq numbers}. The spectrum of normalized eigenvalues of the $\tilde\mat S$ matrix ($\la_{\rm N}$) as a function of geometry. The geometries considered have been created from the nominal one-station geometry by applying rotation misalignment to each sensor. The misalignments have been randomly generated according to a normal distribution with zero mean and variance $\si^2_\rh$ -- the quantity on the horizontal axis. The case $\si_\rh = 0$ corresponds to the nominal geometry, which contains two read-out groups. Increasing $\si_\rh$ then represents a smooth transition to configurations with three and more read-out groups.
The colors correspond to the eigenvalue order (bottom-top). Therefore, when two curves representing two alignment modes cross, they exchange their colors (e.g. the mode (4.47) in the bottom plot).
% Theta 10 mrad, geometry 2.7x3.3, overlap=f
}

\caption{Pathological track distributions}

Besides the singular modes derived above, there could be singular modes arising from special track distributions. One could immediately think of a case where the sensors split into several groups such that no track can go through sensors of different groups at a time. Regarding RPs, this situation may appear when the pots are not inserted close enough and therefore no track can go through the overlap between the vertical and horizontal pots. In this case, the alignment task would split into several smaller tasks (one per group) and for each of them one could write an alignment equation like \Eq{al fit equation}. Each of these equations contains a $\tilde\mat S$ matrix with its singular modes as discussed above. Therefore the number of singular modes gets multiplied by the number of groups.

The LHC proton tracks are very parallel (see e.g. \Tb{al lhc datasets}). In other words the distribution of the track slopes $\vec a^n$ is peaked about a mean value $\bar\vec a$ with a spread (in either projection)
\eqref{\si_a = \O{10^{-4}\un{rad}}\ .}{al sm a dist}
Let us reexamine the condition for rotation singular modes \Eq{al sm rotz con} when the spread $\si_a\to 0$:
\eqref{
	\De\rh_i\, \vec \de_{\perp_g} \cdot (\bar\vec a z_i + \vec b^n - \vec c_i) =
	\vec\de_g \cdot (
		\De\vec a^n\, z_i +
		\De\vec b^n +
		\De\vec c_i
	)\ .
}{al sm rotz con la}
Again, we will compare the coefficients to the vectors vectors $(z_1, z_2, \ldots)^\T$, $(c_{x,1}, c_{x,2}, \ldots)^\T$, $(c_{y,1}, c_{y,2},\ldots)^\T$ and $(1, 1, \ldots)^\T$ on both sides. For \Eq{al sm rotz con} there was only one way to do it, here is another one. It comes from the fact that $\bar\vec a z_i$ term is not $n$-dependent. Therefore it can be absorbed into the $\De\vec c_i$ term on the \rhs Then, $\vec b^n$ can be linked to $\De\vec a^n$, this requires $\De\rh_i$ to be proportional to $z_i$. Denoting $\al_i$ the proportionality constant, the condition splits into three:

\eqref{
	\al_i\, \vec\de_{\perp_g} \cdot \vec b^n = \vec\de_g \cdot \De\vec a^n\ ,\qquad
	0 = \vec\de_g \cdot \De\vec b^n\ ,\qquad
	\al_i z_i\, \vec\de_{\perp_g} \cdot (\bar\vec a - \vec c_i) = \vec\de_g \cdot \De\vec c_i\ .
}{al sm rotz con sep la}
The first relation is very similar to the first one from \Eq{al sm rotz con sep} and so the interpretation is. First, the proportionality constant must be constant within each read-out group: $\al_i\to \al_g$. Second, for one group only, there are infinitely many solutions $\vec\De a^n$ for every $\vec b^n$. For two groups, there is exactly one solution. For three and more groups the problem is generally over-constrained, having a solution only if $\al_g$ is the same for all the read-out groups.

The third relation in \Eq{al sm rotz con sep la} determines the read-out-direction shift that accompanies the rotation singular modes. They can be summarized:
\eqref{
\pmatrix{
	\De\rh_i \cr
	\De s_i
	}
= \al_g z_i \pmatrix{
	1\cr
	\vec d_{\perp_i} \cdot (\bar\vec a - c_i)
}
\ .}{al sm rotz sol la}
For two groups, the $\al_g$ parameters are independent for both groups. There are thus two more singular modes. For three and more groups, the values of $\al_g$ must be the same for all groups, therefore there is just one new singular mode.

Note that the $\De s_i$ part of the rotation singular vectors \Eq{al sm rotz sol,al sm rotz sol la} can be compensated by the read-out-direction shift singular modes \Eq{al sm shr sol}. It is therefore sufficient to consider the rotation singular modes in a form:
\eqref{\De\rh_i = \al_g z_i + \be_g\ .}{al sm rotz full sol}
$\be_g$ (equivalent of $\De\rh_g$ in \Eq{al sm rotz sol}) represents a global rotation, $\al_g$ leads to a linearly-progressive rotation.

\TODO{ref to \Fg{al sing modes rotz al}, for a summary see \Tb{al sing mode overview}.}

\fig{fig/pdf/al_sing_modes_rotz_al.pdf}{al sing modes rotz al}{An illustration of the singular modes \Eq{al sm rotz full sol}: a front view on a sensor package. Each sensor is represented by a dot (its center $\vec c$) and an arrow (its read-out direction $\vec d$), cf.~the front view in \Fg{al proton sensor interaction}. The hollow dots and dashed arrows/lines show the nominal positions and read-out directions, the full dots and solid arrows represent the misalignment mode. For simplicity we have considered a simplified package of six sensors only (see the numbers), moreover with an equal spacing $\De z$. There are two read-out groups (like in the nominal geometry): the $U$ (red) and $V$ (blue) sensors are regularly alternating, the first one is at $z = 0$. For three and more groups the situation looks very similar, only the $\al$ and $\be$ parameters are the same for all the groups. To keep the figure simple, we assumed $\bar\vec a = 0$, that is why all detector centers lie on a circle.
}

Unlike the geometrical singular modes, the modes \Eq{al sm rotz sol la} are singular for $\mat S$ and $\bar\mat S$ only, not for $\tilde\mat S$. The reason is that the transition \Eq{al tau linearization} from true to estimated track parameters brings in an experimental error (finite resolution of the sensors and their misalignment). Thus even if the true tracks were perfectly parallel, the reconstructed ones would be not. Consequently the modes \Eq{al sm rotz sol la} correspond only to small, but non-zero eigenvalues of $\tilde\mat S$ -- they are weak modes.

\Fg{al eig theta} shows a transition from almost parallel (low $\si_a$) to non-parallel tracks (high $\si_a$). There are two alignment modes very sensitive to the track angular spread $\si_a$ -- see the labelled blue and red curves. Regarding the $\mat S$ matrix (left panel), these modes become singular in the parallel limit $\si_a \to 0$. Not so for the $\tilde\mat S$ matrix (right panel). The eigenvalues stop reducing at about $\si_a\approx 1\un{\mu rad}$, which compares well to the angular resolution of a station. This saturation demonstrates the effect discussed in the previous paragraph. A similar saturation takes place (in both panels) at the high-$\si_a$ end. It starts about $\si_a \approx 10\un{mrad}$, which corresponds well to the maximal angle which can be detected by both near and far units. Thus tracks with higher angles can not contribute to an alignment of a station.

\fig{fig/pdf/al_eigenvalues_theta.pdf}{al eig theta}{Spectra of normalized eigenvalues ($\la_{\rm N}$) of the $\mat S$ (left) and $\tilde\mat S$ (right) matrices, with only the read-out-direction shifts and rotations included. The spectra are plotted as a function of the track-angle spread $\si_a$ (same values used for $x$ and $y$ projections). Zero value of the spread means perfectly parallel tracks. The simulations have been done for a RP station of the nominal geometry (two read-out groups). For more groups there would be just one line instead of the labeled red and blue curves. The vertical dotted line marks the typical slope spread in the LHC data.
% shr\_rotz=0. 2.7x33, overlap=f
}

It would be possible to draw a similar line of arguments for a situation where $\vec b^n\to \bar\vec b = $ const. But keeping in mind the application to LHC proton tracks, this turns out to be very unrealistic scenario.

At the end, let us remark that parallel tracks do not introduce new weak modes for the shifts in the read-out directions. The corresponding singular mode condition \Eq{al sm shr sol} is completely independent of the track slopes $\vec a^n$, therefore their vanishing spread makes no difference. Regarding the shifts in $z$, it makes little sense to consider the case of parallel tracks -- there the $z$-shifts can not be determined at all. The singular and weak modes for all considered configurations are summarized in \Tb{al sing mode overview}.

\htab{al sing mode overview}{An overview of the singular and weak modes. The bold numbers give the number of singular/weak modes that are listed afterwards. The ''gl.'' abbreviation stands for ''global'' which is used to refer to a mode with constant coefficients for every sensor. The ''l.p.'' stands for ''linearly-progressive'' which means a mode the coefficients of which are proportional to $z_i$. Note that the additional modes for the parallel-track case are weak only.
\TODO{one read-out group: such det. apparatus can determine one track projection only}
\TODO{parallel tracks make no sense for $z$-shifts}
}{
\omit&\multispan{4}\bhrulefill\cr
\omit			&\multispan2\bvrule\strut\hfil two read-out groups\hfil &\multispan2\strut\vrule\hfil three and more read-out groups\hfil\cr
\omit\bhrulefill&\multispan{4}\hrulefill\cr
						& \hbox{non-parallel tracks} & \hbox{parallel tracks} & \hbox{non-parallel tracks} & \hbox{parallel tracks} \cr\bln
\hbox{read-out shifts}	&\multispan4\bvrule\hfil {\bf 4}: $x$ and $y$ global and linearly progressive shifts\hfil\cr\ln
%
&\hbox{{\bf 2}: gl. rot.}  &\hbox{{\bf 4}: gl. and l.p. rots.} &\hbox{{\bf 1}: gl. rot.} &\hbox{{\bf 2}: gl. and l.p. rot.} \cr
\omit\vbox to 0pt{\vss\hbox{ rotations about $z$ }\vss}&\multispan4\cr
& \hbox{for $U$ and $V$ indep.} & \hbox{for $U$ and $V$ indep.}&&\cr\ln
%
& \hbox{{\bf 4}: gl. and l.p.} &  & \hbox{{\bf 2}: gl. and l.p.}  & \cr
\hbox{shifts in }z	& \hbox{shifts in }z &-& \hbox{shift in }z&-\cr
& \hbox{for $U$ and $V$ indep.} &&&\cr\bln
}

\subsection[al constr]{Constraints}

In the previous section we have shown that the $\tilde\mat S$ is singular, which expresses the fact that some alignment modes are inaccessible to the track-based alignment. To accomplish the alignment task -- to solve the \Eq{al fit equation} -- one thus needs to supply the information about the inaccessible (singular) modes from another source. This information may be formulated in a form of additional \em{constraints} (to the alignment solution $\hat\vec\ch$) which compensates the singularity of the $\tilde\mat S$ matrix. We will consider a set of linear constraints

\eqref{
	\mat C^\T \hat\vec\ch = \vec V\ ,\qquad
	\mat C = \pmatrix{
		\vdots\cr
		\vec c_i^\T\cr
		\vdots\cr
	}\ ,\qquad
	\vec V = \pmatrix{
		\vdots\cr
		v_i\cr
		\vdots\cr
	}
\ ,}{al constraints}
where $\vec c_i$ and $v_i$ are the constraint vectors and values ($i$ enumerates the constraints). The constraints require that the ``projection'' of the solution $\hat\vec\ch$ to the ``direction'' $\vec c_i$ is $v_i$. We have used the quotes since we do not require the vectors $\vec c_i$ to be of unit size. Still we find this interpretation of the constraints quite intuitive.

The fit equation \Eq{al fit equation} has been derived by pursuing a least squares fit (simultaneously for misalignment and track parameters, see \Sc{al sim fit}). Now introducing the constraints, we have to deal with a constrained least squares problem. It can be addressed by the Lagrange-multipliers technique (see e.g.~\bref{Millepede}):
\eqref{
	\pmatrix{
		\tilde\mat S & \mat C \cr
		\mat C^\T & 0\cr
	}
	\pmatrix{
		\hat\vec\ch \cr
		\vec\La \cr
	}
	=
	\pmatrix{
		\tilde\vec T\cr
		\vec V\cr
	}
\ ,}{al alignment equation}
where $\vec\La$ is a vector of Lagrange multipliers. The bottom row expands to the constraints requirement \Eq{al constraints}. Expanding the upper row yields the fit equation \Eq{al fit equation} with a small modification:
\eqref{
	\tilde\mat S \hat\vec\ch = \vec T' \equiv \tilde\vec T - \mat C \vec\La\ .
}{al mod fit eq}
The presence of the Lagrange multiplier term can guarantee the existence of a solution for any $\tilde\vec T$.

The $\tilde\mat S$ matrix is symmetric and thus, according to theorem 7.3.1 in \bref{anton}, it can be diagonalized by $\tilde\mat S = \mat Q \mat D \mat Q^\T$. The columns of the $\mat Q$ matrix form an orthonormal set of eigenvectors of $\tilde\mat S$, with the corresponding eigenvalues on the diagonal of the $\mat D$ matrix (which is thus diagonal). Moreover we can order the eigenvectors such that the first columns in $\mat Q$ correspond to zero eigenvalue. We will denote this part $\mat Q_{\rm s}$, the other eigenvectors form the $\mat Q_{\rm r}$ sub-matrix. Denoting $\mat L$ the matrix containing all non-zero eigenvalues on the diagonal, one can write
\eqref{\tilde\mat S = (\mat Q_{\rm s}, \mat Q_{\rm r})
	\pmatrix{0& 0\cr 0& \mat L\cr}
	\pmatrix{\mat Q_{\rm s}^\T\cr \mat Q_{\rm r}^\T}\ ,
}{al S diag}
Applying this decomposition to \Eq{al mod fit eq} yields
\eqref{
	\pmatrix{0 & 0 \cr 0 & \mat L\cr} \pmatrix{\mat Q_{\rm s}^\T\cr \mat Q_{\rm r}^\T} \hat\vec\ch
	= \pmatrix{\mat Q_{\rm s}^\T\cr \mat Q_{\rm r}^\T} \vec T'
}{al mod fit eq eig}
Since the upper row is identically zero on the \lhs, so must be on the \rhs., thereby
\eqref{
	\mat Q_{\rm s}^\T \tilde\vec T = \mat Q_{\rm s}^\T \mat C\vec\La\ .
}{al T no sing cont}
Note that it would not be possible to fulfill this condition without the Lagrange multipliers. In order to find a vector $\vec\La$ for any vector $\tilde\vec T$, one gets a requirement on the set of constraints:
\eqref{\mat Q_{\rm s}^\T \mat C\hbox{ must be regular}\ .}{al ETC reg}
Then
\eqref{\vec\La = (\mat Q_{\rm s}^\T \mat C)^{-1} \mat Q_{\rm s}^\T \tilde\vec T\ .}{al cnst La}

The \Eq{al mod fit eq eig} sets no requirements on $\mat Q_{s}^\T \hat\vec\ch \equiv \vec\ch_{\rm s}$, but it requires $\mat Q_{\rm r}^\T \hat\vec\ch = \mat L^{-1} \mat Q_{\rm r}^\T \vec T'$. Therefor every solution of \Eq{al mod fit eq eig} can be written
\eqref{\hat\vec\ch =  \mat Q_{\rm r} \mat L^{-1} \mat Q_{\rm r}^\T \vec T' + \mat Q_{\rm s} \vec\ch_{\rm s}\ ,}{al sol param}
which corresponds to the standard form with a particular solution (first term) and a parameterization of solutions of the homogeneous equation $\tilde\mat S \vec\ch_{\rm s} = 0$ (second term), see theorem 5.5.2~in \bref{anton}. The matrix in the first term could be interpreted as the inversion of the non-singular part of the $\tilde\mat S$ matrix.

Another requirement we want to imply on the set of constraints $\mat C$ is to determine the solution $\hat\vec\ch$ uniquely. By inserting the above parameterization to \Eq{al constraints}, one finds that the requirement is equivalent to asking $\mat C^\T \mat Q_{\rm s}$ to be regular. This is further equivalent to the previous requirement \Eq{al ETC reg}. Then, the homogeneous-equation solution $\vec\ch_{\rm s}$ is fixed to
\eqref{\vec\ch_{\rm s} = (\mat C^\T \mat Q_{\rm s})^{-1} (\vec V - \vec C^\T \mat Q_{\rm r} \mat L^{-1} \mat Q_{\rm r}^\T \vec T')\ .}{al cnst chi s}

Putting altogether yields
\eqref{\hat\vec\ch =
	\mat P^\T \mat Q_{\rm r} \mat L^{-1} \mat Q_{\rm r}^\T \mat P \tilde\vec T
	+ \mat Q_{\rm s} (\mat C^\T \mat Q_{\rm s})^{-1} \vec V\ ,\qquad
	\mat P = 1 - \mat C (\mat Q_{\rm s}^\T \mat C)^{-1} \mat Q_{\rm s}^\T\ .
}{al cnst sol}
It will be the starting point for discussing the alignment uncertainty in \Sc{al err}.

\iffalse
If we neglect the experimental errors that enter the $\tilde\vec T$ vector, it can be calculated from \Eq{al exact fit equation}: $\tilde\vec T = \bar\mat S \vec\ch$. We recall that $\vec\ch$ is the vector of true (not estimated) residuals. As the singular modes of $\tilde\mat S$ and $\bar\mat S$ are the same, one finds that $\bar\mat S E = 0$ too. Consequently, \Eq{al T no sing cont} leads to $\La = 0$. If the experimental errors are not neglected, the relation becomes only approximative: $\La \approx 0$. This provides a consistency check that may be used during real data analysis.
\fi

So far we have focused on constraining the singular modes. Looking back to \Eq{al fit eq eigen one} we remind that there might be also weak modes, which would still allow for a solution, however they would severely deteriorate the accuracy. In such a situation, one might want to constrain some of the weak modes too. In principle, the alignment equation \Eq{al alignment equation} still may be used. But since the number of constraints and the number of singular modes is different now, the results \Eq{al ETC reg} and onwards can not be applied. A discussion on a general level would be complicated, let us focus on a practical case instead.

As already mentioned, singular modes look like weak to any computer program (due to limited calculation accuracy their eigenvalues are not exactly zero). In such a situation, the $\tilde\mat S$ matrix is not singular, but has a number of weak modes that need to be constrained. Then two important results can be derived for the solution of \Eq{al alignment equation} \footnote{%
The derivation is lengthy, here we just sketch the main steps. Since $\tilde\mat S$ is regular now, the solution of \Eq{al alignment equation} can be written
$$
	\hat\vec\ch = \tilde\mat S^{-1} \big[ 1 - \mat C \mat M \mat C^\T \tilde\mat S^{-1} \big]
		+ \tilde\mat S^{-1} \mat C \mat M \vec V\ ,\qquad
	\mat M = (\mat C^\T \tilde\mat S^{-1} \mat C)^{-1}\ .
$$
In order to evaluate the $\mat M$ matrix, one may employ a diagonalization similar to \Eq{al S diag}. With the only exception that $\tilde\mat S$ has weak and regular modes now: $\mat Q = (\mat Q_{\rm w}, \mat Q_{\rm r})$ and $\mat D = \diag(\mat l, \mat L)$. The matrices $\mat l$ and $\mat L$ are diagonal and contain the weak and regular eigenvalues respectively. Using the diagonalization, one finds that $\mat M^{-1}$ receives two contributions -- from weak and regular modes:
$$ \mat M = \big[
	(\mat Q_{\rm w}^\T \mat C)^\T \mat l^{-1} (\mat Q_{\rm w}^\T \mat C)
	+ (\mat Q_{\rm r}^\T \mat C)^\T \mat L^{-1} (\mat Q_{\rm r}^\T \mat C)
\big]^{-1}\ .$$
Since $\mat l$ contains the weak eigenvalues, which are small, the first term is dominating. To perform the inversion, one may use the recurrent relation:
$$(\mat A+\mat B)^{-1} = \mat A^{-1} - \mat A^{-1} \mat B \mat A^{-1} + \mat A^{-1} \mat B \mat A^{-1} \mat B \mat A^{-1} \ldots\ ,$$
identifying $\mat A$ with the first term. Since each $\mat A$ contains a $\mat l$ factor, the above expansion represents a sort of power series. Then it is just a matter of some algebra to obtain this type of power expansion for the solution $\hat\vec\ch$. Provided that $\mat Q_{\rm w}^\T \mat C$ is regular, the expansion involves non-negative powers of $\mat l$ only. Thus, it is finite in the limit $\mat l\to 0$. Moreover, in this limit, the solution gains the form of \Eq{al cnst sol}.
}.
First, the weak modes are constrained if
\eqref{\mat Q_{\rm w}^\T \mat C \hbox{ is regular} \ .}{al WTC reg}
Similarly to the definition of $\mat Q_{\rm s}$, the $\mat Q_{\rm w}$ matrix contains the weak modes as columns.
Second, in the limit where the weak modes become singular (their eigenvalues tend to zero), the solution becomes equivalent to \Eq{al cnst sol}. Hence there is no difference between extremely weak and singular modes. This is important since computer programs treat singular modes as weak.

We have seen that a set of constraints must fulfill certain conditions, see \Eq{al ETC reg,al WTC reg}, but one has still a lot of freedom in their choice. Let us list a few common options.

\> Probably the most natural choice is to take the singular modes as the constraints, that is $\mat C = \mat Q_{\rm s}$. We will refer to this option as \em{homogeneous constraints} (the role of all detectors is equal, in contrary to the next options). In principle, this set of constraints may be extended by the weak modes, then $\mat C = (\mat Q_{\rm s}, \mat Q_{\rm w})$.
\par\parindent\itindent\indent\hang As an example, we write explicitly the constraint matrix $\mat C$ for the read-out-direction shifts only. Taking the singular modes from \Eq{al sm shr sol}, the matrix can read
\eqref{\mat C = \pmatrix{
	\vdots		& \vdots		& \vdots	& \vdots \cr
	d_{ix} z_i	& d_{iy} z_i	& d_{ix}	& d_{iy} \cr
	\vdots		& \vdots		& \vdots	& \vdots \cr
}\ ,\qquad i\hbox{ going through all detectors.}}{al hom cnst ex}

\> Another natural choice is to select a subset of reference sensors, fix their positions and let the other sensors align with respect to the reference ones. We will call this option \em{fixed-sensors constraints}. Both singular and weak modes can be constrained in this way.
\par\parindent\itindent\indent\hang For an example, we consider again the case with read-out-direction shifts only. There we face four singular modes, thus we need to fix the positions of four detectors, let us take first two and last two:
\eqref{\mat C = \pmatrix{
	1		& 0			& 0 		& 0 		\cr
	0		& 1			& 0 		& 0 		\cr
	\vdots	& \vdots	& \vdots	& \vdots	\cr
	0		& 0			& 1 		& 0 		\cr
	0		& 0			& 0 		& 1 		\cr
}\ .}{al fix-det cnst ex}

\> The last option is called \em{final constraints} since it has been used for the final alignment analysis (and the elastic scattering analysis presented in \Sc{felm}). In the \Sc{al mc tests} we will show the shifts in $z$ can not be resolved with a satisfactory precision (see \Fg{al err shz theta}). That is why they have been dropped from the alignment procedure. As shown in \Fg{al err rotz rho,al err rotz theta}, the rotation between $U$ and $V$ sensors and between far and near units can only be determined with a large uncertainty. It is thus necessary to constrain these two additional weak modes, increasing the number of rotation constraints to four (cf. \Tb{al sing mode overview}). The final constraints require zero mean rotation $\De\rh$, separately for $U$ and $V$ sensors and separately for each unit.
\par\parindent\itindent\indent\hang
The read-out-direction shift constraints are similar to those from the homogeneous-constraints set. With the exception that the horizontal pots are not included in the constraints. The reason is that we trust the beam position determination by the collimation alignment (see \Sc{al collim}) and therefore we want to preserve the vertical center between the top and bottom pots. This is achieved by asking no mean shift of the top and bottom pot.
\par\parindent\itindent\indent\hang
To summarize, for every unit, these four constraints are imposed:
\eqref{\eqnarray{
\hbox{shifts:}\qquad && \sum_{i\ \in\ \hbox{top, bottom}} \De s_i \, d_{x_i} = \sum_{i\ \in\ \hbox{top, bottom}} \De s_i \, d_{y_i} = 0\ ,\cr
\hbox{rotations:}\qquad && \sum_{i\ \in\ \hbox{U detectors}} \De \rh_i = \sum_{i\ \in\ \hbox{V detectors}} \De \rh_i = 0\ .
}}{al final constraints}

\iffalse
\eqref{
	C = \pmatrix{
		\vdots	&\vdots	&		&		& \cr
		d_{ix}	&d_{iy}	&		&		& \cr
		\vdots	&\vdots	&		&		& \cr
				&		&\vdots	&\vdots	& \cr
				&		&d_{ix}	&d_{iy}	& \cr
				&		&\vdots	&\vdots	& \cr\ln
				&		&		&		&1		&0		& 		& 	\cr
				&		&		&		&0		&1		& 		& 	\cr
				&		&		&		&1		&0		& 		& 	\cr
				&		&		&		&0		&1		& 		& 	\cr
				&		&		&		&\vdots	&\vdots	& 		& 	\cr
				&		&		&		&		&		&1		&0		\cr
				&		&		&		&		&		&0		&1		\cr
				&		&		&		&		&		&1		&0		\cr
				&		&		&		&		&		&0		&1		\cr
				&		&		&		&		&		&\vdots	&\vdots	\cr
	}
}{al final cnst ex}
\fi

As the last comment, let us recall that the $\mat S$ scales approximately with the number of events $N_{\rm events}$ (see \Eq{al S norm eig val}). In order to simplify the interpretation of the eigenvalues of the alignment matrix (the \lhs~of \Eq{al alignment equation}), it is advantageous to let the $\mat C$ matrix scale with $N_{\rm events}$ too. This effectively means to use
\eqref{\mat C = N_{\rm events}\, \mat C^0\ ,}{al C scaled}
where $\mat C^0$ is one of the constraint choices above.


\subsection[al err]{Errors}

In the previous section we have derived \Eq{al alignment equation} that solves the alignment task. In this section we will focus on the uncertainties of the alignment solution.

First of all, let us remark that any error that might appear can be of few origins only: the \em{approximations} used to derive the alignment equations and \em{neglected effects} (e.g. multiple scattering) and \em{data corruption} (electronic noise, problems in the DAQ chain etc.). One can check the presence of problems of the latter type by dividing the sample into a number of subsamples. Were the results obtained from the subsamples not compatible with each other, these effects would be important (however, in \Sc{al lhc res} we will see it is not the case). Now, let us review the approximations we have made.

\> We have neglected the \em{resolution error} (formally the $\De m$ term in \Eq{al measurement}). According to the discussion in \Sc{al psi}, it is reasonable to assume that ($P$ denotes the pitch of the sensors)
\eqref{\eqnarray{
	&\bullet \hbox{ the resolution error is a random variable uniformly distributed on } (-P/2, +P/2) \hbox{ and}\cr
	&\bullet \hbox{ the resolution errors in different detectors are independent.}\cr
}}{al resolution err mod}
Given the low angles of the LHC tracks, the second point might look doubtful. But we will show (in \Fg{al stat fixDet} for instance) that it leads to satisfactory results though.

\par\parindent\itindent\indent\hang
An important property of the resolution error is that the errors in different events are independent. If one doubles the sample, the errors in the second half will be independent of those in the first one. Consequently, one may expect better results with increasing sample size. As rule of thumb, the error should be proportional to $1/\sqrt{N_{\rm events}}$ where $N_{\rm events}$ is the sample size. That is why the pitch rounding error is of \em{statistical nature}.

\> In deriving \Eq{al effective measurement} we have neglected a number of terms. Most notably all \em{terms non-linear in the alignment corrections}. Since the corrections are small, the non-linear terms lead only small relative errors. For example for the rotation-matrix linearization, see \Eq{al small rotation approximation}, the order of the neglected terms over the retained terms is $10^{-5}$. Following the order estimates from \Sc{al psi} we can conclude that the error due to dropping the non-linear terms is negligible.

\> In the definition of the $\hat\mat\Ga$ matrices, we have replaced the true track parameters by the track fits neglecting the misalignments (see \Eq{al tau linearization}). Thus the fitted \em{track parameters are biased} by the presence of misalignments. In contrary to the resolution error, here it makes sense expecting the error to be independent of the sample size. In other words, this is a  \em{systematic error}.

\vskip\itskip

In \Sc{al psi} when describing the interaction of a proton with a silicon sensor, we wrote the proton creates electron-hole pairs along its trajectory. But in addition the proton can interact with silicon nuclei, leading to an angular deflection. This phenomenon is often called \em{multiple scattering}. In Sc.~7.5.2 in \bref{hubert} the size of this effect was evaluated for the TOTEM RPs. The result for a single RP is that the angular deflection is a random variable with zero mean and standard deviation about $0.55\un{\mu rad}$. With the lever-arm given by the length of a $220\un{m}$ station (the worst situation), it leads to an elongation about $3\un{\mu m}$. Since the multiple scattering is a random process, the resulting error would add up with the resolution error. From \Eq{al resolution err mod} we get the standard deviation $19.05\un{\mu m}$, when the multiple scattering is added it increases to $19.29\un{\mu m}$. The multiple scattering is thus a minor effect for the alignment application.

Now let us rewrite some of the above statements a bit more formally. The $\tilde\vec T$ vector that appears on the \rhs.~of the alignment equation \Eq{al alignment equation} can be calculated from \Eq{al exact fit equation}. Considering that the measurements $\vec\mu$ gain a contribution $\De\vec\mu$ from the resolution error, we can write
\eqref{\tilde\vec T =
	\tilde\mat S \vec\ch
	+ \underbrace{(\bar\mat S - \tilde\mat S) \vec\ch}_{\De\vec T_{\rm syst}}
	+ \De\vec T_{\rm stat}
\ .}{al vec T error}
The $\De\vec T_{\rm stat}$ is induced by the resolution error ($\De\vec\mu$) and corresponds to the statistical error component. The $\De\vec T_{\rm syst}$ reflects the difference between the $\mat S$ matrices with bar and tilde which differ only by using true or fitted track parameters (see \Eq{al fit equation,al exact fit equation}). It therefore corresponds to the systematic error due to the biased track parameters. By construction it is proportional to the true alignments $\vec\ch$, that is the difference between the true and thought geometry. Note also that it receives no contribution from the singular-mode components of $\vec\ch$.

This can be inserted into \Eq{al cnst sol} which describes the solution of the alignment task. It yields a long expression
\eqref{\hat\vec\ch =
	\underbrace{\mat Q_{\rm r} \mat Q_{\rm r}^\T }_{\mat P_{\rm r}} \vec \ch
	+ \underbrace{\mat Q_{\rm s} (\mat C^\T \mat Q_{\rm s})^{-1} \vec V}_{\vec\ch_{\rm C}}
	+ \underbrace{\mat P^\T \mat Q_{\rm r} \mat L^{-1} \mat Q_{\rm r}^\T \mat P \De\vec T_{\rm syst}}_{\De\vec\ch_{\rm syst}}
	+ \underbrace{\mat P^\T \mat Q_{\rm r} \mat L^{-1} \mat Q_{\rm r}^\T \mat P \De\vec T_{\rm stat}}_{\De\vec\ch_{\rm stat}}
}{al sol err}
with an easy interpretation. The first term is the projection of the true misalignments to the regular modes. The second term is what is added by the constraints. The last two terms correspond to the systematic and statistical errors of the alignment solution $\hat\vec\ch$. We have started with \Eq{al cnst sol} which is only valid when no weak modes are constrained. If they are, the result differs slightly from \Eq{al cnst sol}. However, the difference is the smaller the weaker the weak modes are. In any case, the difference can be included in the systematic error $\De\ch_{\rm syst}$.

Using the fitted track parameters in the $\Ga$ matrices (see \Eq{al tau linearization}) is unavoidable. We can only hope that the induced (systematic) error is not too large and applying the alignment corrections (as resolved from \Eq{al alignment equation}) will improve our knowledge about the geometry. If so, one may perform several alignment iterations. In the second iteration we would have a better knowledge of the geometry, thus the track fits would be closer to the true track parameters, consequently there would be less bias and one might expect more accurate alignment results.

This iteration process is sketched in \Tb{al iter} in a greater detail. In the beginning we think the geometry is $g_0$ while the true one is $g$. In the first iteration we would like to obtain alignment corrections $\hat\vec\ch = g - g_0$. This is not possible for two reasons: the presence of singular and weak modes and the alignment errors. The expectation for the alignment correction is shown in the right-most column. It comes from \Eq{al sol err}, in addition we have written explicitly the dependence of the systematical error $\De\vec\ch_{\rm syst}$ on $g - g_0$, that is ``how far we are from the true geometry.'' During the iteration this ``distance'' is reducing and so does the systematic error. On contrary, the statistical errors is almost constant -- the resolution error $\De m$ does not change, only the $\tilde\mat\Ga$ are altered slightly. The iterations stop when no further improvement can be made, formally $|\hat\vec\ch^n|$ is smaller than a certain limit.

\tab[\strut\quad\hfil#\qquad&#\hfil\qquad&#\hfil\quad\cr]{al iter}{A scheme of alignment iterations with the error evolution.}{\ln
	& geometry & alignment correction \cr\ln
	& $g_0$ \cr
iteration $1$ & $\downarrow$ & $\hat\vec\ch_1 = \mat P_{\rm r} (g - g_0) + \vec\ch_{\rm C} + \De\vec\ch_{\rm syst}(g - g_0) + \De\vec\ch_{\rm stat}$ \cr
	& $g_1 = g_0 + \hat\vec\ch_1$ &   \cr
iteration $2$ & $\downarrow$ & $\hat\vec\ch_2 = \mat P_{\rm r} (g - g_1) + \vec\ch_{\rm C} + \De\vec\ch_{\rm syst}(g - g_1) + \De\vec\ch_{\rm stat}$ \cr
	& $g_2 = g_1 + \hat\vec\ch_2$ &   \cr
	& $\vdots$	\cr
last iteration $n$ & $\downarrow$ & $|\hat\vec\ch_n| < \hbox{limit}$ \cr
	& $g_n = g_{n-1} + \hat\vec\ch_n$ &  \cr\ln
}

After the iteration process has converged, the final error still have two components -- the statistical and the systematic. The latter one is proportional to the distance to the actual geometry. The distance can not vanish because of two reasons -- the statistical error and the constraints (in case the constraints are not compatible with the true geometry). Let us recall that the singular modes do not contribute to the systematic error, see \Eq{al vec T error}. Hence only the constrained weak modes can lead to an error of this type (they are both constrained and non-singular). For the RP sensors this is the case of $U$-$V$ and far-near unit rotations, they will be studied in the next section, see \Fg{al syst err uv rot,al syst err fn rot}.

As the last comment related to the iterations, let us mention the typical number of steps. Both in MC tests (\Sc{al mc tests}) and real data analyses (\Sc{al int res,al lhc res}), two iterations have turned out to be sufficient. That is, the correction in the third one has already been negligible.

Now, let us estimate the size of the statistical error $\De\ch_{\rm stat}$. It originates in the resolution error $\De m$, thus one may take the model \Eq{al resolution err mod} and propagate the error to the alignment result $\hat\vec\ch$. For the error propagation one may use formula (4.19) from \bref{barlow}:
\eqref{\vec y = \mat M\vec x \qquad \Rightarrow \qquad \Var\vec y = \mat M\ \Var\vec x\ \mat M^\T\ .}{error prop}
In \Eq{al mat V} we have used the symbol $\mat V$ for the variance matrix of the sensor measurements. Since in our model the only random variables are the resolutions errors, $\mat V$ corresponds to their variance matrix. Then, from the definition of $\tilde\vec T$ \Eq{al fit equation} and following the properties of the $\si^n$ matrix \Eq{al prop sigma n} it is straightforward to see that $\Var \tilde\vec T = \tilde\mat S$. Propagating the error further through \Eq{al alignment equation}, one obtains the variance of the alignment result
\eqref{\Var \pmatrix{\tilde\vec\ch\cr \vec\La} = 
	\pmatrix{
		\tilde\mat S & \mat C \cr
		\mat C^\T & 0\cr
	}^{-1}
	\pmatrix{
		\tilde\mat S& 0\cr
		0 & 0}
	\pmatrix{
		\tilde\mat S & \mat C \cr
		\mat C^\T & 0\cr
	}^{-1}\ .
}{al stat err}
The standard deviation of the $\hat\vec\ch$ vector components can be found on the diagonal of the matrix above:
\eqref{\si(\hat\ch_i) = \sqrt{\Var \pmatrix{\hat\vec\ch\cr  \vec\La}_{i\,i}}}{al stat err chi}



\subsection[al mc tests]{Monte-Carlo tests}

In the previous sections we have developed a method of track-based alignment and have derived some of its characteristics (inaccessible alignment modes, error behavior, etc.). Before applying the method to real LHC data, we have made a handful of MC tests to validate our theoretical findings.

Using the full simulation chain including Geant4 (see \Fg{sr sw structure}) is very resource demanding and severely limits the number of simulations that can be done in a reasonable time. For our alignment studies we have used a fast simulation (see \Sc{fast simu}) instead. The fast simulation skips the detailed simulations of charge creation and collection and electronic response, it uses an effective model instead. The neglected effects shall be much smaller that the resolution error, which is the leading statistical error that we consider for the alignment (see \Eq{al resolution err mod}). Therefore within this error, we expect the fast simulation to provide a good tool for alignment studies.

All our simulations have been done with the $220\un{m}$ station in sector 56. We have tried a number of track distributions, but the impact on the alignment results has been small, as it should. For the figures in this section, we have used a track distribution which resembles the one observed in the LHC data -- both track slopes $a_x$ and $a_y$ generated according to a normal distribution with zero mean and standard deviation $\si_a$, the intercepts $b_x$ and $b_y$ (at $z=217\un{m}$) according to normal distributions with zero means and standard deviations $\si_x = 6\un{mm}$ and $\si_y= 8\un{mm}$.

Our first MC tests have been to verify the statistical behavior of the alignment results. In particular how the statistical and systematic errors depend on the size of a track sample used. For each sample size we have generated 20 samples (with different random seeds). For each sample we have made two simulations, each followed by an alignment analysis. For the \em{realistic simulation}, the sensor hit positions have been rounded to the nearest (inter-)strip position (see \Sc{fast simu}) and fitted track parameters (that is $\hat\mat\Ga$) have been used. This corresponds to the alignment application to real data. The second simulation has been made to get a reference alignment result with no errors included. For this \em{reference simulation} the sensor hit positions have not been rounded and true track parameters (that is $\mat\Ga$) have been used. The alignment analyses have been performed in three iterations, yielding a result and its uncertainty estimate. Then, from the 20 samples of the same size we have calculated the following quantities.
\eqref{\vbox{\halign{\quad\strut#\qquad&#\quad\cr\ln
quantity & label\cr\ln
\hbox{mean (realistic result - reference result)}	& \hbox{systematic error} \cr
\hbox{standard deviation (realistic result - reference result)}& \hbox{statistical error} \cr
\hbox{mean (realistic uncertainty)} 					& \hbox{estimated uncertainty} \cr\ln
}}}{al stat quan def}
Two examples of such statistical studies can be found in \Fg{al stat fixDet} (with a set of fixed-sensors constraints) and \Fg{al stat final} (with the final constraints).

For \Fg{al stat fixDet} the rotations have been fixed only in the near unit. Following our discussion of weak modes in \Sc{al sing modes} (see especially \Fg{al eig theta}), we may expect high alignment uncertainties for the far unit. The figure confirms this expectation (compare top and bottom two rows). Let us make three more observations. First, the systematic error is compatible with zero within the its uncertainty (semi-transparent band). Second, the estimated uncertainty falls as $1/\sqrt{N_{\rm events}}$, which confirms our rule of thumb mentioned below \Eq{al resolution err mod}. Third, the ratio of the statistical error to the estimated uncertainty is almost flat, but below the expected value of one. This means that the uncertainty estimated with the help of the model \Er{al resolution err mod} is slightly overestimated. This might, in fact, partly compensate our neglecting the multiple scattering.

The same three observations could be made for \Fg{al stat final}, which corresponds to the final constraints. These fix the far-near unit rotation weak mode, thereby the alignment uncertainties are similar for both near and far unit.

\fig{fig/pdf/al_stat_fixDet.pdf}{al stat fixDet}{A statistical study of the alignment method with a set of fixed-sensors constraints. For the read-out-direction shifts we have fixed the first two sensors in the top-near and the last two sensors in the top-far pot. Regarding the rotations, the first two sensors have been fixed. The simulations have been performed with a realistic track slope spread $\si_a = 0.1\un{mrad}$. Each curve corresponds to the third sensor of a pot, see the legend. The near/far unit plots are at the top/bottom of the figure. The semi-transparent areas show $1\si$ error bands. The vertical dotted line marks a typical number of events in LHC runs.
%[20 repetitions, geometry 2.7x3.3, misalignment rotz4, gauss6,8, 0.1mrad, 3 iterations]
}

\fig{fig/pdf/al_stat_final.pdf}{al stat final}{A statistical study of the alignment method with the final constraints. The simulations have been performed with a realistic track slope spread $\si_a = 0.1\un{mrad}$. Each curve corresponds to the third plane of a RP, see the legend. The semi-transparent areas represent $1\si$ error bands. The vertical dotted lines mark the typical number of tracks in LHC runs.
%(20 repetitions), geometry 2.7x3.3, misalignment rotz4, gauss6,8, 0.1mrad.
}

In the above figures we have not showed any plots for the $z$ shifts. The reason is their quite special behavior. We recall that the $\ga$ coefficients related to the shifts in $z$ are proportional to the track angle (see \Tb{al alignment quantities}). Thus if the variation of the track angles is too small, one can expect a bad resolution. This fact is documented in \Fg{al err shz theta}. For the LHC-typical angular spread (see \Tb{al lhc datasets}), the uncertainty is about $1\un{m}$, which is much worse than the misalignment expectation, see \Tb{al exp misal}. Since the track-based alignment can not improve our knowledge about the detectors' $z$-positions, we will focus on read-out-direction shifts and rotations only it in what follows.

\fig{fig/pdf/al_err_shz_theta.pdf}{al err shz theta}{The estimated uncertainty of $z$ shifts as a function of the track angular spread $\si_a$. Simulations performed with $10^5$ events and a set of fixed-sensors constraints (for $z$ shifts: first two sensors in near-top and last two sensors in far-top pot). For each pot, the uncertainty of 2nd or 3rd sensor (according to its $U$ or $V$ orientation) is shown, for the other sensors the situation is very similar. The vertical dotted line shows the typical track angular spread for LHC data.
%No misalignment, fix-ext constraints (rotz in 1200, 1201; shz in 1200, 1201, 1248 and 1249), extfit=f.
}

Let us have a more detailed look at the weak modes now. From \Sc{al sing modes} we know there are two of them for TOTEM RPs: the rotation between $U$ and $V$ sensors and the rotation between far and near units. The former is, in fact, a singular mode for the nominal geometry (which comprises two read-out groups only), see \Tb{al sing mode overview}. However, because of rotation misalignments, the true geometry is not a two-group one and consequently the singular mode becomes weak only, see the transition in \Fg{al eig rho} (middle row).

A similar transition is shown in \Fg{al err rotz rho}. Again, a series of geometries has been considered, starting with the nominal one and adding rotation misalignments, randomly generated according to a normal distribution with zero mean and standard deviation $\si_\rh$ (on the horizontal axis). For this plot, we have fixed the rotation of one sensor only: the first one in near-top RP. It is a $V$ sensor. Consequently, the alignment uncertainties for all $V$ sensors (left plot) are practically $\si_\rh$-independent. In contrary for the $U$ sensors (right plot), the uncertainties grow with decreasing $\si_\rh$, as expected. The uncertainties for the far unit pots are higher than for the near unit (we recall that the fixed sensor is in the near one). This is related to the other weak mode (far-near unit rotation), which will be discussed later, in \Fg{al err rotz theta}. Focusing on the near unit, we see that for the expected misalignments (the dotted vertical line) the uncertainty for $V$ sensors is between $10^{-2}$ and $10^{-1}\un{mrad}$, while for the $U$ sensors it grows over $1\un{mrad}$. For this plot a sample of $10^5$ tracks has been used, it is a typical number for the LHC runs. Now, comparing the results to the expected misalignments in \Tb{al exp misal}, drawn as the dotted horizontal line, we conclude that the track-based alignment can not provide reasonable corrections to the $U$ sensors' rotations. We recall that the $U$ sensors turn out to be problematic in this example as we have fixed the rotation of the $V$ sensors by the constraints. Generally, the troublesome mode is the relative rotation between $U$ and $V$ sensors.

\fig{fig/pdf/al_err_rotz_rho.pdf}{al err rotz rho}{The estimated uncertainty of the rotations about $z$ as a function of geometry. The geometries considered have been derived from the nominal one by applying rotation misalignments to each sensor. These misalignments have been randomly generated according to a normal distribution with zero mean and variance $\si_\rh^2$ (the quantity on the horizontal axis). The expected misalignment is marked by the vertical dotted line. The simulations have been done with $10^5$ events, track angular spread $\si_a = 0.1\un{mrad}$ and a set of fixed-sensors constraints. The only sensor with fixed rotation has been the first one in the top-near RP (it is a $V$ sensor). Each line corresponds to the uncertainty of the 2nd or 3rd sensor (according to its $U$ or $V$ orientation) of a RP. Solid lines have been used for RPs in the near, dashed for RPs in the far unit. The lines for far vertical pots are overlapping. The horizontal dotted line shows the expected rotation misalignment.
%No misalignment (misaligned and real geometry contain detectors rotated with the given rho distribution).  Extfit=false. fix-bas, 1 iteration, 1 repetition
}

The other weak mode is related to the rotation between far and near units. In fact we have derived in \Eq{al sm rotz sol la} is a linearly progressive rotation. But looking at the structure of $220\un{m}$ stations (see \Tb{ttm rp station}), we see that the RPs cluster into two units -- each about $45\un{cm}$ in length and separated by about $5.5\un{m}$. Since their distance is much higher than their size, the linearly progressive rotation effectively reduces to a rotation of the near unit and a rotation of the far unit. Moreover, the global rotation must be constrained (see the singular mode \Eq{al sm rotz sol}), thus what is left for the weak mode is the relative far-near unit rotation.

As indicated by \Fg{al eig theta}, the weakness of this mode depends on the angular spread $\si_a$ of the tracks used. Since the weakness is related to the alignment uncertainty, see \Eq{al fit eq eigen one}, one may expect growing uncertainty with reducing angular spread (and increasing weakness of the weak mode). This trend is well confirmed by \Fg{al err rotz theta}. For that figure we have fixed the rotations of the first two sensors in the near-top pot. The uncertainties of near top and bottom pot (having the same $z$ position) are practically flat. The near horizontal pot, being about $45\un{cm}$ from the constrained sensors, already feels some dependence on the angular spread. For the pots in the far unit the effect is even stronger. Considering a realistic value of the angular spread $\si_a =  0.1\un{mrad}$ and a typical number of tracks $10^5$ (see \Tb{al lhc datasets}), the uncertainty is $10^{-1}$ to $10^{-2}\un{mrad}$ for near and about $1\un{mrad}$ for far unit RPs. This is to be compared with the expected misalignments from \Tb{al exp misal} (drawn as the horizontal dotted line). One may conclude that the track-based alignment can hardly improve our knowledge on the far-near unit rotation.

Let us repeat the reason for the saturation at the high-$\si_a$ side of \Fg{al err rotz theta}. The maximal angle that can be detected by both units simultaneously is about $10^{-2}\un{rad}$, therefore even if the track sample includes higher angle tracks, they do not contribute to the alignment analysis.

\fig{fig/pdf/al_err_rotz_theta.pdf}{al err rotz theta}{The estimated uncertainty of the rotations about $z$ as a function of the track angular spread $\si_a$. The simulations have been performed with $10^5$ tracks and fixed-sensors constraints (rotations fixed for the first two sensors in the near-top pot). Each curve corresponds to a sensor (2nd or 3rd, depending on its $U$ or $V$ orientation) in a RP. The curves for near-unit pots are drawn solid, those for far-unit pots dashed. The vertical dotted line shows the realistic angular spread, the horizontal dotted line marks the expected rotation misalignment.
%No misalignment, extfit=f. , 1 iteration, 1 repetition
}

\Fg{al err rotz rho,al err rotz theta} tell us to constrain the $U$-$V$ and far-near rotations in order to get reasonable alignment results. However, in \Sc{al err} we have mentioned that constraining weak modes may lead to a systematic error (when the constraints are not compatible with the true misalignment). Although we have not observed any sizable effect of that sort in \Fg{al stat final} (the final constraints include both types weak modes), let us verify it explicitly.

To study the effect of the $U$-$V$ rotation we have considered a series of misalignment scenarios. They have contained no shifts in the read-out direction, the rotations have been generated randomly according to a normal distribution with a standard deviation of $1\un{mrad}$. For $V$ sensors the distribution have had zero mean, for $U$ sensors the mean has been set to $\rh_{U-V}$ (which therefore corresponds to the relative $U$-$V$ rotation). For each misalignment scenario we have made the realistic and reference simulations (defined above \Eq{al stat quan def}) with the final constraints. The difference in results (systematic error) is plotted in \Fg{al syst err uv rot}. One can conclude that for the expected $U$-$V$ rotation of the order $1\un{mrad}$, the systematic errors are negligible.

%One can see the errors are practically zero for $U$ detectors (solid lines), but non-zero for $V$ (detectors). This can be easily understood from the form of the $\ga$ coefficient for rotations (see \Tb{alignment quantities}) -- it contains the direction perpendicular to the read-out one. And in the nominal geometry, the direction perpendicular to $U$ is $\pm V$ and vice versa. Hence by neglecting (not determining) the rotations of $U$ detectors, one biases the $\ga$ coefficients for $V$ detectors (and gives rise to the related error).


\fig{fig/pdf/al_syst_err_uv_rot.pdf}{al syst err uv rot}{The systematic errors due the $U$-$V$ rotation $\rh_{U-V}$ (the final constraints fix it to zero). The simulations have been carried out with the typical track angular spread $\si_a = 0.1\un{mrad}$ and the final constraints. Each curve corresponds to a sensor in the near top RP (the errors in the other RPs have been only smaller). The curves corresponding to the $V$ sensors are dashed, to $U$ sensors solid.
%geometry 2.7x3.3, Jan(round=f, extfit=f) - Jan(f, t), 3 iterations, 1E4 events
}

A similar study has been made for the systematic error induced by constraining the far-near rotation. Again, the misalignment scenarios have had no read-out-direction shifts and the rotations two components. All sensors have had the random component, the far-unit sensors also a rotation $\rh_{\rm F-N}$ (which gives the rotation between the units). The resulting systematic errors are shown in \Fg{al syst err fn rot}. For the expected far-near rotations of order of few milliradians, the systematic errors are negligible.

\fig{fig/pdf/al_syst_err_fn_rot.pdf}{al syst err fn rot}{The systematic errors due to the far-near unit rotation $\rh_{\rm F-N}$ (the final constraints fix it to zero). The simulations have been done with a realistic track angular spread $\si_a = 0.1\un{mrad}$ and the final constraints. Each curve corresponds to a sensor in the far top RP (the errors in the other RPs have been only smaller). The curves corresponding to the $V$ sensors are dashed, to $U$ sensors solid.
%geometry 2.7x3.3, 3 iterations, Jan(round=f, extfit=f) - Jan(f, t), 1E4 events
}


\subsection[al data sel]{Input data selection}

In the above MC tests we have simulated one proton track per event. Thus, we have known a priory that all detector hits have belonged to the track of a given event. For real LHC data, the situation is more complicated. There might be a number of tracks per event, moreover different in every RP. There might be a number of hits that do not belong to any particle track -- noise, data corruption, etc. That is why it is necessary to select carefully the input for the track-based alignment. It is better to loose some statistics than to bias the results with false input.

The selection algorithm has two components: \em{hit selection} and \em{event selection}. In the first one, doubtful hits may be removed. In the latter, entire events may be dropped, if found suspicious..

The hit selection is integrated with track fitting and can be described as follows.
\bitm
\itm Input: the collection of all hits from all sensors (in all RPs).
\itm Local track fit -- a least squares fit with track parameterization \Eq{al local track}. This is the fit which determines the $\hat\vec a$ and $\hat\vec b$ (or shortly $\hat\vec\ta$) track parameters used in \Eq{al tau linearization}.
\itm Outlier removal. From the hit collection, we remove all hits for which
\eqref{m_i - \hat m_i > \pmt{maxResidualToSigma}\ \si(m_i)\ ,}{al outlier cond}
where $\hat m_i = \vec d_i \cdot (\hat\vec a^n z_i + \hat\vec b^n)$ is the track-fit interpolation and $\si(m_i)$ is the corresponding measurement uncertainty (i.e. a square root of the corresponding diagonal element of a $\mat V^n$ matrix, see \Eq{al mat V}). \pmt{maxResidualToSigma} is a parameter of the algorithm.
\itm Suspicious pot removal. We remove all hits from the RPs where there are less hits per projection than a given limit \pmt{minimumHitsPerProjectionPerRP}.
\itm If some points have been removed in steps 3 or 4, go back to step 2; stop otherwise.
\eitm

The event selection starts with a $\ch^2$ cut, that is all events with track fits the $\ch^2/\hbox{n.d.f.}$ of which exceeds a limit \pmt{chiSqPerNdfCut} are discarded. Then, there are several optional checks, see \Tb{al alg flags}. They can control which event types are used for the alignment.

\tab[\strut\hfil#\ &#\hfil\cr]{al alg flags}{The options of the event selection algorithm.}{\ln
parameter							& meaning\cr\ln
\pmt{removeImpossible} 				& remove events with signal in a top and a bottom pot simultaneously\cr
\pmt{requireBothUnits} 				& require signal in both units\cr
\pmt{requireOverlap} 				& require signal in the overlap between horizontal and vertical pots\cr
\pmt{requireAtLeast3PotsInOverlap}	& if there is signal in the overlap, require signal in at least three pots\cr\ln
}

The values of the selection algorithms parameters will be given for each analysis in \Sc{al int res,al lhc res}. Let us only remark that some of these parameters will be changed during iterations. For example the parameter of the outlier cut \Eq{al outlier cond} or the one of the $\ch^2$ cut will be made increasingly strict, as we will approach the expected alignment corrections (see the \rhs~column in \Tb{al iter}) decrease.


\subsection[al rp fac]{Alignment of detector packages}

As anticipated in the introduction, the elementary objects for the track-based alignment are the sensors. However, the sensors are stacked into rigid detector packages (see \Sc{ttm}) and therefore each sensor inherits the misalignment of its package. Moreover, the expected misalignments of sensors within a package are much smaller than the misalignments of packages within a station (see \Tb{al exp misal}). Thus it makes sense to extract the common misalignments of all sensors of a package and call them \em{detector-package (DP) misalignments}. The residual misalignments, that means the misalignments of a sensor with respect to its package, will be called \em{internal misalignments}.

It is straight-forward to introduce the DP misalignments in the formalism presented in \Sc{al psi}. One has to split the global (station) to local (sensor) coordinate transformation \Eq{al global to local} into two steps: station to DP and DP to sensor. This naturally defines the placement of a DP in a station and the placement of a sensor in a DP. Each placement may in reality differ from the nominal, in other words there may be an misalignment (c.f.~\Eq{al misalignments}). Then, repeating the steps leading to \Eq{al hit v} yields
\eqref{
	\De\rh_{z_i} = \De\rh_z^{\rm DP} + \De\rh_{z_i}^{\rm S}\ ,\qquad
	\De\vec c_i = \De\vec c^{\rm DP} + \De\vec c_i^{\rm S} - \pmatrix{\rh_y^{\rm DP}\cr \rh_x^{\rm DP}} \ze_i\ .
}{al dp sen al}
For the rotations about $z$, the interpretation is simple: the full rotation of the $i$-th sensor is given by a sum of the rotation of the DP within the station (DP rotation $\De\rh_z^{\rm DP}$) and the rotation of the sensor within the DP (internal rotation $\De\rh_z^{\rm S}$). The transverse-shift part can be interpreted similarly, with the exception of the last term, which reflects the fact that the entire DP is rotated about the $x$ and $y$ axes (see the illustration in \Fg{al rp misalignment}). The rotations angles $\rh_x^{\rm DP}$ and $\rh_y^{\rm DP}$ are defined by \Eq{al rotation parameterization}. The parameter $\ze_i$ gives the $z$ distance of the $i$-th sensor from the center of the DP (see \Tb{ttm det package}).

\fig{fig/pdf/al_rp_misalignment.pdf}{al rp misalignment}{A side view on a detector package (for simplicity only four sensors are drawn). The DP is shifted by $\De c^{\rm DP}$ (green arrow) and rotated by $\De\rh_x^{\rm DP}$ about the $x$ axis. The shifts of the individual sensors $\De\vec c_i$ are drawn as the blue arrows. For simplicity we have not considered the sensor shifts within the package, that is $\De\vec c_i^{\rm S} = 0$. The hollow/full dots mark the nominal/actual positions of sensor centers.}

As already mentioned, the internal misalignments are expected to be much smaller that the DP misalignments. This means that the $\De\rh_{z_i}^{\rm S}$ and $ \De\vec c_i^{\rm S}$ terms play a minor role in \Eq{al dp sen al}. Treating these terms as perturbations, one may apply the least squares method (see e.g.~Sec.~6.6 in \bref{barlow}) in order to estimate the DP misalignments. For the rotations about $z$, the fit equation may be written
\eqref{
	\De\rh_z^{\rm DP} = (\mat F^\T \mat V^{-1} \mat F)^{-1} \mat F^\T \mat V^{-1} \pmatrix{\vdots\cr \De\rh_i\cr\vdots}\ ,\qquad
	\mat F = \pmatrix{\vdots\cr 1\cr\vdots}\ .
}{al dp rotz}
The $(\ldots, \De\rh_i, \ldots)^\T$ vector groups the rotation alignment results (from \Eq{al alignment equation}) for all sensors of a given DP. The covariance matrix of this vector is denoted $V$. In fact, this formula corresponds to a weighted mean over the $\De\rh_i$ values. We have used this form as the same one can be used for the other DP alignment parameters.

The other two DP rotations and the transverse DP shift components can be extracted from the vector of the read-out-direction shifts:
\eqref{
	\pmatrix{c_x^{\rm DP}\cr c_y^{\rm DP}\cr \rh_x^{\rm DP}\cr \rh_y^{\rm DP}\cr} =
	(\mat F^\T \mat V^{-1} \mat F)^{-1} \mat F^\T \mat V^{-1}
	\pmatrix{\vdots\cr \De s_i\cr \vdots}\ ,\qquad
	\mat F = \pmatrix{
		\vdots & \vdots & \vdots & \vdots\cr
		d_{x_i} & d_{y_i} & -d_{y_i}\,\ze_i & -d_{x_i}\,\ze_i\cr
		\vdots & \vdots & \vdots & \vdots\cr
	}\ .
}{al dp shr}
Again, the index $i$ goes through all the sensors of a given package. The $\mat V$ symbol represents the covariance matrix of the $(\ldots, \De s _i, \ldots)^\T$ vector.

Two examples of extracting DP alignments can be found in \Fg{al comp rp all rot} and \Tb{al rp rot}.


\subsection[al int res]{Internal alignment results}

In this section we are going to compare the alignment of sensors within a package (the internal alignment) obtained from three sources. The first one is an optical metrology, the other two are track-based alignment analyses of two types of data -- beam test/cosmic rays and LHC runs.

\caption{Optical metrology}

The position of the sensors with respect to their hybrids was measured during the package assembly. A high-zoom camera was used to photograph each sensor on hits hybrid, which allowed to measure the positions of two sensor fiducial marks with respect to a reference point on the hybrid frame, see \Fg{al optical metrology}. Each coordinate was determined with a precision about $10\un{\mu m}$.

\fig{fig/pdf/al_optical_metrology.pdf}{al optical metrology}{An illustration of the optical metrology measurement. The two fiducial marks are drawn in blue, the reference mark on the hybrid frame in red. The numbers give the ideal measurements.}

From the four measured coordinates one can extract the (misalignment) shift and rotation of each sensor:
\eqref{
	\hbox{rotation} = {y_2 - y_1\over x_2 - x_1}\ ,\qquad
	x\hbox{ shift} = {x_1 + x_2\over 2} - \bar x\ ,\qquad
	y\hbox{ shift} = {y_1 + y_2\over 2} - \bar y\ ,
}{al metrology ext}
where $\bar x$ and $\bar y$ are the arithmetic means of the ideal values shown in \Fg{al optical metrology}. The shifts in the read-out direction $\De s$ then follow from \Eq{al shr def}. By error propagation, the expected uncertainty is about $7\un{\mu m}$ for shifts and about $0.3\un{\mu rad}$ for rotations.

Most of the results will be presented later in comparison with the other methods. But let us remark that this is the only method sensitive to the rotation between $U$ and $V$ sensors. See \Tb{al opt uv rot} for a list of the $U$-$V$ rotations determined per pot. These rotation are about an order of magnitude smaller that the expected rotation misalignments (\Tb{al exp misal}).

\htab{al opt uv rot}{The rotation between $U$ and $V$ sensors (in $\rm mrad$) as determined from the optical metrology.}{
\omit&\multispan6\hrulefill\cr
\omit&\multispan3\strut\bvrule\hfil near unit\hfil&\multispan3\bvrule\hfil far unit\hfil\cr
\omit&\multispan6\hrulefill\cr
\omit&\omit\bvrule\strut\hfil\hbox{top} &\hbox{bot} & \hbox{hor} & \hbox{top} & \hbox{bot} & \hbox{hor}\cr\ln
\hbox{sector 45, $220\un{m}$ station} & 0.03 & -0.08 & -0.08 & -0.13 & -0.10 & 0.06\cr\ln
\hbox{sector 56, $220\un{m}$ station} & 0.19 & -0.24 & -0.21 & 0.51 & 0.00 & -0.07\cr\bln
}

Let us repeat that this metrology measured the relative position of a sensor on its hybrid frame. It is only one component of the internal alignment. The other component is the alignment of the hybrid frames in the detector packages. The latter contribution is expected small, but is unfortunately unknown.

\caption{Beam test and cosmic rays runs}

Every RP was tested with a muon beam or cosmic rays prior to its installation in the LHC. These tests took place at a TOTEM test facility at H8 (a CERN test hall). That is why we will refer to these tests as H8 tests. There was always one RP studied at a time. What will turn out important is that, at H8 the RPs were kept a different temperature than in the LHC.

\tab{al H8 stat}{Statistics on the H8 data used for alignment. $\si(a_x)$ and $\si(a_y)$ denote the standard deviations of track angles in the horizontal and vertical direction.}{\bln
\hbox{sector}	&\hbox{unit}	&\hbox{pot}	&\hbox{detector}	&\hbox{particle}	& \hbox{events}	&\hbox{tracks}	&\si(a_x)	&\si(a_y)\cr
				&				&           &\hbox{package}		&\hbox{type} 		& \hbox{total}	&\hbox{used} 	&\rm mrad	&\rm mrad\cr\bln
%              
&			&\hbox{top} 	&\hbox{11} 		&  \hbox{muons} & 1\cdot10^{4} & 4\cdot10^{3} & 2.6 & 2.1\cr
\multispan2&\multispan7\hrulefill\cr
&\hbox{near}&\hbox{bot} 	&\hbox{12} 		&  \hbox{muons} & 6\cdot10^{4} & 2\cdot10^{4} & 2.3 & 2.1\cr
\multispan2&\multispan7\hrulefill\cr
&			&\hbox{hor} 	&\hbox{8}  		&  \hbox{muons} & 2\cdot10^{5} & 3\cdot10^{4} & 2.6 & 2.7\cr
\omit\hss\vbox to0pt{\vss\hbox{45}\vss}\hss&\multispan8\hrulefill\cr
&			&\hbox{hor}  	&\hbox{1}  		&  \hbox{muons} & 2\cdot10^{4} & 1\cdot10^{4} & 4.9 & 4.9\cr
\multispan2&\multispan7\hrulefill\cr
&\hbox{far}	&\hbox{top}  	&\hbox{9}  		&  \hbox{muons} & 3\cdot10^{4} & 1\cdot10^{4} & 2.1 & 2.1\cr
\multispan2&\multispan7\hrulefill\cr
&			&\hbox{bot}  	&\hbox{10} 		&  \hbox{muons} & 2\cdot10^{4} & 7\cdot10^{3} & 2.2 & 2.8\cr
\bln                           
&			&\hbox{top} 	&\hbox{5}  		&  \hbox{cosmics} & 4\cdot10^{2} & 2\cdot10^{2} & 54.6 & 53\cr
\multispan2&\multispan7\hrulefill\cr
&\hbox{near}&\hbox{bot} 	&\hbox{6}  		&  \hbox{cosmics} & 4\cdot10^{2} & 1\cdot10^{2} & 56.7 & 45.2\cr
\multispan2&\multispan7\hrulefill\cr
&			&\hbox{hor} 	&\hbox{7}  		&  \hbox{cosmics} & 6\cdot10^{2} & 3\cdot10^{2} & 52.8 & 54.5\cr
\omit\hss\vbox to0pt{\vss\hbox{56}\vss}\hss&\multispan8\hrulefill\cr
&			&\hbox{hor}  	&\hbox{2}  		&  \hbox{muons} & 4\cdot10^{4} & 2\cdot10^{4} & 3.4 & 3.9\cr
\multispan2&\multispan7\hrulefill\cr
&\hbox{far}	&\hbox{top}  	&\hbox{4}  		&  \hbox{cosmics} & 6\cdot10^{2} & 3\cdot10^{2} & 55.3 & 53.5\cr
\multispan2&\multispan7\hrulefill\cr
&			&\hbox{bot}  	&\hbox{3}  		&  \hbox{cosmics} & 9\cdot10^{2} & 4\cdot10^{2} & 53.4 & 53.1\cr\bln
}

\Tb{al H8 stat} presents some statistics about the data sample collected for each pot. Besides the total number of events and the number of tracks used for alignment, we have indicated the track angular spreads. The trend is clear, muon beams compared to cosmic rays provided many more events, but with more collimated tracks.


The track-based alignment has been performed with the following settings (for their explanation see \Sc{al data sel}).

\hbox to\hsize{\hss\vtop{\halign{\quad\strut\hfil#\quad&#\hfil\quad\cr\ln
\pmt{minimumHitsPerProjectionPerRP} & 4\cr
\pmt{removeImpossible} & True\cr
\pmt{requireBothUnits} & False\cr
\pmt{requireOverlap} & False\cr
\pmt{requireAtLeast3PotsInOverlap} & False\cr\ln
}}\hss\vtop{\halign{\strut\quad\hfil#\ &\ \hfil#\hfil\ &\ \hfil#\hfil\ &\ \hfil#\hfil\ &\ \hfil#\hfil\ &\ \hfil#\hfil\quad\cr\ln
iteration 				& 1 & 2  & 3   & 4   & 5\cr\ln
quantities optimized	& s & s  & s+r & s+r & s+r\cr
\pmt{maxResidualToSigma}&10 & 7  & 3   & 3   & 3\cr
\pmt{chiSqPerNdfCut}	&50 & 25 & 5   & 5   & 5\cr\ln
}}\hss}

\noindent Five iterations have been performed in total. In the first two, only the read-out shifts have been optimized (s in the table), in the rest the rotations about $z$ have been added (s+r). This has been done in order to increase stability -- during our MC tests the shifts have proven to be more resistant to any sort of problems. That is why we have added the rotations only when the cuts could have been set tight enough to remove all pathological events. On contrary, in the first iteration we have started with rather loose cuts -- hit residuals allowed up ten times the hit uncertainty. Since the uncertainty is most often $19\un{\mu m}$, the residual cut has been $190\un{\mu m}$. Comparing this value to the expected internal misalignments (the first column in \Tb{al exp misal}), one may find it exaggerated, but our goal has been to allow for even very large misalignments.

%The value of the \pmt{chiSqPerNdfCut} parameter has been derived from \pmt{maxResidualToSigma}. 
%\TODO{chi**2 cut = maxResidualToSigma**2 / 2. Why the division}

We have used a set of fixed-sensors constraints. First we have tried to constrain for rotations only one sensor per projection. But as it is shown in \Fg{al comp det per pot dp1 ext}, the linearly-progressive rotation (the weak mode \Eq{al sm rotz sol la}) is poorly determined and spoils the results. Therefore we have constrained two more sensors -- eventually constraining first and last two sensors (for both shifts and rotations).

\fig{fig/pdf/al_comp_det_per_pot_dp2_ext.pdf}{al comp det per pot dp1 ext}{An attempt to determine the rotations about $z$ with one fixed sensor per projection (first two sensors of the RP).}

As suggested in \Sc{al err}, it is a good idea to split the event sample into several subsamples and compare the alignment results among them. In this way one can check for present of problems like data corruption. For the H8 data, we have performed alignment analyses with full sample and two subsamples -- with even and odd event numbers. All results will be shown later, in a comparison to the optical metrology and LHC data alignment analyses.


\caption{LHC runs and comparisons}

We have also analyzed data from a number of 2010 LHC runs (see the list in \Tb{al lhc datasets}). The track-based alignment has been performed in very much the same way as for the H8 runs. \Fg{al comp det per pot dp1 ext2,al comp det per pot dp2 ext2} show the comparisons of the tree internal-alignment methods for two RPs. The former figure represents the RPs with a bad match between the methods, the former shows one of the best matches the we have observed.

\fig{fig/pdf/al_comp_det_per_pot_dp1_ext2.pdf}{al comp det per pot dp1 ext2}{Internal alignment comparison for RP 45-220-far-hor (an example of bad match among the tree methods). All LHC points are overlapping (one can see only the top orange) and thus only few data-sets are shown.
%ext2 constraints
}

\fig{fig/pdf/al_comp_det_per_pot_dp2_ext2.pdf}{al comp det per pot dp2 ext2}{Internal alignment comparison for RP 56-220-far-hor (an example of good match among the three methods). All LHC points are overlapping (one can see only the top orange) and thus only few data-sets are shown.
%ext2 constraints
}

Let us conclude the comparisons (generally, even for the ten other comparisons not shown). First observation is that all LHC results are perfectly consistent with each other. Similarly, there is no remarkable difference between the H8 analyses using the full sample or either from the subsamples. These are, with no doubt, pleasant outcomes. Generally, the match is better for the rotations. Regarding the optical metrology this might be partially understood by noticing that a part (systematic shift) of the measurement errors cancels for rotations in \Eq{al metrology ext}. Also quite generally, the H8 and LHC results match better together than with the optical metrology. This might be an indication that the alignment component not taken into account in the optical metrology, is in fact important. The difference between the H8 and LHC results may be understood as a consequence of different temperatures of the pots. Due to the thermal expansion, the mechanical stresses might lay out slightly differently, provoking slightly different misalignments.


\subsection[al lhc res]{LHC alignment results}

In this section we will present our alignment analyses of the LHC data collected in 2010. To start with, we have selected the runs where also the horizontal pots were inserted, see \Tb{al lhc datasets}. They are essential for the alignment as they provide the overlap with the top and the bottom pots, allowing for the alignment among each other. Usually the horizontal RPs were inserted at the end of a run, not to distort the measurements done with the vertical pots only. This is the reason for the relatively low number of runs with horizontal pots included.

\Tb{al lhc datasets} includes also the track angular spreads. One can see that they were very similar in both projections. The typical order is $\O{0.1\un{mrad}}$ -- this estimate has been used many times in the preceding sections.

\tab{al lhc datasets}{The list of 2010 LHC runs used for alignment. The RP approach gives the planned distance of the vertical pot thin windows from the beam in terms of the beam sigmas. The symbols $\si(a_x)$ and $\si(a_y)$ denote the track angular spread in the horizontal and vertical projection.}{
\multispan3&\multispan6\bhrulefill\cr
\multispan3&\multispan3\bvrule\strut\hfil\hbox{sector 45}\hfil&\multispan3\vrule\hfil\strut\hbox{sector 56}\hfil\cr
\multispan3&\multispan6\hrulefill\cr
\multispan3&\omit\bvrule\hfil\strut\hbox{events}\hfil & \si(a_x) & \si(a_y) &\hbox{events} & \si(a_x) & \si(a_y)\cr
\multispan3\bhrulefill&&&&&&\cr
\hbox{date} & \hbox{RP approach} & \hbox{run numbers} & \times 10^5 & \rm mrad & \rm mrad  & \times 10^5 & \rm mrad & \rm mrad \cr\bln
\hbox{24 Aug}    & 20\,\sigma & 2762,2763,2770,2772 & 7 & 0.3 & 0.3 & 8.4 & 0.3 & 0.3\cr\ln
\hbox{26 Aug}    & 20\,\sigma & 2896,2895,2892,2891 & 2.5 & 0.5 & 0.5 & 3.8 & 0.3 & 0.4\cr\ln
\hbox{21 Sep}    &  8\,\sigma & 3230,3231 & 0.5 & 0.2 & 0.2 & 0.6 & 0.3 & 0.2\cr\ln
\hbox{28 Sep}    & 18\,\sigma & 3285,3286,3287,3288 & 5 & 0.4 & 0.4 & 6.2 & 0.3 & 0.3\cr\ln
\hbox{05 Oct}    & 18\,\sigma & 3336,3337 & 10.8 & 0.4 & 0.4 & 15.1 & 0.3 & 0.3\cr\ln
\hbox{07 Oct}    & 18\,\sigma & 3359,3360,3361 & 4 & 0.3 & 0.3 & 4.6 & 0.3 & 0.3\cr\ln
\hbox{14 Oct}    & 18\,\sigma & 3457,3459,3460 & 0.8 & 0.4 & 0.4 & 1.1 & 0.3 & 0.3\cr\ln
\hbox{24 Oct}    & 18\,\sigma & 3609 & 6.2 & 0.2 & 0.2 & 6.6 & 0.2 & 0.2\cr\ln
\hbox{26 Oct}    & 18\,\sigma & 3634,3635 & 5.2 & 0.4 & 0.3 & 6.5 & 0.3 & 0.3\cr\ln
\hbox{29-30 Oct} &  7\,\sigma & 3723,3725,3728 & 2.5 & 0.2 & 0.2 & 2.9 & 0.3 & 0.2\cr\bln
}

The track-based alignment has been applied with the following parameters (for explanation see \Sc{al data sel}).

\hbox to\hsize{\hss\vtop{\halign{\quad\strut\hfil#\ &#\hfil\quad\cr\ln
\pmt{minimumHitsPerProjectionPerRP} & 4\cr
\pmt{removeImpossible} & True\cr
\pmt{requireBothUnits} & True\cr
\pmt{requireOverlap} & False\cr
\pmt{requireAtLeast3PotsInOverlap} & True\cr\ln
}}\hss\vtop{\halign{\strut\quad\hfil#\ &\ \hfil#\hfil\ &\ \hfil#\hfil\ &\ \hfil#\hfil\ &\ \hfil#\hfil\ &\ \hfil#\hfil\quad\cr\ln
iteration 				& 1 & 2  & 3   & 4   & 5\cr\ln
quantities optimized	& s & s  & s+r & s+r & s+r\cr
\pmt{maxResidualToSigma}&100 & 10 & 10 & 3 & 3\cr
\pmt{chiSqPerNdfCut}	&5000 & 50 & 50 & 5 & 5\cr\bln
}}\hss}

\vskip1mm\noindent The strategy for iterations has been chosen similar to the H8 data analyses (see the previous section). With the exception of a much higher residual cut at the beginning. This is because the initial misalignment between RPs may go up to about a millimeter. The \pmt{maxResidualToSigma} parameter set to $100$ corresponds roughly to a residual cut of $1.9\un{mm}$.

At first we have tried to run the alignment without fixing the linearly progressive rotation (the weak mode \Eq{al sm rotz sol la}). However, as can be seen in \Fg{al comp det per unit weak}, the results exhibit the typical signs of a poorly determined alignment mode. Looking at the far unit (the rotations have been fixed in the near one), one can see the three results having ``a similar shape'', but being shifted with respect to each other. This can be understood with the help of \Eq{al fit eq eigen one}. For each dataset the effect of errors (the term $\De\tilde\T$) is different, moreover magnified by the low eigenvalue $\la$. Since the weak modes (one for $U$ and one for $V$ sensors) correspond to the far-near unit rotations and since the global rotation has been fixed in the near unit, the weak modes manifests as uniform (rotation) shifts in the far unit.

Here all runs are used (even those without horizontal RPs). The standard results (i.e. with final constraints) are taken as the starting point, then 5 iterations made with fixed-detectors constraints (fixed rotation planes: 1200 and 1201). 

\fig{fig/pdf/al_comp_det_per_unit_weak.pdf}{al comp det per unit weak}{Alignment results with rotations constrained only in the near unit (first two sensors in the top RP). This figure shows the results for the $220\un{m}$ station in the sector 45. The top row shows the near unit, the bottom the far unit. Each color corresponds to a data set: October 24 (black), 26 (red), 29-30 (blue).}

For further analyses we have used the final constraints, where the far-near rotation is fixed. An example is in \Fg{al comp det per unit}. On the first look, one can see that the read-out-direction shift results (top row) form lines. It is a consequence of the detector packages being rotated about $x$ and $y$ axes, see \Sc{al rp fac}. Generally (also for the rotations -- see the bottom row), this figure supports our expectation that the sensor misalignments within a package (deviations from the fit lines) are much smaller that the misalignments of DP packages within a station. The other interesting (and pleasant) outcome is the stability of the results -- the misalignments do not seem to change in time.

\fig{fig/pdf/al_comp_det_per_unit.pdf}{al comp det per unit}{Alignment comparison with the final constraints for 56-220-near unit. Each color corresponds to a data set from \Tb{al lhc datasets}. The dashed lines show fits of \Eq{al dp sen al} and correspond to the DP alignments. For example for rotations, the position of the dashed line marks the DP rotation $\De\rh^{\rm DP}_z$.
%The internal rotation is the rotation of a sensor with respect to its DP (i.e. the full rotation minus $\De\rh^{\rm DP}_z$).
%overlap=f
}

Let us investigate the stability a bit deeper. Since a comparison for all sensors would require too much space, we have decided to compare only the DP alignments calculated according to \Eq{al dp rotz,al dp shr}. These are the misalignments shared by all sensors of a given DP. A comparison for 56-220-far unit can be found in \Fg{al comp rp all rot}. There are two types of points representing two types of analyses. Circles are used for alignment with all tracks (\pmt{requireOverlap} = False), squares for alignment with overlap tracks only (\pmt{requireOverlap} = True). There have been several reasons to do so. First, the overlap requirement selects a subsample from all the tracks. As we have said, this is a way to check for presence of pathological events that might bias the alignment results. Furthermore, the overlap tracks have a rather different distribution of the rotation $\ga$ factors (see \Tb{al alignment quantities}). Therefore, this is also a test of the stability of the rotation determination. Looking at the results, one finds both types of analyses compatible -- this suggests that the results are robust.

\fig{fig/pdf/al_comp_rp_all_rot.pdf}{al comp rp all rot}{DP alignment comparison for 56-220-far unit (with the final constraints.) The order of points is the same as in \Tb{al lhc datasets} (top-down direction in both cases). The circles corresponds to the analyses with all tracks, the squares with overlap tracks only. For certain data sets the latter type of analyses have not yielded reasonable results, which have thus not been drawn. The vertical dashed lines mark the mean misalignments.
}

Returning to \Fg{al comp rp all rot}, we see that most shift corrections lie in a band of $\pm 10\un{\mu m}$ around the mean (the dashed vertical line). This a good result which could have been achieved only thanks to the reliability of the (corrected) LVDT measurements -- they establish the initial geometry for the track-based alignment. In fact, the mean values could be used to improve the corrections obtained in \Sc{al collim}. There, we have determined the corrections for the vertical positions of the vertical pots. Looking at the second row, one finds that track-based alignment brings a modification of $90\un{\mu m}$, which is of the expected order \Eq{al window to dp uncer}.

Regarding the rotations, the situation is less homogeneous. Some rotations are very stable (all rotations of the horizontal pot), some less ($\rh_z$ of the two vertical pots). For the latter, the fluctuations go up to $\pm 0.4\un{mrad}$ from the mean. This is already an important rotation, which can not be neglected. A quantitative summary of all the DP rotations can be found in \Tb{al rp rot}. One can see $\rh_z$ rotations tend to have higher standard deviation, which means they tend to change more from a data set to another.

\tab{al rp rot}{A summary of the DP rotations results (values in milliradians). Only the analyses with all tracks (\pmt{requireOverlap} = False) have been included. The analyses with overlap tracks only have yielded rather large errors for sector 45. The values in the mean column correspond to the vertical dashed line in \Fg{al comp rp all rot}. $\si$ stands for the standard deviation of the results obtained from the individual data sets.
% weighted mean, non-weighted sigma
}{
\multispan3&\multispan6\bhrulefill\cr
\multispan3&\multispan2\bvrule\strut\hfil$\rh_x^{\rm DP}$\hfil&\multispan2\vrule\strut\hfil$\rh_y^{\rm DP}$\hfil&\multispan2\vrule\strut\hfil$\rh_z^{\rm DP}$\hfil\cr
\multispan3\bhrulefill&\multispan6\hrulefill\cr
\hbox{sector}&\hbox{unit}&\hbox{pot} & \hbox{mean} & \si  & \hbox{mean} & \si  & \hbox{mean} & \si \cr\bln
&			&\hbox{top} & -10.8& 0.08 & -4.8 & 0.15 & -4.5 & 0.07\cr
&\hbox{near}&\hbox{bot} & -6.1 & 0.08 & +5.9 & 0.06 & +6.1 & 0.09\cr
&			&\hbox{hor} & +1.7 & 0.02 & -1.4 & 0.04 & -1.6 & 0.03\cr
\omit\hss\vbox to0pt{\vss\hbox{45}\vss}\hss&\multispan8\hrulefill\cr
&			&\hbox{hor} & -1.9 & 0.03 & -9.9 & 0.04 & -2.4 & 0.06\cr
&\hbox{far}	&\hbox{top} & -7.3 & 0.04 & -0.9 & 0.04 & -2.4 & 0.15\cr
&			&\hbox{bot} & -9.2 & 0.04 & +4.7 & 0.02 & +4.8 & 0.15\cr
\bln
&			&\hbox{top} & -7.7 & 0.05 & +2.3 & 0.03 & -3.5 & 0.08\cr
&\hbox{near}&\hbox{bot} & -7.3 & 0.05 & \phantom{+}0.0 & 0.05 & +5.0 & 0.09\cr
&			&\hbox{hor} & +1.2 & 0.03 & -5.7 & 0.03 & -1.6 & 0.03\cr
\omit\hss\vbox to0pt{\vss\hbox{56}\vss}\hss&\multispan8\hrulefill\cr
&			&\hbox{hor} & -0.2 & 0.03 & -3.4 & 0.02 & -2.6 & 0.05\cr
&\hbox{far}	&\hbox{top} & -3.8 & 0.06 & \phantom{+}0.0 & 0.02 & -4.5 & 0.19\cr
&			&\hbox{bot} & -4.4 & 0.07 & +2.5 & 0.02 & +7.1 & 0.17\cr
\bln
}

One may object that such a direct comparison of shifts as in \Fg{al comp rp all rot} is slightly misleading. Every rotation misalignment induces a shift correction, see \Eq{al sm rotz shr}. This correction is proportional to the distance of the sensor from the origin ($\vec c$). Thus, the correction would be different for different RP approaches (different data sets). From \Tb{al lhc datasets} one can read that the thin-window positions varied approximately from $2.6$ to $8\un{mm}$ for the vertical pots and from $2.3$ to $5.6\un{mm}$ for the horizontal pots. The span was thus of the order of $\De c \approx 4\un{mm}$. From the \Eq{al sm rotz shr} one could expect the rotation-induced shifts to vary by $\De c \De \rh$. Taking $\De\rh \approx 10 \un{mrad}$ (the scale of rotation results in \Fg{al comp det per unit}) yields a variation by about $40\un{\mu m}$. However, the variation is actually much smaller, MC simulations give the order of $5\un{\mu m}$. There are two reasons for this. First, the final constraints have been used. The mix shift and rotation degrees of freedom and make the calculation more complex. And second, the rotations of the top and bottom pots have nearly the same values but opposite signs. This leads to partial cancellations. To summarize, the shifts obtained from different data-takings can be compared with an uncertainty of $5\un{\mu m}$.


In this section we have (re)verified that one can not determine the rotation between the near and far unit with a reasonable precision. Then we have compared alignment results from a number of 2010 LHC runs. We have found the DP shifts very stable -- the change from run to another is of the order of $10\un{\mu m}$. The vertical DP shifts, as corrections to the collimation alignment presented in \Sc{al collim}, have the expected order \Eq{al window to dp uncer}. We have also determined all three DP rotations. Those about $x$ and $y$ axes have shown to be very stable too. However, the rotation about $z$ has presented non-negligible run-to-run variations (standard deviation up to about $0.2\un{mrad}$).


\section[al prof]{Profile methods}

In \Sc{al} we have shown that the track-based alignment is a very powerful tool, however, there are certain alignment modes that cannot be accessed (see e.g.~\Tb{al sing mode overview}). Regarding transverse shifts, these alignment modes can be described as horizontal ($x$) and vertical ($y$) shifts of the vertical RPs in each station (cf.~the final constraints \Eq{al final constraints}). In this section we will discuss some methods that can determine the misalignment in these modes (an alternative using elastic events will be shown in \Sc{al elast}).

We recall that our goal is the alignment of RP sensors with respect to the beam. One can approach the task from the opposite side and determine the beam position as measured by the RP sensors (that is the beam's position with respect to the RP sensors). The position of the beam can be deduced from measured hit distributions with the aid of some symmetry considerations.


\iffalse
\> beam position is the theoretical hit position of an elastic proton with $\th = 0$ and the mean vertex position.
\> (in our OfflineSW convention, the beam is at zero)
\> misalignements $\De q$ and RP positions $q^{\rm RP}$ (in accordance with \Eq{misalignment definition})
$$q^{\rm RP}(\hbox{actual/misaligned}) = q^{\rm RP}(\hbox{thought/nominal}) + \De q$$
therefore for hit positions $q$
$$q(\hbox{reconstructed with thought geometry}) = q(\hbox{real}) - \De q$$
\fi

The expected hit-distribution symmetries depend on the considered physical processes as well as on the optics. However, for the nominal optics, the vertical symmetry around the beam can be assumed quite generally. Was there no crossing-angle, the scattering at the IP would have full azimuthal symmetry (about the beam axis). Even if the crossing-angle is non-zero, it acts in the horizontal plane, thus the up-down symmetry is preserved. On the optics side, most of them are designed with vanishing vertical dispersion $D_y$ (however this will turn out false for the optics from 2010, see e.g.~\Fg{al prof hits}), therefore the vertical symmetry would remain also after the proton transport, that means it would be observable in the hit distribution (see e.g.~\Fg{al prof simu}).

\fig{fig/pdf/al_prof_simu.pdf}{al prof simu}{A MC simulation of hit distributions at the RPs of the 56-220-near unit. The green points represent elastic scattering (Elegent, PPP3), blue SD (Pythia) and red DPE (Phojet). All processes simulated at $\sqrt s = 3500\un{GeV}$ and with nominal optics with $\be^* = 2.5\un{m}$. The black solid lines are the contours of the sensors. The dotted lines represent symmetry axes, the middle black dot marks the position of the beam.}

\Fg{al prof simu} shows typical hit distributions (MC simulation) for several important forward-physics processes. Beyond the aforementioned vertical symmetry, one can immediately spot the horizontal symmetry of the green points (elastic scattering). The entire next section \Sc{al elast} will be devoted to the alignment with elastic scattering events, here let us comment on how to use this symmetry even without selecting a sample of elastic events. The ``hot spot'' close to $x\approx 0\un{mm}$ contains also a contribution from non-elastic processes with low $\xi$ (see for example the red dots -- DPE). Due to the horizontal dispersion, the hits are shifted to the right and the symmetry is broken. This can be well seen in \Fg{al prof x dists}: the elastic peak is narrow and close to zero, but the non-elastic events form a broad structure shifted towards higher $x$ values. But what is important is that around $x=0$, the distribution is dominated by elastics events and their peak can be well fitted, as illustrated in the figure.

\fig{fig/pdf/al_prof_hits.pdf}{al prof hits}{Typical hit distributions at the $220\un{m}$ stations (scoring planes at $\pm 217\un{m}$). Data from 21 September. The colorful lines represent fits of low-$\xi$ (green) and large-$\xi$ (magenta) hits.
% analysis with vsym2 geometry
}

\Fg{al prof simu} shows the hit distributions as expected from simulations, however, the distributions from 2010 data look quite different, see \Fg{al prof hits}. On the first sight, we one can two major differences.

\> The \em{tilt of the elastic peak} (around the green line). Our current understanding is that due to LHC magnet misalignments (mainly rotations) a coupling between the $x$ and $y$ components of the optics, \Eq{ttm lin par}, was created. The net effect is the tilt of the elastic hit axis.

\> The \em{unexpected distribution of the diffractive hits} (around the magenta line), mainly for the sector 56. The distribution is tilted and deformed. We understand the tilt as an effect of non-zero vertical dispersion $D_y$. The dispersion terms in the proton transport \Eq{ttm lin par} then move diffractive protons along the axis with tilt $D_y/D_x$.

\fig{fig/pdf/al_prof_x_dists.pdf}{al prof x dists}{Horizontal hit distributions in the 56-220-near-top pot (data from 29-30 October). Left: all hits, Right: one horizontal slice only. The peaks are mostly formed by elastic events, the background is dominated by DPE. The orange curves are Gaussian fits with a parabolic background.}

As a consequence of the elastic peak tilt, the peaks in the $x$ distributions widen up, see \Fg{al prof x dists} left, and their centers do not correspond to the $x$ position of the beam. To mitigate this problem, one may divide the scatter plot into horizontal slices and fit the $x$ distributions for each of them (an example is in \Fg{al prof x dists} right; note the thinner peak). The results can then be plotted versus the center of the slice, see an example in \Fg{al prof fits} left. The vertical error bands represent the slice widths, the horizontal are coming from the peak fit uncertainty. The points in gray have been removed as outliers. The green line represents a common fit through the top and bottom pot data and it confirms that the tilt is the same for both RPs. Then, the green line shall go through the beam center.

\fig{fig/pdf/al_prof_fits.pdf}{al prof fits}{The results of the sliced fits for the 56-220-near unit and the data from 21 September. Left: fit of the elastic peaks, Right: fit of the diffractive hit distribution. The fit lines use the same color code as in \Fg{al prof hits}. The gray points have been excluded from the fit as outliers.}

In principle, the same slicing method could be applied to the diffractive hits (see \Fg{al prof fits} right), however it is complicated by two facts. First, the profile is non-linear. It can be seen already in \Fg{al prof hits} right, if inspected carefully.  Then, it is quantified in \Fg{al prof fits} right. Thus the linear-fit result will be burdened by an important systematic error. Since the diffractive hit distribution is wide and therefore the statistical error of the fit is large as well. This all is made even worse by the second complication: since there is just one horizontal pot, the fit must be extrapolated to the beam position. At the end, the error propagation makes this method uninteresting.


%\TODO{broken symmetries - asymmetric acceptance, optics deviations, aperture limitations, trigger and detector efficiency bias}

To conclude, only the low-$\xi$ hit fit can be performed with an interesting precision. Since the elastic peak tilts are below $40\un{mrad}$ (see \Tb{al el yx}) and the residual vertical misalignment is of the order of $100\un{\mu m}$, the fit determines the beam position with an uncertainty below $10\un{\mu m}$.


Practically, the profile alignment is used as the step prior to the alignment with elastic tracks. It simplifies the selection of the elastic sample, but the precision is superseded by the elastic alignment.

\section[al elast]{Elastic Alignment}

In this section we will show how elastic events may be used for alignment purposes. In particular to determine the misalignments in the modes inaccessible to the track-based method. Regarding transverse shifts, these inaccessible modes can be described as horizontal ($x$) and vertical ($y$) shifts of the vertical RPs in each station (cf.~the final constraints \Eq{al final constraints}).

Before discussing the alignment application, let us show how to select elastic events. Elastically scattered protons retain the incident momentum, thus their momentum loss $\xi = 0$ and consequently the transport equation \Eq{ttm lin par} simplifies to
\eqref{x(s) = L_x(s)\, \th_x^* + v_x(s)\, x^*\ .}{al el transport}
We have written (and we will do so in what follows) the relation explicitly for the $x$ projection only, but for the $y$ projection an equivalent relation holds too. Now focusing on the standard 2010 optics with $\be^*=3.5\un{m}$ one may conclude that the second (vertex) term is negligible compared to the first one. Moreover the optics was almost left--right symmetric: $L_x(s) \approx - L_x(-s)$. Furthermore, the horizontal effective length $L_x$ was rather low, thus the elastic hits cumulated around $x\approx 0\un{mm}$ (cf.~\Fg{al prof hits}).


Now considering that the RPs are actually shifted by $\De x$, a measurement by a RP can be written
\eqref{x \approx L_x\, \th_x^* - \De x\ ,}{al el meas}
where $L_x$ is the effective length corresponding to the RP. Then one can relate far-near hit difference measured in the sectors 45 and 56:
\eqref{
	x^{56}_{\rm F} - x^{56}_{\rm N} \equiv \De_{\rm F-N} x^{56} \approx
	\underbrace{ {L_{x,\rm F}^{56} - L_{x,\rm N}^{56}\over L_{x,\rm F}^{45} - L_{x,\rm N}^{45} } }_{-1 + a} \De_{\rm F-N} x^{45} + 
	\underbrace{ {L_{x,\rm F}^{56} - L_{x,\rm N}^{56}\over L_{x,\rm F}^{45} - L_{x,\rm N}^{45} } (\De x_{\rm F}^{45} - \De x_{\rm N}^{45}) - (\De x_{\rm F}^{56} - \De x_{\rm N}^{56}) }_{b} \ .
}{al el dxdx}
The small slope correction $a$ arises from small asymmetries in the optics, the intercept $b$ is an effect of the RP misalignments. This relation provides our first elastic-selection cut, see \Fg{al el selection} top row.

To obtain our second cut, one may relate the far-near hit difference to the hit position in the RP:
\eqref{
	\De_{\rm F-N} y \approx
	\underbrace{ {L_{y, \rm F} - L_{y, \rm N}\over L_{y, \rm N}} }_a y_N 
	+ \underbrace{ {L_{y, \rm F} - L_{y, \rm N}\over L_y^N} \De y_N - (\De y_F - \De y_N) }_b \ .
}{al el dyy}
The parameters $a$ and $b$ give the slope and intercept of the line along which the elastic events cumulate, see \Fg{al el selection} middle row.

Our third cut comes out from the observation that the elastic hits cumulate along a slightly-tilted vertical axis (see \Fg{al prof hits}). Denoting the tilt $a$ and considering that this axis shall go through the beam center $(-\De x, -\De y)$ yields the following relation.
\eqref{x \approx ay + \underbrace{a \De y - \De x}_{b}\ .}{al el x} 

All the cuts \Eq{al el dxdx,al el dyy,al el x} require ``the distance from a certain line to be smaller than a given threshold'', see the summary \Tb{al el cuts}. The distribution of events around the cut lines has two sources: the beam smearing and the proton transport approximations that we have made. These distributions have a Gaussian shape as can be seen in \Fg{al el selection} and their standard deviations have been used to set the cut thresholds -- they correspond to about $2.5$ multiple of the standard deviation.


\tab[\quad\bstrut\hfil\ #\ \hfil&\ #\ \hfil&\hfil\ #\ \hfil\qquad\cr]{al el cuts}{The cuts used for the elastic-event selection. The threshold gives the maximal permitted distance from the line. This permitted region is drawn as a green band in \Fg{al el selection}.}{\ln
cut & line & threshold\cr\ln
1 & $\De_{\rm F-N} x^{56} = (-1+a) \De_{\rm F-N} x^{45} + b$	& $80\un{\mu m}$ \cr
2 & $\De_{\rm F-N} y = a y_{\rm N} + b$							& $45\un{\mu m}$ \cr
3 & $x = ay + b$ 										& $400\un{\mu m}$ \cr\ln
}

The $a$ and $b$ cut parameters have been determined empirically in a few iterations. Starting with the values given by the ideal optics and no misalignments, a tentative selection has been made with very liberal thresholds. Then, new values of the $a$ and $b$ have been obtained by fitting the selected events. In the next iteration, stricter thresholds have been used.

Let us note that one could build more elastic-selection cuts than presented in \Tb{al el cuts}. A more complete list has been used for the first elastic scattering analysis as presented in \Sc{felm}, see in particular \Fg{felm cuts}. Still, the cuts used in this section provide a reasonably pure sample of elastic events, good for alignment purposes.


\fig{fig/pdf/al_el_selection.pdf}{al el selection}{An illustration of the elastic-event selection cuts (data from 21 September). Each row corresponds to one cut. Left: the events before (black) and after (red) the cut. The light-green area represents the cut requirement. Right: the distribution of events around the cut line (black) with a Gaussian fit (red). The dotted lines show the cut thresholds, cf. \Tb{al el cuts}.}

Since the relations \Eq{al el dxdx,al el dyy,al el x} involve the misalignment parameters, they can be used for alignment too. In the rest of this section we will present several alignment methods. The first one is dedicated to the horizontal alignment, methods two to four to the vertical alignment.

\caption{Method 1}

This method is based on fitting the parameterization \Eq{al el x} through $y$ vs. $x$ data for each top-bottom RP pair. The fit yields the $a$ and $b$ parameters and the horizontal shift can be calculated as $\De x = a \De y - b$. In principle the vertical shift $\De y$ can be obtained from methods 2-4. If not, one does not make a large error by neglecting the $a\De y$ term. The vertical misalignment is of the order $100\un{\mu m}$ (see \Eq{al window to dp uncer}), the tilts $a$ are about $40\un{mrad}$ (see \Tb{al el yx}). Thus one may expect
\eqref{\De x = - b + \O{4\un{\mu m}}\ .}{al el m1 de x}

The fits have been done in several iterations. To reduce the impact of any possible outliers (including non-elastic events passing the selection cuts), every hit has been assigned a weight (one of the suggested outlier treatments from \bref{millepede})
\eqref{w = \left\lbrace \matrix{
\strut 1				& \de < c_H \cr
\strut c_H^2\over\de^2	& \de > c_H \cr
}\right.\ ,\qquad c_H = 1.345\ .}{al el fit weight}
Here, $\de$ stands for the ratio of the hit distance from the fit line and the standard deviation of the distance distribution. Before the first iteration all weights have been set to 1. In total five iterations have been performed. The uncertainty of the final fit has been determined as from the LS method (see e.g.~(6.24) in \bref{barlow}). As the error input, we assigned the standard deviation of the cut-line distance to the error of the $x$ component of the hits. The results are summarized in \Tb{al el yx}.

An illustration of the method is shown in \Fg{al el plots yx}. Similar illustrations will be shown for all other methods. The left plot will always refer to the data from 5~October (an $18\si$ run with low statistics), while the right plot to data from 29-30~October (a $7\si$ run with high statistics).

The \Fg{al el plots yx} shows the separate top (red) and bottom (blue) RP fits are compatible with each other and the global fit (green). This means that the shifts of the top and bottom RPs are the same. This confirms that the track-base alignment, the previous step, has worked correctly.


\fig{fig/pdf/al_el_plots_yx.pdf}{al el plots yx}{An illustration of the alignment method 1 (unit 45-220-far). The red (blue) line corresponds to a fit of top (bottom) RP hits only, the green line represents a global fit.}

\htab{al el yx}{The results of the elastic alignment method 1. The slope $a$ (elastic peak tilt) is in milliradians, the intercept $b$ (related to $\De x$ via \Eq{al el m1 de x}) in micrometers. The errors are statistical only.}{
\omit&\multispan8\bhrulefill\cr
\omit&\multispan2\strut\bvrule\hfil 45-near\hfil & \multispan2\vrule\hfil 45-far\hfil  & \multispan2\vrule\hfil 56-near\hfil & \multispan2\vrule\hfil 56-far\hfil\cr
\omit&\multispan8\hrulefill\cr
\omit\strut & a & b & a & b & a & b & a & b\cr\bln
\hbox{21 Sep}    & -35.3 \pm    0.8& -33 \pm    4& -29.8 \pm    0.7& -46 \pm    3&  43.5 \pm    0.7& -23 \pm    3&  38.7 \pm    0.7& -26 \pm    3\cr\ln
\hbox{05 Oct}    & -36.7 \pm    0.5& -43 \pm    4& -32.4 \pm    0.5& -36 \pm    4&  44.0 \pm    0.5& -24 \pm    4&  38.4 \pm    0.5& -24 \pm    4\cr\ln
\hbox{07 Oct}    & -37.7 \pm    0.4&   2 \pm    4& -31.3 \pm    0.4& -12 \pm    3&  42.7 \pm    0.4& -38 \pm    4&  39.4 \pm    0.4& -27 \pm    4\cr\ln
\hbox{24 Oct}    & -37.7 \pm    0.2& -46 \pm    2& -32.2 \pm    0.2& -34 \pm    2&  43.8 \pm    0.2& -38 \pm    2&  39.5 \pm    0.2& -15 \pm    2\cr\ln
\hbox{26 Oct}    & -37.2 \pm    0.2& -24 \pm    2& -32.4 \pm    0.2& -20 \pm    2&  43.9 \pm    0.2& -35 \pm    2&  38.9 \pm    0.2& -68 \pm    2\cr\ln
\hbox{29-30 Oct} & -36.4 \pm    0.2& -26 \pm    1& -31.7 \pm    0.2&  -4 \pm    1&  44.3 \pm    0.1&  -7 \pm    1&  38.4 \pm    0.1&  -6 \pm    1\cr\bln
}


\caption{Method 2}

The vertical distribution of elastic hits shall be symmetric around the position of the beam. This remains true even if the axis of elastic scattering is tilted with respect to $y$ axis. In this case, the $y$ distribution shrinks, but symmetrically around the beam's position.

\fig{fig/pdf/al_el_shift_test.pdf}{al el shift test}{A method to determine the center of a symmetrical distribution from its tails. The bottom-pot distribution (blue) is flipped around the vertical axis and shifted with respect to the top-pot distribution (red) until the best match is found (green).}

Due to the limited acceptance, the central part of the $y$ distribution is missing, see \Fg{al el plots y shift}. Still, one can determine the center of the distribution with a method illustrated in \Fg{al el shift test}. The bottom-pot tail (blue) is flipped about the vertical axis and shifted left and right with respect to the top-pot tail (red) until the best match is found (green). The match can be measured by
\eqref{S^2/N = {1\over N} \sum_{i\ \in \hbox{overlapping bins}} \left[ C_{\rm red}(i) - C_{\rm green}(i) \right]^2\ ,}{al el shift test}
where for example $C_{\rm red}(i)$ stands for the contents of the $i$-th bin of the red distribution and $N$ is the number of overlapping bins. The best match corresponds to a minimum of this function. Then the center of the distribution is given ($s$ denotes the best-match shift)
\eqref{-\De y = \hbox{center} = {f_{\rm B} + f_{\rm T} + s\over 2}\ .}{al el shift center}
The center corresponds to the beam position which is related to the vertical misalignment $\De y$ as stated above. To see that, one shall consider that the center of the beam corresponds to $\th_y^* = 0\un{mrad}$. Then the relation to the misalignment follows from \Eq{al el meas}. $f_{\rm B}$ and $f_{\rm T}$ represent the edges of the bottom and top-pot distributions. However, one shall first remove the parts that may be affected by the acceptance and thus be asymmetric. These parts are, indeed, those close to $y=0\un{mm}$. Taking the beam divergence $18\un{\mu rad}$ and effective length $L_y\approx 22\un{m}$ yields a cutoff ($3\si$) of about $1.2\un{mm}$ (this number has been verified with a MC simulation).

We performing the shift tests, we have used this cutoff, moreover only bins with 10 and more hits have been used. For the distribution matching we have only considered shifts providing 10 and more overlapping bins. It has turned out that only the data set from 29-30 October had enough data to perform this test, see \Fg{al el plots y shift} and \Tb{al el y shift}.

\fig{fig/pdf/al_el_plots_y_shift.pdf}{al el plots y shift}{An illustration of the alignment method 2 (unit 56-220-far). The color code is the same as in the explanation \Fg{al el shift test}.}

Since the shift $s$ can only be a multiple of the bin size, a half of the bin size can be taken as a measure of the uncertainty. For the data from 29-30 October we have used a bin size of $30\un{\mu m}$ (for smaller values the bin fluctuation became too excessive), thus the uncertainty estimate is $15\un{\mu m}$. Of course, this does not reflect any systematic errors. 

\htab{al el y shift}{The results of the elastic alignment method 2: the vertical misalignment $\De y$ in micrometers. Our uncertainty estimate (half of the bin size) is $15\un{\mu m}$.}{
\omit&\multispan4\bhrulefill\cr
\omit&\strut\hbox{45 near}&\hbox{45 far}&\hbox{56 near}&\hbox{56 far}\cr\bln
\hbox{29-30 Oct} &   60 & -60 & 120 & 150 \cr\bln
}


\caption{Method 3}

Since the method 2 could determine the vertical misalignments only for the data from 29-30 October, we had to search for additional methods usable for the other data sets. These additional methods do not resolve completely the vertical misalignments, for example this method (method 3) determines only the relative vertical shift of the near unit with respect to the far one.

This method is based on \Eq{al el dyy}. Since the slopes $a$ turn out to be of the order of $20\un{mrad}$ (see \Fg{al el plots dyy}) and the scale of the vertical misalignments is about $100\un{\mu m}$ (see \Eq{al window to dp uncer}), we may write
\eqref{b = \De y_{\rm N} - \De y_{\rm F} + \O{2\un{\mu m}} \ .}{al el m2 b}
Clearly, the last term is rather negligible and the intercept gives the relative near-far vertical misalignment.

The parameters $a$ and $b$ have been obtained by fitting the parameterization \Eq{al el dyy} through $\De_{\rm F-N} y$ vs. $y_{N}$ data, see \Fg{al el plots dyy}. The fits have been performed in a similar manner as in method 1, with the difference that the standard deviation of the distance-to-cut-line distribution is assigned to the error of $\De_{\rm F-N} y$.

As shown in \Fg{al el plots dyy}, the one-side fits (red and blue) are compatible with each other and the global fit (green), with the exception of the red fit in the left-hand side plot. This means that the far-near misalignments are the same for top and bottom pots, a result that is expected after the track-based alignment step.

\fig{fig/pdf/al_el_plots_dyy.pdf}{al el plots dyy}{An illustration of the alignment method 3 (sector 45). The red (blue) line corresponds to a fit of top (bottom) RP hits only, the green line represents a global fit.
}

The results are summarized in \Tb{al el dyy}. For the data set from 29-30 October, one can make comparison to method 2 -- this yields near-far shifts $(-30\pm 20)\un{\mu m}$ for sector 45 and $(120\pm 20)\un{\mu m}$ for sector 56. This means a discrepancy of $1.5\si$ (sector 45) and $2\si$ (sector 56). Keeping in mind that the errors are statistical only, it is a reasonable agreement.

\htab{al el dyy}{The results of the elastic alignment method 3: the relative near-far vertical shift $\De y_{\rm N} - \De y_{\rm F} $ in micrometers. The errors are statistical only.}{
\omit&\multispan2\bhrulefill\cr
\omit&\strut\hbox{sector 45}&\hbox{sector 56}\cr\bln
\hbox{21 Sep}    &  69.3 \pm    0.7&  37.9 \pm    0.7 \cr\ln
\hbox{05 Oct}    &  70.4 \pm    0.8&  12.8 \pm    1.3 \cr\ln
\hbox{07 Oct}    &  72.0 \pm    0.9&   6.0 \pm    1.0 \cr\ln
\hbox{24 Oct}    &  86.8 \pm    0.4&  -1.4 \pm    0.6 \cr\ln
\hbox{26 Oct}    &  79.6 \pm    0.5&  -7.9 \pm    0.6 \cr\ln
\hbox{29-30 Oct} &  79.6 \pm    0.2&  -1.1 \pm    0.2 \cr\bln
}

\caption{Method 4}

\Eq{al el meas} predicts a correlation between $y$ hit positions in sectors 45 and 56:

\eqref{y^{56} =
	\underbrace{ {L_y^{56}\over L_y^{45}} }_{-1+a}  y^{45}
	+ \underbrace{ {L_y^{56}\over L_y^{45}} \De y^{45} - \De y^{56} }_b\ .
}{al el yy}
Since the optical length differ between the sectors about $10\percent$ and the vertical misalignment scale is about $100\un{\mu m}$  (see \Eq{al window to dp uncer}), one can write
\eqref{b = \De y^{45} - \De y^{56} + \O{10\un{\mu m}}\ .}{al el yy b}

The parameters $a$ and $b$ have been obtained by fitting as in methods 1 and 3. But before fitting, we have removed tips of the distributions that are affected by the acceptance. These are tips of triangular shape extending towards the point $y_{45} = y_{56} = 0\un{mm}$. Since they are not symmetric about the fit line and since they are densely populated, they might bias the fit results. The fits are illustrated in \Fg{al el plots ylyr}.

\fig{fig/pdf/al_el_plots_ylyr.pdf}{al el plots ylyr}{An illustration of the alignment method 4 (220-far units). The green line is a fit according to the parameterization \Eq{al el yy}.}

Unfortunately, the results are not compatible with methods 2 and 3. We have verified the correctness of the method with a MC study including angular, vertex and energy smearing (as described in \Sc{beam smearing}). However, for this MC we have considered elastic events only. The experimental sample obtained with cuts from \Tb{al el cuts} contains also a background contribution. Since the point distribution for method 4 (see \Fg{al el plots ylyr}) is rather wide (more than a millimeter), we assume that the background might have a non-negligible effect on the fit results. Consequently, the systematic error of this method is such that the method is not of a practical use.

\iffalse
\htab{al el ylyr}{
Left: the results of the elastic alignment method 4 -- fits of $y^{56}$ vs.~$y^{45}$ data. \TODO{}}{
\omit&\multispan2\bhrulefill\cr
\omit&\strut\hbox{near units}&\hbox{far units}\cr\bln
\hbox{21 Sep}    & -40.4 \pm   21.3&  78.6 \pm   21.4\cr\ln
\hbox{05 Oct}    & 170.8 \pm   26.3& 264.5 \pm   28.6\cr\ln
\hbox{07 Oct}    &  95.4 \pm   20.2& 181.9 \pm   21.8\cr\ln
\hbox{24 Oct}    & -75.8 \pm   10.7&  15.3 \pm   11.4\cr\ln
\hbox{26 Oct}    &  19.5 \pm    9.1&  98.6 \pm    9.9\cr\ln
\hbox{29-30 Oct} &-240.9 \pm    4.9&-158.3 \pm    4.9\cr\bln
}

-80		\pm30	&27		\pm31
377		\pm69	&477	\pm77
192		\pm55	&284	\pm61
-84		\pm30	&5		\pm33
56		\pm24	&145	\pm27
-270	\pm7	&-187	\pm7
\fi



\section[al sum]{Summary}

In the introduction of this chapter, we have drawn the alignment procedure in three steps -- collimation alignment, track-based alignment and alignment with physics processes. Let us now review the tasks accomplished for each of them.

\> During the collimation alignment, the LVDT offsets have been resolved, see \Tb{al lvdt off}. Consequently, the distances of the RP thin windows from the beam can be determined with a precision about $50\un{\mu m}$. Due to the poorly known distance between the thin windows and the sensors, the position of the sensors with respect to the beam has an uncertainty of the order of $100\un{\mu m}$, see \Eq{al window to dp uncer}.

\> The track-based alignment has eventually been applied with the final constraints (see \Sc{al constr}) that fix 4 shift and 4 rotation singular/weak modes per station. Regarding shifts, these modes can be interpreted as horizontal and vertical shifts of each unit. They have been addressed by the alignment with physics processes, see below. The rotation singular/weak modes comprise the rotations between $U$ and $V$ planes and the rotations between near and far units. Regarding the former, the optical alignment has shown no indication for non-negligible misalignments, see \Tb{al opt uv rot}. The near-far unit rotations are linked with the optics perturbations, see for example the elastic peak tilts in \Fg{al prof hits}. In principle, if the optics was well known, the near-far rotations can be determined from the tilt measurements.

\par\parindent\itindent\indent\hang
The typical precision of the track-based alignment applied to the 2010 LHC runs (see \Tb{al lhc datasets}) has been about $1\un{\mu m}$ for shifts and about $0.1\un{mrad}$. We have noticed very little run-to-run variations (see \Fg{al comp rp all rot} and \Tb{al rp rot}) except the rotation about $z$ for several RPs. There the run-to-run standard deviation have been up to $0.2\un{mrad}$.

\> With the elastic alignment we have determined some of the misalignments inaccessible to the track-based alignment. Using method 1 (see \Sc{al elast}), we have determined the horizontal shifts of each unit with an uncertainty about $5\un{\mu m}$. With method 2 we have determined the vertical shifts of each unit with an uncertainty about $15\un{\mu m}$ (systematic error not included), but only for one data set. For the others there has not been enough data. We have used the method 3 to determine the vertical shifts between far and near units with an uncertainty of few micrometers.

\vskip\itskip

To conclude, let us demonstrate the magnitude of the residual misalignment by evaluating its impact on the elastic scattering reconstruction. We will consider the standard 2010 LHC optics (see \Sc{felm}) and the reconstruction formulae \Eq{felm xy reco}. For the horizontal projection, the alignment error for one-arm reconstruction can be estimated as follows:
\eqref{
	\De\th_x^*
	= {\De x_{\rm F} - \De x_{\rm N} \over L_{x, \rm F} - L_{x, \rm N} }
	= {\sqrt 2\, \si(\De x) \over L_{x, \rm F} - L_{x, \rm N} }
	\approx {\sqrt{2} \cdot 5\un{\mu m}\over 1.6\un{m}} \approx 5 \un{\mu rad}\ ,
}{al sum th x err}
where $\si(\De x) \approx 5\un{\mu m}$ denotes a typical horizontal (residual) misalignment. For a scale feeling, the result may be compared to the (one-arm) beam divergence which was about $17\un{\mu rad}$, see \Sc{felm unfold}. Indeed, these two errors are different in nature -- beam divergence leads to a statistical error, whereas a misalignment to a systematic shift in $\th_x^*$. For the vertical direction, the alignment induced error may be estimated:
\eqref{
	\De\th_y^*
	= {1\over 2} \left( {\De y_{\rm F}\over L_{y, \rm F}} + {\De y_{\rm N}\over L_{y, \rm N}} \right)
	\approx {\si(\De y)\over L_y \sqrt 2}
	\approx {\si(\De y)\over 21\un{m} \cdot \sqrt 2}\ .
}{al sum th y err}
Again, $\si(\De y)$ denotes a typical vertical misalignment (after the whole alignment procedure). For the data from 29-30 October, where the method 2 could be used, the residual vertical misalignment is of the order $\si(\De y) \approx 15\un{\mu m}$. This leads to an angular shift $\De \th_y^* \approx 0.5\un{\mu rad}$. For the other data sets only the method 3 could be used. There, one may assume that this method reduces the residual misalignment to a half, that is $\si(\De y)\approx 50 \un{\mu m}$. This corresponds to an angular shift of $1.7\un{\mu rad}$.
