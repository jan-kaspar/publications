\input base
\input utf8-csf

\SetFontSizesX

\halign{\hbox to 4cm{\strut#\hss\ }&\vtop{\advance\hsize-4cm\noindent\strut#\strut}\cr
Title:		& Elastic scattering at the LHC\cr
Author:		& Jan Ka\v spar\cr
Departement:& Institute of Particle and Nuclear Physics\cr
Supervisor of the thesis:& RNDr.~Vojt\v ech Kundr\' at, DrSc., Institute of Physics AS CR\cr
%			& Mario Deile, PhD, CERN, Geneva\cr
%Abstract:	& [abstract of 80-200 words in English]\cr
Abstract:	& 
The seemingly simple elastic scattering of protons still presents a challenge for the theory. In this thesis we discuss the elastic scattering from theoretical as well as experimental point of view. In the theory part, we present several models and their predictions for the LHC. We also discuss the Coulomb-hadronic interference, where we present a new eikonal calculation to all orders of $\al$, the fine-structure constant. In the experimental part we introduce the TOTEM experiment which is dedicated, among other subjects, to the measurement of the elastic scattering at the LHC. This measurement is performed primarily with the Roman Pot (RP) detectors -- movable beam-pipe insertions hundreds of meters from the interaction point, that can detect protons scattered to very small angles. We discuss some aspects of the RP simulation and reconstruction software. A central point is devoted to the techniques of RP alignment -- determining the RP sensor positions relative to each other and to the beam. At the end we present the analysis of TOTEM's first elastic scattering measurement at the LHC. The resulting differential cross section is compared to model predictions.\cr
%Keywords:	& [ 3-5 keywords in English]\cr
Keywords:	& elastic scattering of protons, Coulomb-hadronic interference, TOTEM, Roman Pots, alignment\cr
}

\bye
