\def\FC{F^{\rm C}}
\def\FH{F^{\rm H}}
\def\FCH{F^{\rm C+H}}
\def\e{{\rm e}}

\chapter{Elastic scattering of protons}

\section[el models]{Hadronic models}

Normalization as in Cahn \bref{cahn82}

\eqref{{\d\si\over\d t} = {\pi\over s p^2} |F|^2}{el dsdt}
\eqref{\si_{\rm tot} = {4\pi\over p \sqrt{s}} \Im F(t = 0)}{el si tot}

\em{The model of Islam et al.}

\bref{islam87,islam03,islam04,islam06,islam07}

\> the structure of proton is very important for low $|t|$ el. scat
\> just putting several ideas together

\em{The model of Petrov, Predazzi and Prokudin}

\bref{ppp03}

\em{The model of Bourrely, Soffer and Wu}

\bref{wu70,bsw79,bsw84,bsw03,bsw10}

\em{The model of Block and Halzen}


\bref{bh99,bh11}

\section[el pred]{Predictions for the LHC}

\eqref{B(t) = {\d\over \d t} |F^{\rm N}|^2}{el B}
\eqref{\rh(t) = {\Re F^{\rm N}\over \Im F^{\rm N}}}{el rh}

\fig{fig/pdf/el_mod_dsdt_large.pdf}{el mod dsdt large}{Differential cross-section as predicted by the four models.}

\fig{fig/pdf/el_mod_dsdt_narrow.pdf}{el mod dsdt narrow}{Differential cross-section as predicted by the four models.}

\fig{fig/pdf/el_mod_B.pdf}{el mod B}{The elastic slope as predicted by the four models.}

\fig{fig/pdf/el_mod_phase.pdf}{el mod phase}{The elastic amplitude phase as predicted by the four models.}

\fig{fig/pdf/el_mod_rho.pdf}{el mod rho}{The $\rho$ parameter as predicted by the four models.}

\section[el impact]{Impact parameter analyses}

\section[el coulomb]{Coulomb interference}

\> Born vs. complete, that is the complete sum of the Born series.
\> Coulomb, hadronic and total amplitudes.
\> We will use high energy approximations to compare the results of several authors.

\def\KL{Kudr\' at-Lokaj\' i\v cek}
\def\KaL{Kudr\' at and Lokaj\' i\v cek}
\def\WY{West-Yennie}
\def\WaY{West and Yennie}

In \Sc{el models} we discussed the contribution of the strong interaction to the elastic scattering of nucleon. The strong interaction is obviously the most important one, however, there are three more fundamental forces: electromagnetic, weak and gravitational. The gravity is too weak on the mass and length scales typical for particle physics experiments and can be safely neglected therefore. The carriers of the weak force are much heavier that a typical four-momentum transfer $t\approx 1\un{GeV^2}$ and thus can be neglected too. The only other relevant interaction is the electromagnetic.

The scattering amplitude due to the electromagnetic force can be calculated in a number of ways. Here we will use two -- eikonal and QED. The former since the eikonal formalism will be used to discuss the Coulomb-hadron interference and the latter since we consider it the most appropriate framework.

\vskip\baselineskip
\em{Electromagnetic scattering in QED}
\vskip\baselineskip

QED is a relativistic QFT that can naturally account for the structure of the proton and its anomalous magnetic moment (see e.g. Sc.~6.2 in \bref{peskin} or Sc.~5.2 in \bref{chyla}). These are described by two form factors, the electric $G_E(t)$ and the magnetic $G_M(t)$, which are related to the distributions of charge and magnetization in the proton \bref{sachs62}. The leading contribution (one photon exchange, $\O{\al^2}$) to the cross-section has been calculated in \bref{block06} (Eq.~(29)). Since we are interested in the LHC energies ($\approx 7\un{Tev}$) and momentum transfers smaller than $|t| = 10\un{GeV^2}$, we may use high energy ($m/p\to 0$) and low-momentum transfer ($t/p^2\to 0$) limits. Then, the result reads (both for $\rm pp$ and $\rm \bar p p$ scattering)
\eqref{{\d\si^{\rm EM}_{\rm QED, OPE}\over\d t} = {4\pi \al^2\over t^2} G_{\rm eff}^4(t)\ .}{el dsdt em}
The symbol $G_{\rm eff}$ stands for an ''effective'' form factor that is a combination of the electric and magnetic form factors:
\eqref{G^2_{\rm eff}(t) = {G_E(t)^2 + \ta G_M(t)^2 \over 1 + \ta},\qquad \ta = - {t\over 4m^2}\ .}{el ff eff}
For very low four-momentum transfers $|t| \ll m^2$, the role of the magnetic form factor is negligible and one can put $G_{\rm eff} = G_E$. This is approximation is used almost universally (\bref{kl97} \TODO{many others}). However, as $|t|$ raises to about $m^2$, the influence of proton's magnetic moment can not be neglected anymore (as already stated by Block \bref{block06}).

The form factors of proton have typically been determined from the measurements of the elastic $\rm ep$ scattering. The electrons serve as a point-like projectile that probes the spatial structure of the proton. \TODO{universality of the form factors}.
This has been a number of analyses done, gradually accounting for the increasing amount of experimental data. One of the first results was the paper of Hofstadter \bref{hofstadter58}, who parameterized both form factors by a dipole formula
\eqref{G_E = F_D (1 - \ta\ka),\quad G_M = F_D (1+\ka),\quad F_D = \left( \La^2\over \La^2 - t \right)^2\ ,\qquad \La^2 = 0.71\un{GeV^2}\ .}{el ff dipole}
$\ka\doteq 1.793$ stands for the anomalous magnetic moment of the proton. In this parameterization, the effective form factor would be
\eqref{G^2_{\rm eff,D}(t) = F_{\rm D}^2 (1 + \ka^2 \ta)\ .}{el ff eff dipole}
Again, the approximation $G_{eff,D}\approx F_D$ can be justified only for $|t| \ll m^2$.

Hoffstadter's parameterization has been improved by Borkowski et al.~\bref{borkowski74,borkowski75}. They have parameterized electric and magnetic form factors by sums of four poles:
\eqref{
G_{E} = \sum_{i=1}^4 {{c_{E, i}}\over w_{E, i} - t},\qquad
G_{M} = \mu\sum_{i=1}^4 {{c_{M, i}}\over w_{M, i} - t}
\ .}{el ff borkowski}
The parameter values can be found in Tb.~4 in \bref{borkowski75}.

Kelly \bref{kelly04} proposed a class of form factor parameterization that is consistent with dimensional scaling at high $|t|$. His parameterization is given by a ratio of two polynomials
\eqref{G(t) = {\sum_k^n a_k \ta^k \over \sum_k^{n+2} b_k \ta^k}\ .}{el ff kelly}
Since the degree of the polynomial in the denominator is higher by two compared to the one in the numerator, the form factor falls off as $|t|^{-2}$ as $|t|\to\infty$. Kelly's parameterization has been used by Arrington \bref{arrington07} and Puckett \bref{puckett10} who has included more recent data sets. \TODO{TPE effect}.

\Fg{el ff comparison} compares the discussed form factors. We observe very little difference among Kelly's, Arrington's and Puckett's form factors. On the other hand, for $|t| \gs 0.1\un{GeV^2}$, there is significant difference between the effective (solid lines) and electric (dashed lines) form factors. The same holds for the difference between Hofstadter's effective form factor and the dipole function $F_D$ (dotted line).

\fig{fig/pdf/el_ff_comparison.pdf}{el ff comparison}{Comparison of electric (dashed) and effective (solid) form factors. The dotted line represents dipole expression $F_D$.}

\vskip\baselineskip
\em{Electromagnetic scattering in eikonal description}
\vskip\baselineskip

\iffalse
The terms ''eikonal'' or ''impact-parameter'' descriptions are used in a number of meanings, thus let's make it clear. Adachi et. al \bref{adachi65} formulated an \em{impact-parameter representation} of a scattering amplitude. That is an amplitude $F(s, t)$ is transformed into a function $A(s, b)$, where $b$ has the meaning of impact-parameter. This procedure is mathematically consistent, however, does not present an algorithm to calculate either of the amplitudes.

Islam developed his own \em{eikonal formalism} as a relativistic generalization of the eikonal approximation to QM. He defines an ''optical potential'' $V(s, r)$ (depending on the energy), which is the starting point in his algorithm. However, it is far from evident, what properties does this new ''potential'' shares with the one used in the traditional QM. Most importantly, whether the optical potential is additive.
\fi

The eikonal approximation is a high-energy and low-scattering-angle (i.e. the kinematic domain of elastic scattering at the LHC) in QM (see e.g. Sc.~2.3 in \bref{barone} or Sc.~4.2.7 in \bref{formanek qm}). It relates the scattering amplitude $F(t)$ to the eikonal function $\de(b)$
\eqref{F(t) = {s\over 2i}\int\limits_0^\infty b\,\d b\,J_0(b\sqrt{|t|})\,\left( \e^{2i\de(b)} - 1 \right)}{el eik rep}
(using the normalization of Cahn \bref{Cahn82}). The \em{eikonal} function can be calculated from a potential $V(x, y, z)$
\eqref{\de(b) = - {1\over 4 p_{\rm lab}} \int\limits_{-\infty}^{\infty} V(\vec b, z)\ \d z}{el eik app}
or from a Born amplitude
\eqref{\de(b) = {1\over s} \int\limits_0^\infty q\,\d q\,J_0(bq)\, F_{\rm Born}(t = -q^2)\ .}{el eik app born}

With the Coulomb potential $V(r) = \pm \al/r$ one gets
\eqref{F^{\rm C}_{\rm Born}(t) = \pm {\al s\over t}\ .}{el coul born}
This result is to be compared to the one-photon-exchange prediction by QED \Eq{el dsdt em} (both amplitudes are of the order $\O{\al}$). We see that the more general QED result differs only by the presence of the effective form factor $G_{\rm eff}$. We will use the same form factor also within the eikonal formalism:
\eqref{F^{\rm C}_{\rm Born}(t) = \pm {\al s\over t}\, G^2_{\rm eff} \.}{el coul born ff}
In that sense, the lattern amplitude includes more than just pure Coulomb scattering (magnetic influence is accounted for too), but we will keep calling the amplitude ``Coulomb''.

The Coulomb eikonal can, in principle, be calculated by \Eq{el eik app born}. However, a direct application to amplitudes \Eq{el coul born,el coul born ff} is not possible because of their singularity at $t=0$. This issue can be circumvented by introducing temporarily a fictitious photon mass $\la$. This would alter the denominators to $t - \la^2$ and the eikonal can be evaluated. The complete amplitude $\FC$ can then be obtained by \Eq{el eik rep} in a limit $\la \to 0$. An concrete example of such a calculation will be show later, when discussing the Coulomb interference.

\vskip\baselineskip

So far, we have discussed hadronic and electromagnetic interactions separately. Now we turn to the question what is the amplitude describing these forces acting simultaneously. This \em{Coulomb-hadronic} interference has been studied by a number of authors within a number of formalism. We will first review some of the most important results and eventually discuss the differences and provide numerical comparisons. \TODO{eikonal calculation -- new}

\vskip\baselineskip
\em{Interference in non-relativistic QM}
\vskip\baselineskip

One of the first authors to tackle the interference task was Bethe. In his work \bref{bethe58} he studied particle scattering by nuclei. He proposed a model based on two assumptions. First, the spatial distribution of protons and neutrons is the same. He parameterized this distribution by a Gaussian, with parameter $a$ governing its width (and thus the size of the nucleus). The second assumption is that the nuclear potential is proportional to the nucleon density. His total amplitude for $\rm pp$ and $\rm \bar pp$ scattering (upper and lower sign respectively) could be written:
\eqref{F^{\rm C+H}(t) = \pm {\al s\over t} \e^{i \al \Ph(t)} + F^{\rm H}(t)}{el F CH decomp C}
with the Coulomb-hadronic phase $\Phi$
\eqref{\Phi_{\rm Bethe}(t) = \pm 2 \log {1.06\over a\sqrt{|t|}}\ .}{el phase bethe}
We recall that the hadronic amplitude $F^{\rm H}$ reflects Bethe's hadronic model and thus one cannot use his interference formula for a generic hadronic model. This fact may seem as a major disadvantage nowadays. We are mentioning Bethe's work because most of his successors have derived formulae formally very similar to \Eq{el F CH decomp C}. The common feature is that the total amplitude $F^{\rm C+H}$ hides an infinite phase factor which is not observable in differential cross-section. In that sense, \Eq{el dsdt} is valid for $F^{\rm C+H}$ but \Eq{el si tot} not.

\vskip\baselineskip
\em{Interference in QFT / Feynman diagrams}
\vskip\baselineskip

\fig{fig/pdf/el_diagrams.pdf}{el diagrams}{Some of the Feynman diagrams contributing to proton-proton scattering. A: one photon exchange (OPE) diagram. B: a representation of complete hadronic scattering, that is the sum of all diagrams without photon exchange. C and D: the lowest order diagram contributing to the Coulomb-hadronic interference phase.}

%Meister and Yennie \bref{yennie63}

In perturbative QFT one obtains the scattering amplitude by summing all Feynman diagrams. Some of those contributing to proton-proton scattering are shown in \Fg{el diagrams}. The amplitude can be written as sum in powers of the fine structure constant $\al$
\eqref{\FCH = \FH \left(1 + \al f^{\rm H}_1 + \ldots \right) + \FC_{\rm OPE} \left(1 + \al f^{\rm C}_1 + \ldots \right)\ .}{el wy 1}
The second term describes diagrams containing photon exchanges only, $\FC_{\rm OPE}$ refers to the diagram A. Generally speaking, all diagrams with loops containing photon propagator are IR divergent because the zero photon mass. Yennie et al. \bref{yennie61} have shown that these divergences exponentiate into factors $A_{\rm C}$ and $A_{\rm H}$:
\eqref{\FCH = \FH \e^{\al A_{\rm H}} \left(1 + \al g^{\rm H}_1 + \ldots \right) + \FC_{\rm OPE} \e^{\al A_{\rm C}} \left(1 + \al g^{\rm C}_1 + \ldots \right)\ ,}{el wy 2}
where the (residual) functions $g^{\rm H, C}$ are IR-divergence free. The infinite factors $A_{\rm C}$ and $A_{\rm H}$ are caused by the inner bremsstrahlung and are partially compensated by the real soft bremsstrahlung. The latter process brings in factor $B$:
\eqref{\FCH =
\underbrace{\FH \e^{\al (\Re A_{\rm H} + B)} \left(1 + \al g^{\rm H}_1 + \ldots \right)}_{G^{\rm H}}
+
\underbrace{\FC_{\rm OPE} \e^{\al (\Re A_{\rm C} + B)} \left(1 + \al g^{\rm C}_1 + \ldots \right)}_{\FC}
\e^{i\al \Im (A_{\rm C} - A_{\rm H})}
}{el wy 3}
such that the terms $\Re A_{\rm H, C} + B$ are finite. In the last equation an infinite phase factor $\exp(-iA_{\rm H})$ has been absorbed to the amplitude $\FCH$, similarly as in Bethe's formula. We used the braces to identify the complete Coulomb amplitude $\FC$ and Coulomb-modified hadronic amplitude $G^{\rm H}$.

To the lowest order in $\al$ one may approximate
\eqref{G^{\rm H}(t) \approx \FH(t)\ ,\qquad \FC(t) \approx \FC_{\rm OPE}(t)}{el wy 4}
and thus the total amplitude reads
\eqref{\FCH(t) \approx \pm {\al s\over t} \e^{i \al \Ph(t)} + F^{\rm H}(t)\ ,}{el wy 5}
where we have identified the interference phase
\eqref{\Ph = \Im (A_{\rm C} - A_{\rm H})\ .}{el wy 6}
Note that this phase is, by construction, a real function.
\TODO{$g_1$ neglected in $G^{\rm H}$ but not in $\Ph$}

So far we have followed the steps of \WaY\ \bref{wy68}. Let us mention two preceding works of Rix and Thaler \bref{rix66} and Locher \bref{locher67}, both of which pursued a very similar approach. While Locher constrained himself to the case with $\FH(t) = \hbox{const}$, Rix and Thaler made no such an assumption and their final result is equivalent with the one of \WY.

\WY, later on, argued that the dominant contribution to the interference phase $\Ph$ comes from the diagrams \Fg{el diagrams} C (plus the inverted one) and D. While the evaluation of diagram D is, in principle, straight forward, diagram C presents a major difficulty -- its evaluation requires the knowledge of $\FH$ off the mass shell. In contrary, we are not aware of any hadronic model which provides also off shell amplitude (cf.~\Sc{el models}). The authors tried to evaluate at least a contribution to the phase $\Ph$ which is independent on the off shell amplitude. They used a number of simplifying assumptions some of which can be partially justified if the phase of the hadronic amplitude $\FH$ is not rapidly changing. While this point will be discussed later on, the final result of \WY{} reads\footnote{%
Note that in \bref{wy68} there is a number of inconsistencies between Eq.~(17) and (23), the latter one is likely to contain a misprint: the sign of the second term shall read minus, i.e. as in our \Eq{el phase WY}.
}:
\eqref{\Phi_{\rm WY}(t) = \mp \log {t\over t_{\rm min}} \pm \int\limits_{t_{\rm min}} {\d t'\over |t' - t|} \left( {\FH(t')\over \FH(t)} - 1 \right)\ .}{el phase WY}
The authors the uncertainty of this formula to be of the order of $\al$.

In the time when \WY{} published their work, it seem perfectly plausible to describe the hadronic amplitude as $F^H(t) \propto \e^{Bt/2}$, where $B$ is the diffractive slope (constant). Then the interference phase would simplify (in the limit of low scattering angles and high energies) to
\eqref{\Phi_{\rm SWY}(t) = \mp \left(\log {B|t|\over 2} + \ga \right) \ ,}{el phase SWY}
where $\ga \doteq 0.577$ is the Euler's constant.



\vskip\baselineskip
\em{Interference in eikonal description}
\vskip\baselineskip

If electromagnetic and hadronic interactions could be described by potentials $V^{\rm C}$ and $V^{\rm H}$, the total potential would be given by their sum $V^{\rm C} + V^{\rm H}$. Since the eikonal is linear in potential, see \Eq{el eik app}, the total eikonal is a sum of Coulomb and hadronic eikonals
\eqref{\de^{\rm C+H}(b) = \de^{\rm C}(b) + \de^{\rm H}(b)\ .}{el de CH}
The total scattering amplitude can be obtained from \Eq{el eik rep}
\eqref{\FCH(t) = {s\over 2i}\int\limits_0^\infty b\,\d b\,J_0(b\sqrt{|t|})\,\left( \e^{2i\de^{\rm C+H}(b)} - 1 \right)\ .}{el F CH eik}

These steps can be found in the beginning of a number of works, aiming primarily at rederiving the formulae of Bethe \bref{bethe58} and \WY{} \bref{wy68}.
  For example Islam \bref{islam67}\footnote{%
\TODO{His own eikonal formalism}
} rederived the formula of Bethe and found it more accurate than Bethe's original assumptions.
%He also obtained small corrections \TODO{due to ...}.
  Franco \bref{franco73} used the eikonal formalism to derive the formula of \WY{} (in $\O{\al}$ approximation). He also obtained an interference formula at all $\al$ orders, unfortunately only for a hadronic model with a constant phase and a constant diffractive slope.
  Buttimore et al. \bref{buttimore78} extended (postulated) the eikonal formalism also for spin-dependent scattering. Again unfortunately, they limited themselves to a hadronic model with a constant slope.

Cahn \bref{cahn82} also started by rederiving the formula of \WY. By inserting \Eq{el de CH} to \Eq{el F CH eik} he obtained
\eqref{\FCH(t) = \FC(t)
+ \FH(t) \left[ 1 +
{s\over 2i \FH(t)}\int\limits_0^\infty b\,\d b\,J_0(b\sqrt{|t|})\,\left( \e^{2i\de^{\rm C}(b)} - 1 \right) \left( \e^{2i\de^{\rm H}(b)} - 1 \right)
\right ]\ .
}{el cahn 1}
For the complete Coulomb formula $\FC$ he obtained (in the lowest order in $\al$)
\eqref{\FC(t) = \pm {\al s\over t} \e^{i\al \et(t)}\ ,\qquad \et(t) = \log {\la^2\over -t}\ .}{el cahn 2}
The only difference compared to the Born approximation \Eq{el coul born} is the presence of the Coulomb phase $\et(t)$. This phase diverges for vanishing $\la$ and therefore it is not possible to apply the $\la\to 0$ limit in a straight-forward manner. However, Cahn showed that the terms in square brackets in \Eq{el cahn 1} have the same $\log\la^2$ divergence in their phase. Thus it is possible to factor out this term and absorb it into the amplitude $\FCH$ (just as in the treatments of Bethe and \WY). Cahn's final result reads
\eqref{\FCH(t) = \pm {\al s\over t} + \FH(t)\, \big[1 + i \al \Ps(t) \big]}{el F CH decomp H lin}
with the interference phase:
\eqref{\Ps_{\rm Cahn} = \pm \lim\limits_{T \to -\infty} \left[ \log {t\over T} - \int\limits_{T}^{0} {\d t'\over |t' - t|} \left ( {\FH(t')\over \FH(t)} - 1 \right) \right] + \O{\al}}{el phase Cahn}
Within the $\O{\al}$ approximation in the phase, one can complete the term $1+i\al\Ps$ to the exponential:
\eqref{\FCH(t) = \pm {\al s\over t} + \FH(t)\, \e^{i \al \Ps(t)}\ .}{el F CH decomp H}
This expression is similar to \Eq{el F CH decomp C,el wy 5}, only the phase factor is attached to the hadronic amplitude. That is why we used symbol $\Ps$ for the phase here, to distinguish it from phases $\Ph$ that are attached to the Coulomb amplitude.

Then, Cahn turned to the discussion of the influence of the electromagnetic form factors. He calculated the complete Coulomb amplitude with a low-$t$ approximation of $G_{\rm eff}(t) = F_{\rm D}(t)$ (now the calculation starts with Born amplitude \Eq{el coul born ff}). The form was just as in \Eq{el cahn 2}, just the Coulomb phase gained a shift:
\eqref{\et \rightarrow \et + \nu\ ,}{el FF influence}
\eqref{\nu_{\rm Cahn} = - {4t\over \La^2} \log {\La^2\over -4t} + {2t\over \La^2}}{el nu Cahn}
($\La$ is the parameter of the dipole form factor, see \Eq{el ff dipole}).

Cahn eventually derived an interference formula for a generic hadronic amplitude and a generic set of form factors, however his formula was unnecessary complicated. This issue has been solved by \KaL{} \bref{kl94}. Moreover they have identified Cahn's integration bound $T$ (see \Eq{el phase Cahn}) with the boundary of the kinematically allowed region $t_{\rm min}$. Such that the integration in \Eq{el phase Cahn} is constrained to the physical region only. Their result reads:
\eqref{\Ps_{\rm KL} =
	\mp \int\limits_{t_{\rm min}}^{0} \log {t'\over t}\ {\d\phantom{t'}\over\d t'} G_{\rm eff}^2(t')
	\pm \int\limits_{t_{\rm min}}^{0} \d t' \left ( {\FH(t')\over \FH(t)} - 1 \right)\, {I(t, t')\over 2\pi}
	+ \O{\al}
}{el phase KL}
where
\eqref{I(t, t') = \int\limits_{0}^{2\pi}\d\ph\ {G_{\rm eff}^2(t'')\over t''}\ ,\qquad t'' = t + t' + 2\sqrt{t t'} \cos\ph\ .}{el KL I}
The authors used the assumption of vanishing form factors at $t_{\rm min}$. Unfortunately, this makes the connection with Cahn's formula \Eq{el phase Cahn} uneasy. But one can easily recover the lost term and correct the result of \KL{}:
\eqref{\Ps_{\rm CKL} =
	\mp \int\limits_{T}^{0} \log {t'\over t}\ {\d\phantom{t'}\over\d t'} G_{\rm eff}^2(t')
	\pm \int\limits_{T}^{0} \d t' \left ( {F^H(t')\over F^H(t)} - 1 \right)\, {I(t, t')\over 2\pi}
	\pm G_{\rm eff}^2(T)\,\log {t\over T}
	+ \O{\al}
}{el phase CKL}
In the limit $G_{\rm eff}^2\to 1$ one finds (see Eq.~(25) in \bref{Cahn82})
$${I(t, t')\over 2\pi}\to - {1\over |t' - t|}$$
and thus $\Ps_{\rm CKL}$ coincides with $\Ps_{\rm Cahn}$ (indeed, in the limit $T\to -\infty$).

The work of Selyugin \bref{selyugin99} provides a refinement of Cahn's $\nu$ calculation. In contrary to Cahn, Selyugin used the full dipole form factor $F_{\rm D}$. However both calculations have been performed on the same level of approximation -- the phase modification has been extracted from oder $\al$ and $\al^2$ contributions to $\FC$. Selyugin's expression for $\nu$ is rather complicated and thus he has proposed a simple form that well approximates the full result:
\eqref{\nu_{\rm Selyugin} = c_1 \log (1 - c_2^2 t),\qquad c_1 = 0.11,\qquad c_2 = 200\un{GeV^{-1}}\ .}{el nu Selyugin}

One of the most recent works on the field of the Coulomb-interference is from Kopeliovich et al. \bref{kopeliovich01}. They aimed at describing proton-carbon collisions and from the beginning they fixed the electromagnetic form factors (exponential) and hadronic model (constant slope and phase).

\vskip\baselineskip
\em{Critique and discussion}
\vskip\baselineskip

\> two-photon exchange, QFT effects missing in eikonal formalism, running coupling constant.

\> Form-factors in $ep$ and $pp$

\WaY{} used a series of assumptions in deriving their interference phase \Eq{el phase WY}. One of the most limiting was that the phase of the hadronic amplitude is only slowly varying function. They wrote: ``One can hope, therefore, that in the region where $\Ph$ gets its major contribution, corrections due to the variation of the phase of $F^H$ are not large. Nevertheless, this represents another source of our ignorance whose magnitude cannot be estimated without detailed knowledge of the strong interactions.'' Therefore it makes no sense applying WY formula to hadronic models with a rapidly varying phase which, unfortunately, is the case for most phenomenological models -- see \Fg{el mod phase}. In such cases, the interference ``phase'' (\rhs{} of \Eq{el phase WY}) turns out complex, however the phase has been derived as an imaginary part of a quantity (thus purely real) -- see \Eq{el wy 6}. The imaginary component of the phase $\Ph_{\rm WY}$ can be seen in \Fg{el cic noff Psi} (right). In similar words, the interference phase of \WaY{} can only be real if the phase of the hadronic amplitude is constant (this theorem has been proved by \bref{klv07}).

The simplified WY formula, \Eq{el phase SWY}, brings yet another assumption -- constant slope $B$, which is undoubtedly ruled out by experimental data.
No wonder that the prediction of SWY formula deviates significantly the from WY predictions for $|t| \gs 3\cdot 10^{-2}\un{GeV^2}$, see \Fg{el cic noff Psi,el cic diff Psi ff}. We may only wonder why it is still being used, see for instance Eq.~2-7 in ATLAS ALFA TDR \bref{alfa}. May be even higher level of ignorance has been shown in works \bref{ppp03,ppp05}. The authors fit $\rm pp$ and $\rm\bar pp$ scattering data with their own hadronic model and several interference formulae. Even though their models have non-constant slopes (see red curves in \Fg{el mod B}) and non-constant phase (see \Fg{el mod phase}) they have used simplified \WY{} formula. They have included Cahn's formula too, unfortunately the simplified one, which assumes constant slope $B$.

\TODO{Other ignorants? BSW?}

Let's discuss Cahn's interference formula \Eq{el phase Cahn} now. Using his words, he has ``abused'' a relation (used it beyond its domain of validity) in deriving the complete Coulomb amplitude \Eq{el cahn 2}. His interference phase is clearly neglecting terms of order $\O{\al}$. Therefore one may naively expect a precision of $\al\approx 1\percent$. However a question is whether some enhancements may appear. In addition, he has performed rather benevolent manipulations with terms like
$$\al \log {t\over t'}\ ,$$
where $t'$ is an integration variable ranging from $t_{\rm min}$. Taking some values of interest: $|t| = 10^{-3}\un{GeV^2}$ and $|t_{\rm min}| \approx s = (14\un{TeV})^2$ yields a correction of $-24\percent$, which is rather important.

All the above doubts led us to verify Cahn's formula by a numerical calculation based on the eikonal formula \Eq{el F CH eik}. To carry out the calculations we had to use the fictitious photon mass $\la$ non-zero. We have chosen, however, such low value that it has had no influence on the results in withing the $|t|$ range of our interest. To demonstrate this, we have used three values of $\la$: $10^{-2}$, $10^{-3}$ and $10^{-4}\un{GeV}$. Then whenever the three corresponding results overlap, it means no dependence on $\la$. \TODO{red curve often off}

\Fg{el cic noff FC} shows our eikonal calculation of the complete Coulomb amplitude $\FC(t)$ with no form factor ($G_{\rm eff}(t) = 1$). Our calculation is in perfect agreement with Cahn's result \Eq{el cahn 2}. When the form factor of Puckett has been used, we have obtained \Fg{el cic diff FC}. It supports the hypothesis of Cahn, that the presence of a form factor modifies mostly the phase of the complete Coulomb amplitude -- its modulus is indistinguishable from the Born approximation. The form-factor-induced modification of the phase, expressed by function $\nu(t)$, is shown in \Fg{el cic diff nu}.One can see that the modification is rather small (it is to be compared with the phase with no form factor, see \Fg{el cic noff FC} right). Cahn's expression for $\nu$ matches with our calculation rather well for $|t| \ls 10^{-2}\un{GeV^2}$, Selyugin's is not too bad up to $0.2\un{GeV^2}$. Let us recall that both expression have been derived assuming the dipole form factor, thus they shall be compared to the red curve.

\fig{fig/pdf/el_cic_noff_F_C.pdf}{el cic noff FC}{Complete Coulomb amplitude calculated with \Eq{el F CH eik}, no form factor used. Left: the modulus of the amplitude (for three values of $\la$) compared to the Born approximation \Eq{el coul born ff} (overlapped with the green and blue curves). Right: the argument of the amplitude (solid lines) compared to Cahn's $\eta^C$ (dashed lines, often hidden by the solid lines).}

\bmfig
\fig{fig/pdf/el_cic_diff_F_C.pdf}{el cic diff FC}{[7cm]The modulus of the complete Coulomb amplitude calculated with the eikonal formula and Puckett form factor. The Born curve is overlapped with the green and blue curves.}
%
\fig{fig/pdf/el_cic_diff_nu.pdf}{el cic diff nu}{[7cm]The form factor induced Coulomb phase alternation $\nu$. Colorful lines calculated with eikonal formula.}
\emfig

Now we are going to compare the interference phase calculated with the eikonal, \WY{} and (corrected) \KL{} formulae. For simplicity, we will use only one hadronic model. We have chosen the model of Bourrely et al. (it is one of the models giving a prescription for the hadronic eikonal $\de_H(b)$ and therefore \Eq{el F CH eik} can be used directly).

\TODO{comment of $\Ps$ vs. $-\Ph$}

\Fg{el cic noff Psi} compares the interference phases with no form factor used. The simplified \WY{} formula does not provide a reasonable approximation above $|t| \gs 10^{-2}\un{GeV^2}$. \WY{} and \KL{} results are identical, as they should, however, they are significantly different from our eikonal calculation. We believe this is the result of the $\O{\al}$ approximation in \Eq{el phase WY,el phase Cahn}.

\Fg{el cic diff Psi ff} compares the interference phases calculated with different form factors. In addition, it shows again that the SWY approximation is not admissible for higher $|t|$ values.

\Fg{el cic diff Psi eik} compares the \KL{} formula to our eikonal calculation. Whereas the real part shows significant differences (even for rather low $|t|$ values), the both formulae agree well on the imaginary part.


\fig{fig/pdf/el_cic_noff_Psi.pdf}{el cic noff Psi}{Comparison of Coulomb-interference phases with no form factor. Solid curves have been calculated with the eikonal formula. WY and CKL results are entirely overlapping.}

\fig{fig/pdf/el_cic_diff_Psi_ff.pdf}{el cic diff Psi ff}{The Coulomb-interference phases with different form factors and according to different formulae.}

\fig{fig/pdf/el_cic_diff_Psi_eik.pdf}{el cic diff Psi eik}{A comparison of Coulomb-interference phases with Puckett form factor, between CKL and eikonal formulae.}

Cahn has found his result \Eq{el phase Cahn} agree with the one of \WY{} \Eq{el phase WY}. However, we can see two differences. The first, the limit $T\to -\infty$ in Cahn's formula -- we will discuss it a bit later, it will turn out that it is no real issue. The second difference is related to the decompositions \Eq{el F CH decomp C,el F CH decomp H}. In the case of Cahn the phase factor is plugged to the hadronic amplitude, in the case of \WaY{} to the Coulomb amplitude. If the phases were real, then the formulae would be equivalent if $\Ps_{\rm Cahn}(t) = - \Ph_{\rm WY}(t)$ (the difference in phase could be absorbed to $\FCH(t)$). However, as it has been discussed already, the phases have non-negligible imaginary parts.

One of the quantities that can demonstrate the difference between Cahns's (or corrected \KL) and \WY{} formulae is the \em{interference importance} function
\eqref{Z(t) = {|F^{\rm C+H}(t)|^2 - |F^{\rm C}(t)|^2 - |F^{\rm H}(t)|^2\over |F^{\rm C+H}(t)|^2}\ ,}{el Z}
which gives a relative importance of the Coulomb interference in the cross-section. \Fg{el cic noff Z} shows this function plotted for the case of no form factor. Up to $|t| \approx 0.1\un{GeV^2}$ all formulae agree, but then a dramatic difference occurs at the position of the first diffractive dip ($t\approx 0.5\un{GeV^2}$). This is the point where the imaginary parts of the phases reach large values, cf. \Fg{el cic noff Psi}.

\fig{fig/pdf/el_cic_noff_Z.pdf}{el cic noff Z}{Left: the interference importance function $Z(t)$. Right: the $\ze(t)$. Both: no form factor included, solid lines represent the eikonal calculation.}

In fact, the interference importance function can gain non-negligible values if the hadronic and Coulomb amplitudes are comparable in modulus. For example imagine $|\FC(\ta)| \ll |\FH(\ta)|$ at a given point $\ta$. Then, no matter what decomposition \Eq{el F CH decomp C,el F CH decomp H} we take, $|\FCH(\ta)|^2 \approx |\FH(\ta)|^2$ and thus $Z(\ta) \approx 0$. That is why certain features of the $Z(t)$ dependence are given by the relative ratio of both amplitudes. We expect the interference importance function to have a similar modulation as 
\eqref{\ze(t) = {2\over {|\FH(t)|\over |\FC(t)|} + {|\FC(t)|\over |\FH(t)|}}\ .}{el zeta}
$\ze(t) \approx 1$ when both amplitudes are similar and tends to zero if either of them dominates. The $\ze$ function, for the no-form-factor case, is plotted in \Fg{el cic noff Z} and indeed, the peak structure is similar as for $Z(t)$. The first broad peak corresponds to the crossing of Coulomb and hadronic amplitudes, the next ones correspond to the diffractive dips of the hadronic models, where the hadronic amplitude drops closer to the Coulomb one.

\Fg{el cic diff Z ff} shows the interference importance functions for several form factor choices. Again, the largest difference occurs at the position of the first dip and again, SWY approximation is not suitable for higher $|t|$ values. A comparison of \KL{} formula to our eikonal calculation can be found in \Fg{el cic diff Z eik}. Unlike the case with no form factor, there is negligible difference here.

\fig{fig/pdf/el_cic_diff_Z_ff.pdf}{el cic diff Z ff}{The importance of the Coulomb-interference ($Z$ function) for different form factors and different formulae.}

\fig{fig/pdf/el_cic_diff_Z_eik.pdf}{el cic diff Z eik}{Left: the importance of the interference $Z(t)$, Right: the $\ze(t)$ function. Both: the Puckett form factor has been used.}

\TODO{WY wrong, Cahn OK: weird}

\> $\de^{\rm H}$ problem
\>> de H from F H, not know for $t < t_{min}$
\>> most models: de H ~ Gauss, si ~ few fm; oscillation theorem; $F\to 0$ as $t\to -\infty$
\>> ${\FH(t')\over \FH(t)}$ negligible for $t' < T$ then
$$\Psi \approx -\log {t\over T} + \int_{T} \ldots$$
\>> limit in Cahn doesn't make it differ from WY

\hbox{\vrule\hskip1cm\vbox{\advance\hsize-2cm
Cahn's work lies entirely in within the eikonal framework. Regarding the hadronic interaction, the assumed input is the eikonal $\de^H(b)$. However, if a hadronic model is not eikonal-formulated, one would be given the amplitude in the $t$-space $F^H(t)$. Then, the hadronic eikonal $\de_H$ could, in principle, be obtained by means of Fourier-Bessel transform:
\eqref{\de^H(b) = {1\over 2i} \log \left[ 1 + {2i\over s} \int\limits_0^\infty q\ \d q\ J_0(b q)\ F^H(-q^2)  \right ]\ .}{}
However, it is immediately clear that this transform requires the knowledge of the hadronic amplitude also outside the physical region\footnote{There might be an interesting parallel between the complications in adopting eikonal and QFT approaches. In the former, one is missing the hadronic amplitude beyond the physical region, in the latter, one is missing the amplitude off mass shell.}.

\KaL{} tried to solve this issue simply by taking Cahn's final result and truncating the integrals at $Q^2 = - t_{\rm min}$. This step, if unjustified, may seem very controversial. Cahn assumed that $Q^2\to\infty$ throughout his work, e.g. his crucial Eq.~(15) would not hold otherwise.

It is clear that one may hardly argue about $F^H$ behavior in the unphysical region on a general level. Therefore let's focus on the group of eikonal hadronic models, in fact most of those presented in \Sc{el models}. The eikonal amplitudes \TODO{reference} of these models are typically of a Gaussian profile \TODO{reference} \Sc{el impact} with RMS of about \TODO{value}. The $t$-amplitude can be calculated through \Eq{el F CH eik}. Now, for very high $|t|$ values (like in the unphysical region) the factor $J_0(b\sqrt{-t})$ becomes a rapidly oscillating function and from the ``oscillation theorem'' \TODO{reference} it becomes clear that $F^H$ must be heavily suppressed. This can be seen even for much lower $|t|$ values, even in \Fg{el mod dsdt narrow}. Therefore, the contribution to the interference term in Cahn's Eq.~15
$$\int \d^2 \vec q' F^C(\vec q') F^H([\vec q - \vec q']^2)$$
from sufficiently high $q'^2$ will be negligible. This means that would could cut the integration region at $q'^2 = \mu^2$, where $\mu$ is the sufficiently large constant. Now, following the steps of \KaL{}, one would obtain their formula, just with $t_{\rm min}$ replaced with $-\mu^2$.

The last statement could be demonstrated directly in Eq.~(26) in \bref{kl94}. Let's split the integration region into two: $(t_{\rm min}, -\mu^2)$ and $(-\mu^2, 0)$. Again, $\mu$ is such that $F^H(-\mu^2)$ is small (or more precisely negligible wrt. the denominator $F^H(t)$). Then one can show that the integral over the first integration region vanishes. That is, the lower bound is quite arbitrary, it only needs to be value large enough.

\eqref{F^{\rm C+H}_{\rm WY} = F^C(t) \e^{i\al\Ph(t) + F^H(t) }}{el WY}
\TODO{Comment on FC and FH, c.f. Eq.~(6) in \bref{wy68}}

\eqref{\Phi_{\rm WY}(t) =  -\al\log {t\over t_{\rm min}} + \int\limits_{t_{\rm min}}^0 {\d t'\over |t-t'|} \left({F^H(t')\over F^H(t)} - 1\right) }{el phase WY}
\TODO{Check the sign of the second term!!!}
}}

\> problems of KL

\>> add hoc cut off at $t_{min}$
The relation between hadronic $t$- and $b$-amplitudes require the knowledge of the $t$-amplitude beyond the physical region. Cahn assumes this is know, the upper bound of his $q^2$-integrals is $Q^2\to\infty$. \KL{} realized the issue and fixed the upper bound to the edge of the physical region $Q^2 = t_{\rm min}$. This brings some inconsistency.

\>> $\nu$ missing; the modification of the Coulomb phase is not discussed (c.f. Sec.~4 in \bref{cahn82})


\vskip\baselineskip
\em{Coulomb interference for different hadronic models}
\vskip\baselineskip

\> \KL{} formula and Puckett form factor.

\eqref{C(t) = {|F^{\rm C+H}(t)|^2 - |F^{\rm H}(t)|^2\over |F^{\rm H}(t)|^2}}{el C}

\eqref{R(t) = {|F^{\rm C+H}_{\rm CKL}(t)|^2 - |F^{\rm C+H}_{\rm SWY}(t)|^2 \over |F^{\rm C+H}_{\rm CKL}(t)|^2}}{el R}

\fig{fig/pdf/el_mod_C.pdf}{el mod C}{The influence of Coulomb interaction (see \Eq{el C}) as a function of $t$. Calculated with CKL formula and Puckett form factor.}

\fig{fig/pdf/el_mod_Z.pdf}{el mod Z}{The importance of the interference term (see \Eq{el Z}) as a function of $t$. Calculated with CKL formula and Puckett form factor.}

\fig{fig/pdf/el_mod_R.pdf}{el mod R}{The difference between WY and KL formulae as a function of $t$. Calculated with CKL formula and Puckett form factor.}

\vskip\baselineskip
\em{Conclusion}
\vskip\baselineskip

We have verified the numerical accuracy of Cahn and \KL{} interference formulae by comparing them to the eikonal one \Eq{el F CH eik}. We have found only one case of disagreement -- the form factor effect in the interference phase $\Psi$. However, this discrepancy is not visible in cross sections.

