\chapter{Elastic scattering of protons}

\section[el models]{Hadronic models}

Normalization as in Cahn \bref{cahn82}

\eqref{{\d\si\over\d t} = {\pi\over s p^2} |F|^2}{el dsdt}
\eqref{\si_{\rm tot} = {4\pi\over p \sqrt{s}} \Im F(t = 0)}{el si tot}

\em{The model of Islam et al.}

\bref{islam87,islam03,islam04,islam06,islam07}

\> the structure of proton is very important for low $|t|$ el. scat
\> just putting several ideas together

\em{The model of Petrov, Predazzi and Prokudin}

\bref{ppp03}

\em{The model of Bourrely, Soffer and Wu}

\bref{wu70,bsw79,bsw84,bsw03,bsw10}

\em{The model of Block and Halzen}


\bref{bh99,bh11}

\section[el pred]{Predictions for the LHC}

\eqref{B(t) = {\d\over \d t} |F^{\rm N}|^2}{el B}
\eqref{\rh(t) = {\Re F^{\rm N}\over \Im F^{\rm N}}}{el rh}

\fig{fig/pdf/el_mod_dsdt_large.pdf}{el mod dsdt large}{Differential cross-section as predicted by the four models.}

\fig{fig/pdf/el_mod_dsdt_narrow.pdf}{el mod dsdt narrow}{Differential cross-section as predicted by the four models.}

\fig{fig/pdf/el_mod_B.pdf}{el mod B}{The elastic slope as predicted by the four models.}

\fig{fig/pdf/el_mod_phase.pdf}{el mod phase}{The elastic amplitude phase as predicted by the four models.}

\fig{fig/pdf/el_mod_rho.pdf}{el mod rho}{The $\rho$ parameter as predicted by the four models.}

\section[el impact]{Impact parameter analyses}

\section[el coulomb]{Coulomb interference}

\fig{fig/pdf/el_diagrams.pdf}{el diagrams}{Diagrams.}

Born vs. complete, that is the complete sum of the Born series.

Coulomb, hadronic and total amplitudes.

\def\KL{Kudr\' at-Lokaj\' i\v cek}
\def\KaL{Kudr\' at and Lokaj\' i\v cek}
\def\WY{West-Yennie}
\def\WaY{West and Yennie}

In \Sc{el models} we discussed the contribution of the strong interaction to the elastic scattering of nucleon. The strong interaction is obviously the most important one, however, there are three more fundamental forces: electromagnetic, weak and gravitational. The gravity is too weak on the mass and length scales typical for particle physics experiments and can be safely neglected therefore. The carriers of the weak force are much heavier that a typical four-momentum transfer $t\approx 1\un{GeV^2}$ and thus can be neglected too. The only other relevant interaction is the electromagnetic.

The scattering amplitude due to the electromagnetic force can be calculated in a number of ways. Here we will use two -- eikonal and QED. The former since the eikonal formalism will be used to discuss the Coulomb-hadron interference and the latter since we consider it the most appropriate framework.

\vskip\baselineskip
\em{Electromagnetic scattering in QED}
\vskip\baselineskip

QED is a relativistic QFT that can naturally account for the structure of the proton and its anomalous magnetic moment (see e.g. Sc.~6.2 in \bref{peskin} or Sc.~5.2 in \bref{chyla}). These are described by two form factors, the electric $G_E(t)$ and the magnetic $G_M(t)$, which are related to the distributions of charge and magnetization in the proton \bref{sachs62}. The leading contribution (one photon exchange, $\O{\al^2}$) to the cross-section has been calculated in \bref{block06} (Eq.~(29)). Since we are interested in the LHC energies ($\approx 7\un{Tev}$) and momentum transfers smaller than $|t| = 10\un{GeV^2}$, we may use high energy ($m/p\to 0$) and low-momentum transfer ($t/p^2\to 0$) limits. Then, the result reads (both for $\rm pp$ and $\rm \bar p p$ scattering)
\eqref{{\d\si^{\rm EM}_{\rm QED, OPE}\over\d t} = {4\pi \al^2\over t^2} G_{\rm eff}^4(t)\ .}{el dsdt em}
The symbol $G_{\rm eff}$ stands for an ''effective'' form factor that is a combination of the electric and magnetic form factors:
\eqref{G^2_{\rm eff}(t) = {G_E(t)^2 + \ta G_M(t)^2 \over 1 + \ta},\qquad \ta = - {t\over 4m^2}\ .}{el ff eff}
For very low four-momentum transfers $|t| \ll m^2$, the role of the magnetic form factor is negligible and one can put $G_{\rm eff} = G_E$. This is approximation is used almost universally (\bref{kl97} \TODO{many others}). However, as $|t|$ raises to about $m^2$, the influence of proton's magnetic moment can not be neglected anymore (as already stated by Block \bref{block06}).

The form factors of proton have typically been determined from the measurements of the elastic $\rm ep$ scattering. The electrons serve as a point-like projectile that probes the spatial structure of the proton. \TODO{universality of the form factors}.
This has been a number of analyses done, gradually accounting for the increasing amount of experimental data. One of the first results was the paper of Hofstadter \bref{hofstadter58}, who parameterized both form factors by a dipole formula
\eqref{G_E = F_D (1 - \ta\ka),\quad G_M = F_D (1+\ka),\quad F_D = \left( \La^2\over \La^2 - t \right)^2\ ,\qquad \La^2 = 0.71\un{GeV^2}\ .}{el ff dipole}
$\ka\doteq 1.793$ stands for the anomalous magnetic moment of the proton. In this parameterization, the effective form factor would be
\eqref{G^2_{\rm eff,D}(t) = F_{\rm D}^2 (1 + \ka^2 \ta)\ .}{el ff eff dipole}
Again, the approximation $G_{eff,D}\approx F_D$ can be justified only for $|t| \ll m^2$.

Hoffstadter's parameterization has been improved by Borkowski et al.~\bref{borkowski74,borkowski75}. They have parameterized electric and magnetic form factors by sums of four poles:
\eqref{
G_{E} = \sum_{i=1}^4 {{c_{E, i}}\over w_{E, i} - t},\qquad
G_{M} = \mu\sum_{i=1}^4 {{c_{M, i}}\over w_{M, i} - t}
\ .}{el ff borkowski}
The parameter values can be found in Tb.~4 in \bref{borkowski75}.

Kelly \bref{kelly04} proposed a class of form factor parameterization that is consistent with dimensional scaling at high $|t|$. His parameterization is given by a ratio of two polynomials
\eqref{G(t) = {\sum_k^n a_k \ta^k \over \sum_k^{n+2} b_k \ta^k}\ .}{el ff kelly}
Since the degree of the polynomial in the denominator is higher by two compared to the one in the numerator, the form factor falls off as $|t|^{-2}$ as $|t|\to\infty$. Kelly's parameterization has been used by Arrington \bref{arrington07} and Puckett \bref{puckett10} who has included more recent data sets. \TODO{TPE effect}.

\Fg{el ff comparison} compares the discussed form factors. We observe very little difference among Kelly's, Arrington's and Puckett's form factors. On the other hand, for $|t| \gs 0.1\un{GeV^2}$, there is significant difference between the effective (solid lines) and electric (dashed lines) form factors. The same holds for the difference between Hofstadter's effective form factor and the dipole function $F_D$ (dotted line).

\fig{fig/pdf/el_ff_comparison.pdf}{el ff comparison}{Comparison of electric (dashed) and effective (solid) form factors. The dotted line represents dipole expression $F_D$.}

\vskip\baselineskip
\em{Electromagnetic scattering in eikonal description}
\vskip\baselineskip

The terms ''eikonal'' or ''impact-parameter'' descriptions are used in a number of meanings, thus let's make it clear. Adachi et. al \bref{adachi65} formulated an \em{impact-parameter representation} of a scattering amplitude. That is an amplitude $F(s, t)$ is transformed into a function $A(s, b)$, where $b$ has the meaning of impact-parameter. This procedure is mathematically consistent, however, does not present an algorithm to calculate either of the amplitudes.

Islam developed his own \em{eikonal formalism} as a relativistic generalization of the eikonal approximation to QM. He defines an ''optical potential'' $V(s, r)$ (depending on the energy), which is the starting point in his algorithm. However, it is far from evident, what properties does this new ''potential'' shares with the one used in the traditional QM. Most importantly, whether the optical potential is additive.

\em{Eikonal approximation} is ... (see e.g. Sc.~2.3 in \bref{barone} or Sc.~4.2.7 in \bref{formanek qm}). It sais
\eqref{F(t) = {s\over 2i}\int b\,\d b\,J_0(b\sqrt{|t|})\,\left( e^{2i\de(b)} - 1 \right)}{el eik rep}
(using the notation of Cahn \bref{Cahn82}). The \em{eikonal} function is given as
\eqref{\de(b) = - {1\over 4 p_{\rm lab}} \int\limits_{-\infty}^{\infty} V(\vec b, z)\ \d z,}{el eik app}
where $V(x, y, z)$ is a potential. As the eikonal is linear in potential, it retains the property of additivity \TODO{verified by \bref{rix66}}. The eikonal can also be related to the Born approximation of a scattering amplitude
\eqref{\de(b) = {1\over s} \int q\,\d q\,J_0(bq)\, F_{\rm Born}(t = -q^2)}{el eik app born}

With the Coulomb potential $V(r) = \pm \al/r$ one gets
\eqref{F^{\rm C}_{\rm Born} = \pm {\al s\over t}\ .}{el coul born}
This result is to be compared to the one-photon-exchange prediction by QED \Eq{el dsdt em} (both amplitudes are of the order $\O{\al}$). We see that the more general QED result differs only by the presence of the effective form factor $G_{\rm eff}$. We will use the same form factor also within the eikonal formalism:
\eqref{F^{\rm C}_{\rm Born} = \pm {\al s\over t}\, G^2_{\rm eff} \.}{el coul born ff}
In that sense, it is not a Coulomb scattering anymore (magnetic influence is accounted for too), but we will keep calling the amplitude ``Coulomb''.

Interference -- what is the amplitude describing hadronic and electromagnetic forces acting simultaneously? Several approaches...

\vskip\baselineskip
\em{Interference in non-relativistic QM}
\vskip\baselineskip

Bethe derived \bref{bethe58} by WKB approximation. Model: spatial distribution of protons and neutrons is the same. Parameterized by a Gaussian with $a$ giving the ``size''. Nuclear potential proportional to the nucleon density. (low angle and high energy approximation). The result could have been written (it was using lab quantities and for a fixed energy). Upper (lower) sign for $\rm pp$ ($\rm \bar pp$).
\eqref{F^{\rm C+H}(t) = \pm {\al s\over t} e^{i \al \Ph(t)} + F^{\rm H(t)}}{el FCH decomp C}
with the Coulomb-hadronic phase $\Phi$
\eqref{\Phi_{\rm Bethe}(t) = \pm 2 \log {1.06\over a\sqrt{|t|}}}{el phase bethe}

\vskip\baselineskip
\em{Interference in QFT / Feynman diagrams}
\vskip\baselineskip

Yennie et al. \bref{yennie61}, Meister and Yennie \bref{yennie63}

Rix and Thaler \bref{rix66}

Locher \bref{locher67}

\WaY \bref{wy68}


\eqref{\Phi_{\rm WY} = \mp \log {t\over t_{\rm min}} \pm \int\limits_{t_{\rm min}} {\d t'\over |t' - t|} \left( {F^H(t')\over F^H(t)} - 1 \right)}{el phase WY}
Note that in \bref{wy68} there is a number of inconsistencies between Eq.~(17) and (23), the latter one is likely to contain a missprint: the sign of the second term shall read minus, i.e. as in our \Eq{el phase WY}

The authors write: ``One can hope, therefor, that in the region where $\Ph$ gets its major contribution, corrections due to the variation of the phase of $F^H$ are not large. Nevertheless, this represents another source of our ignorance whose magnitude cannot be estimated without detailed knowledge of the strong interactions.''

If $F^H(t) \propto e^{Bt/2}$, then the phase would simplify (for low scattering angles and high energies) to
\eqref{\Phi_{\rm SWY} = \mp \left(\log {B|t|\over 2} + \ga \right) \ ,}{el phase SWY}
where $\ga \doteq 0.577$ is the Euler's constant.



\vskip\baselineskip
\em{Interference in eikonal description}
\vskip\baselineskip

Islam \bref{islam67}


Cahn \bref{cahn82}

The author used decomposition
\eqref{F^{C+H}(t) = F^C(t) + F^H(t) [1 + i \al \Ps(t) ]\ .}{el FCH decomp H lin}
However, within the approximations used in the paper, this decomposition could be written in an exponential form
\eqref{F^{C+H}(t) = F^C(t) + F^H(t) e^{i \al \Ps(t)}\ .}{el FCH decomp H}
$F_C$ is merely a symbol for $\pm\al s G_{\rm eff}^2 / t$

No FF

\eqref{\Ps_{\rm Cahn} = \pm \lim\limits_{t_{\rm min} \to -\infty} \left[ \log {t\over t_{\rm min}} - \int\limits_{t_{\rm min}}^{0} {\d t'\over |t' - t|} \left ( {F^H(t')\over F^H(t)} - 1 \right) \right]}{el phase Cahn}
Note: without the limit, $\Ps_{\rm Cahn} = - \Ph_{\rm WY}$

With FF, a simplification of $G_{\rm eff} = F_D$ leads to a shift in phase
\eqref{\Ps \rightarrow \Ps + \nu}{el FF influence}
For the phase shift Cahn derived
\eqref{\nu_{\rm Cahn} = - {4t\over \La^2} \log {\La^2\over -4t} + {2t\over \La^2}}{el nu Cahn}
Note: independent of $s$

\KaL{} \bref{kl94}:

\eqref{\Ps_{\rm KL} =
	\mp \int\limits_{t_{\rm min}}^{0} \log {t'\over t}\ {\d\phantom{t'}\over\d t'} G_{\rm eff}^2(t')
	\pm \int\limits_{t_{\rm min}}^{0} \d t' \left ( {F^H(t')\over F^H(t)} - 1 \right)\, {I(t, t')\over 2\pi}
}{el phase KL}
where
\eqref{I(t, t') = \int\limits_{0}^{2\pi}\d\ph\ {G_{\rm eff}^2(t'')\over t''}\ ,\qquad t'' = t + t' + 2\sqrt{t t'} \cos\ph}{el KL I}

\eqref{\Ps_{\rm CKL} =
	\mp \int\limits_{T}^{0} \log {t'\over t}\ {\d\phantom{t'}\over\d t'} G_{\rm eff}^2(t')
	\pm \int\limits_{T}^{0} \d t' \left ( {F^H(t')\over F^H(t)} - 1 \right)\, {I(t, t')\over 2\pi}
	\pm G_{\rm eff}^2(T)\,\log {t\over T}
}{el phase CKL}

Note that in the limit $G_{\rm eff}^2\to 1$ one finds (see Eq.~(25) in \bref{Cahn82})
$${I(t, t')\over 2\pi}\to - {1\over |t' - t|}$$
and thus $\Ps_{\rm CKL}$ coincides with $\Ps_{\rm Cahn}$ (indeed, in the limit $T\to -\infty$)

Selyugin \bref{selyugin99}

\eqref{\nu_{\rm Selyugin} = c_1 \log (1 - c_2^2 t),\qquad c_1 = 0.11,\qquad c_2 = 200\un{GeV^{-1}}\ .}{el nu Selyugin}
Again, independent of $s$

Kopeliovich and Tarasov \bref{kopeliovich01}

Franco \bref{franco73}

Buttimore et al. \bref{buttimore78}

\vskip\baselineskip
Data fits with different interference formulae: \bref{ppp03,ppp05}

\vskip\baselineskip
\em{Critique and discussion}
\vskip\baselineskip

\hrule

\fig{fig/pdf/el_cic_noff_F_C.pdf}{el cic noff FC}{(Full) Coulomb amplitude calculated with \Eq{el F CH eik} ($\de_H$ set to zero, no form factor). Left: the modulus of the amplitude (for three values of $\la$) compared to the Born approximation (hidden by the green and blue curves). Right: the argument of the amplitude (solid lines) compared to Cahn's $\eta^C$ (dashed lines, often hidden by the solid lines).}

SWY all right up to $3\cdot 10^{-2}$

\fig{fig/pdf/el_cic_noff_Psi.pdf}{el cic noff Psi}{Comparison of Coulomb-interference phases with no form factor. Solid curves have been calculated with the eikonal formula.}

\fig{fig/pdf/el_cic_noff_Z.pdf}{el cic noff Z}{Left: the interference importance function $Z$. Right: $\zeta$ function reaches one if $|F^C| = |F^H|$ and tends to zero if one of the amplitudes is much larger that the other one. Both: no form factor included.}

Now with form factors.

\bmfig
\fig{fig/pdf/el_cic_diff_F_C.pdf}{el cic diff FC}{[7cm]The modulus of the (full) Coulomb amplitude calculated with the eikonal formula and Puckett form factor. The Born curve is hidden behined green and blue curves.}
%
\fig{fig/pdf/el_cic_diff_nu.pdf}{el cic diff nu}{[7cm]The form factor induced Coulomb phase alternation $\nu$. Colorful lines calculated with eikonal formula.}
\emfig

\fig{fig/pdf/el_cic_diff_Psi_ff.pdf}{el cic diff Psi ff}{The Coulomb-interference phases with different form factors and according to different formulae.}

\fig{fig/pdf/el_cic_diff_Psi_eik.pdf}{el cic diff Psi eik}{A comparison of Coulomb-interference phases with Puckett form factor, between CKL and eikonal formulae.}

\fig{fig/pdf/el_cic_diff_Z_ff.pdf}{el cic diff Z ff}{The importance of the Coulomb-interference ($Z$ function) for diffrent form factors and different formulae.}

\fig{fig/pdf/el_cic_diff_Z_eik.pdf}{el cic diff Z eik}{Left: the importance of the interference $Z$, Right: the $\ze$ function. For both plots the Puckett form factor has been used.}

\hrule

\iffalse
\> problems of Cahn formula (inherited and not solved in \KL{} treatment)
\>> PROBLEM 2: The case without form factors gives just the \WY{} formula. If \WY{} is wrong, then Cahn is wrong too. The formula with form factors (and thus \KL too) is just a trivial generalization which does not remove the basic issue.
\>> PROBLEM 3: Some approximations may be a bit too crude (e.g. neglecting $\al \log {t\over t_{\rm min}}$)

\> problems in \KL{} formula
\>> PROBLEM 4: The inclusion of form-factors is not complete -- the modification of the Coulomb phase is not discussed (c.f. Sec.~4 in \bref{cahn82})
\>> PROBLEM 5: The relation between hadronic $t$- and $b$-amplitudes require the knowledge of the $t$-amplitude beyond the physical region. Cahn assumes this is know, the upper bound of his $q^2$-integrals is $Q^2\to\infty$. \KL{} realized the issue and fixed the upper bound to the edge of the physical region $Q^2 = t_{\rm min}$. This brings some inconsistency.

\> comments
\>> to be really consistent, one should not mix amplitudes obtained by different approaches, they might be incompatible (especially their phases) 
\>> We will use high energy approximations to compare the results of several authors.

\vskip1cm


Before discussing the problems in a greater detail, let us make a comment about the Coulomb-hadronic interference phase. Many authors (including Bethe \bref{bethe58} and \WY{} \bref{wy68}) decomposed the full amplitude in the following way
\eqref{F^{C+H}(t) = F^C(t) e^{i \al \Ph(t)} + F^H(t)\ .}{el FCH decomp}
Since the phase $\Ph$ is real in their treatments, the phase factor can be moved to the hadronic amplitude $F^H$:
\eqref{F^{C+H}(t) e^{-i \al \Ph(t)} = F^C(t) + F^H(t) e^{-i \al \Ph(t)}\ .}{el FCH decomp2}
The phase shift of the full amplitude is not observable. This form is more similar to the formulae of Cahn \bref{cahn82} and \KL{} \bref{kl94}, which can be written as
\eqref{F^{C+H}(t) = F^C(t) + F^H(t) e^{i \al \Ps(t)}\ .}{el FCH decomp2}
Here we took the liberty to exponentiate the factor $1+ia\Ps$. This modification introduces corrections of the order $\O{\al^2}$, that is the order which is commonly neglected in the work of Cahn \bref{cahn82}. The ``phase'' $\Ps$ may generally be complex (and thus it does not qualify to be called phase) and that is why it cannot be moved to the Coulomb amplitude. Still, it will be interesting to compare the phases:
\eqref{-\Ph\quad\hbox{vs.}\quad\Ps\ .}{}

\TODO{Here, $F^C$ is merely a symbol for $\al s/t$, it is real!}


{\bf PROBLEM 2}

Cahn finds his result Eq.~(26) agree with the result of WY. The latter one has beed shown to be inconsistent, see e.g. \bref{klv07}. Then, a question raises immediately -- can the result be still correct?

Despite Cahn's comment, we find the results different for two reasons. The first is the limit in Cahn's Eq.~(26). In the result of \WaY{}, $Q^2$ is fixed at $-t_{\rm min}$. As will show (at least for a class of hadronic models) in the discussion of PROBLEM 5, this difference is not important.

The second issue is that Cahn moved his phase factor from $F^H$ term to $F^C$ term, simply by inverting the sign in the exponent (cf. decompositions \Eq{el FCH decomp,el FCH decomp2}). This would be permitted if the phase $\Ps$ would be purely real -- the overall phase is not observable. But it is not the case as can be seen e.g. in \Fg{el cic noff Psi,el cic diff Psi}. The imaginary part is very small up to $|t|\approx 0.1\un{GeV^2}$, but then it gains importance. Because of this, Cahn's comparison to the \WY{} formula looks doubtful. \TODO{One should look at $|F^{C+H}|$, this is unambiguous. TO BE DONE}.

{\bf PROBLEM 3}

The derivation of Cahn \bref{cahn82} is clearly only the first order in $\al$, the fine structure constant. Therefore, one would naively expect the error to be of the order $\al\approx 1\percent$. However, looking closer one would find a rather benevolent manipulation with terms like
$$\al \log {t\over t' - \la^2}\ ,$$
where $t'$ is an integration variable ranging from $t_{\rm min}$ to $0$. $\la\to0$ is merely an infrared regulator. Taking some values of interest: $|t| = 10^{-3}\un{GeV^2}$ and $|t_{\rm min}| \approx s = (14\un{TeV})^2$ yields a correction of $-24\percent$, which is rather important. Analytical estimation of the influence of these rather large correction is not an easy task. Instead, we decided to verify the accuracy of the Cahn's formula numerically. For that we reverted to his starting point (his Eq.~(12)):
\eqref{F^{C+H}(t) = {s\over 2i} \int\limits_0^\infty b\ \d b\ J_0(b\sqrt{-t}) \left( e^{2i [\de^C(b) + \de^H(b)] } - 1\right)}{el F CH eik}

The Coulomb eikonal $\de_C$ can be determined from a Born Coulomb amplitude $F^C_{\rm Born}(t)$:
\eqref{\de^C(b) = \int\limits_0^\infty q\ \d q\ J_0(b q)\ F^C_{\rm Born}(-q^2)}{}
Consistently with Cahn, we took the Born Coulomb amplitude
\eqref{F^C_{\rm Born} = - {\al s\over -t + \la^2}\ .}{}
It is missing the form-factors, which we will plug in later on. The factor $\la$ represents a fictious mass of the photon. It will be kept low throughout our calculations, moreover we will use a set of values. At the end we will demonstrate that the final results, in a region of interest, are independent of the value of $\la$.

\Fg{el cic noff FC} shows that one can well recover the Coulomb amplitude in the $|t|$ range $10^{-5}\hbox{ to } 10\un{GeV^2}$ with $\la \le 10^{-3}\un{GeV}$.

\fig{fig/pdf/el_cic_noff_F_C.pdf}{el cic noff FC}{Coulomb amplitude calculated with \Eq{el F CH eik} ($\de_H$ set to zero). Left: the modulus of the amplitude (for three values of $\la$) compared to the Born approximation. Right: the argument of the amplitude (solid lines) compared to Cahn's $\eta^C$ (dashed lines).}

For simplicity, we will use only one hadronic model in what follows. We have chosen BSW model, because the corresponding eikonal function is known.

\Fg{el cic noff Psi} compares interference phases obtained with Cahn/KL formulae and via the eikonal one, \Eq{el F CH eik}. For the latter one, the phase extraction is somewhat delicate since the $F^{C+H}$ involves and phase factor that is ignored in the decomposition \Eq{el FCH decomp2}. It is ignored as it is unobservable and it diverges as $\la\to 0$. The phase can be extracted
\eqref{\Psi = {1\over i} \log {F^{C+H} e^{-i \arg F^C + i\pi} + |F^C| \over F^H} \ .}{}
Looking at the real part of the phase, we may conclude that the eikonal calculation (with $\la \le 10^{-3}\un{GeV}$) agrees perfectly with the Cahn/KL formula. Regarding the imaginary part, there is a disagreement for $|t|$ below $\approx 0.1\un{GeV^2}$. This is, in fact one of the $\al^2$ effects, coming from approximating $\exp i\al\ph$ with $1+i\al\ph$. However the discrepancy is about $100$ times smaller than the typical values of the phase magnitude, hence quite negligible.

\fig{fig/pdf/el_cic_noff_Psi.pdf}{el cic noff Psi}{Comparison of Coulomb-interference phases.}

The importance of the interference can be measured with the help of this function
\eqref{Z(t) = {|F^{\rm C+H}|^2 - |F^{\rm C}|^2 - |F^{\rm H}|^2\over |F^{\rm C+H}|^2}\ .}{el Z}
It is plotted in \Fg{el cic noff Z} together with function
\eqref{\zeta = {2\over {|F^H|\over |F^C|} + {|F^C|\over |F^H|}}}{el zeta}
which measures the relative difference between the hadronic and Coulomb amplitude. If they are equal, $\zeta = 1$, if one is much larger, $\ze$ tends to $0$. If one substitutes the decomposition \Eq{el FCH decomp2} to \Eq{el Z}, one easily finds that $Z$ must be very small (therefore the interference unimportant) if $F^C$ is much larger than $F^H$. A similar argument can be made for the opposite situation, where $F^H \gg F^C$, but there one must employ also the fact that the phase is almost real. Thus it becomes more a rule of thumb than an exact statement. Still, looking at the \Fg{el cic noff Z} one can find quite good relation between the peaks of the importance function $Z$ and the ``closeness'' function $\zeta$.

\fig{fig/pdf/el_cic_noff_Z.pdf}{el cic noff Z}{Left: the interference importance function $Z$. Right: $\zeta$ function reaches one if $|F^C| = |F^H|$ and tends to zero if one of the amplitudes is much larger that the other one.}

{\bf PROBLEM 4}

To describe the Coulomb interaction accurately, one must include the form factors. They are to be included to the Born Coulomb amplitude. When the full Coulomb amplitude is calculated via \Eq{el F CH eik} (with $\de_H=0$), the effect can be two-fold (as discussed by Cahn in his Sec.~4): modification of modulus and argument of the full Coulomb amplitude. 

\Fg{el cic diff FC} compares various quantities related to the Coulomb amplitude. Cahn used dipole form factor and in oder to make comparisons to his results, we used it so. The left plot demonstrates that the modulus alternation provoked by the inclusion of form factors is negligible. In contrary, there is a sizeable effect on the phase (this has been concluded already by Cahn). The detail of this effect compared to Cahn's correction function $\nu$ can be found in \Fg{el cic diff FC darg}. We must conclude that our results do not fully agree with Cahn's correction function, most likely because of oversimplified nature of Cahn's derivation.

\fig{fig/pdf/el_cic_diff_F_C.pdf}{el cic diff FC}{Full Coulomb amplitude for several values of $\la$. Left: the modulus compared to the Born term (black). Right: the argument (solid lines) compared to the case with no form factor (dashed).}

\fig{fig/pdf/el_cic_diff_F_C_darg.pdf}{el cic diff FC darg}{The difference in argument of the Coulomb amplitude between dipole and no form factors. The colored lines are our (eikonal) calculation (see preceding figure for legend), the black line is Cahn's $\nu$ function.}

We have seen that the form factors have non-negligible influence on the phase of the Coulomb amplitude. Therefore one should check what is the effect on the interference phase $\Psi$. Still, we will be using BSW model. Looking at \Fg{el cic diff Psi} (left) one can find the gap between the black line (KL) and the colorful lines, this is the effect not accounted for by \KL{} formulation.

\fig{fig/pdf/el_cic_diff_Psi.pdf}{el cic diff Psi}{The Coulomb-interference phase with dipole form factors.}

The error of the \KL{} formula reaches maximum about $|t|\approx 0.1\un{GeV^2}$. This is a region, where the hadronic amplitude dominates (see \Fg{el cic diff Z} right). Moreover the error applies to the real part of the phase only. As a consequence, one may expect very low importance of the interference and thus the error gets heavily suppressed. The left plot of \Fg{el cic diff Z} fully supports our arguments.

Conclusion. We have verified the numerical accuracy of Cahn and \KL{} interference formulae by comparing them to the eikonal one \Eq{el F CH eik}. We have found only one case of disagreement -- the form factor effect in the interference phase $\Psi$. However, this discrepancy is not visible in cross sections.

\fig{fig/pdf/el_cic_diff_Z.pdf}{el cic diff Z}{Left: the importance of the interference.}

{\bf PROBLEM 5}

Cahn's work lies entirely in within the eikonal framework. Regarding the hadronic interaction, the assumed input is the eikonal $\de^H(b)$. However, if a hadronic model is not eikonal-formulated, one would be given the amplitude in the $t$-space $F^H(t)$. Then, the hadronic eikonal $\de_H$ could, in principle, be obtained by means of Fourier-Bessel transform:
\eqref{\de^H(b) = {1\over 2i} \log \left[ 1 + {2i\over s} \int\limits_0^\infty q\ \d q\ J_0(b q)\ F^H(-q^2)  \right ]\ .}{}
However, it is immediately clear that this transform requires the knowledge of the hadronic amplitude also outside the physical region\footnote{There might be an interesting parallel between the complications in adopting eikonal and QFT approaches. In the former, one is missing the hadronic amplitude beyond the physical region, in the latter, one is missing the amplitude off mass shell.}.

\KaL{} tried to solve this issue simply by taking Cahn's final result and truncating the integrals at $Q^2 = - t_{\rm min}$. This step, if unjustified, may seem very controversial. Cahn assumed that $Q^2\to\infty$ throughout his work, e.g. his crucial Eq.~(15) would not hold otherwise.

It is clear that one may hardly argue about $F^H$ behavior in the unphysical region on a general level. Therefore let's focus on the group of eikonal hadronic models, in fact most of those presented in \Sc{el models}. The eikonal amplitudes \TODO{reference} of these models are typically of a Gaussian profile \TODO{reference} \Sc{el impact} with RMS of about \TODO{value}. The $t$-amplitude can be calculated through \Eq{el F CH eik}. Now, for very high $|t|$ values (like in the unphysical region) the factor $J_0(b\sqrt{-t})$ becomes a rapidly oscillating function and from the ``oscillation theorem'' \TODO{reference} it becomes clear that $F^H$ must be heavily suppressed. This can be seen even for much lower $|t|$ values, even in \Fg{el mod dsdt narrow}. Therefore, the contribution to the interference term in Cahn's Eq.~15
$$\int \d^2 \vec q' F^C(\vec q') F^H([\vec q - \vec q']^2)$$
from sufficiently high $q'^2$ will be negligible. This means that would could cut the integration region at $q'^2 = \mu^2$, where $\mu$ is the sufficiently large constant. Now, following the steps of \KaL{}, one would obtain their formula, just with $t_{\rm min}$ replaced with $-\mu^2$.

The last statement could be demonstrated directly in Eq.~(26) in \bref{kl94}. Let's split the integration region into two: $(t_{\rm min}, -\mu^2)$ and $(-\mu^2, 0)$. Again, $\mu$ is such that $F^H(-\mu^2)$ is small (or more precisely negligible wrt. the denominator $F^H(t)$). Then one can show that the integral over the first integration region vanishes. That is, the lower bound is quite arbitrary, it only needs to be value large enough.

\eqref{F^{\rm C+H}_{\rm WY} = F^C(t) e^{i\al\Ph(t) + F^H(t) }}{el WY}
\TODO{Comment on FC and FH, c.f. Eq.~(6) in \bref{wy68}}

\eqref{\Phi_{\rm WY}(t) =  -\al\log {t\over t_{\rm min}} + \int\limits_{t_{\rm min}}^0 {\d t'\over |t-t'|} \left({F^H(t')\over F^H(t)} - 1\right) }{el phase WY}
\TODO{Check the sign of the second term!!!}


The \KL{} formula.
\eqref{F^{\rm C+H}_{\rm KL}(t) = \pm {\al s\over t} f^2(t) + F^H(t)\ e^{i\al\Ps(t)}\ ,}{el FCH KL}
where
\eqref{\Ps_{\rm KL}(t) = \mp \int\limits_{t_{\rm min}}^0 \d t' \left[ \log {t'\over t} \left( f^2(t') \right)' - {1\over 2\pi} \left({F^H(t')\over F^H(t)} - 1\right) I(t, t') \right]}{el phase KL}
\eqref{I(t, t') = \int\limits_0^{2\pi} \d\ph {f^2(t'')\over t''}\ ,\qquad t'' = t + t' + 2\sqrt{t t'} \cos\ph\ .}{el KL I}

Compared to \bref{kl94} we wrote the correction to the hadronic amplitude in a exponentiated way. Both are, in principle, equivalent since the phase $\Phi_{\rm KL}$ has been derived under the assumption that $\O{\al^2}$ terms can be neglected. Another assumption leading to this phase was that $f(t_{\rm min}) = 0$. Although this is a very good assumption for any real form factor, it makes it uneasy to compare the phases of \KL{} and \WY. Fortunately, it is easy to correct \Eq{el phase KL} to allow for any form factor dependence:
\eqref{\Ps_{\rm KL,corr}(t) = \Ps_{\rm KL}(t) \pm f^2(t_{\rm min}) \log {t\over t_{\rm min}}\ .}{el phase KL corr}
Now it is easy to verify that the WY phase corresponds to KL phase, if no form factors are used:
\eqref{\Ps_{\rm KL,corr}(t) \mathop{\longrightarrow}\limits^{f(t)\to 1} - \Ph_{\rm WY}(t)}{el KL WY correspondance}

\fi

\vskip1cm
\hrule

\eqref{C(t) = {|F^{\rm \rm C+H}_{\rm CKL}|^2 - |F^{\rm H}|^2\over |F^{\rm H}|^2}}{el C}

\eqref{Z(t) = {|F^{\rm \rm C+H}_{\rm CKL}|^2 - |F^{\rm C}|^2 - |F^{\rm H}|^2\over |F^{\rm C+H}_{\rm CKL}|^2}}{el Z}

\eqref{R(t) = {|F^{\rm \rm C+H}_{\rm CKL}|^2 - |F^{\rm C+H}_{\rm SWY}|^2 \over |F^{\rm C+H}_{\rm CKL}|^2}}{el R}

\fig{fig/pdf/el_mod_C.pdf}{el mod C}{The influence of Coulomb interaction (see \Eq{el C}) as a function of $t$. Calculated with CKL formula and Puckett form factor.}

\fig{fig/pdf/el_mod_Z.pdf}{el mod Z}{The importance of the interference term (see \Eq{el Z}) as a function of $t$. Calculated with CKL formula and Puckett form factor.}

\fig{fig/pdf/el_mod_R.pdf}{el mod R}{The difference between WY and KL formulae as a function of $t$. Calculated with CKL formula and Puckett form factor.}

Eq.~2-7 (page 14) in ATLAS ALFA TDR \bref{alfa} is wrong.


