\iffalse
RP, MC, IP5, CMS, T1, T2, VFAT, DIGI, RECO, LAB, LHC, LS, DPE, MC, TOTEM, RMS, ndf
\fi


\chapter{RP simulation and reconstruction methods}

In \Sc{rp measurement} we have described how scattered protons can be detected by the RP detectors. This processes can be simulated by computer programs which is essential for development and tuning of reconstruction techniques. That for processes where the scattered proton kinematics is deduced from the RP hits. The structure of the simulation and reconstruction software is sketched in \Fg{sr sw structure}. Beyond the aforemetined \em{simulation} and \em{reconstruction} blocks, you can see three more: \em{alignment} (special reconstruction focused to the alignment of RP detectors), \em{raw-data} (experimental data preprocessing) and \em{trigger} (trigger related analyses).

\fig{fig/pdf/sr_sw_structure.pdf}{sr sw structure}{The structure of the RP simulation and reconstruction software.}

Our contribution to the software mostly complements the work described in \bref{hubert}. Below we will briefly describe all the essential components and later will give details for our modules.

\em{Event generator}.
The simulation process starts with generation of particles at the interaction point. This can be either a \em{particle gun} or a \em{physics event generator}. Particle guns generate particles according to some given distributions and are useful for specialized tests. Physics event generators simulate physics processes such as elastic scattering, single diffraction etc. We use common generators like Pythia \bref{pyhia6,pythia8} or Phojet \bref{phojet} and a custom developed Elegent (see \Sc{elegent}).

\em{Smearing}.
Most Monte Carlo generators simulate head-to-head collisions of particles of a fixed energy at a given point. However, in reality there are two bunches collided under a certain crossing angle. Within a bunch, particles do not have identical energy (energy smearing) and they are not all collinear (angular smearing). The bunches have non-zero dimensions and therefore the collision may take place at various points (vertex smearing). This module introduces these smearing effects in order to obtain realistic simulations. It will be discussed in detail in \Sc{beam smearing}

\em{Geant4 and proton transport}.
This module performs two actions: simulates energy depositions in sensitive volumes and propagates protons in between. The sensitive volumes are located around the IP5 (CMS, T1 and T2) and at the RP stations at $\pm 147\un{m}$ and $\pm 220\un{m}$. Both actions are implemented within the Geant4 \bref{geant4} simulation kit. The proton transport uses a polynomial parameterization model described in Chap.~6 of \bref{hubert}. For details on the implementation see Sec.~7.3 in \bref{hubert}.

\em{Detector response and electronics simulation}. The energy deposited in the silicon detectors gives raise to electron-hole pairs, which are collected by the strips and then processed by the VFAT2 chips \bref{vfat}. Details can be found in Sec.~7.4 in \bref{hubert}.

\em{DIGI}.
Since the VFAT chips are digital, their output is a boolean value per silicon sensor channel. We call these data DIGI. This is the meeting point of the simulation and raw-data chains and is the start point for reconstructions and analyses.

\em{Raw data}.
Raw data are those saved by the DAQ system.

\em{Raw-data validation}.
The form of raw data is given by the DAQ and can change during time according to needs. Hence before the data is used, they must be converted to a common format -- DIGI in our case. The DAQ system has only a limited possibility to run error checks online, therefore most of then must of run offline. This is the main duty of this module.
\TODO{Standalone section?}

\em{Fast simulation}
\Sc{fast simu}


\em{Clusterization}
is the first step of the reconstruction chain. Due to charge-sharing effects, it is possible that one track fires two or more neighbouring strips. Such strip clusters need be found and later treated as a single hit.

\em{RECO}.
The position of a cluster can be turned into the distance (in the read-out direction) from the center of the sensor. We call such data RECO.


\em{Pattern recognition}.
In this step we would like to find RECO hits belonging to (a) track(s) and suppress those which are result of noise of errors. For that we may use the fact that we have five planes in every projection and that particle tracks are straight lines. Hence hits forming line patterns are likely to come from a particle track, isolated hits are likely to be noise. Such an algorithm will be discussed in \Sc{pattern reco}. An alternative to this method -- road search algorithm -- is presented in Sec.~8.1.4 in \bref{hubert}. It is based on an additional piece of knowledge that physics protons are always very parallel to the beam. However, a possibility to reconstruct tracks with high angles will turn out important for alignment applications (see \Fg{al eig theta}). \TODO{problem with multiple tracks}

\em{One-RP track fitting}.
In this step, hits from $U$ and $V$ projections are combined into one XYZ fit, see Sec.~8.1.5 in \bref{hubert}.

\em{Station track fitting} is similar to one-RP track fit, but here all hits from an entire (or a part) station are combined. This module has been developed mainly for alignment purposes. In order to select really clean tracks, it also comprises outlier removal procedure. It will be discussed in \Sc{al data sel}.

\em{Physics reconstruction}.
In this step, the one-RP track fits are used to reconstruct proton kinematics at IP5. There are two versions -- \em{general} suitable for any diffractive protons and \em{elastic} which is a lighter version for elastic events. The latter one will be discussed in \Sc{el reco}.

\em{Alignment} modules perform a special analyses focused on the alignment of RP sensors wrt.~each other and wrt.~the beam. The entire Chapter~\sref{al} will be devoted to RP alignment. \TODO{Standalone section with implementation details??}

Various \em{coincidence-chip and trigger analyses} have been performed to validate the performance of the trigger system.

\section[elegent]{Elegent}

Elegent is acronym for ELastic Event GENeraTor. It generates elastic $\rm pp$ and $\rm\bar pp$ collisions with $\d\si/\d t$ distributions according to the four phenomenological models discussed in \Sc{el models}.

It uses the inversion method (see \bref{wikipedia} key ''Inverse transform sampling''). This means one has to prepare cumulative distribution functions (cdfs.) first. The Elegent package contains a standalone program to do so. For a chosen energy and interaction type ($\rm pp$ or $\rm\bar pp$), the program calculates (samples in a given range) cdfs.~for every models and saves them as a ROOT file. In fact, for each hadronic model, there are three cdfs.~with various levels of Coulomb interaction inclusion: pure hadron (Coulomb effects are completely neglected), West-Yennie (\Eq{el WY}) and Kundr\' at-Lokaj\' i\v cek (\Eq{el KL}).

The generation program itself loads a selected cdf., can reduce the available $t$-region and inverts the cdf. The generated events are saved in the standard HepMC \TODO{ref} format. Technical details and usage hints can be found in \bref{elegent}.


\section[beam smearing]{Beam smearing}

The aim of this module is to include beam smearing effects that are usually not considered by event generators. As it has been mentioned above, the beam smearing comprises three components: \em{angular smearing} (the particles in a bunch are not all parallel), \em{energy smearing} (there is energy fluctuation within a bunch) and \em{vertex smearing} (the bunches have non-zero dimensions, hence the interactions can take place in various space points).

This section is an updated version the internal note \bref{smearing}.


\subsection{Angular and energy smearing}

Let's discuss angular and energy smearing first. A collision in LAB frame (a frame bound to the accelerator). On the other hand, MC generators use a different frame to describe events -- a frame where incident particles have same momenta and opposite directions parallel to $z$ axis. This frame will be referred as the MC frame. Obviously, we want to find a transformation between these two frames. It will be done in two steps. 

First, we find the Lorentz boost which makes incident particles have equal momenta and opposite directions. If we write the transformation of a four-momentum $(E|\vec p)$ to $(E'|\vec p')$ in the following form
\eqref{\eqalign{
E'      &= \ga\,(E - \vec p\cdot\vec\be)\cr
\vec p' &= \vec p  +  (\ga - 1) {\vec p \cdot \vec \be\over \be^2}\vec\be - \ga E \vec\be\cr
}\qquad\eqalign{
\be &= |\vec\be|\cr
\ga &= {1\over\sqrt{1 - \be^2}}\cr}
\ ,}{sm lorentz}
one can find that the $\vec\be$ needed for our purpose is
\eqref{\vec\be = {\vec p_1 + \vec p_2\over E_1 + E_2}\ ,}{sm beta}
where $\vec p_1, \vec p_2$ are the momenta and $E_1, E_2$ are the energies of the incident particles in the LAB frame. Let's denote this transformation $L(\vec\be)$ and the boosted momenta of the incident particles $\vec p_1'$ and $\vec p_2'$.

In the second step, we rotate the vectors $\vec p_1'$ and $\vec p_2'$ to be parallel to the MC frame $z$ axis. We use Rodrigues' formula to describe the rotation of a vector $\vec v$ around a unit vector (axis) $\vec a$:
\eqref{R(a, \om)\, \vec v = \vec v\,\cos\om + \vec a\times\vec v\,\sin\om + \vec a\cdot\vec v\,\vec a\,(1 - \cos\om)\ .}{sm rotation}
For our purpose, one may identify\footnote{In fact, this is just one of the possible solutions, rotation around $\vec a$ axis is arbitrary.}
\eqref{\vec a = {\vec p_1' \times \hat z\over |\vec p_1' \times \hat z|}, \qquad \cos\om = {\vec p_1' \cdot \hat z\over |\vec p_1'|}}{sm axis angle}
where $\hat z$ is the unit vector in the $z$ direction in the frame after the boost. Obviously, the coordinate representation is $\hat z = (0, 0, 1)$.

Unfortunately, this is not the full story. MC generators usually produce events for a fixed center-of-mass energy $\sqrt{s}$. However, in the real case, $\sqrt{s}$ varies slightly due to the smearing effects. Since the variation is small it might be neglected. But for consistency reason, one had better make sure that the scattering products sum up to the same $\sqrt{s}$ as for which the event has been produced. There is generally no correct way how to accomplish that. Nevertheless, as the variations are small, it does not matter much. Therefore, let's scale the energy of the outgoing particles
\eqref{E_i^{\rm MC} \rightarrow \chi\, E_i^{\rm MC},\qquad \chi = \sqrt{s_{\rm LAB}\over s_{\rm MC}}\ .}{sm energy scaling}
The momenta of the particles are scaled such that their masses are preserved.

To summarize, if $(E|\vec p)_{\rm MC}$ is the four-momentum for a particle produced in a MC event, then the LAB four-momentum can be written
\eqref{(E|\vec p)_{LAB} = L(-\vec\be)\,R(\vec a, -\om)\, S(\chi)\, (E|\vec p)_{MC}\ ,}{sm mc to lab}
where $S(\chi)$ stands for the energy scaling procedure.


\fig[8cm]{fig/pdf/smearing_angular_energy.pdf}{smearing angular energy}{Sketch of angular and energy smearing in XZ plane.}

The above formula provides a method to apply angular and energy smearing to MC particles. The missing step to have a smearing MC generator is to know the distributions of the quantities $\vec\be$, $\vec a$, $\om$ and $\ch$. But these can be calculated from beam smearing parameters as discussed in \Sc{pr transport}. Assuming the crossing angle $\al$, see \Fg{smearing angular energy}, one can parameterize the momenta of the incident particles as
\eqref{\vec p_i = \pm p_{\rm nom}\,(1 + \xi_i)\,\pmatrix{\cos\al & 0 & \mp \sin\al\cr 0 & 1 & 0 \cr \pm\sin\al & 0 & \cos\al\cr}\,\pmatrix{C_i\cr S_i\cr \sqrt{1 - C_i^2 - S_i^2}}\ ,}{sm momenta par}
where $i\in {1, 2}$ and the upper/bottom sign corresponds to particle 1/2. $C_i$ and $S_i$ describe the angular smearing in X and Y and according to \Eq{to beam div} one assign them distributions $N(0, \si_\th^2)$. In fact, the tails of the normal distribution must be cut off since the parameterization requires $S_{1,2}^2 + C_{1,2}^2 \leq 1$. But as any realistic $\si_\th \ll 1$, the effect is negligible. The $\xi$ parameter accounts for the energy smearing, the expected distribution is $N(\bar\xi, \si_\xi^2)$ (see \Eq{to energy fl}).

The smearing effects and the crossing angle are rather small: $\si_\th \sim \O{10^{-6}}$, $\al \sim \O{10^{-4}}$ and $\si_\xi \sim \O{10^{-4}}$ (see \Sc{pr transport}). Therefore, one may simplify \Eq{sm mc to lab} by keeping just the first oder terms. It will yield an easily iterpretable forumula which will become useful reconstruction uncertainty estimates. Also, bearing in mind the LHC energies, we will approximate $E\approx p$ for all the particles involved.

In this approximation \Eq{sm beta} gives
\eqref{\vec\be \simeq {1\over 2}\pmatrix{-2\al + C_1 - C_2\cr S_1 - S_2\cr \xi_1 - \xi_2}}{sm beta expl}
and thus $\ga\simeq 1$ (all other terms are of second and higher orders). Then, the Lorentz boost \Eq{sm lorentz} and rotation \Eq{sm rotation} simplify to
\eqref{\vec p' \simeq \vec p - |\vec p| \vec\be,\qquad E' \simeq E - \vec p\cdot\vec\be\ ,}{sm lorentz expl}
\eqref{R(\vec a, \om)\,\vec v \simeq \vec v + {1\over 2}\pmatrix{S_1 + S_2\cr -C_1 - C_2\cr 0} \times \vec v\ .}{sm rotation expl}
The energy scaling factor turns out
\eqref{\ch \simeq 1 + {\xi_1 + \xi_2\over 2}\ .}{sm energy scaling expl}

Now we are ready to apply \Eq{sm mc to lab} to an outgoing (MC generated) particle with momentum $\vec p_{\rm MC}$. Let's limit ourselves to particles that can reach the RP stations, that means particles scattered to small angles $\th \ls 10^{-3}$. A convenient parameterization, thus, is
\eqref{\vec p_{\rm MC} = \pm p\,\pmatrix{\th\cos\ph\cr \th\sin\ph\cr 1\cr} \.}{sm diff mc}
The particles with $+$ sign can be detected in the right arm stations, with $-$ sign in the left arm. Still working in the leading approximation, one finds
\eqref{\vec p_{\rm LAB} = \pm p\,\pmatrix{\th\cos\ph + \De\th_x\cr \th\sin\ph + \De\th_y\cr 1 + \xi\cr},
\quad \xi = \left\{\matrix{\xi_1\cr\cr \xi_2}\right.,
\quad \De\th_x = \left\{\matrix{C_1 - \al\cr\cr C_2 + \al\cr}\right.,
\quad \De\th_y = \left\{\matrix{S_1\qquad\hbox{for right arm}\cr\cr S_2\qquad\hbox{ for left arm}\cr}\right.\ .}{sm diff lab} %}}}

The interpretation is quite intuitive. Let us take, for instance, a forward proton targeting the right arm and let us focus on the X projection. Its full scattering angle is a sum of three components: the original (MC) scattering angle ($\th\cos\ph$), crossing angle ($-\al$) and the X component of the beam divergence ($C_1$). The proton inherits the energy shift $\xi_1$ from the first incident proton.

\iffalse
Using the statistical properties suggested in the previous section yields the following relation between variances (recalling $\si_\th$ refers to the beam divergence)
\eqref{\si_{\De\th_x} = \si_{\De\th_y} = \si_{C_1} = \si_{C_2} = \si_{S_1} = \si_{S_2} = \cases{
{\si_\th\over\sqrt2}\qquad\hbox{for parameterization \Eq{mom par 1}}\cr
\si_\th\qquad\hbox{for parameterization \Eq{mom par 2}}\cr
}\ .}{delta th sigma}
\fi


\subsection{Vertex smearing}

\fig[8cm]{fig/pdf/smearing_vertex.pdf}{smearing vertex}{Bunch collision with crossing angle $\al$.}

\Eq{in luminosity}, one of the luminosity definitions \TODO{ref formanek?}, suggests that function
\eqref{h(x, y, z; t) = {1\over {\cal L}_{int}}\,j(x, y, z; t)\ \rh_T(x, y, z; t)}{sm pdf}
can be interpreted as the probability density function of finding an interaction (vertex) at position $(x, y, z)$ and time $t$. Recalling that $j$ stands for the flux of bombarding particles and $\rh_T$ is the density of target particles. This formula can well be adatped for a collision of two bunches, as depicted in \Fg{smearing vertex}. The two bunches, each propagating (in the LAB frame) with speed $v$, are collided under a crossing angle $\al$. The flux $j$ in \Eq{sm pdf} can be expressed as $v_{\rm rel}\,\rh$, where $v_{\rm rel} = 2v\cos\al$ is the relative velocity of the two bunches and $\rh$ is the density of one of them. Therefore one can write
\eqref{h(x, y, z; t) = {2v\cos\al\over {\cal L}_{int}}\ \rh_1(x, y, z; t)\, \rh_2(x, y, z; t)\.}{sm pdf2}
The $\rh_{1,2}$ densities correspond to the two bunches (note that the formula is symmetric against swap of the two bunches, as it should be).

If $\rh_0(\tilde x, \tilde y, \tilde z)$ is a (time-independent) particle density in the rest frame of bunch 2 (the tilded frame in \Fg{smearing vertex}), then the (time-dependent) particle density $\rh_2(x, y, z)$ in the LAB frame (the non-tilded one) can be obtained by means of coordinate transformation (we assume that the origins of the tilded and the non-tilded frame coincide in $t=t'=0$)
\eqref{\eqnarray{
\tilde x &= x\cos\al\ - z\sin\al \cr
\tilde y &= y\cr
\tilde z &= \ga \left( z\cos\al\ + x\sin\al + vt \right),\qquad \ga = {1\over\sqrt{1 - v^2}}\ .\cr
}}{sm coord trans}
The result reads
\eqref{\rh_2(x, y, z; t) = \ga\, \rh_0\Big(x\cos\al - z\sin\al, y, \ga(z\cos\al + x\sin\al + vt) \Big)\ .}{sm density2}
The overall factor $\ga$ is needed to preserve normalization:
\eqref{\int \rh_0(\tilde x, \tilde y, \tilde z)\ \d\tilde x\d\tilde y\d\tilde z = \int \rh_2(x, y, z)\ \d x\d y\d z\ .}{sm rho normalization}
The density for bunch 1 can be obtained analogically
\eqref{\rh_1(x, y, z; t) = \ga\, \rh_0\Big(x\cos\al + z\sin\al, y, \ga(z\cos\al - x\sin\al - vt) \Big)\ .}{sm density1}

The Lorentz contracted particle density of each bunch can be well approximated by a Gaussian \TODO{reference}
\eqref{\ga \rh_0(x, y, \ga z) = {n_B\over (2\pi)^{3/2}\si_x\si_y\si_z}\,\exp\left(-{ x^2\over 2\si_x^2}-{ y^2\over 2\si_y^2}-{ z^2\over 2\si_z^2}\right),}{sm density}
where $n_B$ is the number of protons in a bunch and the variances $\si_x, \si_y$ and $\si_z$ refer to the bunch dimensions in the LAB frame.

Inserting \Eq{sm density1,sm density2,sm density} to \Eq{sm pdf2} yields
\eqref{h(x, y, z; t) \propto \exp\left[- {\cos^2\al\over\si^2_x} x^2 - {y^2\over\si^2_y} - \left({\sin^2\al\over\si^2_x} + {\cos^2\al\over\si^2_z}\right)z^2 - {(vt + x\sin\al)^2\over\si^2_z}\right]\.}{sm vertex full distribution}
This means that the random variables $x$ and $t$ are not independent. But as we are not interested in the time of collision, we can integrate over the time $t$ and obtain the \hbox{p.d.f.} only for the spatial coordinates of the vertex
\eqref{h(x, y, z) \propto \exp\left[- {\cos^2\al\over\si^2_x} x^2 - {y^2\over\si^2_y} - \left({\sin^2\al\over\si^2_x} + {\cos^2\al\over\si^2_z}\right)z^2\right]\.}{sm vertex distribution}
We can see that the vertex distribution retains a Gaussian form. The mean values of $x, y$ and $z$ are zero while the effective variations are
\eqref{\si_{x,\rm ef\!f} = {\si_x\over\sqrt{2}\cos\al},\quad \si_{y,\rm ef\!f} = {\si_y\over\sqrt{2}},\quad \si_{z,\rm ef\!f} = {\si_z\over\sqrt{2}}\,{\si_x\over\sqrt{\si_z^2\sin^2\al + \si_x^2\cos^2\al}} \.}{sm effective variations}

\section[fast simu]{Fast simulation}

In order to verify the functionality of our track-based alignment (see Chapter~\sref{al}) we had perform a number of MC simulations. With full Geant4 simulation it would have had taken an enormous time. Instead we developped a fast simulation modules. As indicated in \Fg{sr sw structure}, there are two versions: \em{Fast full simulation} and \em{fast station simulation}. The first one takes as input smeared particles at IP5 and transport them to the RP station with the polynomial optics approximation (the one used later for reconstruction). The latter version generates random tracks at the beginning of a chosen station. Subsequent processing is in common for both versions. Within a station, the tracks are interpolated linearly and intersections with all sensors are calculated. If \pmt{roundToPitch} is set to True, then the hit point is rounded to the nearest strip or inter-strip position. The mechanism is illsutrated in \Fg{sr fast simulation scheme}. If the intersection falls in the blue region, the creation of a double-strip cluster is assumed and the hit position is rounded to the inter-strip position (dash-dotted line). Otherwise one-strip clusters are assumed and the position is rounded to the nearest strip. The strip pitch can be set by \pmt{pitch} parameter, the size of the double-strip-cluster region (blue) by \pmt{dscrWidth} parameter. All hits are assigned errors of pitch$/\sqrt{12}$ unless \pmt{dscReduceUncertainty} = True. In that case, the error of double-strip cluster hits reduced to half.

\fig{fig/pdf/sr_fast_simulation_scheme.pdf}{sr fast simulation scheme}{One and two-strip cluster regions as used in the fast simulation. The solid lines represent the strip centers, the dash-dotted lines mark the half way between the strips. The blue areas correspond to the double-strip-cluster regions, the white ones to the regions where one-strip clusters are created.}


\section[pattern reco]{Pattern recognition}

The input to this module is a list of RECO hits among which we would like to find those lying around particle tracks. This search is performed pot per pot, where any physics track is a straight line. Thus our task is to find straight line patterns within the data. The fact, that there are just two read-out directions (U and V) makes it special. In a way simpler -- one 2-dimensional task can be split into two 1-dimensional ones (one per projection). In a way more complicated -- if there are several tracks in a RP, we may find line patterns it both projections, but there is now clue how to combine them togher.
%For example, imagine two tracks in a pot. We would find two patterns in both projections, yielding four possible combinations. However, two of them are real and two are fake (ghosts).
Even if we applied rotational alignment corrections, the deviations from the U and V read-out directions would be insufficient to perform a 2D pattern search. 

Our algorithm is an optimization of the Hough transform pattern search (see e.g. \bref{wikipedia} key ''hough transform''). Let us parameterize the line patterns in the following way ($q$ stands for either $u$ or $v$)
\eqref{q = a (z - z_0) + b\ .}{pr line par}
$z_0$ refers to the center of the RP (this is the ''nearest'' place to all the planes, therefore it minimizes the extrapolation error in $b$ determination). The Hough transform assignes to each point $(z, q)$ a line in $ab$ space:
\eqref{(z, q) \rightarrow b = - (z - z_0) a + q \ .}{pr transform}
This process is shown in \Fg{sr pattern reco}: each of the points in the left plot is transformed to a line (of the same color) in the right plot. The effect is evident. The intersections of colorful lines (corresponding to the colorful points that lie around a line) cumulate together, but the intersections of the black line (representing a noise hit) are scattered around. Hence the taks of finding line patters is reduced to a (intersection) cluster search.

Since the RECO hit positions are multiples of pith $P$, it is natural to measure $b$ in units of $P$. Similarly, we will give slopes $a$ in multiples of $a_0 \equiv P / d$, where $d = 9 \un{mm}$ is the nominal distance between adjacent planes of the same strip orientation.

\fig{fig/pdf/sr_pattern_reco.pdf}{sr pattern reco}{Illustration of the pattern recognition method. Left: sideview on the five sensors (black vertical lines) of identical strip orientation (U or V). The small horizontal ticks represent the strip positions. The thick inclined line shows a track, the colorful points around mark the measurments by the five sensors. The black dot in the upper-left corener represents an error, due to noise for instance. Right: the corresponding Hough diagram. The colors of the lines correspond to the colors of the points in the left plot. The line intersections are marked with black dots. The gray region represents a cluster (of the optimal size -- see \Tb{pattern reco par}).}

Now, let us describe the cluster-search algorithm. Every RECO hit is assigned a weight $w = \si_0/\si$, where $\si_0 = 66/\sqrt{12} \un{\mu m}$ is the uncertainty of a 1-strip cluster. The weight of a line crossing is given by the sum of the two contributing hit weights. The algorithm can be described as follows.

\bitm
\itm Build clusters. Go throught all crossings, try to match each crossing to any of the already existing clusters. A crossing falls into a cluster if
\eqref{|a - a_c| < \pmt{clusterSize\_a}/2 \quad \hbox{ and }\quad |b - b_c| < \pmt{clusterSize\_b}/2\ ,}{pr cluster match}
where $a$ and $b$ are the coordintates of the crossing and $a_c$ and $b_c$ give the position of the cluster. The position is calculated as a weighed mean over all contributing crossings: $a_c = \sum a_i w_i / \sum w_i$.
If match, add the crossing to the cluster. If no match, create a new cluster containing the crossing only.
\itm Calculate cluster weights as the sum of its hit weights.
\itm Take the cluster with the highest weight. If its weight is lower than \pmt{threshold}, stop. If not, the cluster corresponds to a recognized track. Remove its points from the list and return to the step 1) to recognize more line patterns.
\eitm

The \pmt{threshold} parameter gives an effective minimal number of hits for a cluster to be trusted as a reasonable line. The other two critical parameters are the cluster sizes. If set too low, real tracks would not be recognized, if set too high, noise hits may get selected or even multiple tracks may be merged. The needed cluster size depends on the expected deviations of the individual measurements from the track. These deviations have two major ingredients: pitch rounding and misalignments. To find the optimal cluster sizes we made a simple MC test and a real data analysis, see \Fg{sr pattern reco tune}. For the MC test, we simulated  1000 (internal) misalignment scenarios compatible with \Fg{al comp det per unit} (slope $50\un{\mu m}$ per detector, deviations with the variation of $20\un{\mu m}$). For each misalignment, we simulated 1000 tracks through 5 planes at nominal distances $d$, the angular spread of tracks was $\si_\th = 5\un{mrad}$. For each track we calculated the cluster sizes and filled the values in the blue histograms. The red curves correspond to an analysis of run 3728, file 1. For both projection we required five planes with just one hit -- a perfect track. There are non-negligible differences between the two histograms at lower cluster sizes, but both agree that there is practically nothing above $3a_0$ and $4P$.

\fig{fig/pdf/sr_pattern_reco_tune.pdf}{sr pattern reco tune}{Cluster-size distributions: MC simulation (blue) and run 3728 (red).}

To optimize the performance, we apply the cuts below before performing the search.
\bitm
\itm Planes with more than \pmt{maxHitsPerPlaneToSearch} hits per plane are tagged as unusuable. We recall that only events with one track can be fully reconstructed.
\itm Events with less reasonable (i.e.~non-empty and not unusable) planes per projection than\break \pmt{minPlanesPerProjectionToSearch} are skipped.
\eitm

\Tb{pattern reco par} summarizes the parameters of the algorithm and gives their default/optimal values. \Fg{sr pattern reco ex} provides an example of the recognition results.

\tab[\strut\quad\pmt{#}\hfil&\quad\hfil$#$\hfil\quad\cr]{pattern reco par}{The parameters of the pattern recognition algorithm with their default values.}{\ln
clusterSize\_a& 3a_0\cr
clusterSize\_b& 4P\cr
maxHitsPerPlaneToSearch& 4\cr
minPlanesPerProjectionToSearch& 3\cr
threshold& 2.99\cr\ln
}


\fig{fig/pdf/sr_pattern_reco_ex.pdf}{sr pattern reco ex}{An example of the patter recognition. Run 3230, file 0, event 111, unit 45-220-near. Left: a track plus noise in the bottom pot (V sensors). Right: many tracks in the horizontal pot (U sensors). The red lines correspond to the recognized patterns (cluster centers).}

\section[elr]{Reconstruction of elastic events}

In the note \bref{el reco} we described a module performing a reconstruction of elastic events. It is was light-weight alternative to the full (inelastic proton) reconstruction, with the primary aim to quantify the reconstruction power of the RP detector system. Here in this section, we will present an update to that note. Of of the main differences is the use of a more accurate beam smearing description (as is done in \Sc{beam smearing}). Compared to the old study, the beam divergence has increased by a factor of $\sqrt 2$ (cf. Eq.~(17) in \bref{smearing}).

We will constrain ourselves to the most interesting results, for the others we refer the reader to the original note \bref{el reco}. We will skip the studies of the hypothetical situations without beam smearing. We will also skip the discussion of the peak structures appearing in certain histograms. These only appear with perfectly aligned detectors, therefore they correspond to a hypothetical situation too.

As shown in diagram \Fg{sr sw structure}, the elastic reconstruction follows the one-RP-track fit step. In the note \bref{el reco} we showed that the lever-arm of single RP is too short for a reasonable angular reconstruction. That is why we take only the hit position from the one-RP-track fit.

Elastic protons have, by definition, no momentum loss, that is $\xi \equiv 0$. For such protons, the transport equation \Eq{to lin par} simplifies to:
\eqref{x(s) = L_x(s)\, \th_x^* + v_x(s)\, x^*\ ,\qquad y(s) = L_y(s)\, \th_y^* + v_y(s)\, y^*\ .}{elr track param}
We have written the optical functions as functions of $s$ only since in a very good approximation they do not depended on proton's scattering parameters $x^*$, $y^*$, $\th^*_x$ and $\th^*_y$ \TODO{ref to note by Valentina etc?}. 

The hit positions $x$ and $y$ are known from the one-RP-track fits. Assuming that we know the optical functions corresponding to all RPs involved in the event, we may use LS (Least Squares) estimation to obtain the scattering parameters (\TODO{ref})
\eqref{\pmatrix{x^*\cr\th_x^*} = {1\over \sum v^2 \sum L^2 - \sum vL \sum vL} \pmatrix{\sum L^2 \sum x v - \sum vL \sum x L\cr -\sum vL \sum x v + \sum v^2 \sum x L}\ ,}{elr fit}
where for example $\sum x L$ abbreviates $\sum_{i} x(s_i) L_x(s_i) / \si_x^2(s_i)$ with $\si_x$ representing the uncertainty of the $x$ measurement (one-RP-track fit). A similar equation can be written for the $y$ projection. In fact, since the proton transport does not mix $x$ and $y$ projections, they decouple in the fit and one can perform $x$ and $y$ fits separately. The treatment is equal for both projections, thus we will write any formulae for the $x$ projection only in what follows.

The LS estimator gives also the covariance matrix for the fit parameters (see e.g. Eq. (6.24) in \bref{barlow}):
\eqref{\mathop{\rm Var}[x^*, \th^*] = {1\over \sum v^2 \sum L^2 - \sum vL \sum vL} \pmatrix{\sum L^2 & - \sum vL \cr -\sum vL & \sum v^2 }\ .
}{elr fit err}

The module includes a simple \em{hit selection} (to suppress noise and background hits) and an \em{event selection} (to distinguish elastic events from other processes like DPE). The algorithm can be described by the three steps below.

\bitm
\itm \em{Hit selection}. The aim is to suppress hits (one-RP tracks) that do not belong to the elastic event. The implementation is based on the fact that the $L\th^*$ terms dominate in the proton transport \Eq{elr fit} (\TODO{more justification}). When a $x/L_x$ histogram is build, the hits of an elastic event tend to cluster together, as illustrated in \Fg{elr road search}. The size of the cluster is optics-dependent and is thus a parameter of the algorithm, called \pmt{road-size}.

\fig{fig/pdf/elr_road_search.pdf}{elr road search}{An illustration of the hit selection. Left: sample RP hits shown vs\hbox{.} the corresponding effective length. The hits by elastic protons are drawn in blue, background in red. Right: the corresponding histogram of angles $x/L_x$.}

\itm \em{Fitting}. There are three fits performed according to \Eq{elr fit}: left-arm, right-arm and global fit. The left-arm fit is carried out through hits from the left-arm only, analogously for the right-arm fit. The global fit uses all hits available.

\itm \em{Elastic event selection}. Here, the left and right-arm fits are compared. For an ideal elastic scattering event, the left and right scattering angles are identical and, indeed, there is just one vertex. Thus, the left and right fit results should coincide. Practically we set \pmt{tolerance}s for the left-right differences of the scattering angles and vertex coordinates.
\eitm

In total, there are three parameters per projection (\pmt{road-size}, \pmt{angular tolerance} and \pmt{vertex tolerance}). For most optics \TODO{ref}, the dominant smearing effect is the beam divergence. Therefore the \pmt{angular tolerance} should be set to few times the standard deviation of the beam divergence \TODO{a ref?}. Similar argument applies to the \pmt{road-size} parameter. There one should also account for neglecting the $v$ terms in the proton transport \Eq{elr track param}. Thus reasonable values will be larger than those for \pmt{angular tolerance}. In fact, the best way to determine reasonable settings is to run a MC with elastic events only, do the reconstruction with no cuts and look at the distributions of the parameters. This will be done later on in \Sc{elr 1535,elr 90}.

But before turning to optics-dependent results, let us discuss what measurement uncertainties we expect for the quantities of interest -- most notably the scattering angle $\th$ and the momentum-transfer squared $t$. The measurement (fit) uncertainties can be determined from the covariance matrix \Eq{elr fit err}. Moreover, the typical TOTEM optics \TODO{ref} have such properties that will allow us to derive a much simpler formula. First, the effect of the $v$ terms in the proton transport \Eq{elr track param} can be neglected to those of the $L$ terms. Second, let us constrain to the situation where only $220\un{m}$ stations are used \TODO{typical up to 2011}. The variation of the effective lengths $L_x(s)$ with a station is not large, hence we might think of a ``typical effective length'' $L_x$. Under these simplifications, the scattering angle estimator becomes:
\eqref{\th_x^*
	\quad\mathop{\relbar\joinrel\relbar\joinrel\relbar\joinrel\relbar\joinrel\relbar\joinrel\longrightarrow}\limits^{v \hbox{\SmallerFonts\ neglected}}\quad
		{\sum L x\over \sum L^2}
	\quad\mathop{\relbar\joinrel\relbar\joinrel\relbar\joinrel\relbar\joinrel\relbar\joinrel\longrightarrow}\limits^{\hbox{\SmallerFonts\ typical }L}\quad
	{\sum_{\rm R} x - \sum_{\rm L} x\over (N_{\rm R} + N_{\rm L}) L_x}
\ ,}{elr th fit sim} 
where for example $\sum_{\rm R}$ represents the sum over the right hits only and $N_{\rm R}$ is the number of the right-arm hits. In this model, the error contributions to a measurement can be written as
\eqref{x = L_x (\th_x^* + \De\th_x) + \De x\ .}{elr x meas}
$\th_x^*$ represents the true scattering angle, $\De\th_x$ its alternation due to the beam divergence, see \Eq{sm diff lab}, which is independent for left and right arms. The error $\De x$ combines two effects -- misalignment and error due the finite resolution of the silicon detectors. The misalignment leads to a systematic shift a we will not discuss it here. Had we only one sensor per RP (per projection), the finite-resolution error could be well described by an uniform distribution between plus and minus half of the strip pitch $P$, which would give a standard deviation $\si_{\rm P} = P/\sqrt{12}\approx 19\un{\mu m}$. In reality, there are five sensors per projection in each RP, thus one might naively expect the standard deviation to reduce by a factor $\sqrt 5$. But since physics tracks are almost perpendicular to the sensors, they most often hit the same strip in all five sensors and the reduction is not justified. Because of small misalignments of the sensors within a stack, the distribution of $\De x$ is rather Gaussian-like, but the standard deviation $\si_{\rm P}$ is retained \TODO{ref to Hubert's paper?}.

Putting \Eq{elr x meas} to the rhs.~of\Eq{elr th fit sim} yields an estimate for the reconstruction error (reconstructed value minus the original one)
\eqref{\De\th_x^* = \De_{\rm B}\th_x^* + \De_{\rm R}\th_x\ ,
	\quad \De_{\rm B}\th_x^* = {N_{\rm R} \De\th_x^{\rm R} - N_{\rm L} \De\th_x^{\rm L}\over N_{\rm R} + N_{\rm L}}\ ,
	\quad \De_{\rm R}\th_x^* = {\sum_{\rm R} \De x - \sum_{\rm L} \De x\over  N_{\rm R} + N_{\rm L}}\ ,
}{elr th err proj}
where we have introduced the error components due to the beam divergence $\De_{\rm B}\th_x^*$ and due to the finite resolution $\De_{\rm R}\th_x^*$. To a good approximation, both components have zero mean value and their standard deviations can be estimated to
\eqref{
	\si_{\rm B} = {\sqrt{N_{\rm R}^2 + N_{\rm L}^2} \over N_{\rm R} + N_{\rm L}}\, \si_\th\ ,\qquad
	\si_{\rm R} = {1\over L} {1\over \sqrt{N_{\rm R} + N_{\rm L}}}\, \si_{\rm P}\ ,
}{elr th err comp sig}
where $\si_\th$ is defined below \Eq{sm momenta par}. The standard deviation of the $\th_x^*$ reconstruction is then
\eqref{\si(\th_x^*) = \si_{\rm B}^2 + \si^2_{\rm P}\ .}{elr th err proj sig}

The error estimate for $\th^*$ measurement can be obtained by error propagation ($\th_x^*$ and $\th_y^*$ are independent variables)
\eqref{\si^2(\th^*) \equiv \si^2(\sqrt{\th_x^{*2} + \th_y^{*2}}) = {\th_x^{*2}\over \th^{*2}} \si^2(\De\th_x^*) + {\th_y^{*2}\over \th^{*2}} \si^2(\De\th_y^*)\ .}{elr th err sig}
If the beam-smearing error contribution dominates the one from finite resolution, that is $\si_{\rm B} \gg \si_{\rm R}$, then we find $\si(\th^*) = \si(\th_x^*) = \si(\th_y^*) = \si_{\rm B}$.

The error propagation can be used to calculate the standard deviations of the momentum-transfer components (\TODO{ref to definition})
\eqref{{\si(t_x)\over |t_x|} = {2p\over \sqrt{|t_x|}}\, \si(\th_x^*)\ .}{elr tx err sig}
A similar result can be for the full momentum-transfer $t$ (averaged over all values of the azimuthal angle $\ph$)
\eqref{{\si(t)\over |t|} = {2p\over \sqrt{|t|}}\,\sqrt{\si^2(\th_x^*) + \si^2(\th_y^*)\over 2}\ .}{elr t err sig}

We will test these error estimates with MC simulations in the two sub-sections below.


\subsection[elr 1535]{1535 m optics}

\> the nominal optics, \TODO{ref}
\> reconstruction with no cuts

\Fg{elr 1535 ndf} shows that the most frequent numbers of degrees of freedom of the fit \Eq{elr fit} are 4 and 6. The former corresponds to the situation with 2 vertical pots hit on both sides. The latter is created when a track hits also 2 horizontal pots on a side.

\Fg{elr 1535 rs} illustrates that it is safe to set the \pmt{road size} parameter to $\approx 2$ for both projections.

\bmfig
\fig{fig/pdf/elr_1535_ndf.pdf}{elr 1535 ndf}{[7cm]A histogram of the number of degrees of freedom in fit \Eq{elr fit} (the histograms are overlapping).}
\fig{fig/pdf/elr_1535_rs.pdf}{elr 1535 rs}{[7cm]A histogram of the road sizes.}
\emfig

The angular resolutions are practically the same, as shown in \Fg{elr 1535 dth}. The dominant error contribution is the beam smearing with $\si_\th = 0.3\un{\mu rad}$, which gives rise to $\si_{\rm B} = 0.21 \un{\mu rad}$ for the 2+2 configuration (i.e. 2 pots active in both arms) or $\si_{\rm B}$ for the 2+4 configuration.

As a consequence of the low values of the optical functions $v$ the vertex resolution is rather bad, see \Fg{elr 1535 dvtx}.

\bmfig
\fig{fig/pdf/elr_1535_dth.pdf}{elr 1535 dth}{[7cm]The angular reconstruction error.}
\fig{fig/pdf/elr_1535_dvtx.pdf}{elr 1535 dvtx}{[7cm]The reconstruction error of the vertex position.}
\emfig

\Fg{elr 1535 res t} show the relative $t$-resolution with a fit of the form of \Eq{elr t err sig}. The fit describes the data well, the analytic estimate for the numerator is $3.08\un{GeV}$ (for the 2+2 configuration) and $3.18\un{GeV}$ (for the 4+2 configuration), which agrees well with the fit result.

\fig{fig/pdf/elr_1535_res_t.pdf}{elr 1535 res t}{[7cm]The $t$-resolution.}


\subsection[elr 90]{90 m optics}

\> again, nominal optics and reco with no cuts

For this optics, the hits are concentrated around $x=0\un{mm}$ (the effective length $L_x$ is very small, \TODO{ref to picture}?). This means that horizontal pots can not contribute to the measurement of elastic scattering. Therefore we expect most frequent hit-configuration to be 2 vertical pots on both sides, that is $\hbox{ndf}=2$, which is confirmed by \Fg{elr 90 ndf}.

Since the effective length $L_x$ is too small, the road-search algorithm is not efficient for the $x$ projection (the road sizes are excessively large), see \Fg{elr 90 rs}. For the $y$ projection a value of about $12$ seems safe.

\bmfig
\fig{fig/pdf/elr_90_ndf.pdf}{elr 90 ndf}{[7cm]A histogram of the number of degrees of freedom in fit \Eq{elr fit} (the histograms are overlapping).}
\fig{fig/pdf/elr_90_rs.pdf}{elr 90 rs}{[7cm]A histogram of the road sizes.}
\emfig

The $\th_y$ angular resolution is dominated by the beam divergence which corresponds to $\si_{\rm B} = 1.7\un{\mu rad}$. For the $x$ projection it is deteriorated by the low $L_x$ values.

The $v$ optical functions are too small for a reasonable vertex resolution, see \Fg{elr 90 dvtx} (the curve for the $y$ projection is not even drawn).

\bmfig
\fig{fig/pdf/elr_90_dth.pdf}{elr 90 dth}{[7cm]The angular reconstruction error.}
\fig{fig/pdf/elr_90_dvtx.pdf}{elr 90 dvtx}{[7cm]The reconstruction error of the vertex position.}
\emfig

The analytical estimate \Eq{elr t err sig} for the $t$-resolution, drawn in \Fg{elr 90 res t}, gives $4.0\un{GeV}/\sqrt{|t|}$. This differs slightly from the fit value. For this optics, the nominal distance of the detector edges is not negligible anymore, therefore the average over all $\phi$ values becomes a too crude approximation. It gives too much weight to the $t_x$ component, which is burdened by a much larger error. That is why the analytical approach overestimates the error.

\fig{fig/pdf/elr_90_res_t.pdf}{elr 90 res t}{[7cm]The $t$-resolution.}

\input elasticReco.tex
