\chapter{Simulation and analysis methods}

\section{Principle of the experiment}

\subsection[rp measurement]{Relation between RP measurements and a proton}

A collision takes place (the vertex is) in
\eqref{(x^*, y^*, z^*)^\T}{IP vertex}
and a proton is created. The proton has energy $E=\sqrt{p^2 + m^2}$ and momentum components
\eqref{\vec p = p \pmatrix{\sin\th \cos\ph\cr \sin\th\sin\ph \cr \cos\th} = 
p \pmatrix{\th_x\cr \th_y\cr \sqrt{1 - \th_x^2 - \th_y^2}}}{IP proton momentum}

\fig{fig/pdf/protonTransport.pdf}{protonTransport}{Proton transport.}

Then, the proton is transported by the lattice of the LHC magnets to the detector area (say to $z = z_0$). The transport is schematically depicted in \Fg{protonTransport}. As there are no magnets in the detector area, from $z_0$ on, the proton follows a straight trajectory which might be described as
\eqref{\pmatrix{x\cr y\cr z} = \pmatrix{a_x\cr a_y\cr 1} z + \pmatrix{b_x\cr b_y\cr 0}}{local track}

So far, the coordinates have been described in the global $xyz$ coordinate system. For deriving (the expected measurement value), it is useful to use a local coordinate frame $x_l y_l z_l$. This coordinate system has its origin in the center of the detector. Its $x_l$ and $y_l$ axes parallel to the detector surface and its $z_l$ axis perpendicular to the surface. The global to local transformation can be written
\eqref{\pmatrix{x_l\cr y_l\cr z_l} = \mat R \left[ \pmatrix{x\cr y\cr z}  - \pmatrix{c_x\cr c_y\cr c_z}  \right] \ ,}{global to local}
where the $\vec c$ vector stands for the detector's center position in the global coordinates.

The measurement $m$ is then given by projection of the local surface coordinates $x_l$ and $y_l$ to the readout direction $\vec d$
\eqref{m = \hbox{ROUND}\left( (d_x, d_y) \pmatrix{x_l\cr y_l} \right)}{full measurement}
The vector $\vec d = (d_x, d_y)^\T$ is the readout direction (perpendicular to strips). The ``ROUND'' comprises charge sharing effects etc.

The rotation $\mat R$ can be parameterized as
\eqref{\mat R =
\pmatrix{
\cos\rh_z  & \sin\rh_z & 0\cr
-\sin\rh_z & \cos\rh_z & 0\cr
0		   & 0         & 1\cr
}
\pmatrix{
\cos\rh_y  & 0 & \sin\rh_y\cr
0		   & 1 &          \cr
-\sin\rh_y & 0 & \cos\rh_y\cr
}
\pmatrix{
1 & 0		   & 0        \cr
0 & \cos\rh_x  & \sin\rh_x\cr
0 & -\sin\rh_x & \cos\rh_x\cr
}
}{rotation parameterization}

\TODO{} The rotation $\mat R$ only describes the misalignment rotations. As follows from the \Sc{al exp misal}, the estimate for such rotations is of the order of $1\un{mrad}$. For such angles, it is possible to use the following approximation
\eqref{\cos\rh \approx 1,\qquad \sin\rh \approx \rh\ .}{small rotation approximation}
Furthermore, terms as $\sin\rh_x \sin\rh_y$ can be neglected to $\sin\rh_x$. Therefore, the rotation $\mat R$ can be approximated by
\eqref{\mat R \approx \pmatrix{
1 & \rh_z & \rh_y\cr
-\rh_z & 1 & \rh_x\cr
-\rh_y & -\rh_x & 1\cr
}}{rotation parameterization approximated}

\TODO{keep $\rh_z$ without the limit}

Now, let us find the point where the track \Eq{local track} hits the detector. This is the point on the track with $z_l = 0$. Inserting \Eq{local track} to \Eq{global to local}, one can find that the condition is fulfilled for 
\eqref{z = z_i \equiv { c_z + \rh_y (b_x - c_x) + \rh_x (b_y - c_y) \over 1 - \rh_y a_x - \rh_x a_y} \ .}{z i}
Now, lets make order estimate for the terms in the expression for $z_i$.\hfil\break
\eqref{\eqnarray{
c_z &\sim& 1\un{m} \cr
\rh_{x,y} &\sim& 10^{-3}\un{rad} \cr
(b - c)_{x, y} &\sim& 10^{-2}\un{m} \cr
a_{x, y} &\sim& 10^{-2}\un{rad} \cr
}}{z i order estimates}

With these estimates, one can conclude
\eqref{z_i = c_z + \O{10^{-5}\un{m}}\ .}{z i approximated}

With the help of \Eq{global to local}, one can calculate the ``on-surface'' coordinates $x_l$ and $y_l$, for example
\eqref{x_l = 
\underbrace{a_x z_i}_{10^{-2}\un{m}}
+ \underbrace{b_x - c_x}_{10^{-2}\un{m}} 
+ \underbrace{\rh_z}_{10^{-3}} (\underbrace{a_y z_i}_{10^{-2}\un{m}} + \underbrace{b_y - c_y}_{10^{-2}\un{m}})
+ \underbrace{\rh_y}_{10^{-3}} \underbrace{(z_i - c_z)}_{10^{-5}\un{m}}
\ .}{xlyl i}
Keeping only the important terms
\eqref{\pmatrix{x_l\cr y_l} = \pmatrix{1 & \rh_z\cr -\rh_z & 1}
\left[
\pmatrix{a_x\cr a_y} c_z + \vec b - \vec c
\right] + \O{10^{-8}\un{m}}\ .
}{xlyl i approximated}

\section{Simulation and analysis chain}

\fig[15cm]{fig/pdf/offline_sw_structure.pdf}{offline sw structure}{The structure of the TOTEM Offline software.}

\> briefly comment on all steps
\> details only for those I worked on

\section{Beam smearing simulation}

%\input smearing.tex

\section{Pattern recognition}

\fig{fig/pdf/sr_pattern_reco.pdf}{sr pattern reco}{Illustration of the pattern recognition method. Left: sideview on the five sensors (black vertical lines) of identical strip orientation (U or V). The small horizontal ticks represent the strip positions. The thick inclined line shows a track, the colorful points around mark the measurments by the five sensors. The black dot in the upper-left corener represents an error, due to noise for instance. Right: the corresponding Hough diagram. The colors of the lines correspond to the colors of the points in the left plot. The line intersections are marked with black dots. The gray region represents a cluster. Here, the intercept $b$ is measured in strips, the slope $a$ in strips/plane count.}

\fig{fig/pdf/sr_pattern_reco_ex.pdf}{sr pattern reco ex}{An example of the patter recognition. Run 3230, file 0, event 111, unit 45-220-near. Left: a track plus noise in the bottom pot (V sensors). Right: many tracks in the horizontal pot (U sensors). The red lines correspond to the cluster \TODO{centers}.}

\section{Reconstruction of elastic events}

%\input elasticReco.tex
