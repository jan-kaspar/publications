%%% Inputs
\input pdfslides

%%%%%%%%%%%% 1 %%% --------------------------------------------------------------------------------
\vbox to0pt{}\vfil
\Title{Elastic {\fMIxv pp} scattering and TOTEM experiment}
\title{Jan Ka�par}

\newpage %%% 2 %%% --------------------------------------------------------------------------------
\Title{TOTEM experiment}
\vskip-\baselineskip
\centerline{\fPviii(Total Cross Section, Elastic Scattering and Diffraction Dissociation at the LHC)}
\vskip\baselineskip
The smallest, but {\bf independent} experiment at the LHC at CERN. Its physical program \bibref{TDR}

\bitm{\parskip=5pt}
\itm Total cross section -- precise measurement (error $\approx 1\un{mb}$), luminosity independent method
$$\si_{tot}(s) = {16\pi\over 1 + \rh^2} {\d N_{el} / \d t |_{t = 0\un{GeV^2}}\over N_{el} + N_{inel}}\qquad \rh(s) = {\Re T(s, 0)\over \Im T(s, 0)}$$
\itm Elastic scattering ...
\itm Diffraction and the Pomeron -- inelastic processes "not too different" from elastic, various $X \sim$ various physics
$$p + p \to p + X \qquad p + p \rightarrow p + p + X$$
\eitm

\newpage %%% 2b %%%
\centerline{\fig{width8cm}{figures/tdr10a.jpg}}
\centerline{\fig{width8cm}{figures/tdr10b.jpg}}

There is no satisfactory theory of diffraction so far. On the other hand there is a variety of phenomenological models and the TOTEM experiment will provide a tool to sort out the models that are based on true physical nature.

\newpage %%% 3 %%% -------------------------------------------------------------------------------- 
\Title{TOTEM detectors}
It will share location with CMS. The forward {\bf inelastic detectors} will be assembled in CMS.

\centerline{\fig{width7cm}{figures/tdr1.jpg}}

Elastic detectors will be placed far from IP in {\bf Roman pots}

\centerline{\fig{width7cm}{figures/tdr2.jpg}}


\newpage %%% 4 %%% --------------------------------------------------------------------------------
\Title{Roman pots}
\centerline{\fig{width4cm}{figures/tdr25.jpg}}
\centerline{\fig{width3cm}{figures/tdr34.jpg}}


\newpage %%% 5 %%% --------------------------------------------------------------------------------
\Title{TOTEM/CMS acceptance}
\centerline{\fig{width5cm}{figures/tdr9.jpg}}
\vskip\baselineskip
Angle corresponding to pseudorapidity $\et = 10$ is $\th \approx 2e^{-\et} \approx 9\cdot 10^{-5}$.


\newpage %%% 6 %%% --------------------------------------------------------------------------------
\Title{TOTEM elastic scattering aims}
\centerline{\fig{width7cm}{figures/tp22.jpg}}
\line{\fig{width3.5cm}{figures/app541_inter.jpg}\hfil\fig{width6.4cm}{figures/lhc.jpg}}
During one run, $\d\si/\d t$ can be measured in a relatively narrow $t$ region. Finally, these pieces of data will have to be fitted together.

\newpage %%% 7 %%% --------------------------------------------------------------------------------
\Title{What can one find in this thesis?}
\bitm{\parskip=10pt}
\itm Elastic $pp$ scattering analysis
\itmm Islam's model
\itmm Impact parameter point of view
\itmm Coulomb interference
\itm Calibration of detector alignment -- my summer project in CERN
\eitm


\newpage %%% 8 %%% --------------------------------------------------------------------------------
\Title{The model}
\bitm{}
\> Interesting is small $t$ region, where coupling constant is too large to use perturbative approach within QCD $\Rightarrow$ phenomenological models are employed.
\> Model of Islam et al. (\bibref{Islam1987}, \bibref{Islam2003}, \bibref{Islam2004}) was used. It is based on idea of proton structure. They start from $SU(3)_L\times SU(3)_R$ gauged nonlinear sigma model.

\centerline{\fig{width5cm}{figures/lag_sigma_model.jpg}}
\centerline{\fig{width4cm}{figures/nucleon.jpg}}
\eitm


\newpage %%% 8 %%% --------------------------------------------------------------------------------
\parskip=9pt
They distinguish 3 mechanisms (diffraction, core scattering and quark scattering)
\vskip-8pt
$$T(s, t) = T_D(s, t) + T_C(s, t) + T_Q(s, t) \qquad {\d\si\over\d t} = {\pi\over s p^2}\,|T(s, t)|^2$$

{\bf Diffraction} is a consequence of interaction of the outer clouds. The corresponding amplitude is parametrised
\vskip-8pt
$$T_D(s, t) \sim \int\limits_0^\infty b \d b\, J_0(b\sqrt{-t})\, \left[ {1\over 1 + e^{b-R\over a}} + {1\over 1 + e^{-b-R\over a}} - 1 \right ]$$

Scattering one core off the other gives birth to the {\bf core scattering}. It should be go via $\om$ exchange, therefore 
\vskip-8pt
$$T_C(s, t) \sim {s\,F^2(t)\over m^2_\om - t}$$
\parskip=0pt

\newpage %%% 9 %%% --------------------------------------------------------------------------------
{\bf Quark amplitude} reflects transition to perturbative regime of QCD. They take into account "distribution functions", elastic $qq$ amplitude. The latter is obtained with BFKL theory. The former leads to appearance of form factor ${\cal F}(q_\perp)$. The quark amplitude reads
$$T_Q(s, t) \sim {s\,{\cal F}(q_\perp)\over |t| + r_0^{-2}}$$

\vskip\baselineskip
\centerline{\fig{width7cm}{figures/quark_scattering.jpg}}


\newpage %%% 10 %%% -------------------------------------------------------------------------------
Alltogether, there are 17 (13) energy--independent parameters. Authors determined these parameters by fitting data on $\si_{tot}(s)$, $\rho(s)$ and $\d\si/\d t$ for $\bar pp$ at energies $\sqrt{s} = 541\un{GeV}$, $630\un{GeV}$ and $1.8 \un{TeV}$.

\line{\hskip-2mm\fig{width6.3cm}{figures/app541full.jpg}\hfil\fig{width4cm}{figures/app541.jpg}\hskip-2mm}


\newpage %%% 11 %%% -------------------------------------------------------------------------------
\centerline{\fig{width10cm}{figures/lhc.jpg}}


\newpage %%% 12 %%% -------------------------------------------------------------------------------
\Title{Impact parameter representation}
\bitm{}
\> Basic idea is to replace $T(s, t)$ by $A(s, b)$, where $b$ is impact parameter.
\> It naturally follows from eikonal approximation in QM.
\> There is precise formulation based on Fourier--Bessel transform \bibref{AK1965}
$$U(q) = c \int\limits_0^\infty b\, \d b\, J_0(bq)\,A(b) \qquad A(b) = {1\over c} \int\limits_0^\infty q\, \d q\, J_0(bq)\,U(q)$$
\> However, it requires both $A(b)$ and $U(q)$ to be defined on $(0, \infty)$. We want to identify $U(q)$ with amplitude $T(s, t=-q^2)$, but it is constrained to $t_{min} < t < 0$ $\Rightarrow$ we introduce arbitrary function $\tilde T$
$$U(q) = \left\{ %} for syntax highlighting
\vcenter{\halign{\strut $#$\hfil& \qquad #\hfil\cr
T(s, t=-q^2)& for $t_{min} < t < 0$\cr 
\tilde T(s, t=-q^2)& for $t < t_{min}$\cr
}} \right.$$
\eitm


\newpage %%% 13 %%% -------------------------------------------------------------------------------
\bitm{}
\> $\tilde T$ represents ambiguity in i.p. amplitude $\Rightarrow$ it is necessary to introduce a physical requirement for $\tilde T$, that would remove the ambiguity.
\> $A(s, b) = a(s, b) + \tilde a(s, b)$
$$\int\limits_0^\infty b\, \d b\,b^{2n}\,\tilde a(s, b) = 0 \qquad \int\limits_0^\infty b\,\d b\ \tilde a^*(s, b)\,a(s, b) = 0$$
\> $\displaystyle \sigma_{tot}(s) = 4\pi \int\limits_0^\infty \d^2 \vec b\ \Im A(s, b)\c \qquad \si_{el}(s) = 4\pi \int\limits_0^\infty \d^2 \vec b\ |a(s, b)|^2$
$\Rightarrow$ seems natural to interpret $4\pi\,\Im A(s, b)$ resp. $4\pi\,|a(s, b)|^2$ as total resp. elastic collisions density $\rh_{el,\, tot}(s, b)$. To do so, $\Im A(s, b)$ must be non--negative $\Rightarrow$ physical constraint on $\tilde a$ (or $\tilde T$).
\eitm


\newpage %%% 14 %%% -------------------------------------------------------------------------------
\bitm{}
\> If we adopt the density interpretation we can define mean value of $b^2$ (independent on $\tilde T$ choice)
$$\mean{b^2(s)}_{el,\, tot} = \displaystyle\int \d^2 \vec b\, b^2\, \rh_{el,\, tot}(s, b) \left( \displaystyle\int \d^2 \vec b\, \rh_{el,\, tot}(s, b) \right)^{-1}$$

\centerline{\fig{width8cm}{figures/mean_b2.jpg}}

\parindent=0pt
Islam model is central, with weak $t$ dependence of amplitude phase $\Rightarrow \mean b_{tot} > \mean b_{el}$. It is in contradiction with diffraction, which is described by peripheral models.
\eitm


\newpage %%% 15 %%% -------------------------------------------------------------------------------
\Title{Coulomb--Hadron interference}
\bitm{}
\> Two fundamental forces are essential for $pp$ scattering. Namely, electromagnetic and strong. We know scattering amplitudes for each interaction separately. The problem, how to determine amplitude for both interactions acting simultaneously.
\> One problem, two {\fPviii(approximate)} solutions:
\itcc=96
\itmm eikonal additivity way
\itmm summing some relevant Feynman diagrams
\eitm
\vskip\baselineskip
{\bf Eikonal} approximation in QM gives following prescription
$$A(s, b) = {e^{2\de(s, b)} - 1\over 2i} \qquad \de(s, b) = -{1\over 2p} \int\limits_0^\infty \d z\, V(\sqrt{b^2 + z^2})$$


\newpage %%% 16 %%% -------------------------------------------------------------------------------
Potential is additive $\Rightarrow$ eikonal $\de$ is additive. For $V = V_1 + V_2$, the total amplitude 
$$A(s, b) = A_1(s, b) + A_2(s, b) + 2iA_1(s, b) A_2(s, b)$$
Jump to the final result \bibref{Cahn}, \bibref{Kundrat1994}
$$\eqalign{
T(s, t) = \mp {\alpha s\over t}\,F^2(t)\ +\ T_H(s, t) \left[\vrule width0pt height20pt\right. &1 \pm i \alpha \int\limits_{t_{min}}^0 \!\!\d t' \left( \right. {\rm ln}{t'\over t}\ {\d F^2(t')\over \d t} -\cr
&\left.\left. - \left({T_H(s, t')\over T_H(s, t)}  - 1\right)\,I(t, t') \right) \vrule width0pt height20pt \right]\cr
}$$
$$I(t, t') = {1\over 2\pi} \int\limits_0^{2\pi} \d\ph {F^2(t'')\over t''}\qquad t'' = t + t' + 2\sqrt{tt'} \cos\ph$$
\bitm{}
\>> Two ways of possible use: $T \Rightarrow T_H$ and $T \Leftarrow T_H$
\>> One formula for whole $t$ region.
\eitm

\newpage %%% 17 %%% -------------------------------------------------------------------------------
{\bf Feynman diagrams technique} (West--Yennie formula) \bibref{West1968}, \bibref{Rix1966}
\bitm{}
\>> Uses scalar field for protons.
\>> Needs to know off shell hadron amplitude.
\>> A "model independent contribution" is introduced. It depends on amplitude on the mass shell only and does not suffer from IR divergence.
\>> Final formula is equivalent with simplified formula derived within the previous approach.
\eitm

\vskip\baselineskip
{\bf Conventional analysis} -- simplified WY formula assumes 
\bitm{}
\>> Purely exponential decay of amplitude in whole $t$ region.
\>> Constant phase in whole $t$ region.
\eitm

\vskip\baselineskip
Relative importance of interference
$$f(s, t) = {|T(s, t)|^2 - |T_H(s, t)|^2 - |T_C(s, t)|^2\over |T_H(s, t)|^2}$$


\newpage %%% 18 %%% -------------------------------------------------------------------------------
For $pp$ at energy $\sqrt s = 53\un{GeV}$

\line{\fig{width4.8cm}{figures/pp53_inter.jpg}\hfil\fig{width4.8cm}{figures/pp53_imp.jpg}}

Prediction for LHC has the same character, with 2 quantitative differences. The blue and the red line are closer (diff. cross section is higher) and the peak in $f(t)$ is at $t\approx 5\cdot 10^{-1}\un{GeV^2}$ (position of diffraction dip).


\newpage %%% 19 %%% -------------------------------------------------------------------------------
\Title{My summer project in CERN}
\vskip-\baselineskip
\title{calibration of detector alignment}

\bitm{\parskip=8pt}
\> In every roman pot, there will be a bunch of 10 planar strip detectors (with strip pitch $66\mu m$). One half having the strips perpendicular to the other half.
\> The detectors will be adapted for TOTEM special needs, they will be {\bf edgeless}.
\> This new technology need to be tested. For this pourpose it is crucial to know precise position of every detector.
\> Estimated maximum error of mechanical placement is $20\un{\mu m}$. This induces maximum possible slant $10^{-3}$.
\eitm


\newpage %%% 20 %%% -------------------------------------------------------------------------------
\title{Test detector geometry}
\centerline{\fig{width7cm}{figures/test_detector.jpg}}
\centerline{\fig{width5.8cm}{figures/test_detector2.jpg}}


\newpage %%% 21 %%% -------------------------------------------------------------------------------
\title{The method}
\bitm{\parskip=5pt}
\> We know distribution of tracks from beam source.
$$\mean a = 0,\quad \si_a = 3\cdot 10^{-3}, \qquad \mean b = 0\un{mm}, \qquad \si_b \approx 12\un{mm}$$
\> Difference in $x$ positions measured in $i$-th and $j$-th detector
\eitm
\vskip-5pt
$$\eqalign{D_{ij} = x_i' - x_j' =& \,a_x (z_i - z_j) - (\De x_i - \De x_j) - a_y (z_i s_i - z_j s_j) - \cr
&- b_y (s_i - s_j) + (\De y_i s_i - \De y_j s_j)\cr
}$$
\vskip5pt
\centerline{\fig{width8cm}{figures/order_estimate.jpg}}


\newpage %%% 22 %%% -------------------------------------------------------------------------------
\title{Shifts}
\bitm{\parskip=8pt}
\> Event-averaged value of $D_{ij}$
$$\eqalign{\bar D_{ij} = {1\over N} \sum_{n=0}^N D_{ij}^n =& - (\De x_i - \De x_j) + \bar a_x (z_i - z_j) -\cr 
& - \bar a_y (z_i s_i - z_j s_j) - \bar b_y (s_i - s_j)\cr
}$$
\> Deviation of arithmetic mean
$$\si_{\bar a_x} = {\si_{a_x}\over\sqrt N}$$
\> $\bar D_{ij} \approx - (\De x_i - \De x_j)$ with error $0.7\un{\mu m}$ for $5\cdot 10^4$ events.
\eitm


\newpage %%% 23 %%% -------------------------------------------------------------------------------
\title{Slants}
\bitm{\parskip=10pt}
\> Linear regression applied on data $D_{ij}^n$ versus $b_y^n$
$$T_{ij} = -(s_i - s_j) + F$$
\> It can be shown $\mean F = 0\un{m^{-1}}$ and $\si_F \sim 1/\sqrt{N}$
\> $T_{ij} \approx -(s_i - s_j)$ with error $2.2\cdot 10^{-4}$ for $5\cdot 10^4$ events.
\eitm


\newpage %%% 24 %%% -------------------------------------------------------------------------------
\title{Real situation}
\centerline{\fig{width6cm}{figures/det_size.jpg}}
\def\foo{\hbox to3mm{\hfil}}
$$h(a, b)= {g(b)\over \foo\int\limits_{b_{min}}^{b_{max}} g(b)\foo} \, {f(a)\over \foo\int\limits_{a_{min}(b)}^{a_{max}(b)} f(a)\foo}$$


\newpage %%% 24 %%% -------------------------------------------------------------------------------
\vbox to0pt{}\vfil
\Title{That's all, thank you for attention :-)}
\vskip-\baselineskip


\newpage %%% 25 %%% -------------------------------------------------------------------------------
\Title{References}
\fPviii
\PrintReferences{references.bib}
\end
