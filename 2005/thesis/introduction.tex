\section{Introduction}

Elastic scattering of hadrons is a process, measurement of which enables us to determine one of the basic characteristics of every hadron collider -- the luminosity. Without knowledge of this quantity, it is almost impossible to measure cross sections of other collision processes that are observed. Measurement of the luminosity has always finite precision. And this uncertainty affects precision of the measured cross sections. That is why the aim is to determine luminosity with maximum precision. The rate of elastic scattering is quite high and hence it is possible to gain experimental data with outstanding precision in wide region of momentum transfer.

A new hadron collider, the LHC, is being built at CERN. It will provide colliding protons with center--of--mass energy up to $14\un{TeV}$. And it will be the elastic scattering what will be measured first. However, at such a high energy, one can expect that majority of the collisions will be inelastic. But there is a group of inelastic collisions that have similar characteristics to the elastic processes. These processes are called diffraction production. And it is convenient to divide all processes to diffractive and non-diffractive rather than elastic and inelastic ones. All the diffraction processes exhibit a weak energy dependence and the incident particles scatter with small momentum transfer (i.e., they scatter to small angles). Conversely, the non--diffraction processes are strongly energy--dependent and the final state particles have large transverse momenta. The diffractive processes will be studied by the TOTEM experiment. The experiment itself will be able to measure and analyze about 60 \%
of diffractive channels including the elastic scattering. In cooperation with the CMS detector the ratio of measured diffraction channels can be extended to approximately 95 \%.

As already indicated, the diffractive final state contains initial particles beside the newly created particles. And the initial particles are scattered to very small angles. The angles are so small that these particles stay in the accelerator pipe. That is why the TOTEM experiment involves special detector systems called Roman pots. To measure particle tracks in the near forward direction, there will be 3 stations of Roman pots at distances of hundreds of meters from the interaction point. The Roman pot itself is a device that can insert detectors to the pipe and move them close to the beam, when the beam is stabilized. Edge of the detectors will be about $1\un{mm}$ far from the beam axis. The Roman pots must keep vacuum in the accelerator pipe during their operation.

While some of the inelastic processes are successfully described by perturbation QCD, there is no satisfying theory explaining diffraction phenomena. Hence, one has to use phenomenological models to describe diffraction. These models are built so as to obey rigorous theorems, such as Froissart, Pomeranchuk or Martin theorems. These theorems are usually derived within quantum field theory or analytic $S$--matrix framework and they usually present a requirement for infinite (asymptotic) energy behavior.

The submitted thesis is devoted to two problems. In the first chapter we analyze a model for $pp$ and $\bar pp$ elastic scattering. The model was developed by M.M.Islam and coworkers in the past 25 years. Our aim was to make prediction for differential cross section of $pp$ scattering at energy of $14\un{TeV}$ which will be measured by the TOTEM experiment. Since protons carry electromagnetic charge, we had to take into account electromagnetic interaction and effects of interference between electromagnetic and hadron forces. We also analyze the model in the impact parameter representation. It enables us to gain information about the range of hadron forces responsible for elastic, inelastic and total $pp$ and $\bar pp$ scattering. In the second chapter we present our alignment method for detectors inside the Roman pots. It was used during Roman pot tests on the SPS beam last year.

In the first chapter we will use natural units, i.e.,
$$\hbar = c = 1 \.$$
At the end, we will show how to correct our formulae to satisfy SI unit system.


\iffalse Pruzny rozptyl hadronu pri vysokych energiich je proces, jehoz provedeni umoznuje stanovit u kazdeho urychlovace vstricnych svazku, jako je napr. LHC v CERNu, jeho zakladni charakteristiky: totalni ucinny prurez a luminositu. Bez znalosti hodnot techto charakteristik neni prakticky mozne urcit ucinne prurezy jinych typu srazkovych procesu, ktere mohou na tomto urychlovaci nastat. Cetnost pruzneho neboli elastickeho procesu je pomerne velka, takze je mozno takovyto proces zrealizovat s pomerne vysokou  statistikou, coz umoznuje ziskat experimentalni data s mimoradne vysokou presnosti v pomerne sirokem intervalu kvadratu prenosu impulsu  t .

Je zrejme, ze pri tak vyskojych energiich bude dochazet i k inelastickym procesum, ve kterych bude dochazet k produkci mnoha castic. Mezi temito inelastickymi kanaly se vsak budou vyskytovat i takove procesy, ktere maji podobne charakteristiky jako elasticky proces. A takoveto procesy se nazyvaji difrakcne produkcni procesy. Takze misto deleni procesu na elasticky a inelasticke procesy, muzeme je rozdelit na difrakcni a nedifrakcni procesy. Jejich zakladni rozliseni spociva v tom, ze difrakcni procesy jsou slabe energeticky zavisle a jsou realizovany pri malych $|t|$, nedifrakcni procesy jsou silne zavisle na energii a k jejich realizaci dochazi pri velkych $|t|$. Mluvi se o tak zvanych hluboce nepruznych procesech. Prave studiu techto procesu se bude zabyvat zmineny experiment TOTEM, ktery je sam schopen krome studia pruzneho rozptylu provest i analyzu asi 60 \% difrakcnich kanalu a ve spolupraci s experimentem CMS pak dalsich 35 \% difrakcnich kanalu.

Pri difrakcnich rozptylech mame v koncovem stavu krome nove vyprodukovanych castic take ty castice, ktere byly v puvodnim pocatecnim stavu. V elestackem kanalu mame v koncovem stavu pouze tytez castice, jako v pocatecnim stavu. Protoze ale difrakcni procesy jsou realizovany s malym prenosem impulsu, jsou sekundarni castice, ktere jsou identicke s pocatecnimi casticemi, rozptyleny pod velmi malymi uhly a zustavaji uvnitr urychlovacich trubic. K jejich detekci se proto pouziva specialnich detektoru, zvanych rimske hrnce. To jsou zarizeni, ktera umoznuji zasunout do urychlovacich trubic primo detektory, ktere, kdyz jsou svazky stabilizovany, se priblizi tesne k temto svazkum a meri polohu rozptylenych castic, ktere jsou na vzdalenosti stovek metru od interakcniho bodu vzdaleny pouze o nekolik milimetru. Pritom tyto rimske hrnce musi byt konstruovany tak, aby behem zasunovani detektoru ke svazku nenarusovaly vakuum uvnitr urychlovacich trubic.

Zatimco hluboce nepruzne procesy jsou pomerne uspesna popsany pomoci poruchove QCD, pro popis difrakcnich procesu se doposud nenasla zadna teorie, ktera by byla schopna takoveto procesy, uskutecnovane prevazne pri malych $|t|$, vubec popsat. Proto se k popisu techto procesu pouziva prevazne fenomenologockych modelu, konstruovanych tak, aby mely charakteristicke rysy, ktere splnuji kriteria, ziskana pomoci tak zv. asymptotickych teoremu. Ty byly odvozeny budto v ramci polnich teorii, nebo na zaklade analytickych vlastnosti S matice, platnych v oblasti asymptotickych energii.

Na LHC bude nejprve studovan pruzny rozptyl protonu na protonech a to pri teztstove energii az 14 TeV. Protoze se jedna o rozptyl nabitych nukleonu, tak k nemu dochazi pusobenim jak hadronovych, tak i coulombickych interakci. Zatimco hadronovy rozptyl je realizovan pri vsech hodnotach  t, ke coulombickemu rozptylu dochazi prakticky pri malych $|t|$ .

Predladana diplomova prace se zabyva dvema problemy: predevsim se zabyva vypracovanim metodiky pro stanoveni vzajemne polohy rimskych hrncu vuci svazku, ktera byla zapotrebi pri testech rimskych hrncu na SPS svazku v CERNu minuleho roku. Druha cast pak obsahuje analyzu modelu pruzneho rozptylu protonu pri energii 14 TeV, navrzeneho Islamem a spolupracovniky behem minulych 25 let k popisu pruzneho hadronoveho rozptylu protonu na protonech. Cilem teto casti diplomove prace je ziskat predikce daneho modelu - t.j. stanovit volne parametry Islamova modelu prislusne hadronove amplitudy tak, aby bylo mozno tuto amplitudu jednoznacne urcit v cele kinematicke oblasti, ktera bude odpovidat experimentu TOTEM. dale pak provest analyzu tohoto modelu v prostoru srazkoveho parametru a purcit tak dosah hadrnovych sil zodpovednych za totalni, elasticky a inelasticky rozptyl protonu v oblatsi energii TOTEMu.
\fi

