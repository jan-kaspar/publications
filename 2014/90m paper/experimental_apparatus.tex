\section{Experimental apparatus}

While the full TOTEM apparatus is described elsewhere \cite{totem-jinst}, here only the sub-detector
relevant for elastic scattering measurement is outlined. The Roman Pots (RPs) are movable beam pipe
insertions that approach the LHC beam very closely in order to detect particles scattered to very 
small angles. They are organised in two stations, one on the left (sector 45), one on the right
(sector 56) side of the interaction point (IP). Each station is formed by two units: near ($214\un{m}$
from the IP) and far ($220\un{m}$). Each unit consists of three RPs: one approaching the beam from top,
one from bottom and one horizontally. Since elastic scattering events contain two anti-parallel protons,
the detected events can have two topologies, called diagonals: 45 bottom -- 56 top and 45 top -- 56 bottom.

This report will use a reference frame where $x$ denotes horizontal axis (pointing out of the LHC ring) and $y$ vertical axis (pointing against gravity).
