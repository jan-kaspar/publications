\section{Experimental apparatus}

The TOTEM experiment is located at the LHC interaction point (IP) 5 together with the CMS experiment. The full experimental apparatus is described elsewhere \cite{totem-jinst}, here only the sub-detector relevant for elastic scattering measurement is outlined. The Roman Pots (RPs) are movable beam-pipe
insertions that approach the LHC beam very closely in order to detect particles scattered to very small angles. They are organised in two stations placed symmetrically around the IP: one on the left side (in LHC sector 45), one on the right (sector 56). Each station is formed by two units: near ($214\un{m}$ from the IP) and far ($220\un{m}$). Each unit includes three RPs: one approaching the beam from top, one from bottom and one horizontally. Each RP hosts a stack of 10 silicon strip sensors (pitch $66\un{\mu m}$) with strongly reduced insensitive margin at the edge facing the beam (few tens of micrometres). The sensors are equipped with trigger-capable electronics. Since elastic scattering events consist of two anti-parallel protons, the detected events can have two topologies, called diagonals: 45 bottom -- 56 top and 45 top -- 56 bottom.

This report will use a reference frame where $x$ denotes horizontal axis (pointing out of the LHC ring) and $y$ vertical axis (pointing against gravity).
