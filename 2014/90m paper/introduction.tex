\section{Introduction}
%
The differential cross-section $\d \sigma/\d t$ of nuclear proton-%\linebreak
(anti)proton 
scattering at low $|t|$ has traditionally been parameterised with a simple 
exponential function, ${\rm e}^{-B |t|}$, giving a satisfactory description of 
all past experimental data.
Note however, that a few publications have already reported about hints at 
slight deviations from this simple behaviour. At the ISR, for $\sqrt{s}$ 
between 21.5\,GeV and 52.8\,GeV, elastic pp and partly $\rm\bar{p}$p data show a 
change of slope~\cite{plb39,plb115} or are better parameterised with quadratic 
exponential functions, ${\rm e}^{-B |t|-C t^2}$~\cite{npb141,npb248}. 
At the S$\rm\bar{p}$pS, for 
$\sqrt{s} = 546\,$GeV, a change of slope at $|t| \approx 0.14\,\rm GeV^{2}$ 
has been observed, while the inclusion of a quadratic term in the exponent did
not improve the fit significantly~\cite{plb147}.
At the LHC, at 7\,TeV as well as at 8\,TeV, all data published so 
far~\cite{epl96,epl101-el,prl111,alfa} have been
compatible with a pure exponential shape.

This report presents a new data sample of elastic scattering at the energy of $\sqrt s = 8\un{TeV}$. Thanks to its very high statistics,
%and an improved analysis technique
an unprecedented precision has been reached in the region of four-momentum transfer squared $|t| \lesssim 0.2\un{GeV^2}$. Both statistical and systematic components of the differential cross-section uncertainty are controlled 
at a level below $1\un{\%}$, except for the normalisation 
(Section~\ref{sec:normalisation}). Consequently, the functional form of the cross-section can be strongly constrained, thus having more impact on theoretical model building and, in particular, extrapolation to $t=0$, used for total cross-section determination. The often used purely exponential extrapolation has been found inadequate, and extended models are provided, still yielding total cross-section values compatible with previous TOTEM results~\cite{prl111}.
