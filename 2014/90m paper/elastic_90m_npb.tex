\documentclass[3p,twocolumn]{elsarticle}

\usepackage{amsmath}
\usepackage{amssymb}
\usepackage{graphicx}
\usepackage{color}
\usepackage{mathptmx}

%------------------------------------------------------------------------------

\def\d{{\rm d}}
\def\un#1{\,{\rm #1}}
\def\ung#1{\quad[{\rm #1}]}
\def\unt#1{[{\rm #1}]}
\def\e{{\rm e}}
\def\I{{\rm i}}
\def\T{{\rm T}}
\def\vec#1{\mathbf{#1}}
\def\mat#1{\mathsf{#1}}
\def\etal{et al.}
\def\todo#1{{\color{red} #1}}

\setbox123\hbox{$0$}
\setbox124\hbox{$.$}
\def\S{\hbox to\wd123{\hss}}
\def\.{\hbox to\wd124{\hss}}

\def\Name#1{\textsc{#1}, }
\def\REVIEW#1#2#3#4{{\it #1} {\bf #2} (#3) #4}

%------------------------------------------------------------------------------

\journal{Nuclear Physics B}

\begin{document}

\begin{frontmatter}

\title{Evidence for Non-Exponential Elastic Differential Cross-Section at Low $|\hbox{t}|$ and $\sqrt{\hbox{s}}$ = 8 TeV by TOTEM at the CERN LHC}

\input authorlist_npb

\date{\today}


\begin{abstract}
The TOTEM experiment has made a high-precision measurement of the elastic 
proton-proton differential cross-section at the centre-of-mass energy 
$\sqrt s = 8\un{TeV}$ based on a new high-statistics data sample obtained with 
the $\beta^* = 90\un{m}$ optics. 
%Furthermore, thanks to an improved analysis procedure, an unprecedented precision has been reached: 
Both statistical and systematic uncertainties remain below $1\un{\%}$, except 
the contribution from the overall normalisation which was obtained from 
another data set. This unprecedented precision allows to exclude a purely exponential cross-section in the range of four-momentum transfer squared $0.027 < |t| < 0.2\un{GeV^2}$ with a significance greater than $7\un{\sigma}$. Consequently, the extrapolation to $t=0$ is reviewed and two extended models are presented, yielding total cross-section estimates $(100.8 \pm 2.1)\un{mb}$ and $(101.2 \pm 2.1)\un{mb}$, well compatible with previous TOTEM measurements.
%
%Analysis of a new dataset obtained at $\sqrt{s} = 8\un{TeV}$ and $\beta^* = 90\un{m}$ optics.
%High statistics (more than $7\cdot10^{6}$ elastic events)
%$\Rightarrow$ strongly reduced statistical and systematic uncertainties (each below $1\un{\%}$) % excluding normalisation unc.
%$\Rightarrow$ purely exponential behaviour of differential cross-section excluded at more than $7\un{\sigma}$ significance
%$\Rightarrow$ new determination of the total cross-section (non-exponential extrapolation), values, but still compatible with the previous TOTEM measurements.
\end{abstract}

\begin{keyword}
elastic scattering \sep total cross-section \sep LHC \sep TOTEM
\PACS 13.60.Hb % Total and inclusive cross sections (including deep-inelastic processes)
%\preprint{CERN-PH-EP-2014-XXXX}
\end{keyword}
\end{frontmatter}

%--------------------------------------------------

\input introduction

\input experimental_apparatus

\input data_taking

\input differential_cross_section

%--------------------------------------------------

\section{Conclusions and Outlook}
%
Thanks to a very-high statistics data set
%and an improved analysis technique,
TOTEM has experimentally excluded a purely exponential differential 
cross-section for elastic proton-proton scattering with more than $7\,\sigma$
in the $|t|$ range from 0.027 to 0.2\,GeV$^{2}$ at $\sqrt{s}=8\,$TeV. The data
are described satisfactorily with an exponent linear or quadratic in $t$.
Using this refined parameterisation for the extrapolation to the optical point,
$t = 0$, yields a total cross-section compatible with the previous measurement.

In the next article~\cite{1km}, this proof of non-exponentiality in a 
$t$-domain strongly
dominated by hadronic interactions will be combined with a measurement of 
elastic scattering in the Coulomb-nuclear interference region, providing strong
constraints on the phenomenological description of the hadronic phase and of 
the interference mechanism.


%--------------------------------------------------

\section*{Acknowledgements}
We are grateful to the beam optics development team for the design and the 
successful commissioning of the high $\beta^{*}$ optics and to the LHC machine 
coordinators for scheduling the dedicated fills.

We thank B. Alberski, P. Anielski, M. Idzik, I. Jurkowski, R. Lazarz, 
B. Niemczura for their help in software development.

This work was supported by the institutions listed on the front page and 
partially also by NSF (US), the Magnus Ehrnrooth Foundation (Finland), the 
Waldemar von Frenckell Foundation (Finland), the Academy of Finland, the 
Finnish Academy of Science and Letters (The Vilho, Yrj\"o and Kalle 
V\"ais\"al\"a Fund), the OTKA grant NK 101438, 73143 (Hungary) and the 
NKTH-OTKA grant 74458 (Hungary).


%----------------------------------------------------------------------------------------------------
\begin{thebibliography}{99}
%
\bibitem{plb39} \Name{G. Barbiellini et al.} \REVIEW{Phys.~Lett.}{B39}{1972}{663}

\bibitem{plb115} \Name{M. Ambrosio et al.} \REVIEW{Phys.~Lett.}{B115}{1982}{495}

\bibitem{npb141} \Name{L. Baksay et al.} \REVIEW{Nucl.~Phys.}{B141}{1978}{1}

\bibitem{npb248} \Name{A. Breakstone et al.} \REVIEW{Nucl.~Phys.}{B248}{1984}{253}

\bibitem{plb147} \Name{M. Bozzo et al.} \REVIEW{Phys.~Lett.}{B147}{1984}{385}

\bibitem{epl96}
    %First measurements of the total proton-proton cross-section at the LHC energy of $\sqrt s =7\,\rm TeV$ CERN-PH-EP-2011-158
	\Name{G.~Antchev~\etal{}~(TOTEM Collaboration)}
	\REVIEW{Europhys.~Lett.}{96}{2011}{21002}

\bibitem{epl101-el}
	\Name{G.~Antchev~\etal{}~(TOTEM Collaboration)}
	\REVIEW{Europhys.~Lett.}{101}{2013}{21002}

\bibitem{prl111} 
	\Name{G.~Antchev~\etal{}~(TOTEM Collaboration)}
	\REVIEW{Phys.~Rev.~Lett.}{111}{2013}{012001}

\bibitem{alfa} \Name{The ATLAS Collaboration} ATLAS-CONF-2014-040.

\bibitem{totem-jinst}
    %The TOTEM Experiment at the CERN Large Hadron Collider, JINST 3 S08007, 2008
	\Name{G.~Anelli~\etal{}~(TOTEM Collaboration)}
	\REVIEW{JINST}{3}{2008}{S08007}


\bibitem{totem-ijmp}
	% Performance of the Totem Detectors at the LHC
	\Name{G. Antchev~\etal{}~(TOTEM Collaboration)}
	\REVIEW{Int.~J.~Mod.~Phys.~A}{28}{2013}{1330046}

\bibitem{totem-optics}
	\Name{G.~Antchev~\etal{}~(TOTEM Collaboration)}
	LHC Optics Measurement with Proton Tracks Detected by the Roman Pots of the TOTEM Experiment, 
	arXiv:1406.0546, accepted by New Journal of Physics.

\bibitem{compete} 
	\Name{J.R.~Cudell~\etal{} (COMPETE Collaboration)}
	\REVIEW{Phys.\ Rev.\ Lett.}{89}{2002}{201801}

\bibitem{lafferty94}
 	% Where to stick your data points: The treatment of measurements within wide bins
	\Name{G.D.~Lafferty~and T.R.~Wyatt}
	\REVIEW{Nucl.\ Instrum.\ Meth.}{A 355}{1995}{541}

\bibitem{1km}
	\Name{G.~Antchev~\etal{}~(TOTEM Collaboration)}
        Measurement of Elastic pp Scattering at $\sqrt{s}=8\,$TeV in the 
        Coulomb-Nuclear Interference Region by the TOTEM Experiment at the 
        CERN LHC, this journal.

\iffalse

\bibitem{epl95}
    %Proton-proton elastic scattering at the LHC energy of \sqrt{s} = 7 TeV, Europhys. Lett. 95 (2011) 41001,CERN-PH-EP-2011-101 
	\Name{Antchev G.~\etal{}~(TOTEM Collaboration)}
	\REVIEW{Europhys.~Lett.}{95}{2011}{41001}


\bibitem{epl101-inel}
	\Name{Antchev G.~\etal{}~(TOTEM Collaboration)}
	\REVIEW{Europhys.~Lett.}{101}{2013}{21003}

\bibitem{epl101-tot}
	\Name{Antchev G.~\etal{}~(TOTEM Collaboration)}
	\REVIEW{Europhys.~Lett.}{101}{2013}{21004}

\bibitem{jan_thesis}
	\Name{Ka\v spar J.}
	PhD Thesis, CERN-THESIS-2011-214, {\tt http://cdsweb.cern.ch/record/1441140}

\bibitem{mario_ipac_2011}
	\Name{Deile M.}
	{\it The First 1 1/2 Years of TOTEM Roman Pot Operation at LHC}, in
	{\it Proceedings of the 2nd International Particle Accelerator Conference (IPAC 2011), San Sebastian, Spain}. 
	%{\tt http://accelconf.web.cern.ch/AccelConf/IPAC2011/papers/mopo011.pdf}
	arXiv:1110.5808v1

%\bibitem{pdg} 
%	\Name{Nakamura K.~\etal{} (Particle Data Group)}
%	\REVIEW{J.~Phys.}{G37}{2010}{075021}

%\bibitem{B_vs_s}
%	\Name{ISR (CR Collaboration)} \REVIEW{Phys.~Lett.}{B62}{1976}{460}; 
%	\Name{ISR (ACHGT Collaboration)} \REVIEW{Phys.~Lett.}{B39}{1972}{663}; 
%	\Name{ISR (R-211)} \REVIEW{Nucl.~Phys.}{B262}{1985}{689}; 
%	\Name{ISR (R-210)} \REVIEW{Phys.~Lett.}{B115}{1982}{495}; 
%	\Name{UA4} \REVIEW{Phys.~Lett.}{B147}{1984}{385}; 
%	\Name{UA4} \REVIEW{Phys.~Lett.}{B127}{1983}{472} and \REVIEW{Phys. Lett.}{B198}{1987}{583}; 
%	\Name{UA4/2} \REVIEW{Phys.~Lett.}{B316}{1993}{448}; 
%	\Name{CDF} \REVIEW{Phys.~Rev.}{D50}{1994}{5518}; 
%	\Name{E710} \REVIEW{Phys.~Rev.~Lett.}{68}{1992}{2433} and \REVIEW{Nuovo Cimento}{A106}{1992}{123}; 
%	\Name{D0} D0 Note 6056-CONF; 
%	\Name{pp2pp} \REVIEW{Phys.~Lett.}{B579}{2004}{245}

\fi

\end{thebibliography}



\end{document}
