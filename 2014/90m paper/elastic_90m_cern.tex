\documentclass[TOTEM]{cern/cernphprep}

\def\d{{\rm d}}
\def\un#1{\,{\rm #1}}
\def\ung#1{\quad[{\rm #1}]}
\def\unt#1{[{\rm #1}]}
\def\e{{\rm e}}

\setbox123\hbox{\small$0$}
\def\S{\hbox to\wd123{\hss}}
\setbox124\hbox{\small$_{0}$}
\def\s{\hbox to\wd124{\hss}}

\def\etal{et al.}
\def\acknowledgments{\section*{Acknowledgements}}
\def\Name#1{\textsc{#1}, }
\def\REVIEW#1#2#3#4{{\it #1} {\bf #2} (#3) #4}

\def\Instline#1#2{%
	\expandafter\write1{\string\newlabel{#1}{{#1}{}}}%
	\hbox to\hsize{\strut\hss$^{#1}$#2\hss}
}

\def\hang{\hangindent=\parindent}
\catcode`\>=11
\newskip\itskip \itskip2mm
\newskip\iitskip \iitskip0mm
\newdimen\itindent \itindent3mm
\newdimen\iitindent \iitindent5mm
\def\>{\par\vskip\itskip\parindent\itindent\indent\hang\llap{\hbox to3mm{$\bullet$\hss}}}
\def\>E{\par\vskip\itskip\parindent\itindent\indent\hang\llap{\hbox to3mm{\hss}}}
\def\>>{\par\vskip\iitskip\parindent\iitindent\indent\hang\llap{\hbox to\iitindent{\hss--\ }}}



%----------------------------------------------------------------------------------------------------

\begin{document}

\begin{titlepage}

\renewcommand{\EXPLOGO}{fig/logo_totem_black.pdf}

\PHnumber{XXXX}
\PHdate{XXXX}

\EXPnumber{XXXX}
\EXPdate{XXXX}

%\title{Measurement of Elastic pp Scattering at $\sqrt{\hbox{s}} = \hbox{8}$\,TeV in the 
%Coulomb-Nuclear Interference Region by the TOTEM Experiment at the CERN LHC}

\title{TODO -- title}

\ShortTitle{TODO -- short title}

\Collaboration{The TOTEM Collaboration}
\ShortAuthor{The TOTEM Collaboration (G.~Antchev \emph{\etal})}

\begin{Authlist}
TODO -- names
\iffalse
G.~Antchev\Aref{a},
P.~Aspell\Iref{8},
I.~Atanassov\IAref{8}{a},
V.~Avati\Iref{8},
J.~Baechler\Iref{8},
V.~Berardi\IIref{5b}{5a},
M.~Berretti\Iref{7b},
E.~Bossini\Iref{7b},
M.~Bozzo\IIref{6b}{6a},
P.~Brogi\Iref{7b},
E.~Br\"{u}cken\IIref{3a}{3b},
A.~Buzzo\Iref{6a},
F.~S.~Cafagna\Iref{5a},
M.~Calicchio\IIref{5b}{5a},
M.~G.~Catanesi\Iref{5a},
C.~Covault\Iref{9},
M.~Csan\'{a}d\IAref{4}{e},
T.~Cs\"{o}rg\H{o}\Iref{4},
M.~Deile\Iref{8},
K.~Eggert\Iref{9},
V.~Eremin\Aref{b},
R.~Ferretti\IIref{6a}{6b},
F.~Ferro\Iref{6a},
A. Fiergolski\Aref{c},
F.~Garcia\Iref{3a},
S.~Giani\Iref{8},
V.~Greco\IIref{7b}{8},
L.~Grzanka\IAref{8}{d},
J.~Heino\Iref{3a},
T.~Hilden\IIref{3a}{3b},
R.~A.~Intonti\Iref{5a},
J.~Ka\v{s}par\IIref{1a}{8},
J.~Kopal\IIref{1a}{8},
V.~Kundr\'{a}t\Iref{1a},
K.~Kurvinen\Iref{3a},
S.~Lami\Iref{7a},
G.~Latino\Iref{7b},
R.~Lauhakangas\Iref{3a},
T.~Leszko\Aref{c},
E.~Lippmaa\Iref{2},
M.~Lokaj\'{\i}\v{c}ek\Iref{1a},
M.~Lo~Vetere\IIref{6b}{6a},
F.~Lucas~Rodr\'{i}guez\Iref{8},
M.~Macr\'{\i}\Iref{6a},
T.~M\"aki\Iref{3a},
A.~Mercadante\IIref{5b}{5a},
N.~Minafra\Iref{8} ,
S.~Minutoli\Iref{6a},
F.~Nemes\IAref{4}{e},
H.~Niewiadomski\Iref{8},
E.~Oliveri\Iref{7b},
F.~Oljemark\IAref{3a}{3b},
R.~Orava\IIref{3a}{3b},
M.~Oriunno\IAref{8}{f},
K.~\"{O}sterberg\IIref{3a}{3b},
P.~Palazzi\Iref{7b},
J.~Proch\'{a}zka\Iref{1a},
M.~Quinto\Iref{5a},
E.~Radermacher\Iref{8},
E.~Radicioni\Iref{5a},
F.~Ravotti\Iref{8},
E.~Robutti\Iref{6a},
L.~Ropelewski\Iref{8},
G.~Ruggiero\Iref{8},
H.~Saarikko\IIref{3a}{3b},
A.~Santroni\IIref{6b}{6a},
A.~Scribano\Iref{7b},
J.~Smajek\Iref{8},
W.~Snoeys\Iref{8},
J.~Sziklai\Iref{4},
C.~Taylor\Iref{9},
N.~Turini\Iref{7b},
V.~Vacek\Iref{1b},
M.~V\'itek\Iref{1b},
J.~Welti\IIref{3a}{3b} and
J.~Whitmore\Iref{10}
\fi
\end{Authlist}

\Instline{1a}{Institute of Physics of the Academy of Sciences of the Czech Republic, Praha, Czech Republic.}
\Instline{1b}{Czech Technical University, Praha, Czech Republic.}
\Instline{2} {National Institute of Chemical Physics and Biophysics NICPB, Tallinn, Estonia.}
\Instline{3a}{Helsinki Institute of Physics, Finland.}
\Instline{3b}{Department of Physics, University of Helsinki, Finland.}
\Instline{4} {MTA Wigner Research Center, RMKI, Budapest, Hungary.}
\Instline{5a}{INFN Sezione di Bari, Italy.}
\Instline{5b}{Dipartimento Interateneo di Fisica di Bari, Italy.}
\Instline{6a}{Sezione INFN, Genova, Italy.}
\Instline{6b}{Universit\`{a} degli Studi di Genova, Italy.}
\Instline{7a}{INFN Sezione di Pisa, Italy.}
\Instline{7b}{Universit\`{a} degli Studi di Siena and Gruppo Collegato INFN di Siena, Italy.}
\Instline{8} {CERN, Geneva, Switzerland.}
\Instline{9} {Case Western Reserve University, Dept. of Physics, Cleveland, OH, USA.}
\Instline{10}{Penn State University, Dept.~of Physics, University Park, PA, USA.}

\Anotfoot{a}{INRNE-BAS, Institute for Nuclear Research and Nuclear Energy, Bulgarian Academy of Sciences, Sofia, Bulgaria.}
\Anotfoot{b}{Ioffe Physical - Technical Institute of Russian Academy of Sciences.}
\Anotfoot{c}{Warsaw University of Technology, Poland.}
\Anotfoot{d}{Institute of Nuclear Physics, Polish Academy of Science, Cracow, Poland.}
\Anotfoot{e}{Department of Atomic Physics, E\"otv\"os University, Hungary.}
\Anotfoot{f}{SLAC National Accelerator Laboratory, Stanford CA, USA.}

%\newpage

\begin{abstract}
TODO
\end{abstract}
\end{titlepage}


%----------------------------------------------------------------------------------------------------
\section{Introduction}

\> new: high-statistics data sample
\>> eventually beyond dip/bumb
\>> for the moment only what is relevant for extrapolations to $t=0$

SOMEWHERE: x = horizontal, y = vertical

%----------------------------------------------------------------------------------------------------
\section{Experimental apparatus}

Detailed description elsewhere \cite{totem-jinst} - TODO, focus on RPs 220m only
\> structure: 2 arms, near and far units, 2 diagonals, ...

\> Minimal description of optics -- or in later sections ??

%----------------------------------------------------------------------------------------------------
\section{Data taking}

July 2012, typical bunch population, typ. inst. luminosity, beta star, trigger settings

TODO: why DS1 not used, TODO: why DS3 and DS4 separated -- different trigger

\begin{table}
\caption{Description of the three datasets available. The RP position gives the RP approach to the beam in multiples of the beam size ($\sigma_{\rm beam}$). The third column summarizes the numbers of elastic events reconstructed from both diagonals. $\mathcal{L}_{\rm int}$ is the integrated
luminosity for each dataset, accounting for the data-acquisition (DAQ) inefficiency. The last column shows the lowest $|t|$ values observed.
}
\label{tab:datasets}
\begin{center}
\vskip-3mm
\begin{tabular}{ccccccc}\hline\hline
    & LHC  & RP                       & bunch   & elastic         & $\mathcal{L}_{\rm int}$ & $|t|_{\rm min}$     \cr
\omit\hss\vbox to 0pt{\vss\hbox{\ dataset\ }\vss}\hss &\multispan4 \cr
    & fill &  position                & pairs   & events          & $\unt{\mu b^{-1}}$      & $\unt{GeV^2}$       \cr\hline
DS2 & 2815 & $6.0\,\sigma_{\rm beam}$ & 1       & $0.40\cdot10^6$ & $34$                    & $0.85\cdot10^{-2}$  \cr
DS3 & 2836 & $9.5\,\sigma_{\rm beam}$ & 1       & $0.22\cdot10^6$ & $25$                    & $1.85\cdot10^{-2}$  \cr
DS4 & 2836 & $9.5\,\sigma_{\rm beam}$ & 2 to 3  & $6.74\cdot10^6$ & $735$                   & $1.83\cdot10^{-2}$  \cr\hline\hline
\end{tabular}
\end{center}
\end{table}


%----------------------------------------------------------------------------------------------------
\section{Analysis}

\> Independent analyses: datasets, diagonals
\>> some steps done independent between bunches

\> iterations ??

%--------------------------------------------------
\subsection{Event reconstruction}

\begin{equation}
t = p^2 ({\theta_x^*}^2 + {\theta_y^*}^2)
\end{equation}
TODO: projections of scattering angle

One-arm reconstruction -- minimise uncertainties due to optics imperfections
\begin{equation}
\theta_x^* = {v_x^{\rm N} x^{\rm F} - v_x^{\rm F} x^{\rm N}\over v_x^{\rm N} L_x^{\rm F} - v_x^{\rm F} L_x^{\rm N}}\ ,\qquad
\theta_y^* = {1\over 2} \left( {y^{\rm N}\over L_y^{\rm N}} + {y^{\rm F}\over L_y^{\rm F}} \right)\ ,\qquad
x^* = {L_x^{\rm N} x^{\rm F} - L_x^{\rm F} x^{\rm N}\over L_x^{\rm N} v_x^{\rm F} - L_x^{\rm F} v_x^{\rm N}}
\end{equation}
x formula -- suppresses the impact of vertex.
Double-arm reconstruction -- TODO.
Single arm for cuts, double-arm for physics.

{\bf alignment} Standard three-step procecure, details \cite{totem-ijmp}. Uncertainty: shifts $2\un{\mu m}$ (horizontal), $100\un{\mu m}$ (vertical) and rotations $0.2\un{mrad}$ (for each unit -- common for top and bottom RPs). Impact on angles (double-arm), horizontal shift $0.8\un{\mu rad}$, vertical shift $0.2\un{\mu rad}$ (effect disappears when diagonals combined). % rotation th_x^reco = th_x^true + A * th_y, si[A] = 0.02

{\bf optics} Matching \cite{totem-optics}, before: scale uncertainty (double-arm) $0.48\un{\%}$ (hor), $0.90\un{\%}$ (ver). After: scale uncertainty $0.21\un{\%}$ (hor), $0.25\un{\%}$ (ver). [not so spectacular difference: formula chosen to minimise the impact].

{\bf resolutions} Beyond systematic uncertainties mentioned above, there are also statistical fluctuations -- beam divergence and for horizontal projection also detector resolution (low Lx).

%--------------------------------------------------
\subsection{Differential cross-section reconstruction}

For a given $t$ bin, the value of differential cross-section is evaluated by selecting and counting elastic events as follows
\begin{equation}
{\d\sigma\over \d t}(\hbox{bin}) =
	{\cal N} {\cal U} {\cal B}\ 
	{\sum\limits_{t \in \hbox{bin}} {\cal A}(\theta_x^*, \theta_y^*) {\cal E}(\theta_y^*)\over \Delta t}\ ,
\end{equation}
where $\Delta t$ is the width of the bin, ${\cal N}$ is a normalisation factor, ${\cal U}$ is an unfolding correction,
${\cal B}$ is a backround subtraction factor, ${\cal A}$ is an acceptace correction and ${\cal E}$ is an efficiency correction.

{\bf Tagging}. The cuts used to select the elastic events are summarized in Tab.~\ref{tab:cuts}. Cuts 1 and 2 require the reconstructed-track collinearity between the left and right arm. Cuts 3 to 6 effectively work as low-$\xi$ cuts ($\xi$ being the fractional momentum loss of a proton).
Cuts 5 and 6: if $\xi\neq 0$, the correlation between the track position ($y^{\rm N}$) and the track angle (proportional to $y^{\rm F} - y^{\rm N}$) is lost. Cut 7 compares the horizontal vertex position reconstructed from the left and right arms. TODO: tagging inefficiency ?

\begin{table}
\caption{The elastic selection cuts. The superscripts R and L refer to the right and left arm, the N and F corresponds to the near and far units. The constant $\alpha = L_y^{\rm F} / L_y^{\rm N} - 1 \approx 0.107$. The right-most column gives the typical (there is diagonal and dataset dependence) RMS of the cut distribution ($\equiv 1\sigma$), all the cuts are applied at $4\sigma$-level.
}
\label{tab:cuts}
\begin{center}
\vskip-3mm
\begin{tabular}{ccc}\hline\hline
number & cut & RMS\cr\hline
diagonal &\multispan2 \hss track reconstructed in all 4 diagonal RPs \hss \cr
1 & $\theta_x^{*\rm R} - \theta_x^{*\rm L}$				& $9.5\un{\mu rad}$	\cr
2 & $\theta_y^{*\rm R} - \theta_y^{*\rm L}$				& $3.3\un{\mu rad}$	\cr
%3 & $|x^{*\rm R}|$ 										& $\un{\mu m}$	\cr
%4 & $|x^{*\rm L}|$ 										& $\un{\mu m}$	\cr
5 & $\alpha\,y^{\rm R,N} - (y^{\rm R,F} - y^{\rm R,N})$	& $18\un{\mu m}$	\cr
6 & $\alpha\,y^{\rm L,N} - (y^{\rm L,F} - y^{\rm L,N})$	& $18\un{\mu m}$	\cr
7 & $x^{*\rm R} - x^{*\rm L}$							& $8.5\un{\mu m}$ 	\cr\hline\hline
\end{tabular}
\end{center}
\end{table}

{\bf background}

{\bf acceptance correction}

{\bf efficiency corrections}

{\bf unsmearing}

{\bf normalisation}

{\bf binning}

{\bf final data merging}

{\bf propagation of systematics}

%----------------------------------------------------------------------------------------------------
\section{Results}

\> Give data table (values + uncertainty matrix?), plot(s)
\>> bin representative points \cite{lafferty94}

\> Describe exclusion of pure exponential - several methods

\> new determination of si tot?

%----------------------------------------------------------------------------------------------------
\section{Discussion/Conclusions}

\> Merge to results?

%----------------------------------------------------------------------------------------------------
\section{Outlook}

\> will extend this analysis beyond the dip


%----------------------------------------------------------------------------------------------------
\acknowledgments

\iffalse
We are indebted to the beam optics development team
%({\sc A.~Verdier} in the initial phase, {\sc H.~Burkhardt}, {\sc G.~M\" uller}, {\sc S.~Redaelli}, {\sc J.~Wenninger}, {\sc S.~M.~White})
for the design, the thorough preparations and the successful commissioning of the $\beta^* = 90\un{m}$ optics. We congratulate the CERN accelerator groups for the very smooth operation in 2011. We thank
%{\sc M.~Ferro-Luzzi}
the LHC machine coordinators for scheduling the dedicated fills.

We are grateful to CMS for providing their luminosity measurements.

This work was supported by the institutions listed on the front page and partially also by NSF (US), the Magnus
Ehrnrooth foundation (Finland), the Waldemar von Frenckell foundation (Finland), the Academy of
Finland, the OTKA grant NK 101438, 73143 (Hungary) and the NKTH-OTKA grant 74458 (Hungary).

\fi

%----------------------------------------------------------------------------------------------------
\begin{thebibliography}{99}

\bibitem{totem-jinst}
    %The TOTEM Experiment at the CERN Large Hadron Collider, JINST 3 S08007, 2008
	\Name{Anelli G.~\etal{}~(TOTEM Collaboration)}
	\REVIEW{JINST}{3}{2008}{S08007}

\bibitem{totem-ijmp}
	\Name{Antchev G.~\etal{}~(TOTEM Collaboration)}
	\REVIEW{Int.~J.~Mod.~Phys.~A}{28}{2013}{1330046}

\bibitem{totem-optics}
	\Name{Antchev G.~\etal{}~(TOTEM Collaboration)}
	LHC Optics Measurement with Proton Tracks Detected by the Roman Pots of the TOTEM Experiment, 
	arXiv:1406.0546

\bibitem{lafferty94}
 	% Where to stick your data points: The treatment of measurements within wide bins
	\Name{Lafferty G.~D.~and Wyatt T.~R.}
	\REVIEW{Nucl.\ Instrum.\ Meth.}{A 355}{1995}{541}


\iffalse

\bibitem{epl95}
    %Proton-proton elastic scattering at the LHC energy of \sqrt{s} = 7 TeV, Europhys. Lett. 95 (2011) 41001,CERN-PH-EP-2011-101 
	\Name{Antchev G.~\etal{}~(TOTEM Collaboration)}
	\REVIEW{Europhys.~Lett.}{95}{2011}{41001}

\bibitem{epl96}
    %First measurements of the total proton-proton cross-section at the LHC energy of $\sqrt s =7\,\rm TeV$ CERN-PH-EP-2011-158
	\Name{Antchev G.~\etal{}~(TOTEM Collaboration)}
	\REVIEW{Europhys.~Lett.}{96}{2011}{21002}

\bibitem{epl101}
	\Name{Antchev G.~\etal{}~(TOTEM Collaboration)}
	\REVIEW{Europhys.~Lett.}{101}{2013}{21004}

\bibitem{prl111}
	\Name{Antchev G.~\etal{}~(TOTEM Collaboration)}
	\REVIEW{Phys.~Rev.~Lett.}{111}{2013}{012001}

\bibitem{P2} 
	\Name{Antchev G.~\etal{}~(TOTEM Collaboration)}
	%\REVIEW{Europhys.~Lett.}{TODO}{2012}{TODO}
	CERN-PH-EP-2012-352

\bibitem{P3} 
	\Name{Antchev G.~\etal{}~(TOTEM Collaboration)}
	%\REVIEW{Europhys.~Lett.}{TODO}{2012}{TODO}
	CERN-PH-EP-2012-353

\bibitem{jan_thesis}
	\Name{Ka\v spar J.}
	PhD Thesis, CERN-THESIS-2011-214, {\tt http://cdsweb.cern.ch/record/1441140}

\bibitem{mario_ipac_2011}
	\Name{Deile M.}
	{\it The First 1 1/2 Years of TOTEM Roman Pot Operation at LHC}, in
	{\it Proceedings of the 2nd International Particle Accelerator Conference (IPAC 2011), San Sebastian, Spain}. 
	%{\tt http://accelconf.web.cern.ch/AccelConf/IPAC2011/papers/mopo011.pdf}
	arXiv:1110.5808v1

%\bibitem{pdg} 
%	\Name{Nakamura K.~\etal{} (Particle Data Group)}
%	\REVIEW{J.~Phys.}{G37}{2010}{075021}

\bibitem{B_vs_s}
	\Name{ISR (CR Collaboration)} \REVIEW{Phys.~Lett.}{B62}{1976}{460}; 
	\Name{ISR (ACHGT Collaboration)} \REVIEW{Phys.~Lett.}{B39}{1972}{663}; 
	\Name{ISR (R-211)} \REVIEW{Nucl.~Phys.}{B262}{1985}{689}; 
	\Name{ISR (R-210)} \REVIEW{Phys.~Lett.}{B115}{1982}{495}; 
	\Name{UA1} \REVIEW{Phys.~Lett.}{B147}{1984}{385}; 
	\Name{UA4} \REVIEW{Phys.~Lett.}{B127}{1983}{472} and \REVIEW{Phys. Lett.}{B198}{1987}{583}; 
	\Name{UA4/2} \REVIEW{Phys.~Lett.}{B316}{1993}{448}; 
	\Name{CDF} \REVIEW{Phys.~Rev.}{D50}{1994}{5518}; 
	\Name{E710} \REVIEW{Phys.~Rev.~Lett.}{68}{1992}{2433} and \REVIEW{Nuovo Cimento}{A106}{1992}{123}; 
	\Name{D0} D0 Note 6056-CONF; 
	\Name{pp2pp} \REVIEW{Phys.~Lett.}{B579}{2004}{245}

\bibitem{compete} 
	\Name{Cudell~J.~R.~\etal{} (COMPETE Collaboration)}
	\REVIEW{Phys.\ Rev.\ Lett.}{89}{2002}{201801}

\fi

\end{thebibliography}

\end{document}
