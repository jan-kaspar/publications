\section{Data Taking}
\label{sec:data taking}

The measurement presented here is based on data taken in July 2012, during the LHC fill number 2836 providing protons colliding at the centre-of-mass energy $\sqrt s = 8\un{TeV}$. The vertical RPs were inserted at a distance of $9.5$ times the transverse beam size, $\sigma_{\rm beam}$. Initially two, later three colliding bunch-pairs were used, with a typical population of $8\cdot10^{10}$ protons each, yielding an instantaneous luminosity of about $10^{28}\un{cm^{-2}s^{-1}}$ per bunch. The main trigger required a coincidence between the RPs in both arms, combining the near and far units of a station in \textit{OR} to ensure maximal efficiency. During the about $11\un{h}$ long data-taking, a luminosity of $735\un{\mu b^{-1}}$ was accumulated, giving $7.2\cdot 10^6$ tagged elastic events.

%minimal $|t| = 0.027\un{GeV^2}$ % left edge of left most bin


\section{Beam Optics}
\label{sec:beam optics}

The beam optics relates the proton state at the IP to its state at the RP location. At the IP, the direction of a proton can be described by the scattering angle $\theta^*$ (with respect to the $z$ axis) and azimuthal angle $\phi^*$ (about the $z$ axis). Alternatively, the horizontal ($x$) and vertical ($y$) projections of the scattering angle can be used:
\begin{equation}
\label{eq:scatt angle}
\theta_x^* = \theta^* \cos\phi^*\ ,\qquad \theta_y^* = \theta^* \sin\phi^*\ .
\end{equation}
A proton with such scattering angles, emerging from the vertex $(x^*$, $y^*)$ with relative momentum loss $\xi \equiv \Delta p / p$ arrives at the RPs with transverse position
\begin{equation}
\label{eq:prot trans}
	x = L_x\, \theta_x^* + v_x\, x^* + D_x\, \xi\ ,\qquad y = L_y\, \theta_y^* + v_y\, y^* + D_y\, \xi\ ,
\end{equation}
given by the optical functions: effective length $L$, magnification $v$ and dispersion $D$. Since protons lose no momentum in elastic collisions 
(i.e.~$\xi = 0$), the dispersion terms are irrelevant for this process.

For the reported measurement, a special $\beta^* = 90\un{m}$ optics was used, with essentially the same characteristics as at $\sqrt s = 7\un{TeV}$ \cite{epl96}, see Table~\ref{tab:optics} for details. In the vertical plane, it features parallel-to-point focussing ($v_y \approx 0$) and large effective length $L_y$. In the horizontal plane, the almost vanishing effective length $L_x$ simplifies the separation of elastic and diffractive events: any sizeable horizontal displacement must be due to a momentum loss $\xi$.

\begin{table}
\caption{
Optical functions for elastic proton transport. The values refer to the right arm, for the left one they are very similar.
% (after optics matching)
}
\label{tab:optics}
\begin{center}
\vskip-3mm
\begin{tabular}{ccccc}\hline\hline
unit & $L_x$ & $v_x$ & $L_y$ & $v_y$ \cr\hline
near & $\phantom{-}2.45\un{m}$  & $-2.17$ & $239\un{m}$ & $0.040$ \cr
far  & $-0.37\un{m}$ & $-1.87$ & $264\un{m}$ & $0.021$ \cr
\hline\hline
\end{tabular}
\end{center}
\end{table}
