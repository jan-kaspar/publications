\section{Data taking}
\label{sec:data taking}

The measurement presented here is based on data taken in July 2012, during the LHC fill number 2836 providing protons colliding at the centre-of-mass energy $\sqrt s = 8\un{TeV}$. The vertical RPs were inserted at a distance of $9.5$ times the transverse beam size, $\sigma_{\rm beam}$. Initially two, later three colliding bunch-pairs were used, each with a typical population of $8\cdot10^{10}$ protons, yielding an instantaneous luminosity of about $10^{28}\un{cm^{-2}s^{-1}}$ per bunch. The main trigger required a coincidence between the RPs in both arms, combining the near and far units of a station in \textit{OR} to ensure maximal efficiency. During the about $11\un{h}$ long data-taking, a luminosity of $735\un{\mu b^{-1}}$ was accumulated, giving $7.2\cdot 10^6$ tagged elastic events.

%minimal $|t| = 0.027\un{GeV^2}$ % left edge of left most bin


\section{Beam optics}
\label{sec:beam optics}

The beam optics relates the proton state at the IP to its state at the RP location. At the IP, the direction of a proton can be described by the scattering angle $\theta^*$ (with respect to the $z$ axis) and azimuthal angle $\phi^*$ (about the $z$ axis). Alternatively, the horizontal ($x$) and vertical ($y$) projections of the scattering angle can be used:
\begin{equation}
\label{eq:scatt angle}
\theta_x^* = \theta^* \cos\phi^*\ ,\qquad \theta_y^* = \theta^* \sin\phi^*\ .
\end{equation}
A proton emerging from the vertex $(x^*$, $y^*)$ at the angle $(\theta_x^*,\theta_y^*)$ and with momentum  $p (1 +  \xi)$, where $p$ is the nominal initial-state proton momentum, arrives at the RPs in a transverse position
\begin{equation}
\label{eq:prot trans}
	x(z_{\rm RP}) = L_x(z_{\rm RP})\, \theta_x^*\ +\ v_x(z_{\rm RP})\, x^*\ +\ D_x(z_{\rm RP})\, \xi\ ,\quad y(z_{\rm RP}) = L_y(z_{\rm RP})\, \theta_y^*\ +\ v_y(z_{\rm RP})\, y^*\ +\ D_y(z_{\rm RP})\, \xi \quad
\end{equation}
relative to the beam centre. This position is determined by the optical functions: effective length $L_{x,y}(z)$, magnification $v_{x,y}(z)$ and dispersion $D_{x,y}(z)$. 
The relative final-state momentum deviation $\xi$ has the following 
contributions:
\begin{itemize}
\item Beam momentum offsets $\xi_{\rm off}$ relative to the nominal momentum 
and time-dependent variations, $\xi_{\rm var}$, with 
$\sigma(\xi_{\rm off}) \sim 10^{-3}$ and $\sigma(\xi_{\rm var}) \sim 10^{-4}$ 
(see discussion in Section~\ref{sec:beam en unc}).
\item The momentum loss, $\xi_{\rm scatt}$, in diffractive scattering processes.
\end{itemize}
For elastic scattering the dispersion terms, $D_{x,y}\, \xi$, can be ignored: 
\begin{itemize}
\item The protons lose no momentum in elastic collisions 
(i.e. $\xi_{\rm scatt} = 0$).
\item Due to the collinearity of the two elastically scattered protons and 
the symmetry of the optics of the two beams, the effects of 
beam energy deviations ($\xi_{\rm off}$ and $\xi_{\rm var}$) on the 
reconstructed scattering angle (Eq.~(\ref{eq:th t}) in 
Section~\ref{sec:kinematics}) are strongly suppressed. Residual effects from 
optics imperfections have been verified to be negligible compared to all other 
uncertainties.
\end{itemize}


For the reported measurement, a special optics with $\beta^* = 90\un{m}$ was used, with essentially the same characteristics as at $\sqrt s = 7\un{TeV}$~\cite{epl96}, see Table~\ref{tab:optics} for details. In the vertical plane, it features parallel-to-point focussing ($v_y \approx 0$) and large effective length $L_y$. In the horizontal plane, the almost vanishing effective length $L_x$ simplifies the separation of elastic and diffractive events: any sizeable horizontal displacement must be due to a momentum loss $\xi$.

\begin{table}
\caption{
Optical functions for elastic proton transport. The values refer to the right arm; for the left arm the moduli are very similar, but $L_{x}$ and $L_{y}$ have
the opposite sign.
% (after optics matching)
}
\label{tab:optics}
\begin{center}
\vskip-3mm
\begin{tabular}{ccccc}\hline\hline
RP unit & $L_x$ & $v_x$ & $L_y$ & $v_y$ \cr\hline
near & $\phantom{-}2.45\un{m}$  & $-2.17$ & $239\un{m}$ & $0.040$ \cr
far  & $-0.37\un{m}$ & $-1.87$ & $264\un{m}$ & $0.021$ \cr
\hline\hline
\end{tabular}
\end{center}
\end{table}
